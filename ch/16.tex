\chapter{Introductory Chapter}

\begin{chapquote}{Author's name, \textit{Source of this quote}}
``This is a quote and I don't know who said this.''
\end{chapquote}

Una nave da crociera solcava pigramente il Mar Mediterraneo. Era una bellissima giornata, fredda ma soleggiata. Tutti i passeggeri erano intenti a godersi il viaggio, osservando la costa, il mare o dilettandosi in qualche attività sul largo ponte scoperto; tutti, tranne uno.

Fabio fissava pensieroso un minuscolo taccuino. Il vento spiegazzava incurante quelle piccole pagine, gli schizzi delle onde ogni tanto le bagnavano; lui ignorava tutto, immerso nelle parole confuse che aveva scritto.

Regola 27: le buone maniere sono gratis, abusane più che puoi.

Qualcosa non gli tornava. Pensò ancora qualche istante, poi tracciò altri segni confusi.

Regola 28: i tuoi interessi invece sono cari, molto cari; non sacrificarli in nome della cortesia.

Le cose che non gli tornavano continuarono a non tornargli, anzi, si ingarbugliarono ancora di più. L'istinto gli diceva di fondere quelle due regole in una sola, eppure le varie combinazioni che gli venivano in mente somigliavano troppo ad una citazione che aveva letto da qualche parte. Avrebbe quasi voluto riportarla per intero, ma non ricordava la formulazione esatta, tanto meno sapeva a chi attribuirla. Forse sarebbe stato meglio lasciare le due regole divise, rendendo però la ventotto un comma della ventisette\ldots

Ponderò relativamente a lungo la faccenda, poi decise di lasciar perdere. Non c'era bisogno di impazzire per questioni del genere, non quando stava già impazzendo senza fare alcun ulteriore sforzo. Non stava stilando la costituzione mondiale: non avrebbe dovuto esitare nemmeno un secondo nell'andare avanti. Gli venne un'intuizione e prese a scribacchiare.

Regola 29: esitare è facile, ma raramente è utile.

Quella non era decisamente farina del suo sacco, ma non poteva costringersi a ricordare dove la aveva letta. Obbedendo alla regola, non esitò e passò oltre. Gli venne subito un'altra idea.

Regola 30: finché qualcuno non reclama una proprietà intellettuale, è tua.

« Che scrivi adesso? » flautò una voce da un altro mondo.

La bolla di follia in cui Fabio si era isolato scoppiò, rivelando i dintorni. Comparvero una ragazza magra e bionda, il ponte di una nave ed il mare.

« Le stesse cose di stamani » sospirò Fabio, sconsolato.

Strappò l'ultima pagina del suo taccuino e se la cacciò in bocca.

« Ancora le tue regole? Non stai facendo molti progressi, vero? » tubò Vittoria.

Fabio non si disturbò a rispondere: era troppo intento a masticare i suoi scritti.

« Io non me ne intendo », mise le mani avanti lei, « ma\ldots non faresti meglio a lasciar perdere? Va bene pensare, ma figa a una certa anche basta. »

Fabio interruppe la masticazione e la fissò per qualche istante.

« No, non scriverlo, ti prego! » si affrettò ad aggiungere lei. « Dai, molla quel quadernino. È da stamattina che sei in sbatti, oggi non ti ho visto sorridere nemmeno una volta. »

La replica arrivò fin troppo in fretta:

« Vonvollio vovviee\ldots puh! Non voglio sorridere! »

« E invece sorriderai, perché io ti voglio vedere contento! Facciamo qualcosa insieme\ldots è soleggiato, il mare è bello, non fa troppo freddo\ldots facciamoci mezza goccia in due! Ci passa prima ancora di arrivare! »

Fabio rifletté per un istante. Senza pensarci, strappò un'altra pagina di scritti e fece per sbranarla; Vittoria lo bloccò repentinamente. Lui non protestò, ma rispose:

« Tu fai pure, io non credo proprio di volermi buttare in mare dalla disperazione. Non ho dimenticato quanto ho sofferto l'ultima volta. »

Vittoria si incupì.

« Però hai dimenticato quanto siamo stati bene prima. »

Fabio sospirò. Esitò per dei lunghi istanti prima di rispondere.

« Ti sbagli, non l'ho affatto dimenticato. Ma su una cosa hai ragione, devo proprio prendermi una pausa dai miei pensieri: andiamo a prendere una cosa da bere e godiamoci il viaggio. »

Il bar era ben fornito e la crociera poteva rivelarsi molto piacevole, ma Fabio non riusciva proprio a stare sereno. C'era qualcosa che non tornava nel loro viaggio; anche se non riusciva a formulare concretamente il suo dubbio, avvertiva distintamente la sensazione di stare commettendo un grave errore.

Eppure sembrava andare tutto per il meglio: all'imbarco nessuno aveva chiesto documenti, avevano preso il biglietto \- caro asserpentato, non aveva mancato di commentare Fabio - e via. Riguardo al loro intento di usare la crociera per emigrare da clandestini, il piano era così semplice da non poter andare storto: quando la nave avrebbe fatto porto per i rifornimenti, loro sarebbero scesi di soppiatto. Se qualcuno li avesse beccati, sarebbe stato messo in condizione di non parlare. Non era certo quello che lo preoccupava; aveva alle spalle esperienze così intense da rendere lo sfuggire alla sorveglianza di una nave veramente una bazzecola.

Ma questa faccenda di Leghorn semplicemente non tornava. C'era un'immagine che continuava a comparire nella testa di Fabio, un ricordo di un cartello bianco con scritto proprio quel nome. Che fosse dannato se fosse riuscito a ricordarsi dove diavolo lo aveva visto. Avrebbe giurato di non aver mai incontrato il nome di quella città in tutta la sua vita. Che poi, Leghorn\ldots Gamba-corno? Che sarebbe dovuto significare?

Il comandante della nave aveva parlato varie volte dagli altoparlanti, illustrando le varie caratteristiche delle località che stavano costeggiando, ma Fabio non aveva capito molto. Sarà stato l'inglese strascicato dell'arcigno catalano, o il rimbombo metallico che accompagnava ogni comunicazione, fatto sta che in media Fabio era riuscito ad intendere una parola su dieci. E quelle poche che era riuscito ad udire correttamente, non gli tornavano per niente.

« Brr - Bzz - Mar -sei - krrrrr »

« hhhhgggg - Naiss - ssssssrrrrr »

Marsiglia? Nizza? Che diavolo ci facevano sulle coste della Francia? Si sarebbe aspettato che gli venisse indicato Valencia, Gibilterra e poi qualche posto in Portogallo, prima che la nave facesse finalmente porto a Leghorn. Avrebbe pensato di aver capito male, da quell'impianto venivano fuori così tanti ronzii che non si sentiva effettivamente proprio niente, però, stando sulla prua della piccola nave, la costa rimaneva a sinistra. Se due indizi facevano una prova, stavano proprio costeggiando la Francia, diretti verso\ldots

Nonostante Fabio cercasse di ignorare tutto questo con tutte le sue forze, l'illuminazione si innescò quando la voce metallica - tra una robotica gracchiata e l'altra - annunciò abbastanza chiaramente che stavano costeggiando il Principato di Monaco. Dovette accettarlo: non stavano andando in Inghilterra. Ma allora, in nome del cielo, dove stavano andando? Che c'era dopo la Francia? Cosa avrebbe potuto chiamarsi Leghorn?

Era appoggiato sul bordo di prua quando finalmente lo capì, e per poco non cadde in mare. Quel cartello nella sua testa\ldots Era un ricordo vecchio di quasi vent'anni. In automobile, al porto di Olbia, lui giocava con il Game Boy, non riusciva a sconfiggere la settima palestra, suo padre che bestemmiava in cerca dell'imbarco corretto...

Non stavano affatto andando in Inghilterra. Quanto era stato stupido, era ovvio, ovvio! Non era proprio possibile andare in Inghilterra da Barcellona in così poco tempo! Come aveva fatto a convincersi del contrario, come era stato possibile ingannarsi così?

« Uuuh, guardaaaa! Io lì ci sono stata! » civettò Vittoria, anche lei abbandonata sul bordo di prua.

« Tra poco saremo in posti dove sono stato anche io » mormorò Fabio con voce funerea.

Lei lo ignorò, completamente presa dal paesaggio.

« Lì c'era l'albergo dove eravamo! Abbiamo visto il Gran Premo dalla terrazza di camera! »

Fabio bestemmiò. L'unica volta che aveva assistito dal vivo ad un gran premio di Formula 1, aveva speso una cifra che per lui era un patrimonio per stare in un prato fangoso, facendo a cazzotti per intravedere mezza curva da una rete.

« Dai, non fare così » gli disse la ragazza, appoggiandosi a lui. « Lo stiamo facendo, stiamo andando in un posto nuovo a rifarci una vita! Sei sempre\ldots triste? »

Fabio scrutò l'orizzonte. La sua espressione era così dura che sembrava scolpita nella roccia. Dopo una lunga pausa meditabonda, rispose:

« Mi dispiace fare il bastian contrario, ma ogni cosa che hai detto è sbagliata. Non stiamo andando in un posto nuovo. Non ci rifaremo una vita e no, non sono triste. »

Lei lo abbracciò. Lui non ricambiò l'abbraccio, ma neanche lo respinse.

« A me invece non dispiace, e ti dico che non mi importa niente di quello che hai detto. Tranne che non sei più triste. »

Fabio ignorò il moto di insofferenza che lo investì. Replicò secco:

« Non sono neanche felice, se ti può interessare. »

Lei lo strinse a sé ancora di più.

« Vedrai che fra poco starai meglio. Saremo in una città nuova, nessuno ci darà fastidio, saremo solo io e te. Mi prenderò cura di te così bene ti farò passare la voglia di essere cattivo. »

Fabio provò l'impulso di fingere un conato di vomito, ma lo soppresse. Riluttante, ricambiò l'abbraccio e tacque. Lo aveva scritto qualche ora prima, in uno degli innumerevoli foglietti finiti in mare o nel suo stomaco: non c'era alcun vantaggio nel ferire gratuitamente i sentimenti della ragazza. Presto avrebbe capito da sola l'errore che avevano commesso.

***

Le ore trascorsero; il sole sprofondò nel mare, lasciando spazio ad un magnifico cielo stellato. Quasi tutti i passeggeri si erano ormai rifugiati sottocoperta, l'aria era pungente. La voce metallica del comandante gracchiò:

« kkkaaa- Leee - ghorn - drrrggoff the ship for any rrreaaonzzz - brrr »

Enio Filippi dormiva sodo negli alloggi riservati all'equipaggio. Un collega lo svegliò senza tante cerimonie: toccava a lui fare rifornimento. Enio maledisse il clero, il cielo e varie divinità, ma tutto sommato si alzò di buon umore: era nella sua Livorno, e fare porto a casa dava sempre una bella sensazione.

Caricare i rifornimenti era sempre un letale mix di fatica e noia: lo sforzo fisico impediva di fare quattro chiacchiere coi colleghi, la noia costringeva la mente a vagare in cerca di qualcosa con cui intrattenerla. Quella sera poi, con l'equipaggio ridotto al minimo, le sue braccia erano le uniche su cui poteva contare per tirar dentro le casse lasciate lì sula banchina dal fornitore. Ma Enio era un uomo di sostanza, e non si sarebbe certo tirato indietro di fronte ad un compito sì ingrato, ma tutto sommato banale. Così, cuffie nelle orecchie e maniche arricciate nonostante il freddo, il suo corpo si muoveva automaticamente, come un meccanismo ben oliato che scorreva nelle guide scavate da una ferrea memoria muscolare. Immerso in quel trance lavorativo, per Enio sarebbe stato molto facile perdersi qualche dettaglio di ciò che succedeva in quella stiva buia.

Ma una bella biondina che sbirciava oltre una pila di bancali non poteva cerco passare inosservata. Trattenendo a stento un sorriso, mollò sul posto la cassa di bibite che aveva fra le braccia e le andò incontro.

« Bimba, 'sa ci fai vi? Devi tornà di sopra coll'altri! » fece Enio, tirandosi via un auricolare senza tante cerimonie.

Lei lo guardò con un'espressione indecifrabile.

« Sei sporco » gli disse sognante. « Hai un odore strano e parli in modo buffo. Sei interessante. Ora però, foeura di ball!»

Enio rimase completamente senza parole.

Mentre pensava come replicare a delle affermazioni del genere, sentì un tocco sulla spalla che lo fece trasalire; si voltò di scatto e vide una figura maschile, seminascosta dall'ombra. Enio odiava essere spaventato. Sbottò:

« Dé, ma se' scemo? Mi fai veni' un coccolone, te e quer tegame di tu ma'! »

Il figuro rise.

« Bel tentativo, Vittoria. Ma devi parlare nella sua lingua per farti capire. »

... « 'canzati, o piglio 'l remo », ordinò.

...

« Molto bene, io ti avevo avvertito. Ora piglio 'l remo. »

Punto di cista di due mozzi aggrediti.

Ma chi é sto boia?

Alba sul terrazzo mascagni, edicola, bagonghi morto.

Decidono di tornare a casa

