\chapter{Flashback: il Fumo}

\begin{chapquote}{Willie Peyote, \textit{Selezione Naturale}}
	E `sti cazzi la legge di Murphy\newline
	Quando mai vincono i bravi ragazzi?\newline
	A volte il destino è crudele, va bene\newline
	Ma a volte sei tu che ti meriti schiaffi
\end{chapquote}

% pulled
% tex syntax

%     editing in corso:
% troppi punti di sospensione, levane un po' - WIP
% occhio alla punteggiatura

Fabio era totalmente stupefatto. Com'era bello fluttuare fra i meandri della mente! Colori, suoni, brividi, faceva tutto parte di un unico, vaporoso maelstrom psichedelico. Lentamente, quel mondo bizzarro cominciò ad addensarsi in una sorta di etere colorato, che a sua volta si ridusse gradualmente in foschia; non più un mistico artefatto della mente, ma una densa, puzzolente nebbia concreta.

La stanza che la nube nascondeva era così piccola che una sola sigaretta la aveva trasformata in una camera a gas. Fabio odiava il puzzo del tabacco; aprì una finestra e la nebbia sparì. Comparve una ragazza fulva, seduta ad un tavolo, intenta a girare la manovella di una specie di macinino. Il suo volto era confuso, gonfio, stanco, ma i suoi occhi verdi erano vispi.

«Denise» disse Fabio, sedendosi, «Che ho fatto di male?»

«Mah, un sacco di cose direi» rispose vaga lei, mentre rovesciava il contenuto del dispositivo su una specie di tagliere in ceramica. «Bevi, fumi, ti droghi, se puoi rubi, forse uccideresti. Fai tante cose di male, Fabio.»

«Quindi mi merito tutto questo?»

«No, non l'ho detto. Mi hai chiesto cosa hai fatto di male, e io\ldots \space tutto qui.»

Fabio sospirò. Appoggiò la vita alla finestra, si sporse leggermente e guardò fuori. La luna splendeva debolmente in uno specchio d'acqua come nelle migliori cartoline suggestive, totalmente incurante di riflettersi in una pozza rimasta su un piazzale di fronte ad un capannone. Questo contrasto, la bellezza della luce lunare e il grezzo mondo del lavoro, faceva spaziare l'atmosfera dal romantico al sinistro. ``Le zone industriali'' pensò Fabio, incapace di non dare un giudizio a quel luogo, ``sono strane per viverci, e questo condominio è peggio che mai. Di giorno è accerchiato dal trambusto del lavoro, di notte è immerso nell'abbandono\ldots \space mah, oddio\ldots \space in effetti, potrei trovarci del poetico in una casa come questa, piccola, in alto\ldots \space potrebbe sembrare quasi un rifugio, un nido\ldots''

I suoi pensieri erano così profondi che si era addirittura dimenticato che essere in compagnia. Una voce lo riscosse:

«Che guardi?»

«È tetro, fuori» disse Fabio, senza voler significare granché.

«Già», rispose Denise, «è sempre tetro fuori. Ma rintanarsi non serve, quando il tetro è dentro.»

Fabio sospirò ed annuì. Era troppo stanco per distinguere i discorsi sensati dall'aria fritta. Abbandonò il suo avamposto alla finestra e si lasciò andare, inerme, sulla sedia.

«Se si fa questa, a casa non ci torni» disse Denise, mostrando la sua creazione.

Fabio sorrise amaramente, prima di rispondere.

«Non voglio tornare a casa.»

Si frugò in tasca in cerca di un accendino. Denise gliene indicò uno da tavolo, poi lo guardò negli occhi; con voce un po' impastata, ma ferma, gli disse:

«Si capisce. Ho sentito tanta roba, Fabio. Mi dispiacerebbe se fosse vera anche solo la metà. In ogni caso, per me puoi rimanere fino all'ora che vuoi. Ma\ldots \space non lavori domani?»

«No», rispose lui amaramente. «Non ha senso lavorare. Non se non mi pagano, almeno.»

«Allora è proprio definitivo? L'avevo sentito dire dal Bagonghi, ma credevo fosse una cazzata.»

«No, è tutto vero. Si chiude. La crisi, il costo del lavoro\ldots \space le solite cose di tutti. Era ovvio toccasse anche a noi, prima o poi.»

Denise storse la bocca in una smorfia. Sembrava sinceramente preoccupata.

«Bruno non può trovarti qualcos'altro?», chiese. «Così, intanto che ti cerchi qualcos'altro.»

Fabio scosse la testa e sospirò.

«Gli ho già chiesto tanti favori, anche lui è nei guai\ldots \space Non posso mettermici anche io.»

L'accendino cliccò, ed apparve il fuoco.

«Comprensibile», rispose Denise in uno sbuffo di fumo.

Afferrato un posacenere, la ragazza si spostò sul divano. Fabio la seguì, accasciandosi dalla parte opposta.

«Scusa se non sto composto. Spero di non scalfire l'immagine dignitosa che hai di me» scherzò, più per abitudine che per sentimento.

Denise rise, porgendogli il magico artefatto.

«Idiota», gli fece, «negli anni d'oro ti ho visto fare cose che - va beh, lasciamo perdere che mi viene l'urto del vomito.»

«Osi insinuare che talvolta ho avuto dei comportamenti non appropriati?»

«Insinuo, eccome!»

«Allora vai ad insinuare in bagno, che qui c'è gente impressionabile. Piuttosto, senti una cosa\ldots»

% rough editing fino qui

Fabio esitò. Doveva chiederglielo --- anche se la risposta era ovvia --- ma si sentiva un po' in imbarazzo.

«\ldots posso davvero rimanere qui? Tanto tra un po' ribalto per terra e svengo, non ti do noia.»

Prima di rispondere, Denise gli si avvicinò. Mise una gamba sopra le sue, usandolo a mo' di poggiapiedi, e gli fece fare un tiro di mano sua.

«Fabio, te rimani qui e dormi nel letto, non per terra. Dani rimane fuori tutta la notte, c'è tutto il posto che vuoi! La Lavinia è a giro?»

«Ah boh», fece sconsolo lui. «In tutta sincerità non ne ho idea. Ci parliamo il minimo indispensabile, non voglio proprio sapere dove va. Ma non è per lei, né per Dani. Tanto lui sverrà da qualche parte, come sempre» disse Fabio, senza riuscire a trattenersi.

Lei lo guardò interrogativa. Fabio proseguì:

«E' già un paio di volte che lo trovo appoggiato al muro del piscio alle cinque. Una volta quella fava del Gazzi gli ha anche pisciato addosso.»

«Ma che\ldots?»

«Tranquilla, appena mi sono accorto che lo stava facendo davvero gli ho fiondato la testa sul muro!»

Denise passò la gioia, perplessa.

«Credevo che, almeno, andasse a puttane. Mi aveva detto che aveva smesso\ldots»

Fabio si pentì, con qualche frase di ritardo, di aver toccato quel tasto.

«Mi dispiace» disse, desideroso di cambiare discorso il prima possibile.

«Di cosa, di avermi detto la verità?»

«Sì, esattamente. Odio essere l'ambasciatore di queste cose.»

«Tranquillo, non dirò a Dani che te lo sei lasciato sfuggire» sibilò lei, maliziosa.

Fabio tirò un sospiro di sollievo, senza nemmeno premurarsi di nasconderlo. Denise ridacchiò e lo abbracciò.

«Tu sì che sei un ubriacone responsabile. Sai, avrei preferito cento volte te a Daniele» gli disse, mordicchiandogli un orecchio.

«Maledetta» sibilò Fabio, liberandosi stizzito dalla presa. «Infili il dito nella piaga?»

«Tu lo hai infilato nella mia» rispose lei, senza smettere di sorridere.

«Ma non l'ho fatto apposta!»

«Certo che l'hai fatto apposta. Ricordi?»

Fabio fissò stupidamente il vuoto. Parlare con Denise in quelle condizioni era oltremodo faticoso: discorso fine a sé stesso o cosa sensata? L'intuizione lo colpì come un fulmine, ed un sorriso esasperato gli si dipinse sul volto.

«Che arguta allusione, Denise. Così scontata e banale da nascondere sicuramente un messaggio subliminale. Sappilo: non è il caso.»

Neanche questa risposta le spense il sorriso, anzi, la fece ridacchiare.

«E' sempre il caso. Siete schiavi del vostro istinto, voi maschi. Basta solo sapervi accendere e poi\ldots fate il vostro», sentenziò.

La sigaretta di droga esaurì il suo potere, e finì accartocciata senza pietà nel posacenere. Denise si alzò a fatica dal divano, gonfia come una zampogna ma sorprendentemente ben ferma sulle gambe. Con confusa lentezza, riempì due calici con del vino bianco.

«Vermentino» disse, porgendone uno a Fabio, il quale borbottò un grazie con voce funerea.

Lei rimase a fissarlo per un po', in silenzio. Poi sbottò:

«Che hai? Ti è morto il gatto?»

Effettivamente, la faccia di Fabio era più adatta ad un cimitero che ad un after party a due. Tracannò mezzo bicchiere in pochi sorsi. Poi rispose:

«Io sono morto, non il gatto.»

Denise assunse un'espressione dura, come se stesse impiegando ogni suo grammo di forza per evitare di alzare gli occhi al cielo. Piombò accanto a Fabio e lo costrinse a bere il resto del suo bicchiere, per poi riempirglielo di nuovo. Lui, intontito dall'alcool e dalla droga, non riusciva a fermare il suo flusso di pensieri. ``Altro che schiavi dell'istinto'' pensò, ``siamo talmente stupidi da sprecare la vita per una persona, la mettiamo al centro del mondo, lavoriamo sodo per costruire un futuro insieme\ldots e poi\ldots Ma gli altri no, non sono come me'' disse a sé stesso, in una disperata presa di coscienza. Il fumo continuava a fluttuare nella stanza, e lui lo osservava. Gli pareva di vedere ciò che pensava in quelle grigie spire. ``Gli altri maschi non sono così\ldots Sono io quello stupido. Dovrei andare a caccia, invece di fare il nido. Dovrei essere un predatore\ldots Dovrei farlo?``

«Ooh! Fabio! Ce la fai?! E' dieci minuti che parlo da sola!» sbottò Denise, strappando via Fabio dalla sua mente riflessa nel fumo.

«Eh? Oh\ldots Scusa\ldots Mi sento un po'\ldots stupefatto\ldots» rispose, incespicando nelle parole.

«Bene no?»

«Credo di sì\ldots Perché non dovrebbe essere bene?»

Ma non era bene, neanche un po'. La droga lo aveva reso debole, fragile, incapace di proteggersi dal giudizio di sé stesso. Quella situazione, quell'intimità\ldots Non aveva del fracasso in cui nascondersi, una rissa con cui distrarsi o un Gazzi da odiare. Era nudo al cospetto della sua anima. Bevve un ennesimo, generoso sorso dal bicchiere che teneva in mano. Non riusciva a smettere di pensare, di riflettere su tutto ciò che lo aveva reso così miserabile, di tormentarsi per la sua incapacità di impedire il degenerare degli eventi\ldots

«Se almeno avessi ancora qualcuno a cui aggrapparmi\ldots» pensò Fabio ad alta voce.

Denise gli scoccò uno sguardo perplesso.

«Stai pensando a lei?» chiese, dopo una lunga pausa.

«Si\ldots è lei, cazzo, il tassello mancante! Ho investito tutta la mia giovinezza su di lei, per cosa? Per ritrovarmi ad affrontare questa merda da solo? Sarei riuscito a sopportare tutto, ad affrontare tutto! Ma così\ldots è come se mi mancasse la terra sotto i piedi\ldots»

Aveva detto troppo. Non doveva parlare quando era stordito, se l'era ripetuto infinite volte in tutta la sua vita. Ma a che servono gli ammonimenti quando, in preda alla confusione indotta dalle sostanze, si riesce a malapena ad esistere?

Denise gli tolse di mano il bicchiere piuttosto bruscamente.

``Eccoci, ora si è offesa!'' pensò Fabio, con una punta di panico. ``Non potevo stare zitto? Perché non riesco a stare zitto? Sono proprio un coglione\ldots''

Ma non si era offesa. Stava trafficando goffamente intorno ad un mobile a vetro, dal quale riuscì finalmente a tirare fuori una bottiglia di liquore ambrato. Neanche il tempo di tirare un sospiro di sollievo, che Fabio si ritrovò in mano un bicchiere colmo di profumato amaretto.

«Amaretto di Saronno» disse lei, dolcemente. «Non è la cosa più elegante e pregiata del mondo, ma se hai `ste turbe per i' capo ci vuole qualcosa di più forte del vino.»

Fabio ne bevve un sorso. Il buon sapore di amaretto, quel mix di dolce e mandorla amara, gli disegnò sul volto una strana espressione.

«Non ti garba?», chiese Denise.

«Lo devono ancora inventare lo spirito che non mi garba», sospirò sconsolato.

Bevve un altro, generoso sorso.

«Sono proprio un alcolizzato di merda» pensò ad alta voce, fissando per qualche istante il vuoto.

Dopo quelli che gli parvero parecchi minuti, Denise gli parlò all'orecchio:

«Sì che lo sei. Ma l'alcool e la droga non ti bastano, vero? Ne vuoi sempre di più. Non ne hai mai abbastanza per calmare le tue angosce\ldots»

Aveva una voce stranamente diversa dal solito, dolce ma pungente; lo intrigava. Lei gli sfiorò il collo con le labbra. Fabio sentì qualcosa muoversi in zone oscure. Possibile che\ldots?

«Sì» la assecondò Fabio, imitando il suo tono suadente senza neanche rendersene conto. «Non ne ho mai abbastanza\ldots Ne faccio un'altra.»

Non sapeva perché stava rifiutando così stupidamente le avances di Denise, nonostante ai fini pratici le apprezzasse. Era veramente ancora così legato a Lavinia, dopo tutta l'indifferenza e l'incertezza che aveva sopportato? O forse era fuso dalle varie droghe che aveva assunto, legali e non, e quindi semplicemente non capiva più niente? Perplesso da sé stesso, cominciò a produrre. Alzò lo sguardo dalla sua attività solo a lavoro ultimato, accorgendosi che Denise lo stava osservando smarrita. In uno slancio di acidità, Fabio suppose che con tutta probabilità nessuno prima di lui aveva mai rifiutato le attenzioni della bella Denise con così tanta indifferenza. Si sentì immediatamente stupido, ma non poteva farci niente. Ogni boccata di fumo o sorso di liquore lo rendeva sempre più triste ed incapace di controllare sia la sua mente che il suo comportamento.

Come se avesse indovinato i suoi pensieri, Denise ruppe l'imbarazzante silenzio che era sceso nella stanza.

«Capisco. Io\ldots se vuoi ti ascolto».

Non voleva riversare se stesso su di lei, ma non poteva impedirselo. Prima che potesse pensare qualsiasi cosa, cominciò a parlare.

«Cosa vuoi che ti dica? Sai già più o meno tutto, no? Forse ne sai addirittura più di me\ldots L'ho sempre data per scontata, sai? Siamo insieme da millenni, ormai. Senza di lei, tutto mi crolla addosso. Credevo che mi amasse e che tenesse a me. Credevo che ci sarebbe sempre stata. Nonostante tutto\ldots»

Fece una pausa e inspirò. Tracannò disperatamente l'amaretto che gli rimaneva nel bicchiere, quasi strozzandosi. Denise gli si avvicinò per battergli una forte pacca sulla schiena, ma non gli tolse il braccio di dosso quando lui riprese a respirare normalmente. Si accese la sigaretta corretta e, come posseduto, continuò.

«Il lavoro mi ha sempre fatto schifo. Volevo riprendere a studiare, volevo prendere la situazione in mano, volevo un sacco di cose\ldots Ora che assisto impotente al crollo di tutto quello che abbiamo costruito in questi anni, schifo o non schifo, mi sarei persino messo a pregare pur di continuare ad avere una sicurezza economica. Al cesso i sogni e le aspettative\ldots Tanto ci sarebbe stata lei. Avrebbe sostenuto tutte le mie scelte, mi avrebbe dato quella sicurezza che mi serve per affrontare il mondo\ldots»

Fece una pausa per fare un tiro. Denise ne approfittò subito:

«Era tutto rose e fiori? Davvero?»

Le sue mani in qualche modo erano finite sul petto di Fabio e lo stavano accarezzando dolcemente. Lui le ignorò e proseguì:

«Col cazzo. Litigavamo, si\ldots Eccome se litigavamo. Ma poi bastava guardarci per capire che ci si stava scannando per cazzate, e che quello che ci legava era molto più forte di quello che ci stava dividendo. Ma ora è tutto diverso. Ora è tutto freddo, statico, indifferente\ldots Preferirei che ci prendessimo a pedate, piuttosto.»

«Si capisce» rispose Denise con dolcezza. «Dai, vieni a stenderti, sei un pezzo in là. Continuo ad ascoltarti, se vuoi\ldots»

Ma Fabio non si mosse né parlò. Il pensiero proibito, quello che tanto a lungo aveva provato a nascondere a sé stesso, prese forma. Perché sì, era come aveva detto, lei era fredda, statica, indifferente; ma al puzzle della realtà mancavano ancora diversi tasselli, uno dei quali era che Lavinia, la sua Lavinia, si comportava in quel modo solo con lui. Mentre la sua coscienza si spegneva, corrosa fino all'osso dalla combinazione di sostanze che aveva assunto, la sua immaginazione gli dipinse con crudele realismo l'incubo che lo tormentava ormai da tempo: la sua amata Lavinia e la sua nemesi Gazzi, insieme. Tese le mani avanti, cercando di afferrare i due spettri per dividerli, per riappropriarsi di ciò che gli apparteneva, per strozzare chi aveva osato tanto. Si sentì pervaso da un tremendo senso di impotenza\ldots

Udì una voce in lontananza:

«Che fatica, `sti òmini disperati\ldots»

Si accorse di essere trascinato, incapace di opporsi a qualunque evento stesse accadendo. Accettando l'oblio come unica salvezza, si arrese: tutto sparì, inghiottito da un profondo abisso oscuro.
