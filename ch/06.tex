\chapter{Rinascita}

\begin{chapquote}{Author's name, \textit{Source of this quote}}
``This is a quote and I don't know who said this.''
\end{chapquote}

% pulled
% tex syntax

%     editing to do:
% sezione in cui descrive il volto scorre poco, migliorare
% ripetzioni: momento, ora, rinascita

Il sole era sorto da pochi minuti e Fabio già non lo sopportava più.

«Abbozzala  di splendere, coso giallo!» inveì, la voce ancora impastata dal sonno.

Schiuse con cautela gli occhi, tentando di abituarsi alla luce che inondava quella che adesso era casa sua.

A chiamarla casa ci voleva un bel coraggio: era un vecchio, osceno appartamento dove caos, sporcizia e trascuratezza regnavano sovrani. Il pavimento era costellato da un inquietante misto di cianfrusaglie e rifiuti; poteva far invidia ad un accumulatore seriale come ribrezzo ad una persona più o meno sana. L'arredo principale era un vecchio tavolo traballante, che occupava da solo quasi metà dell'unica stanza; un bell'oggetto, finemente decorato con scatole di medicine, buste d'erba vuote e riviste ormai imparate a memoria.

Fabio richiuse gli occhi, immaginando di trovarsi su un lettino in riva al mare. Il trucco non funzionò: rinunciò all'idea di riaddormentarsi e si costrinse ad accettare di essere sveglio.

Sconsolato, si stiracchiò lentamente e fece per sedersi, ma imprecò dal dolore e si alzò di scatto: si era involontariamente grattato l'orecchio, togliendosi la crosta del taglio. Corse subito verso lo specchio per controllarsi la ferita, ma inciampò nei suoi pantaloni, abbandonati per terra ormai da qualche giorno.

«Mmmhrrraaaw!»

Ringhiando furioso, si rimise in piedi. Ma la repentina ascesa infierì sulla sua pressione ancora bassa: barcollò goffamente, prima di piombare di nuovo sul pavimento.

«Mapporcodd\ldots!»

Dopo essersi sfogato offendendo varie divinità alle quali non credeva, si calmò. ``Non è proprio giornata'' pensò, rialzandosi a fatica e raggiungendo finalmente la parete opposta a quella del suo letto; lì aveva appeso un minuscolo calendario promozionale di una farmacia. Prese un pennarello da una mensola piuttosto sudicia e tracciò la nona, microscopica croce.

Rimase ad osservare stupidamente le caselle segnate per qualche momento, poi, all'improvviso, si sentì pervaso da una potente eccitazione: la convalescenza era finita!

Erano stati nove giorni terribili, costellati da dolore, fastidio, buio, cibo in scatola, puzza, antibiotici e soprattutto dalla più intensa sensazione di noia che avesse mai provato. Certo, i postumi dell'operazione erano stati molto più leggeri di quanto si aspettava, ma restare chiuso in un sudicio appartamento per nove giornate, senza potersi lavare e riuscendo a malapena a dormire era stato più duro del previsto. Si era organizzato: aveva fatto scorta di erba e di riviste da leggere per ammazzare il tempo. Ma la droga, forse per l'interazione con gli antibiotici di profilassi o forse per la solitudine, non lo aveva divertito come avrebbe dovuto. E le riviste, beh, ad un lettore digitale incallito come Fabio erano durate sì e no mezza giornata.

Sospirò. Ormai era finita. Almeno ora, con le ferite ormai del tutto rimarginate, avrebbe potuto concedersi una doccia.

Di umore non proprio terribile per la prima volta dopo tanto tempo, corse nel bagno e si specchiò. Il suo riflesso avrebbe spaventato chiunque, ma lui aveva imparato a direzionare lo sguardo con chirurgica precisione, ignorando la visione d'insieme. Vide che le ferite alle orecchie ormai si erano cicatrizzate. Sorrise debolmente, senza smettere di osservarsi. La garza sul naso non sanguinava ormai da qualche giorno ed il mento era già a posto da un pezzo, nonostante la sensazione di rigidezza che si irradiava a tutta la mandibola. Si tastò delicatamente nel punto in cui qualcosa di molto simile ad uno scalpello aveva plasmato l'osso: vide le stelle dal dolore, ma a parte quello sembrava abbastanza solido, per quanto ne poteva capire lui. ``È davvero finita'', pensò radioso. Si tolse definitivamente la garza e si concesse finalmente un momento di narcisismo.

Un'incudine si formò nel suo stomaco. Trattenne a malapena una smorfia di disgusto, mentre un'amara delusione lo avviluppava. Non riusciva a credere che quello specchio potesse riflettere una tale bruttura senza rompersi per protesta. Nove giorni senza curare il suo aspetto lo avevano reso assolutamente inguardabile. Inoltre, il finto chirurgo aveva fatto veramente un pessimo lavoro con il naso: sembrava aver subito una bella spianata con una lima da ferro, tanto era evidente la fresca cicatrice della rimozione della gobba. Il mento sembrava davvero aver ricevuto un trattamento a suon di martello e scalpello, ma tutto sommato non era sgradevole alla vista; se solo non avesse provocato così tanto dolore ad ogni minimo contatto, avrebbe potuto essere un lavoro quasi accettabile. Le orecchie invece erano qualcosa di inquietante: senza le croste, il taglio della cartilagine non si notava, ma si percepiva comunque un qualcosa di artificiale in quella forma, quasi fosse un bizzarro esperimento di body modification.

Senza pietà, un'antica angoscia colpì Fabio. Prese a tormentarsi la barba, incapace di fermare il flusso di pensieri che lo investiva. Il suo bel volto, molto equilibrato e tutto sommato di bell'aspetto, era stato vituperato da un tizio inquietante per quattrocento euro. Non era tanto il risultato, che alla fine era all'altezza delle aspettative e più che sufficiente a garantirgli quel minimo di camouflage di cui aveva bisogno, ma il fatto di essere stato costretto a sacrificare la sua immagine per continuate a vivere. Aveva ventiquattro anni, era nel fiore della gioventù, nel momento di massimo splendore; aveva sacrificato senza esitazione ciò di cui sarebbe andato fiero da vecchio, il ricordo della bellezza giovanile! Ma in nome di cosa?

La verità di quello che aveva fatto lo colpi come una mazzata sulle gengive. Si appoggiò a peso morto alla parete, lasciandosi sfuggire un penoso lamento. Perché non era stato in grado di reagire alla situazione? Perché era dovuto scappare? Serrò la bocca e respirò profondamente, ma non riuscì a trattenersi: scoppiò a piangere. Guardò il proprio riflesso, sprofondando nei suoi occhi umidi riflessi. Si era fatto mutilare, non era possibile che quella gli fosse sembrata un'idea sensata! Era questo che i suoi cari gli avevano insegnato? Cosa avrebbero detto se l'avessero visto in queste condizioni?

Lo specchio, come un perverso quadro raffigurante il più inquietante dei deformi, rifletteva impassibile il Fabio straziato. ``Perché non ti rompi, specchio?'', pensò lui. ``Perché non protesti per la mia disperazione?''. Fiumi di lacrime gli scorrevano sulle guance, scavando solchi argentei che scintillavano alla luce del sole. Era come su un palcoscenico, esposto al giudizio dell'unico spettatore presente. Quella visione, densa di significato, innescò la bomba accesa ormai da troppo tempo. Un barlume rosso fremette nei suoi occhi\ldots

Sferrò un pugno al suo riflesso, deciso a cancellare dalla storia la sua immagine piangente. Il suo dolore non aveva più ragione di esistere. La sua educazione, la sua visione del mondo, le sue esperienze: tutto cancellato. Era per questo che se ne era andato, che aveva fatto strappare via pezzi di sé stesso. Lui non esisteva più. Lui era la sua volontà, non la sua storia. Sorrise follemente a ciò che rimaneva del suo riflesso. Fino ad allora, senza averlo scelto coscientemente, aveva vissuto osservando rispetto per il prossimo, provando solidarietà per i bisognosi, riponendo fiducia nell'amore. Esattamente come tutti gli altri, come tutti gli stupidi\ldots Ma lui non era stupido. Era arrivato il momento di ammettere l'errore, di voltare pagina, di abbandonare la sua coscienza, la sua moralità! Folle debolezza, gratuito punto debole! Da quel momento, sarebbe diventato un Dorian Gray, un'entità dedita solo al proprio piacere e tornaconto. Un perfetto egoista dawkinsiano, ma in pura chiave edonistica. Da quel momento, tutto sarebbe stato diverso.

Riprese il controllo di sé e si infilò sotto la doccia. L'acqua che gli scorreva addosso sembrò purificarlo, in una sorta di bizzarro battesimo che consacrava la sua rinascita. Si lavò con estrema cura, togliendo ogni traccia della sua convalescenza, e quando ebbe finito si ravversò la barba, cresciuta incolta fino a quel momento, riducendola ad un dignitoso pizzo. Si riflesse in ciò che era rimasto dello specchio, osservando il suo nuovo volto. D'ora in avanti avrebbe mostrato questa faccia al mondo: ne prese atto senza il minimo rimpianto. Si vestì ed uscì, pronto a compiere l'ultimo gesto necessario per la sua rinascita.

Uscire fuori dopo tutta quella prigionia fu un'esperienza fantastica: ogni aspetto del mondo esterno destava in Fabio molto più interesse di quanto non avrebbe fatto in condizioni normali. Il sole che bucava i suoi poveri occhi avvezzi al buio, la perenne brezza salmastra che gli rendeva barba e capelli stopposi, la gente da scansare sui marciapiedi: tutti i piccoli aspetti di Barcellona che aveva odiato appena arrivato, adesso gli apparivano meravigliosi. Sorrise ad un passante e quello gli restituì la cortesia. Era fin troppo felice, pensò. Prese ad aggeggiarsi la barba, ma non compulsivamente come aveva fatto appena un'ora prima: ora accarezzava i peli per il verso, cercando di farli convergere tutti in un punto con movimenti lenti e fluidi. Preso dalla sua nuova forma mentis, avvistò ciò che stava cercando senza quasi rendersene conto.

Era una casottino per le fototessere. ``Al fianco di un palazzo sul lungomare all'angolo di Selva de Mar. Deve essere questo!'' pensò Fabio, con tutta la strana calma che lo contraddistingueva. Cercò nel suo borsello una moneta ottenuta molto tempo prima, simile ai due euro ma costruita con un materiale molto più pesante. Avrebbe funzionato? L'amico del Bagonghi era sembrato affidabile, nonostante fosse un mafioso.

Entrò nella cabina e manca poco svenne per l'aria soffocante che vi era dentro. Era talmente stretta che quasi non riusciva a sedersi sullo sgabellino. Si trovò davanti ad uno schermo in stand-by. Curiosamente, le istruzioni per ottenere le fototessere non erano scritte in catalano e spagnolo, ma in inglese, arabo e cinese. Sempre più fiducioso che quella non fosse una normale macchinetta, Fabio inserì la moneta nera nella fessura per gli spiccioli.

Non successe niente. ``\ldots nna maiala!'' pensò furioso, prendendo a calci e pugni ogni cosa che riusciva a raggiungere. ``Ora t'accomodo a pedate se un tu parti!''

Forse stuzzicato dai colpi, qualcosa dietro lo schermo si mosse. Si sentì un rumore metallico e, con immenso sollievo di Fabio, sulla schermata comparve un conto alla rovescia. Un flash, un rumore di stampante anni novanta ed eccolo: un documento di identità catalano con la sua foto appiccicata e timbrata.

Si chiamava Jorge Pedrosa.
