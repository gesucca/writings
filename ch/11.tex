\chapter{Casa (dolce?) Casa}

\begin{chapquote}{Author's name, \textit{Source of this quote}}
``This is a quote and I don't know who said this.''
\end{chapquote}


% pulled

A Prato il tempo non era mai stato così bizzarro. Aveva piovuto per buona parte della mattina, ma sull'ora di pranzo il sole era tornato prepotente, asciugando tutto in un baleno. Col proseguire della giornata, si era formata una cappa di nubi alte e grigie, dalle quali ogni tanto osavano cadere dei radi goccioloni; l'aria era così rovente e umida da essere irrespirabile.

Giacomo Gazzi stava bollendo. Si sbottonò il colletto della camicia, sperando che il sudore non lo avesse macchiato. Non era solo il caldo a farlo sudare: era molto teso. Aveva sentito voci strane, voci che non gli piacevano affatto. Doveva saperne di più a tutti i costi, ma non poteva dare l'impressione che gli importasse più di tanto. Era in gioco una parte troppo grossa del suo mondo: doveva essere impeccabile per scongiurare il crollo di tutto quello che si era costruito.

Il Bagonghi aveva capito, e già questo di per sé era un enorme problema. Ma non aveva ancora aperto bocca, o almeno così pensava Giacomo. Se l'avesse fatto, le sue parole avrebbero avuto un peso enorme. Insomma, l'amico più intimo di Fabio rende pubblica la sua verità, ma nessuno gli dà credito? Impossibile. Ogni singola sfaccettatura della storia che Giacomo era riuscito a spargere in giro era pura fantasia, senza fonti né referenze. Nel pensiero comune dei pettegolezzi, non avrebbe assolutamente retto il confronto con le confidenze del migliore amico dello scomparso Fontanelli. Il Bagonghi non poteva aver parlato con nessuno, non c'erano dubbi. Però, ora che in giro c'erano delle nuove, scottanti dicerie, il rischio che lui finalmente rivelasse al mondo la reale sorte di Fabio era assai più alto di prima. Una sua parola avrebbe potuto rovinare definitivamente ogni cosa: doveva trovare un modo di metterlo a tacere.

Il cielo si esibì in un fragoroso tuono. I pensieri di Giacomo accelerarono: e se il Brogelli avesse scritto la verità, sulla chat di gruppo? Se Fabio fosse davvero a Barcellona con lui, e stesse sul serio per tornare? Soffocò un'imprecazione. Gli mancava così poco... Ancora qualche settimana e Lavinia avrebbe accettato la scomparsa dell'ormai ex compagno e sarebbe stata pronta a farsi consolare da lui. Il suo sogno non era mai stato così vicino all'essere realizzato. Non poteva permettere alle circostanze di impedirgli di ottenere la felicità. Ma come fare? Si era esposto troppo negli ultimi tempi, non poteva riparare in alcun modo ad un eventuale ritorno del Fontanelli. Non dopo aver giurato di aver avuto contatti con lui. Non dopo averlo dato per tossico, pazzo e pericoloso. Non dopo aver spifferato al mondo che aveva perso la testa per una russa con chissà quante malattie veneree. Non avrebbe avuto modo di salvare la faccia ad una smentita collettiva di tutto ciò che aveva raccontato su di lui, sarebbe dovuto come minimo emigrare dall'altra parte del mondo. La prospettiva era inaccettabile: non lo avrebbe permesso.

Cominciò finalmente a piovere. Violenti goccioloni si infrangevano fragorosi contro l'asfalto. Non c'era ombrello in grado di dare riparo da un simile nubifragio: l'unica soluzione era stare sotto ad un tetto. Lavinia chiuse la tenda, tirando un sospiro. Non ci sarebbe stato bisogno di chiamare il Bagonghi per disdire l'appuntamento: era palese che con quel temporale non sarebbero andati proprio da nessuna parte. Prese lo stesso il telefono, decisa a vederlo; lo avrebbe invitato a casa, o sarebbe andata lei da lui. Doveva incontrarlo ad ogni costo, parlarci faccia a faccia. Doveva vedere la sua reazione all'ultima notizia su Fabio. Sarebbe stato sorpreso? Avrebbe finto di esserlo? Oppure avrebbe bollato l'ultima nuova come semplice pettegolezzo, scaturito dalla fantasia di un povero tossico, come diceva Giacomo? In ogni caso, qualunque reazione avesse avuto Bruno, sarebbe stato un altro prezioso tassello da aggiungere al puzzle generale, alla cui soluzione le sarebbe stata chiara la fine del suo Fabio. Perché era ovvio che il Bagonghi sapesse tutto, tanto ovvio quanto il fatto che il Gazzi non sapesse niente di niente, checché ne dicessero loro. Erano entrambi così scarsi a mentire che inducevano quasi a chiedersi se si stessero davvero impegnando. Per lei, ragazza smaliziata, le espressioni sul volto di quelle due parodie di uomini erano come pagine di libri. Ah, se solo avesse saputo come fare per consultarli a piacimento...

E il pensiero le corse a Fabio, alla persona che era diventato nei mesi precedenti alla sua scomparsa: una mera ombra di quello che un tempo era l'uomo che amava. Magari avesse saputo intravedere anche solo un barlume di emozione sincera dal suo volto! Un messaggio sentito dalle sue parole! Una velata richiesta d'aiuto dal suo linguaggio del corpo, qualsiasi cosa! Era questo che la tormentava più della solitudine, più della mancanza del suo amato, più di ogni altra cosa: e se non fosse stata in grado di capire che il suo Fabio aveva bisogno di lei? Non avrebbe potuto perdonarsi. Ma andava sempre tutto bene: mai un discorso strano, un comportamento insolito, anche solo una smorfia. L'umore di Fabio non era mai stato così... stabile come negli ultimi tempi. Sembrava aver imparato, finalmente, a gestire tutto lo stress e le difficoltà di una vita lavorativa tutt'altro che serena. Le sue reazioni erano sempre proporzionate a quello che gli accadeva, mai una volta sopra le righe, mai esuberante, sempre coerente. Cazzate. Per quanto potesse sperarlo, per quanto potesse piacerle l'idea, il Fabio con cui si era messa insieme anni prima, il suo Fabio, non era una persona normale. Quello doveva essere il segnale che qualcosa non stava andando per il verso giusto. Imprevedibile, incontrollabile, incoerente, inopportuno: così era il ragazzo con cui si era fidanzata. Si era detta tante cose, che stava crescendo, che stava maturando... No, col senno di poi era chiaro che stava impazzendo. E lei non se ne era accorta.

« Pronto, Bagonghi. » disse la voce all'altro capo del telefono, ripescandola dal torbido dei suoi pensieri.

« Bruno, sono io. Piove. »

« Ciao Lavinia. Altro che pioggia, questo è un bombardamento! Ma lo sai com'è... »

« Eh, lo so, lo so... Com'era? Prato, 'io bove? »

« Esatto, esatto! Se 'un tira vento e 'un sona a morto... »

« ...e' piove! »

« Bravissima. »

« Mi fai buttare via! Comunque... che si fa? »

« Io sarei per fare passo, non credo che con un meteo del genere si possa... »

« Un bicchiere a casa mia? »

« Grazie per l'invito Lavinia, ma credo di preferire un sigaro sul divano e poi dritto a letto. »

« Per favore, anche solo mezz'ora! »

« Grazie lo stesso, ma sono proprio stanco. Ci sentiamo presto, devo lasciarti. »

« Aspetta... »

La voce di Lavinia fu silenziata alla pressione di un tasto. Bruno Bagonghi mise stancamente il telefono in tasca. Per quanto avrebbe ancora dovuto evitare la sua amica? Non se la sentiva di bere qualcosa da solo con lei, non era bravo a mentire ai suoi amici tanto quanto lo era a recitare le sue battute davanti alla mafia o alle forze dell'ordine. Per riuscirci, doveva entrare in uno stato mentale totalmente incompatibile con lo stare in compagnia allegramente. Non poteva proprio farlo, non avrebbe voluto a nessun costo mischiare le varie formae mentis richieste dal suo lavoro con quelle del suo svago. Era il Fontanelli quello forte nel modificare al volo il suo comportamento, non lui. Fabio non avrebbe avuto problemi a spacciarsi per cattolico di fronte al Papa in persona. Fabio era sempre vero, anche quando faceva finta. Lui, Bruno, invece, aveva bisogno di schemi, di rigide strutture alle quali uniformarsi per spacciarsi credibilmente per ciò che non era. A Fabio bastava il puro, sfacciato talento!

Quel pensiero lo stava irritando. Lo allontanò dalla mente con stizza. Guardò fuori dalla finestra, giusto per distrarsi: pioveva proprio forte. Un lampo spettacolare squarciò il cielo. Ah, la forza della natura! Incredibile, inarrestabile. Per quanti sotterfugi potesse architettare, niente avrebbe potuto alterare il naturale corso degli eventi. Non credeva nel destino, il Bagonghi, ma lo scorrere del tempo era una forza della natura, come il fulmine, come la pioggia. Come avrebbe potuto un patetico umano controllare gli eventi? Voltò le spalle al temporale e cercò un sigaro nella tasca della giacca. Un senso di impotenza lo avvolse. Permise ad una lacrima di bagnargli la guancia. Giorno dopo giorno, la sua decisione sembrava sempre più incastrarsi alla perfezione nel naturale corso degli eventi.

Un altro lampo saettò nel cielo plumbeo. Un altro ancora. Una vera e propria tempesta di fulmini! Il rumore dei tuoni era assordante, le vibrazioni incredibili. Tutto vibrava, scosso dall'enorme forza della tempesta. Anton Leka chiuse le persiane di casa sua, nel vano tentativo di isolarsi dall'armageddon che si stava consumando sopra di lui. Non che avesse chissà cosa da proteggere: forse l'unica cosa di valore che gli restava erano proprio le persiane. Chiusa fuori la poca luce naturale che riusciva a filtrare attraverso le nuvole, il buio scese impietoso sulla stanza. Anton si fece luce con l'accendino e brancolò verso il tavolo, in cerca della candela. L'elettricità era un lusso che, da disoccupato, non si poteva più permettere. Un paio di click ed una luce timida balenò, mostrando tetra la miseria rimasta su quel tavolo. Chiunque sarebbe rabbrividito al pensiero che quello, solo quello, era tutto ciò che possedeva. Non ci avrebbe potuto pagare neanche mezz'ora di affitto, figurarsi un mese! Le chiavi della macchina scintillarono alla luce della candela. Ormai l'auto non gli serviva più, era schedato come tossicodipendente, non avrebbe rivisto la patente per decenni. La sua povera Fiat Punto di terza mano avrebbe potuto fruttare cinque o seicento euro, non senza un folle ottimismo. Probabilmente non valeva neanche la pena di provarci. No, non c'era proprio modo di mantenersi un tetto sulla testa.

Ma al Leka non importava un granché. Era apatico, impassibile di fronte alla sua miseria. Aveva accettato il suo destino, non aveva nessun impulso a fare niente per risollevarsi dal suo disperato pantano. Solo un pensiero riusciva ad animarlo: vendetta, tremenda vendetta contro il responsabile di tutto questo.

NOTA DELL'AUTORE:

Mi sono rotto le palle di mettere le note a tutto. Come disse il mio professore di progettazione e disegno meccanico all'ITI, siete vostri!

Se non capite qualcosa, beh, avete internet: arrangiatevi!


Il prossimo, estatico capitolo verrà pubblicato il 30 novembre 2017.

-Simone



