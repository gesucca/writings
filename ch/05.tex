\chapter{Flashback: la Cenere}

\begin{chapquote}{Author's name, \textit{Source of this quote}}
``This is a quote and I don't know who said this.''
\end{chapquote}


Grigio. Niente più forme eteree, ma solo plumbeo, monolitico grigio. Fabio attese paziente, sicuro che qualcosa sarebbe accaduto. Forse una rivelazione? Magari una premonizione? Qualcosa di così bizzarro da non poter essere neanche immaginato? No, niente di tutto questo. L'operazione era già conclusa da un pezzo, la droga aveva quasi esaurito il suo potere. Restò deluso quando cominciò a capire che cosa lo stesse aspettando: l'ennesimo, banale flashback.

Il vento gelido soffiava impietoso per le viuzze di Prato. Raramente il centro storico era stato così deserto di domenica pomeriggio. Niente ragazzi a bighellonare in San Francesco, niente vecchietti a lamentarsi in Piazza del Comune: in giro c'erano solo i temerari imbacuccati che tiravano dritto per la loro strada, oltre a qualche tossico in disperata ricerca di eroina. Fabio non aveva proprio idea del perché si trovasse lì invece che in un qualsiasi luogo chiuso.

« S'ha a fa' festa? » urlò al suo compagno di disavventura, sovrastando il rumore del vento.

Quello annuì, ben lungi da aprire bocca. Ma perché diavolo erano usciti?

Il sollievo li pervase non appena varcarono la porta di casa Bagonghi. Il tepore, il silenzio, il divano: che piacere! Svestito ed accomodato, Fabio offrì una sigaretta al padrone di casa. Lui gentilmente rifiutò, preferendo un suo sigaro.

« Ora sì, che dianzi no! » sentenziò Bruno, seminascosto in una nuvola di fumo profumato.

« C'arebb'a correre\ldots» confermò Fabio.

Aveva le mani congelate. Resse la sigaretta fra le labbra e mise a scaldare le sue estremità sotto le cosce.

« Prato, città dello sgomento! », sentenziò a mezza bocca.

« Se \textsc{\char0}un piove e \textsc{\char0}un sòna ammorto, tira vento! Attento, ti sta cadendo la cenere addosso », gli rispose Bruno.

Fabio si scrollò e riprese la sigaretta fra le dita. Guardò il suo amico attraverso la foschia: la lotta della sera precedente aveva lasciato molti segni.

« Bada come sei conciato » gli disse con amarezza, « non sei ancora andato a rendergliele? »

Bruno non rispose, ma era chiaro che non provava rancore per il suo aggressore. Difficilmente il buon Bruno Bagonghi si lasciava mettere i piedi in testa, ma era facile al perdono.

« Hai avuto tempo di ascoltare quel disco? » incalzò Fabio, incapace di tacere.

« Book of Souls? »

« Sì, il libraccio delle anime. Cosa significa, poi? »

Bruno sbuffò assente un'altra nuvola di fumo. Sembrava altrove con il pensiero, ma rispose:

« Boh, cose di sciamani, credo. Comunque, ti dico la verità, non mi sta piacendo granché. È fin troppo\ldots »

« Prolisso? »

« Già. I pezzi sarebbero potuti durare la metà, senza perdere nessun significato. »

« Sono d'accordo, ma nonostante questo a me piace. »

Bruno sospirò. Pareva non essere dell'umore adatto per discutere, ma l'argomento era certo di suo interesse. Non si sarebbe sottratto al confronto di idee. Fece un altro tiro dal sigaro e disse:

« Ti dirò che non mi ha deluso. È perfettamente in linea con i lavori precedenti. Sono sincero, non è che avessi chissà quale aspettativa da un lavoro dei Maiden degli ultimi decenni. »

« Dai, Brave New World è oggettivamente un ottimo disco! Non puoi\ldots »

Improvvisamente, suonarono alla porta. Senza la minima esitazione, Bruno abbandonò il sigaro nel posacenere e si alzò.

« Anche di domenica, berva di'C\ldots! » esclamò.

« Dobbiamo proprio? » mugolò Fabio mentre si imbacuccava per uscire di nuovo.

« Sì, meglio di sì, a meno che tu non voglia vedermi lavorare anche oggi. Era scontata prima o poi una loro visita, venerdì avevo i Carabinieri nel magazzino. Ma ora proprio non mi va di parlare di queste faccende. »

Uscirono dalla porta sul retro, e si ritrovarono di nuovo investiti dal vento gelido. Imbestialito per il freddo, Fabio si prodigò in una sfilza di maledizioni verso la razza asiatica, rea di costringerlo a sottrarsi al caldo per evitare la mafia cinese. Ogni sillaba che pronunciava gli faceva entrare in bocca sorsi di aria ghiacciata, alimentando la sua ira.

« E basta, che sarà mai! », urlò Bruno, stufo del continuo borbottare del suo compagno di sventura.« È inutile, non sei credibile, non pensi quello che dici! Il razzismo non è cosa per te. Lascialo a chi ne può fare buon uso, ai neofascisti e agli ignoranti »

« Tipo quel neofascista ignorante che ha trasformato la tua faccia in un campo minato? » lo stuzzicò Fabio.

« Tipo lui, sì. »

« In effetti ti ha chiamato --- com'era? --- ``ebreo di merda'', o qualcosa del genere. Ti sei sentito perseguitato per la tua razza? No, perché sai, checché ne dica lui, tu non sei ebreo! »

« Stai continuando, smettila, non starò a questo gioco. Odio le discriminazioni di ogni sorta. »

« Che palle che fai, per due anatemi di estinzione! Che vuoi che sia il mondo senza i musi gialli? »

« Un mondo senza un quarto della sua popolazione. Personalmente, ritengo i cinesi ottime persone. »

Fabio sogghignò. Avrebbe riso, se non avesse temuto di inghiottire per disgrazia un sorso d'aria ghiacciata. A furia di parlare, aveva l'ugola gelata.

« Così ottime che non vuoi farle entrare in casa la domenica, ma preferisci parlarci lunedì, in ufficio, con un fucile a pompa sotto la scrivania! » disse a denti stretti, beffardo.

« Questi sono cinesi\ldots un po' più cattivi degli altri », replicò Bruno senza scomporsi.

« Maledetti loro, la loro stirpe e il giallo della loro pelle! Cerchiamo un posto al chiuso, piuttosto\ldots uno qualsiasi\ldots questo gelo mi ha già fatto venire il mal di gola! »

« Se tu avessi tenuto sigillata quella cazzo di bocca, invece di maledire gente\ldots Oh, ma che te lo dico a fare? »

Camminarono e camminarono, sfidando il vento impervio. Si era quasi fatto buio quando un pub che faceva angolo, finalmente, aprì. Ci si fiondarono dentro ed ordinarono la roba più calda che trovarono sul menù.

« Senti una cosa », eruppe Bruno da dietro la sua coppa di caffè irlandese, « ieri sei stato con Denise? »

Fabio non rispose, ma a Bruno sembrava bastare il suo silenzio per capire.

Proseguì:

« Avrei piacere che tu non cascassi fra le braccia della prima troia che capita. Metaforicamente, intendo. Cosa intendi fare con la Lavinia? Certe persone stanno dicendo cose\ldots La situazione non mi è chiara. »

« Non è chiara neanche a me, non so proprio che cosa dirti » rispose Fabio, controvoglia.

« Tranquillo, non devi spiegarmi niente. Volevo solo sapere se hai un'idea di cosa stai facendo, oppure agisci del tutto casualmente. »

« Agisco a caso. Contento? »

« Le mie scuse, non volevo essere inopportuno. »

« Ma che inopportuno! Madonna, quanto cazzo sei pomposo! Mi dai un fastidio bestia! »

« Allora va'a' caare, mi voleo fa \textsc{\char0}azzi tua! Contento? »

Fabio sorrise controvoglia. L'argomento lo aveva fatto piombare in un pessimo umore. Non aveva voglia di rimestare ancora nel calderone dei suoi problemi, ma solo di svagarsi insieme a uno dei pochi veri amici che gli erano rimasti. Bruno lo aveva di certo capito, ma riprese comunque l'argomento:

« So che ti do noia a parlare di queste cose. Solo che\ldots beh, sta andando tutto ai maiali, no? Non solo a te, a tutti quanti! A me per primo! Ora ho anche le forze dell'ordine che gironzolano per il magazzino in cerca di droga! Ci mancava solo questa. Non mi resta altro che sperare che trovino solo i soldi del nero dietro la cimatrice, che sono la cosa meno illegale che tengo lì dentro! Io mi rifiuto di credere a cose ridicole come la sfortuna, ma di questo passo non saprò più cosa pensare! »

Fabio sospirò.

« Dove vuoi arrivare? » chiese senza il minimo entusiasmo.

« D'accordo, te la faccio breve. Io non ho più nessuno. E anche te mi pare di capire che non sei messo tanto bene. Va bene, c'è il Brogelli\ldots ma mi è parso di capire che gli trombi la donna. C'è l'Arrighi, c'è il Gazzi\ldots »

 « No, il Gazzi proprio non c'è!  »

« Invece c'è, qualunque cosa questo comporti. Ma il mio discorso è un altro, Fabio\ldots A questo punto, dopo tutto ciò che ci sta accadendo, nelle nostre vite chi rimarrà? Te lo dico io\ldots rimarrò io, e spero proprio che rimarrai te! Quindi, in estrema sintesi\ldots se hai bisogno di qualcosa, qualsiasi cosa, cercherò di fartela avere. »

Fabio rimase perplesso. Che offerta era mai quella?

« Grazie per\ldots beh, il pensiero e tutto quanto » rispose con cautela. « Mi servirebbero tante cose, molte delle quali non so neanche che cosa siano. Non credo che tu possa darmi qualcosa che non sai cos'è. E, comunque, visto che alla fine sei un po' ebreo per davvero\ldots che cosa vuoi in cambio? »

« Non ti sto offrendo qualcosa perché voglio altro in cambio! » ribatté Bruno, teatralmente offeso. « Ti sto offrendo il mio aiuto incondizionato, così che quando avrò bisogno di te, tu ci sarai! Putacaso, avrei proprio qualcosa da chiederti\ldots ma è una mera coincidenza! »

Fabio sospirò.

« Dimmi cosa devo fare e lo farò. Anche io ho soltanto te, credi forse che senza una contropartita non ti farei un favore? »

« Dimmi prima tu. »

« Te l'ho detto, non saprei\ldots L'unica cosa che so di desiderare, tu non puoi darmela. »

« Su, parla. »

« Beh, voglio una nuova vita. Voglio lasciarmi tutto alle spalle, voglio imparare dai miei errori. »

Bruno contrasse il volto, probabilmente per trattenere una risata di scherno.

« Cosa ti impedisce di farlo? », bofonchiò.

Fabio apprezzò lo sforzo che il suo amico stava facendo per non deriderlo, ma si irritò lo stesso.

« Non hai capito\ldots » gli disse, grave. « Voglio sparire. Vorrei vivere una vita completamente diversa, vorrei diventare un\ldots un selvaggio! La caverna, la clava e tutto quanto! Non ne posso più di farmi il culo per poi vedere tutto andare a puttane! Voglio smettere di voler bene\ldots Voglio uccidere per mangiare, non avere altra preoccupazione se non quella di procacciarmi il cibo! Voglio vedere le persone come risorse o come minacce, non le voglio più mettere sul mio stesso piano emotivo! Se solo potessi\ldots »

Bruno alzò garbatamente una mano, sorridendo, e Fabio tacque.

« So che non ti piace quando te lo dico, ma a volte io e te facciamo pensieri molto simili » disse, affabile.

Fabio sbuffò.

« Consideralo già fatto », riprese Bruno. «Complimenti, adesso hai una nuova vita! Potrai uccidere per mangiare, e tutte quelle diavolerie che hai detto. Certo, dovrai aspettare qualche mese prima di cominciare a viverla, spero tu non pretenda che organizzi tutto per domani! »

« Non ci sto cascando\ldots »

« Non fare così! Torniamo verso casa, ti spiego due cosette. Ormai i cinesi se ne saranno andati. »

Bruno fece per alzarsi, ma Fabio lo bloccò.

« Facciamo finta per un momento che tu non mi stia prendendo per il culo. Cosa vorresti in cambio? »

« Ah, giusto! » esclamò, ostentando una finta sorpresa.

Fabio inarcò le sopracciglia e  non disse niente. Si fidava ciecamente del Bagonghi, non avrebbe esitato un attimo a mettere la sua intera vita nelle mani del suo amico. Ma giocare a fare il sospettoso alimentava la sua curiosità: cosa poteva mai volere Bruno da lui? Aveva molto poco da offrirgli, purtroppo.

« Alla luce di quello che hai detto, ciò che voglio chiederti diventa ancora più facile per te! »

« Stiamo a sentire\ldots »

« Ecco, bravo, stai a sentire. Tra qualche tempo, non so di preciso quando, diciamo appena ti sarai annoiato di giocare all'uomo di Neanderthal, mi piacerebbe che tu tornassi qui, a Prato, anche solo per qualche giorno. Sarà una bella sorpresa per te, non sai quanto pagherei per assistere alla tua reazione! »

« Facciamo che non ho voglia di contraddirti, perché o non hai capito che cosa ti ho chiesto, o non ho capito io cosa mi stai chiedendo te. »

« Abbiamo capito entrambi, ne sono certo. »

« Ho capito che dovrò tornare a Prato tra qualche tempo. Vuole dire che hai un modo per farmi sparire dalla circolazione? Ti avverto, se mi dai in pasto a qualche tuo amico del cazzo e finisco a lavorare su una piattaforma nel mezzo al Pacifico, o in Uzbekistan a difendere i pozzi di petrolio dai terroristi, o qualunque altra cazzata del genere, quando torno ti strozzo! »

« Uhm\ldots hai mandato in frantumi una buona metà delle soluzioni che stavo per proporti. Ma non preoccuparti, ho tante frecce al mio arco! »

« Mi inquieti, non posso mai stare tranquillo quando sono nelle tue mani. Ma facciamo che, fra tutte le stronzate che puoi inventare, tu abbia una buona idea. Vuoi solo che prima o poi torni a Prato? Davvero? »

« A dire la verità, no. So che è una cosa bizzarra, ma ti prego, tienilo a mente: qualunque cosa mi sia successa quando tornerai\ldots vienimi a trovare. »

« Eh\ldots tutto qui? »

« Sì. Vienimi a trovare e fuma un sigaro con me. Occhio, un toscano dei miei, non uno qualsiasi! A qualunque costo, non importa cosa tu debba fare per riuscirci. Lo farai? »

Fabio sbatté le palpebre, perplesso. Non capiva proprio dove Bruno voleva arrivare.

« Ma che cazzo di favore è? », sbottò.

« Per favore, dimmi che lo farai », incalzò lui. « A qualunque costo! »

« Sì che lo farò, vai tranquillo! Ma senti questo\ldots Io l'ho sempre detto, te sei tutto matto\ldots »

Ma era Fabio ad essere matto. Tutto ad un tratto, Bruno lo stava strangolando con delle luci da albero di Natale. Nemmeno il tempo di stupirsi che tutto cambiò. Le cose si fecero veramente strane, il senso degli avvenimenti era proprio un mistero. In preda al terrore, Fabio gridò senza emettere alcun suono. Qualcosa nella sua testa gli diceva che era tutto finto, tutta un'allucinazione, ma la paura era così reale\ldots

Senza speranza, attese il risveglio.
