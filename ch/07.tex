\chapter{Lo Sketch}

\begin{chapquote}{Author's name, \textit{Source of this quote}}
``This is a quote and I don't know who said this.''
\end{chapquote}

% pulled
% tex syntax
% rough editing

Forte della sua nuova identità, Fabio era finalmente libero. Avrebbe cominciato una vita completamente slegata dal suo passato. Si sentiva come se avesse preso in mano uno spartito vuoto, dopo mesi passati ad arrangiare un pezzo che proprio non voleva tornare: che gioia abbozzare finalmente idee nuove, fresche, vive! Tutte le possibilità che gli erano precluse nel pezzo cominciato, già rigido nella struttura e chiuso a stravolgenti nuove idee, finalmente tornavano in gioco. Ma da dove avrebbe dovuto cominciare? In quale tonalità avrebbe scritto la sua nuova composizione?

Decise che come prima cosa avrebbe reso vivibile il suo appartamento. Non era sicuramente la casa dei sogni, ma l'affitto era bassissimo e Fabio non intendeva vivere nel lercio. Era un obiettivo modesto, ma realizzabile: poteva partire da lì.

Guardando ad un orizzonte temporale più lontano, prima o poi sarebbe stata ora di trovarsi un impiego. Certo, con una rapina al mese sarebbe campato dignitosamente senza lavorare, ma era fin troppo rischioso; la sua unica esperienza in quel campo non era stata proprio piacevole, non smaniava dalla voglia di ripeterla. Eppure, più pensava all'occupazione da intraprendere, più notava che non aveva intenzione di lavorare onestamente. Il suo desiderio di rivalsa, spettro di una vita passata che doveva in tutti i modi dimenticare, lo avrebbe portato a fare il trafficante, il truffatore o addirittura l'assassino? La prospettiva, invece di preoccuparlo, lo incuriosiva.

La sua prima giornata da Jorge Pedrosa trascorse in un lampo. Il tempo vola quando si hanno fiducia nel futuro e tonnellate di spazzatura da buttare. Decise di andare a godersi quel che restava del giorno sul lungomare. Ormai il sole stava per calare, eppure Fabio sentiva che quel tramonto, in realtà, era la sua alba. Si lisciò la barba ed inspirò profondamente l'aria di mare, prima di inquinarla con un tiro di sigaretta. Stava per piombare in un profondo flusso di pensieri quando la sua attenzione fu catturata da una scena piuttosto bizzarra.

Una coppia distinta, forse sulla sessantina, stava discutendo animatamente con un pachistano, mentre una ragazza riprendeva la scena con lo smartphone. Contrariamente a quanto certi pregiudizi razziali potessero fargli pensare, dal punto di vista di Fabio sembrava proprio che fosse la coppia ad incalzare un'accesa aggressione verbale. Sebbene l'istinto lo supplicasse di farsi gli affari propri, la curiosità lo spinse ad avvicinarsi per scoprire cosa stava effettivamente succedendo.

Quatto quatto, si avvicinò con gattesca destrezza al parapiglia, senza farsi notare. Fu solo un inutile sfoggio di abilità: avrebbe potuto anche avvicinarsi a bordo di una portaerei e nessuno si sarebbe accorto di niente. Ebbe un deja-vu, ma lo ignorò.

Le urla del gruppo si sovrapponevano, creando un frastuono tale da non permettere a Fabio di capire che cosa stesse succedendo. Tuttavia, il suo fenomenale intuito, aiutato dal fatto che il pachistano stava incautamente sventolando una busta piena di roba verde, gli fece pensare che forse la situazione era ai limiti della tollerabilità in quanto a bizzarria.

Uno strano sorriso sbarazzino si dipinse sul volto di Fabio. Non riusciva a credere a quello che stava pensando. Era forse rincretinito? Si lisciò la barba e osservò il tramonto. Quella luce, quei colori\ldots\ era tutto bellissimo. Si volse di nuovo verso il parapiglia, che nel frattempo si era fatto ancora più intenso, e rise. Non sarebbe riuscito a trattenersi: si arrese al pensiero di fare una cosa stupida. Guardò intorno, giusto per scrupolo: non c'erano poliziotti o altre autorità a fare da deterrente alla sua smania di giocare. Buttò via la sigaretta e si avviò a passo svelto verso il gruppo, spaventato dalle conseguenze di quello che stava per fare, ma incapace di starsene nel suo.

«Woh, woh, woh, stop that weed waving, man!» gridò al pachistano appena fu a portata di voce. «That is weed, isn't it?»

Il presunto spacciatore lo degnò di un solo sguardo, prima di riprendere a litigare con la coppia. Era il perfetto stereotipo dello spacciatore a Barcellona: tratti e carnagione dell'est Europa, pantaloni lunghi, infradito, volantini di nightclub nei taschini della camicia e lattine di birra appoggiate per terra vicino a lui.

«Se no vuole tu va via! No urla, no arrabbia!» sbraitava quello, senza smettere di sventolare la busta con la droga.

«Ah ma siete italiani?» lo interruppe Fabio, sorpreso, rivolgendosi alla coppia. Davano proprio l'impressione di essere due vecchi rompipalle: palandrana per entrambi nonostante la piena estate, musi contriti dalla rabbia e qualche accenno di capelli bianchi.

«Impara a farti gli affari tuoi, giovane!» sbottò l'uomo, senza scollare gli occhi dallo spacciatore. «Che qui ci gira la feccia della società!».

«Spacciare in mezzo alla strada!» incalzò la donna, la voce traboccante di disprezzo. «Dovrebbe essere rinchiuso!»

«Scusi l'impertinenza, signora» cominciò Fabio con un tono falsamente sofisticato, «ma spacciare significa vendere la droga alla gente, e gran parte della gente è in giro sul viale. Cosa dovrebbe fare, aprire un negozio in centro? Venderla porta a porta?»

La ragazza eruppe in una risata isterica, ma la vetusta coppia non gradì.

«Giovane, non fare il simpatico!» ruggì l'uomo, infuriato come non mai. «Ora chiamo le guardie! Questa merda farà la fine che si merita: marcirà in galera!»

«Marcirà in galera!» gli fece eco la donna. «Vendere la droga ad una famiglia in vacanza, per di più con la bambina! Ma cosa ha nel cervello!»

Il pachistano, ammutolito ormai da un pezzo, scoccò un occhiata divertita prima alla ragazza, poi a Fabio, che gli sorrise. La bambina, come la moglie dell'obsoleto pisquano l'aveva definita, era proprio una bella figliola, alta, magra e bionda; forse era davvero troppo giovane per farci certi pensieri, ma certo non era una bambina. Non c'era da sorprendersi che lo spacciatore l'avesse avvicinata. Comunque, a lei la situazione non pareva affatto dispiacere, visto che stava piangendo dalle risate mentre riprendeva tutto con il suo smartphone. Fabio le lanciò un ammicco con le sopracciglia e declamò solennemente, imitando i modi dell'antipatico duo:

«Signor spacciatore, la signora ha ragione. Ma cosa le salta in mente di proporre la droga alla bambina! Una dolce, immacolata bambina! La sua candida innocenza dovrebbe intimidirla a tal punto da farle regalare la sua merce. Come osa chiederle dei soldi in cambio! Mi dia retta, se vuole vendere, lei sta sbagliando target commerciale.»

La ragazza fu scossa da risate così violente che quasi lasciò cadere lo smartphone a terra. La coppia invece sembrava aver visto un mostro: tacquero entrambi, pietrificati.

«Guardi me» proseguì Fabio, ormai deciso a sbigottire quei tizi fino in fondo. «Sono un ragazzo losco, con i lineamenti a teppista, do del lei agli spacciatori --- che sono la feccia della società! Tutti elementi che indicano inequivocabilmente che ho bisogno di droga! E che sarei disposto a pagarla profumatamente!»

La situazione improvvisamente degenerò: uomo e donna cominciarono a sbraitare all'unisono, agitandosi come dei folli e cominciando ad attirare l'attenzione dei passanti, rimasti fino ad allora indifferenti al moderato parapiglia. Il pachistano spacciatore fece un breve cenno di saluto a Fabio e, riposta finalmente la busta di droga in tasca, si dileguò veloce come il vento.

Nonostante la scena offerta dalla coppia, che tentava disperatamente di persuadere chiunque gli capitasse a tiro a chiamare le forse dell'ordine, Fabio decise che era arrivato il momento di levarsi di torno. Colto da un improvviso timore, si avvicinò alla ragazza.

«Scusa, ma non vengo bene in video» le disse in fretta, mentre le strappava il telefono di mano.

Lei squittì sorpresa, ma non reagì.

«Dai, non fare così!» le gridò Fabio, mentre si allontanava dalla scena a passi svelti. «Se ti ritrovo in giro te lo rendo, promesso!»

Ma la ragazza non sembrava proprio voler cedere il suo gingillo senza dare battaglia: si scrollò di dosso la madre, che nel frattempo la aveva agguantata per un braccio, forse per mostrare ai passanti la corruzione che aveva subito dal malvagio spacciatore, e trotterellò verso Fabio, decisa a riprendere il maltolto. Con la feroce ragazzina alle calcagna, lui si diresse verso il centro della città, deciso a mettere quanti più metri possibili fra lui e i vecchi chiassosi.

Non pensò neanche un istante a quanto aveva e stava rischiando: si stava divertendo tantissimo.
