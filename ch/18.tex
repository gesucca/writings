\chapter{Mostri}

\begin{chapquote}{Author's name, \textit{Source of this quote}}
``This is a quote and I don't know who said this.''
\end{chapquote}

% pulled

Vittoria seguiva il suo compagno Jorge a passo spedito nella notte. I due si stavano lasciando il porto alle spalle, inoltrandosi rapidamente verso il cuore della città. Non stavano seguendo un percorso regolare, anzi, tutt'altro: era come se cercassero di disorientare eventuali pedinatori, cambiando direzione senza un'apparente senso logico.

In ogni caso, ormai Vittoria non poteva più ignorare il fatto che erano tornati in Italia. Strade e stradine, ora larghe ora fitte senza il minimo senso, serpeggiavano attraverso palazzoni costruiti con sfacciata incuria architettonica per poi sfociare in piazze piuttosto ampie, tutte quante vegliate dall'immancabile marmo raffigurante questo o quel tizio. Tutto quel meraviglioso caos, però, falliva nell'oscurare la diffusa sensazione di bellezza che quella bizzarra accozzaglia di vecchiume e dubbia antichità trasmetteva a chi vi si immergeva. Non era proprio possibile che quella fosse una cittadina anglosassone: l'Inghilterra che aveva visitato Vittoria era completamente diversa. Inoltre, le scritte in italiano sulle insegne di bar e negozi non lasciavano spazio a dubbi, rendendo superflua qualunque arguta deduzione tratta dal paesaggio urbano.

Vittoria però non aveva idea di quale potesse essere la misteriosa città nella quale si trovava. Faceva schifo in geografia, ma non era scema: dalla nave aveva riconosciuto Monaco e poi la Liguria; sotto doveva esserci la Toscana, o il Lazio, o giù di lì. In ogni caso, era abbastanza lontana dalle sue parti per poter stare tranquilla. Jorge, d'altro canto, sembrava essere molto vicino a casa sua. I suoi movimenti spediti, il modo in cui si guardava in giro in cerca di punti di riferimento, la sua strana espressione facciale quando esitava qualche attimo per orientarsi; tutto dava a Vittoria l'impressione che quei luoghi non fossero proprio sconosciuti al suo compagno. Eppure, lui aveva detto di venire da qualche posto vicino a Firenze, ma Firenze mica c'era il mare... Vittoria si inquietò nel fare questi pensieri; cercò di scacciare via la sua preoccupazione osservando distrattamente l'ambiente circostante.

Si erano appena lasciati alle spalle i loggiati di un'ampia via con molti negozi per lei interessanti - ovviamente chiusi, data l'ora - per addentrarsi nuovamente negli stretti viottoli fra palazzo e palazzo. Lo sguardo le cadde su quello che diversi giorni prima era un ratto e lo stomaco le si rivoltò, perciò cercò rapidamente di distrarsi leggendo qualunque cosa le capitasse a portata d'occhio.

« Torta... pizzeria Napoli... pizza e torta... ancora torta? », lesse ad alta voce, interrogando la sua guida. « Jorge, perché ci sono così tanti negozi di torte in questo posto? »

Lui rise a lungo; quando finalmente le rispose, non si voltò nemmeno a guardarla.

« Benvenuta a Livorno, cara mia », fece giocondo. «Se non fossero qui i tortai, dove avrebbero a essere? »

Si fermò di colpo e si girò verso di lei.

« Vuoi un cinque e cinque? », le chiese sorridente. « Non manca tanto all'alba, forse qualcuno ha già aperto. »

« Un che? », fece lei, spaventata.

« Un panino fatto con cinque lire di pane e cinque lire di torta di ceci » spiegò lui. « O comunque, era così qualcosa come un secolo fa. Ora costano di più e ci mettono anche le melanzane. »

« Torta - di ceci!? Con le melanzane!? » fece lei, disgustata. « Ma scherzi? Ma che è? »

Jorge ridacchiò di nuovo e riprese a camminare.

« Sai, la pensavo proprio come te fino a qualche anno fa. » disse, dandole le spalle. « Quando venne a trovarci quel ragazzo, lo presi per il culo tutta la sera, poverino. E invece una volta, dopo una nottata a Cecina col Bagonghi un po'... rocambolesca, diciamo così, non si riuscì a trovare niente di aperto per fare colazione tranne un maledetto tortaio. A lui gl'importava una sega, io avevo una fame bestia. Insomma, ci si fece coraggio e... boia, in tutti quegli anni quanto torto ho avuto! Comunque, non ci pensare, ho detto una cazzata. A Livorno nessuno ha voglia di lavorare, figurati se i tortai aprono all'alba! »

Vittoria rimase muta. Quel breve racconto di vita vissuta, per simpatico che fosse, aveva rinnovato le sue inquietudini, facendole sentire il suo Jorge molto lontano. Era palese che lui avesse nostalgia della sua vita precedente, qualunque essa fosse. Vittoria invece non ne aveva neanche un po', proprio per niente. E se Jorge a un certo punto avesse deciso di riabbracciare il suo passato, abbandonando lei al suo destino? La prospettiva era terrorizzante. Avrebbe dovuto investire ogni grammo della propria energia nello sforzo per tenere Jorge lontano da Firenze o da dove caspita veniva, eppure non riusciva a costringersi a farlo; un eco lontano nella sua testa gridava un terribile avvertimento che rischiava di restare inascoltato.

« Manca molto a dove vuoi andare? », chiese al suo compagno. « E, tra parentesi, dove stiamo andando? Sono tipo venti minuti che giriamo in tondo. »

« Voglio andare alla stazione, chiaramente, ma che sia dannato se mi ricordo dov'è.  Ora siamo... eh, intorno i fossi, quindi vedrai la Fortezza Nuova non deve essere tanto lontana. Insomma, c'è ancora da camminare un bel po'. Sei stanca? »

« No, era solo per sapere se abbiamo tempo prima che tu faccia del male a qualche altro sconosciuto. »

Jorge rise ancora.

«Tranquilla, voglio solo arrivare con discrezione alla stazione. Non credo che assassinerò nessuno nel tragitto, credo che sarebbe poco discreto. »

« Quindi posso prendere mezza goccia o no? »

« Puoi fare tutto quello che vuoi, sei tua. »

Stettero in silenzio per un po', ciascuno immerso nei propri pensieri, continuando a gironzolare sempre più freneticamente nel cuore di Livorno.

***

Il cielo albeggiava; con snervante lentezza, la città cominciava ad animarsi. Un ragazzo alto e una ragazza bionda erano finalmente seduti su una panchina nei pressi della stazione, aspettando pazientemente l'apertura di un'edicola; erano due fuorilegge, ma se volevano prendere il treno tanto valeva comprare i biglietti. Il ragazzo alto si lisciava distrattamente la barba, macchinando chissà cosa in quel suo diabolico cervello; la ragazza bionda fissava semplicemente il vuoto.

Gli effetti della droga erano ormai svaniti da un bel pezzo, ma la mente di Vittoria ancora viaggiava a rilento, intorpidita dagli strascichi lasciati dalla sostanza. Le immagini di ciò che aveva visto quella notte nella stiva della nave la tormentavano; non si poteva proprio definire una ragazza da scrupoli morali, ma il comportamento di Jorge le appariva indubbiamente oltre la sua zona di comfort. Non era stata tanto la crudezza della scena a turbarla, ma quanto il fatto che il suo compagno non si era semplicemente liberato di un ostacolo; lo aveva torturato, costretto in uno stato di totale prostrazione.

Le terribili sensazioni che aveva provato in quella stiva le si presentavano prepotenti e intermittenti all'attenzione: l'arto dilaniato dal colpo di pistola, la pozza di sangue che lentamente si allargava sul pavimento, quel disgustoso ma familiare odore ferroso, le urla dell'uomo, i suoi singhiozzi... Era stata una cosa terribile alla quale assistere.

Tuttavia, ciò che spaventava Vittoria più di ogni altro aspetto della vicenda non era l'inutile crudeltà con cui Jorge aveva trattato quella persona, ma la maestria con cui quel marinaio era stata manipolato nella totale obbedienza. Era precisamente quell'aspetto che suscitava in lei una pericolosa sensazione di disagio. La minaccia, le parole argute, l'illusione di salvezza immediatamente infranta, il tutto sapientemente mescolato per creare quell'orribile sensazione di sottomissione che porta ad accettare passivamente ogni sopruso per inseguire la vana speranza di aver salva la vita. Non era solo un gioco di dominanza o una banale perversione, ma una violenza psicologica disumana, messa in atto con manipolazione a regola d'arte degna di un sociopatico coi fiocchi. Secondo la sensibilità di Vittoria, forse ancora un po' amplificata dalla droga, si trattava di un crimine assai peggiore dell'assassinio: era come negare l'umanità di una persona, renderla meno di una bestia, meno di una pianta, smontarla pezzo per pezzo e ridarle una parvenza di esistenza secondo il proprio comodo e diletto. Un individuo in grado di godere di questo non poteva essere umano. Quello non era il suo Jorge, ma un mostro.

Cosa gli era successo? Perché doveva comportarsi così? La voce interiore che rispose a Vittoria prese il timbro e la cadenza del suo compagno:

"Te lo avevo detto cosa sarei diventato. L'uomo di cui ti sei innamorata non c'è più."

Non aveva senso, non poteva essere. Appena qualche mese prima quel bel ragazzo era coraggioso e intrigante; possibile che un trauma lo avesse cambiato al punto da fargli tormentare un marinaio per il solo gusto di farlo?

Una sensazione sgradevole la pervase. Il suo inconscio aveva formulato una terribile risposta, una verità così orrenda da spaventarla. Non poteva esprimerla con delle parole, non voleva rendersi conto della realtà: cercò di resistere il più possibile, ma alla fine il ricordo che cercava di ignorare irruppe prepotente nello spettro della sua attenzione cosciente. La voce di Jorge tuonò chiara e nitida nella sua memoria:

"Ascoltami bene. Questa situazione è molto pericolosa. Sia per te che per me. Non vorrei proprio ritrovarmi a fare cose che non mi conviene fare..."

"Voglio che tu capisca in che mani sei ..."

"Ora io ti mostrerò una cosa. Ti renderai finalmente conto che seguirmi è stata una pessima idea. Consideralo un favore personale che ti faccio..."

"Ti è andata bene, sei viva e non ti ho nemmeno stuprato ..."

La forza della rivelazione la investì con dirompente violenza. Usò fino all'ultimo grammo della sua forza di volontà per controllarsi; riuscì a mantenere apparentemente la calma, o per lo meno a impedirsi di urlare, ma non poté trattenersi dallo scoppiare silenziosamente a piangere.

Jorge non aveva torturato il marinaio perché era cambiato. Lo aveva torturato perché semplicemente quello era il genere di cose che Jorge faceva. Quell'uomo fragile ma risoluto non era diventato un mostro dopo averlo deciso l'altro giorno, lo era stato fin dal primo momento in cui lo aveva conosciuto. Vittoria stessa aveva subito il suo gioco: attirata in trappola, minacciata, sottomessa ed infine sedotta.

Non voleva crederci, ma nient'altro aveva senso. Lei lo amava davvero, profondamente, sinceramente: quel sentimento le sembrava genuino, non indotto. La realizzazione della verità aveva scosso brutalmente le radici della sua affezione, ma non la aveva fatta immediatamente scoppiare come una bolla di sapone. Il buono che aveva visto nel suo amato non era stata solo un'illusione, era reale; offuscato, soppresso da impulsi nocivi, ma comunque presente. Il suo Jorge... non era sicuramente Jorge. Ma chiunque egli fosse, era sempre una persona, non ancora corrotta al punto di diventare solo un demone. Era un ragazzo debole che combatteva contro la sua malattia mentale. Per quando stare con lui potesse essere pericoloso, la strada di Vittoria era segnata: lei lo amava, lo doveva aiutare.

Ma lui voleva essere aiutato?

Fabio si lisciava la barba. Strani pensieri lo affliggevano; man mano che la sua inquietudine cresceva, il suo povero pizzetto veniva attorcigliato con sempre più veemenza. Avrebbe dovuto essere finalmente libero, ma non lo era. Nonostante tutte le sue risoluzioni, tutte le regole che si era dato, si sentiva esattamente lo stesso di prima, con gli stessi fardelli da portare sulla coscienza. Solo una cosa era cambiata: solo più scosso che mai.

Non erano tanto le esperienze che aveva vissuto a tormentarlo, quanto il fatto che si trovava ad un centinaio di chilometri dalla sua vecchia vita. Non poteva impedirsi di pensare a quello che aveva lasciato alle sue spalle. Aveva avuto tanta bramosia di scappare, ma adesso, in cuor suo, tutto ciò da cui era fuggito gli mancava terribilmente. Chissà cosa era successo a Prato mentre era stato via... Come se la stava passando Bruno? Forse era riuscito a trovare altri clienti... non poteva essersi arreso, non era da lui. Denise aveva scoperto cosa era successo a Daniele? No, come avrebbe potuto? Avrebbe dovuto trovarla e dirglielo lui, era l'unico a sapere. E Lavinia... lei no, non le mancava per niente. Era finalmente caduta nelle grinfie del Gazzi? E soprattutto... aveva pensato a Fabio almeno un po'?

Sospirò in preda all'angoscia e abbandonò la testa fra le mani, lasciando perdere la barba. Chi voleva prendere in giro? Solo sé stesso, a quanto pareva; ma non ne era in grado. Non lo avrebbe ammesso nemmeno sotto tortura, ma a lui Lavinia in realtà mancava moltissimo. Non era certo solo, anzi, Vittoria era una gran bella compagnia femminile, ma... era diverso. Lavinia lo capiva, gli teneva testa, o comunque ci provava come poteva. La giovane milanese, invece, era sua. La aveva rigirata come un calzino, annichilita e posseduta come un qualunque balocco. Non la aveva mai considerata al suo pari, come una volta invece considerava la sua fidanzata.

Fabio si arrese. Avrebbe dato letteralmente qualunque cosa per strappare questi i sentimenti dal suo essere, abbandonando finalmente la sua umanità per imboccare qualunque altra strada fosse disponibile. Ci aveva provato. Lo aveva detto a Vittoria, se lo era raccontato per sé parecchie volte; in realtà, aveva sempre e solo recitato una parte. Gli veniva bene, per carità, riusciva a immaginare ed emulare molto bene una sua versione psicopatica. Ma per quanto si sforzasse, per quanto si immedesimasse, la maschera che abeva infossato era rimasta pur sempre una maschera. Forse aveva convinto Vittoria, ma non aveva ingannato se stesso.

***

L'edicola che il pensieroso duo stava fissando ormai da parecchio tempo, finalmente, aprì.

« Andiamo a comprare 'sti biglietti? », fece Vittoria con voce fin troppo vivace, dopo aver tirato su con il naso.

Fabio non rispose. Il suo volto era deformato da un'emozione indecifrabile, lo sguardo era fisso sulle locandine che l'edicolante aveva appena messo fuori. Si alzò lentamente e si diresse verso l'edicola in uno strano trance, facendosi quasi investire da un marocchino in bicicletta. Entrò dentro l'edicola e, senza dire una sola parola, afferrò una copia de Il Tirreno di Prato.

OMICIDIO BAGONGHI, SCOMPARSO EX DIPENDENTE SOSPETTATO

Custodia cautelativa per l'ex dipendente albanese, che però risulta irreperibile. Mistero sull'autopsia della salma di Bagonghi, disposta e poi revocata.

I fatti che riguardano l'assassinio di Bruno Bagonghi, l'imprenditore proprietario dello storico Lanificio Bagonghi trovato senza vita la settimana scorsa, sprofondano sempre più nel mistero. Il Pubblico Ministero ha finalmente disposto la custodia cautelativa per Anton Leka, l'uomo di nazionalità albanese sospettato fin da subito per l'omicidio, ma egli risulta irreperibile: si teme che abbia già messo in atto una fuga. Ci si chiede come mai non sia stato immediatamente emesso un mandato di cattura appena verificata l'irreperibilita di Leka, ma questa è solo l'ennesima domanda da aggiungere alla...

Solo una voce stizzita riuscì a riscuotere Fabio dallo shock:

« Dé, ma lo 'ompri 'sto giornale o no? Un so' mi'a una biblioteca! »

Il prossimo capitolo, sarà pubblicato il 31 luglio 2018. Sarà corto, fatevelo bastare perché poi vo a i'mmare!

- Simone



