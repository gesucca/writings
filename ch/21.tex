\chapter{Risposte: l'Ingegnere}

% wip

Vittoria malediceva la pioggia e il suo compagno Jorge mentre, incappucciata, si faceva largo in una piccola foresta di ombrelli che sciamavano verso il parcheggio del piccolo cimitero di Chiesanuova.

« Signore! Signoreee! », chiamò a gran voce, cercando di sovrastare il rumore della pioggia.

« Quell'uomo con l'ombrello grigio chiaro! Dico a lei, signore! SIGN - figa, però, eh! Mica posso urlare! »

Se solo ne fosse stata sicura, lo avrebbe chiamato per nome. Govi? Gobi? No, non se lo ricordava proprio. In ogni caso, ne aveva avuto abbastanza di strillare: si fece largo fra la gente con assai poca grazia e, quasi saltando, agguantò per una spalla il signore per cui si era sgolata.

« Scusi, eh! », esordì aggressiva, con un tono di voce che poco si addiceva a delle scuse. « Sono due ore che la chiamo! »

Quello non si scompose affatto. Era un uomo molto alto e snello, dall'età indecifrabile. Vittoria si sentì un po' a disagio al suo cospetto.

« Le chiedo scusa », disse lui a voce alta ma tranquilla. « Fra le persone e la pioggia, non l'ho proprio sentita. Ah, ma non ha l'ombrello, è completamente bagnata! Venga sotto il mio, ci stiamo in due, la accompagno alla sua automobile. Ci conosciamo? »

« Ehm... diciamo che la conosco di parola », rispose Vittoria, togliendosi il cappuccio. Un po' in imbarazzo cominciava a farsi largo in lei; maledì mentalmente Jorge, e si rimproverò per non aver insistito sul non dividersi.

« Mi conosce di parola... », le fece eco quel signore, squadrandola con evidente sospetto.

Passò qualche istante di imbarazzante silenzio, durante il quale Vittoria si accorse che forse approcciare uno sconosciuto in quel modo non era stata poi una grande idea. Cercò di darsi coraggio e decise che, se fosse sopravvissuta alla situazione, avrebbe preso Jorge e gli avrebbe ficcato l'ombrello di quel tizio dritto nel...

« No, mi arrendo », sospirò finalmente il figuro. « Così non riesco proprio ad indovinare. Mi dia qualche indizio, la prego. »

Il disagio di Vittoria diminuì appena.

« Facciamo così », tentò lei. « Io le dico tutto quello che so, ma lei non mi uccide. »

L'uomo rise in un modo assai strano; si poteva capire che era una risata sincera, ma la compostezza con cui rideva era così impeccabile da far pensare che facesse finta.

« Oh, cielo », flautò non appena riprese fiato. « Mi scusi, ma la sua battuta mi ha fatto proprio ridere. »

Vittoria si sciolse un po'.

« Mi scusi lei, ma visto che è un amico di Jorge, pensavo che la tendenza fosse questa »

L'uomo smise di camminare di colpo e Vittoria, che se ne accorse un instante dopo, uscì da sotto l'ombrello e finì sotto l'acqua a capelli scoperti. Con un solo passo egli le si avvicinò rapidamente per coprirla nuovamente con l'ombrello.

« Mi dica subito chi diamine è lei », disse con voce tranquilla, torreggiando davanti a lei e guardandola dall'alto in basso.

Il cuore di Vittoria le saltò in gola.

« Io... Ecco, cioè, lei non mi conosce, che... che le devo dire? Mi manda Jorge, dice che è un suo amico, vuole sapere delle cose, ma... ma io che ne so! »

L'uomo alto la guardò intensamente per dei lunghi momenti. Sembrava che stesse soppesando una decisione.

« Mi descriva il suo mandante, per favore. », chiese dopo un po'.

« E'... è alto, non come lei ma comunque alto, e... »

« E' Fontanelli », fece deciso a se stesso, interrompendo Vittoria e lasciandola interdetta.

« Lei non ha un mezzo, vero? » proseguì cordiale. « Venga con me, così può raccontarmi tutto. Non la uccidefrò, stia tranquilla, quando avremo finito di informarci a vicenda la lascerò andare viva e vegeta. »

Il signore tanto alto quanto distinto accompagnò questo ultimo commento con una risata sommessa. Vittoria sorrise a sua volta, più a disagio che mai in quella situazione, ma acconsentì. Non che avesse avuto altra scelta. Cercò comunque di non perdersi di spirito.

« Può prestarmi il suo ombrello, dopo? », civettò mentre quel signore le apriva la portiera di una elegante auto tedesca. «  Avrei da farci una certa cosa... »

***

In una polverosa e muffita stanza di un capannone, un uomo tanto snello quanto distinto stava finendo di spiegare gli esiti di una brutta situazione ad una giovane ragazza bionda.

« ...così il signor Bagonghi ha offerto al Fontanelli la possibilità di ottenere una nuova identità. Ora le è più chiaro il perché del comportamento del suo amico, signorina Meis? »

Vittoria annuì, sconvolta dal sentirsi confermare per filo e per segno tutte le cose che aveva sempre sospettato sul suo Jorge. Era depresso, eccetera, aveva una fidanzata con cui viveva insieme! ...

« La vedo delusa », proseguì il signor Govidi. « Spero di non averla ferita troppo raccontandole queste vicende in modo così diretto, ma il tempo purtroppo è tiranno e non potevo fare altrimenti. »

« Non - non si preoccupi, si figuri » fece Vittoria con un filo di voce, ad un passo dal pianto.

« Per quello che può valere, sono molto dispiaciuto per lei. E' chiaro che prova qualcosa per il signor Fontanelli, ma - se vuole il mio consiglio - lo lasci perdere. Detto in confidenza, sono rimasto vagamente sorpreso nell'apprendere da lei che fosse ancora vivo. »

Il silenzio scese nuovamente sulla situazione.

Lo sgurado di Govidi vagò lontano da Vittoria. « Maledetto bastardo » sussurrò fra sé, « ha il culo messo peggio di una puttana. » 

Vittoria uscì sorpresa dal suo trance pensieroso. « Cosa? Ma dice a me?  »

Govidi esibì la sua risata intensa ma composta. 

« No, no, si figuri se mi riferivo a lei! » fece divertito.  « Stavo solo ripensando al fatto che ero così sicuro della sorte che avrebbe atteso il Fontanelli, che l'ho dato per suicida entro un mese in una scommessa col signor Bagonghi. Una cena contro un caffé, capisce? Ero così sicuro di averci preso che già pregustavo il caffé che avrei vinto, invece... »

« Almeno ha risparmiato la cena » civettò Vittoria, senza riuscire a trattenersi.

Govidi le scoccò uno sguardo penetrante per un lungo istante. 

« Scusi... » fece lei, la voce piccola piccola.

« Non serve nessuna scusa, la battuta era divertente » replicò Govidi, che non sembrava affatto divertito.

« Anzi, a proposito di questo... » fece, scomparendo in un cassetto della sua scrivania.

Un piccolo aggeggio venne lanciato bruscamente verso Vittoria, che lo prese al volo. Sembrava un telefono cellulare molto vecchio, ma con uno schermo piccolissimo.

« Io ho risposto alle sue domande » disse Govidi, riemergendo dal caos del suo cassetto. « Ora è il mio turno di fare richieste. »

Vittoria si limitò a guardare il suo interlocutore, in attesa. Si sentiva di nuovo un po' a disagio.

« Stanotte, lei convincerà il Fontanelli a visitare la tomba del signor Bagonghi », ordinò Govidi. « Appena sarete nel cimitero, lei userà questo cercapersone per avvertirmi, ed io manderò qualcuno ad aiutarvi. Pensa di poterlo fare? »

Vittoria annuì.

« Molto bene » fece lui in tono definitivo. « Come le ho già detto, il tempo è tiranno. Mi perdoni la sgarbatezza, ma devo congedarla. Non posso proprio accompagnarla in auto, dovrà cavarsela da sola. Prenda questi, chiami un taxi - anzi, fa niente, glielo chiamo io fra un istante, ma la prego di lasciarmi solo adesso. »

Senza dire altro, Vittoria uscì dall'ufficio di quello strano tipo.

Era quasi a metà del corridoio quando la voce di Govidi la raggiunse. « Prenda il mio ombrello! Aveva detto che le sarebbe servito»




Due vecchi amici si stavano raccontando quello che sapevano riguardo alle morti di due uomini a loro cari.

Una giovane donna, (descrizione?) stava riversando mesi di angosce su un ridicolo dandy impomatato.

