\chapter{Risposte}

% wip

Vittoria si faceva largo incappucciata in una piccola foresta di ombrelli che sciamava verso il parcheggio del piccolo cimitero di Chiesanuova.

« Signore! Signoreee! », chiamò a gran voce, cercando di sovrastare il rumore della pioggia.

« Quell'uomo con l'ombrello grigio chiaro! Dico a lei, signore! SIGN - figa, però, eh! Mica posso urlare! »

Se solo si fosse ricordata come Jorge aveva detto che si chiamava, lo avrebbe chiamto per nome. Si fece largo fra la gente con poca grazia e agguantò per una spalla l'uomo per cui si era sgolata.

« Scusi, eh! » gli fece con un tono di voce che poco si addiceva a delle scuse. « Sono due ore che la chiamo! »

Quello non si scompose affatto. Era un uomo molto alto e snello, dall'età indecifrabile. Vittoria si sentì un po' a disagio al suo cospetto.

« Le chiedo chiedo scusa », disse lui a voce alta ma tranquilla. « Fra le persone e la pioggia, non l'ho proprio sentita. Ah, ma non ha l'ombrello, è completamente bagnata! Venga, venga sotto il mio, ci stiamo in due, la accompagno alla sua automobile. Ci conosciamo? »

« Ehm... diciamo che la conosco di parola », fece Vittoria, mentre un po' in imbarazzo cominciava a farsi largo in lei.

« Mi conosce di parola... » le fece eco quell'uomo, squadrandola con evidente sospetto.

Passò qualche istante di imbarazzante silenzio, durante il quale Vittoria si accorse che approcciare uno sconosciuto in quel modo forse non era stata poi una grande idea e decise che, se fosse sopravvissuta alla situazione, avrebbe preso Jorge e gli avrebbe ficcato l'ombrello di quel tizio dritto nel...

« No, mi arrendo », sospirò finalmente il figuro. « Così non riesco proprio ad indovinare. Mi dia qualche indizio, la prego. »

Il disagio di Vittoria diminuì appena.

« Facciamo così », tentò. « Io le dico tutto quello che so, ma lei non mi uccide. »

L'uomo rise in un modo assai strano; si poteva capire che era una risata sincera, ma la compostezza con cui rideva era così impeccabile da far pensare che facesse finta.

« Oh, cielo », fece non appena riprese fiato. « Mi scusi, ma la sua battuta mi ha fatto proprio ridere. »

Vittoria si sciolse un po'.

« Mi scusi lei, ma visto che è un amico di Jorge, pensavo che la tendenza fosse questa »

L'uomo smise di camminare di colpo.

« Mi dica subito chi diamine è lei », fece secco, ma tranquillo.

Il cuore di Vittoriale saltò in gola.

« Io... Ecco, cioè, lei non mi conosce, che... che le devo dire? Mi manda Jorge, dice che è un suo amico, vuole sapere delle cose, ma... ma io che ne so! »

L'uomo giardò intensamente attraverso il suo cappuccio, incrociando direttamente il suo sguardo. Sembrava che stesse soppesando una decisione.

« Mi descriva il suo mandante, per favore. », disse dopo un po'.

« E'... è alto, non come lei ma comunque alto, e... »

« E' Fontanelli. » fece deciso a se stesso, interrompendo Vittoria e lasciandola interdetta.

« Lei non ha un mezzo, vero? » proseguì cordiale. « Venga con me, così mi racconta tutto. Non la uccido, stia tranquilla, quando avremo finito di informarci a vicenda la lascerò andare viva e vegeta. » rise sottovoce.

Vittoria acconsentì, più a disagio che mai. Cercò comunque di non perdersi di spirito.

« Può prestarmi il suo ombrello, dopo? Avrei da farci una certa cosa »
