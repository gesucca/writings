\chapter{Metamorfosi}

\begin{chapquote}{Author's name, \textit{Source of this quote}}
``This is a quote and I don't know who said this.''
\end{chapquote}

Ormai si era fatta notte, non proprio l'ora adatta per gironzolare nelle profondità de La Barçeloneta. Quella zona trasudava degrado e criminalità da ogni casa, vicolo ed angolo buio. ``È solo un'impressione'' si disse Fabio, ed in effetti lo era: tutto taceva tranquillo, come era normale che fosse in un quartiere residenziale di notte. In lontananza si udivano gli schiamazzi della movida, urla di turisti ubriachi intenti a perdere la loro dignità: sembrava tutto regolare. Non sentendosi comunque al sicuro, mise una mano sulla pistola ed affrettò il passo, scrutando nervosamente l'ambiente circostante. Si sentiva costantemente osservato, controllato dal quartiere stesso. Ogni palazzo gli sembrava uguale all'altro. Era già passato di lì?

Finalmente, dopo un lungo peregrinare, trovò l'edificio che stava cercando. Ad un primo sguardo appariva esattamente uguale agli altri, ma sul muro aveva il segno distintivo di cui il suo amico Bruno gli aveva parlato: una svastica con gli uncini al contrario, tracciata a bomboletta sopra un manifesto di propaganda per l'indipendenza catalana. Traboccando di ansia, suonò al citofono etichettato ``Urquinaona'', disse la frase codice e salì al piccolo appartamento.

Era un luogo veramente squallido, degno di un film dell'orrore. Il dottor Gambino, che sembrava un medico tanto quanto Fabio sembrava una teiera, lo ricevette senza alcun convenevole. Prese i soldi faticosamente rubati, li contò e mostrò una specie di menù con i vari ``trattamenti''. Fabio ci pensò un attimo, cercando di far entrare più modifiche possibili nel suo budget.

Alla fine decise:

« Il mento più forte, ma non tocchi la barba, se può. Le orecchie, un po' meno\ldots sì, esatto. E\ldots la gobba dal naso, la tolga, già che c'è. »

« Mento a culo,  `recchie da Legolas e naso senza gobba? » riassunse il sedicente chirurgo plastico. « Molto bene. Stenditi che ti drogo. »

Fabio esitò un istante, poi chiese:

« Secondo lei, dottore, è abbastanza per rendermi irriconoscibile? Legalmente parlando, intendo. »

« Ma quale dottore, io sono perito elettronico! », ribatté lui.

« S'ha a andà bbene\ldots »

« So il fatto mio, devi stare tranquillo. »

« Povero me! »

« E ricco io, che mi hai già pagato! »

Il ceffo eruppe in una beffarda risata, ma si ricompose subito. Mise una mano sulla spalla di Fabio e disse:

« Comunque, stammi a sentire: c'hai pochi soldi, lo capisco, ma non sei messo bene. Certo, puoi metterlo nel culo a qualche guardia che ti cerca con la foto. A quelli, già solo con i capelli li fai fessi. Però\ldots appena si mettono a indagare sei fritto, lo vedono che sei sempre il solito stronzo! Stai accorto e, soprattutto, io non ti ho fatto un bel niente. Ora stenditi e dammi il braccio, ti buco. »

Fabio non era per niente rassicurato. Una plastica totale del viso sarebbe stata molto meglio, ma non se la poteva proprio permettere, nemmeno rapinando altre dieci coppie di ricconi. Con un po' di fantasia, comunque, riusciva a farsi piacere gli interventi che aveva scelto: il mento forte, prominente, con carattere; le orecchie affusolate, aerodinamiche; un naso dritto, armonioso ma integerrimo. Con molta fantasia, invece, si poteva pensare che quei piccoli ritocchi fossero addirittura ben studiati: nessuno che volesse nascondere la sua identità alla società si sarebbe cambiato i connotati con caratteristiche così appariscenti. A nessun poliziotto sarebbe venuto in mente di indagare più approfonditamente sulla sua identità, a meno che non si fosse fatto coinvolgere in un grosso pasticcio. Ma quanto era alta la probabilità che succedesse? Decisamente, pericolosamente alta. Ma non importava: ormai doveva farlo, bene o male che fosse fatto. La modifica del suo aspetto sanciva in modo definitivo il suo cambiamento totale.

E il dubbio tornò, attanagliando ancora la sua coscienza come una pressa idraulica: era proprio sicuro di voler intraprendere questa strada? Pensò a quello che stava lasciando indietro. Erano cose brutte, molto brutte\ldots Non poteva sopportarle.

Si soffermò un po' più del solito su queste ultime riflessioni, sebbene non fossero poi così complesse per i suoi standard. Cercò di chiedersi il perché, ma comporre pensieri era diventato molto più difficile del solito: come pianeti in formazione da una nebulosa, l'astratto diffuso nella sua testa aveva bisogno di molto tempo per combinarsi e solidificarsi in concetti concreti. Si sentiva confuso, obnubilato, distaccato da se stesso. Il braccio gli pizzicava forte: tentò di guardarselo, ma l'impresa di volgere la testa si dimostrò ben oltre le sue momentanee capacità. Si chiese blandamente che cosa gli stesse succedendo.

« Stai buono, che la keta sta facendo effetto », disse un ceffo alla sua destra.

``La keta?'', pensò. La domanda sembrò rimbombare nella sua testa, rimbalzando sulle pareti del cranio. ``Ah, la keta\ldots'', concluse. Non si ricordava più il motivo esatto per il quale era lì. In fondo era un posto piuttosto bruttino per stendersi, perché non era a casa sua? Perché? Non solo non riusciva a rispondersi, ma non ricordava nemmeno la domanda. La comprensione di quel che stava succedendo era ormai un ricordo: si arrese. La sua mente si stava avviluppando su se stessa, lasciandolo passivo in balia dei suoi sensi. Il braccio gli bruciava da morire, si sentiva invadere da uno strano fluido ardente che, inesorabilmente, sopraffaceva ogni parte del corpo che riusciva a raggiungere. Intanto, il soffitto si faceva sempre più grande ogni volta che prestava attenzione a ciò che i suoi occhi mostravano, come se la stanza si espandesse con il ridursi della sua capacità di discernimento. Gradualmente, il mondo si spense: rimase solo con se stesso, ad assistere agli artefatti sensoriali della sua mente.
