\documentclass[12pt]{book}
\usepackage[
	paperwidth=17cm,
	paperheight=24cm,
	% other options
] {geometry}

\usepackage[T1]{fontenc}
\usepackage[utf8]{inputenc}
\usepackage[italian]{babel}
\usepackage{baskervald}

% dedication at the beginning of the book
\newenvironment{dedication} {
	\cleardoublepage\thispagestyle{empty}
	\vspace*{\stretch{1}}
	\hfill\begin{minipage}[t]{0.66\textwidth}
	\raggedright}
{
	\end{minipage}
	\vspace*{\stretch{3}}
	\clearpage
}

% quote at the beginning of the chapter
\makeatletter
\newenvironment{chapquote}[2][2em] {
	\setlength{\@tempdima}{#1}
	\def\chapquote@author{#2}
	\parshape1\@tempdima\dimexpr\textwidth-2\@tempdima\relax}
{
\par\normalfont\hfill--\ \chapquote@author\hspace*{\@tempdima}\par\bigskip
}
\makeatother

% first page stuff
\title{
	\Huge \textbf{Il Senno di Poi}
}
\author{
	\textsc{\textbf{Simone Cipriani}}
}

\begin{document}
\setlength{\parskip}{0.25em}
\setlength{\parindent}{0.5em}

\frontmatter
\maketitle
\begin{dedication}
	A Sara, impavida esploratrice dei più squallidi meandri del multiverso letterario, dedico questo sollievo da tanta schifezza.
\end{dedication}

\mainmatter\chapter*{Prefazione}

Sarò onesto: scrivere questo libro è stato \emph{snervante}.

Ho sempre avuto difficoltà a portare a termine i miei sforzi; non tanto per il fatto di non riuscire a mantenere il mio focus sulla stessa cosa per un tempo sufficiente a finirla, quanto per la mia tendenza a rimandare il più possibile quello sgradevole momento in cui sei costretto a confrontarti con gli altri. 

Io detesto gli altri. Prima di questo, ho scritto un altro libro che avevo serenamente intitolato "Odio la Gente". Fate un po' voi.

Va da sé che la mia misantropia mi ha impedito di godere del rinforzo positivo che complimenti e incoraggiamenti possono fornire, e sì, lo confesso: per quanto ami la solitudine, costringersi a perseverare è stato molto, molto difficile. Ci sono stati parecchi momenti in cui ho pensato di lasciar perdere per un po', di serbare la scrittura per un me del futuro più vecchio e, forse, più paziente. In effetti l'ho fatto, rimandando questo lavoro con la scusa dell'urgenza di altri lavori: ho composto, registrato e prodotto molta musica, ho dato vita a diversi progetti più o meno bislacchi, ho affilato le armi che mi permettono di combattere nel mondo del lavoro, ho progredito la mia vita in ogni aspetto. Questo libro è sempre stato lì ad attendermi, osservandomi da lontano, spronandomi a fare ogni altra cosa che valesse la pena di essere fatta, ma che non fosse dedicargli un pizzico di impegno.

Lasciamo perdere. Se stai leggendo questa prefazione, significa che ho portato a termine questo progetto. Evviva me!

Prima che proseguiate addentrandovi in questo inferno, mi sembra doveroso premettere che non sono propriamente uno scrittore. Insomma, de facto lo sono, visto che state leggendo quello che pare proprio essere un libro; diciamo che posso considerarmi scrittore tanto quanto posso considerarmi musicista e programmatore. Appare evidente che se uno dissipa i suoi talenti in così tante diavolerie, beh, è chiaro che non può mai raggiungere l'eccellenza in nessuna attività in cui si cimenta.

Comunque, ormai eccoci qui.

Vi aspetta una serie strana e malsana di sensazioni e sentimenti, malamente impacchettata in discorsi con poco capo e ancor meno coda. Entrerete nella testa dei miei personaggi, e di riflesso entrerete nella mia. Non vi invidio: io vorrei proprio uscirci dalla mia testa, e se proseguirete nella lettura, temo che molto presto capirete il perché.

\section{Ringraziamenti}

Un'intera sezione per i ringraziamenti? Boia. In realtà, devo ringraziare pesantemente soltanto due persone:

\begin{itemize}
	\item \textbf{Sara Carusi}, ancora ad oggi mia fidanzata nonostante tutto questo, che si è prestata ad una continua discussione costruttiva favorendo lo sviluppo della trama.
	\item \textbf{Silvia Fontanelli}, ancora ad oggi mia madre, che ha pazientemente letto e riletto tutta l'opera in cerca di errori ed orrori, nonostante ne fosse palesemente disgustata.
\end{itemize}

Inoltre, mi sento in dovere quantomeno di citare anche i Quattro Lorenzi, che hanno passivamente foraggiato la mia ispirazione per la creazione di alcuni personaggi: Gianassi, Bordoni, Govi e Lucchetti.

E soprattutto, senza curarmi di quanto questa cosa suonerà come un cliché, ci tenevo a farti sapere che il mio ringraziamento più grande va proprio a te, lettore o lettrice. Spero di essere stato in grado di produrre un qualcosa che ti possa far considerare ben speso il tempo che impiegherai a fruirne.

In questo libro ho riversato molti aspetti di me: idee, paure, o semplici emozioni. Sono grato che tu mi abbia permesso di condividerli con te.

\newpage
\tableofcontents
\newpage

% now the beef
\chapter{Cuore Non Duole}

\begin{chapquote}{Pop X, \textit{La Prima Rondine Venne Ier Sera}}
	Figli di puttana senza padri e senza mamma\newline
	Siamo battiti animali, siamo porci senza le ali
\end{chapquote}

%rough editing done

Fabio si tormentava la barba. L'aria salmastra di Barcellona la rendeva crespa, stopposa, appiccicosa. Non riusciva a sopportarla. ``Che stress'' pensò, ricacciando nel groviglio rossiccio un ricciolo particolarmente ribelle. Non ci poteva fare niente: non riusciva a togliersi le mani da barba e capelli. Erano lunghi e folti, e prudevano da impazzire. Nella sua vecchia routine quotidiana, il rasoio era stato il suo migliore amico. Ma la pacchia adesso era finita, le circostanze gli imponevano un aspetto radicalmente diverso dal passato; sempre nervoso e privo di pazienza, Fabio si trovava in seria difficoltà.

Senza perdere d'occhio la coppia che stava seguendo, cercò di distrarsi dal suo tormento con una sigaretta. Non era affatto un fumatore incallito, anzi, di solito non fumava quasi mai. Tuttavia, per la gioia dei suoi amici, si era sempre concesso il pacchetto in tasca. Nemmeno la solitudine lo aveva convinto a privarsi del suo accessorio e, visto che da qualche tempo nessuno gli scroccava più le sigarette, era finito a fumarsele tutte da solo. Non era abituato ad un tale consumo di tabacco: il saporaccio in bocca, l'arsura costante, la tosse, la sensazione di essere sporco, contaminato\ldots\ Erano disagi sopportabili, confrontati con il piacere di succhiare pulviscolo rovente da un bastoncino incendiato. E poi Fabio era morto, almeno per quanto ne poteva sapere il resto del mondo, ed ai morti fumare non fa male.

I suoi due bersagli finalmente si mossero. Con snervante lentezza, si alzarono dalla panchina che aveva ospitato le loro natiche da almeno un'ora.

«\textit{Oh, e l'era l'ora, serpe d'Id\ldots}» imprecò Fabio, incapace di trattenersi.

Odiava aspettare le persone, anche quando doveva derubarle. Trattenendo a stento l'impazienza, si mise a seguire la coppia con goffa discrezione. Mentre camminava, si tastò la cintura e sospirò. L'arma era ben nascosta alla vista, ma il metallo a contatto con la nuda pelle ricordava costantemente a Fabio di esserne in possesso; ce l'aveva già da qualche giorno, ma si sentiva ancora insicuro nel portarla in giro, soprattutto sapendo di doverla usare: non aveva mai infranto la legge prima di allora, eccetto stupidaggini come il codice della strada ed il divieto di possedere droghe leggere. Non c'era più tempo per vuoti timori o scrupoli morali: era scomparso o addirittura morto, cosa avrebbe dovuto temere? Sospirò di nuovo. ``Non ho nulla da perdere'' si disse, sapendo di mentire. Sapeva anche che la sua coscienza non si sarebbe zittita a suon di menzogne: doveva semplicemente pensare di meno ed agire di più.

Affrettò il passo, avvicinandosi ai due tizi che si era prefissato di rapinare. Li aveva seguiti per tutto il pomeriggio, senza perderli di vista un momento: erano certamente dei turisti, entrambi sui quarant'anni, forse sposati e di certo molto facoltosi. Americani? Inglesi? Qualunque fosse la loro provenienza, il loro abbigliamento, tanto appariscente quanto ridicolo, non lasciava molto all'immaginazione: era la classica coppia di ricconi in vacanza, annoiata e già pentita di aver abbandonato il limbo dorato da cui proveniva. In altre parole, rappresentavano in pieno lo stereotipo di persone che Fabio odiava di più in assoluto. Come primo bersaglio erano perfetti, bastava solo continuare l'inseguimento ad oltranza, fino a che i riccastri non si fossero messi in condizione di essere predati facilmente. All'inizio del pedinamento era parsa proprio questione di momenti, la donna si comportava proprio come una stupida oca e l'uomo non aveva l'aria di saper badare a se stesso. Avrebbe dovuto essere facile, ma il sole era già tramontato e Fabio ormai stava per perdere la pazienza.

Poco prima di crollare di nervi e fare una strage, ecco che li vide entrare in una piccola pineta sul lungomare, vicino a La Barçeloneta. Il crepuscolo ormai stava cedendo il passo alla notte ed il posto era praticamente deserto. I due turisti si accasciarono su una panchina, evidentemente esausti per aver trasportato le loro membra per duecento metri. Fabio fremette: era quella la sua occasione!

Prima di agire, si prese un attimo di tregua. Fu fatale: come se non stesse aspettando altro, il cuore gli saltò in gola. Inspirò profondamente, si morse il labbro inferiore ed assaporò il salmastro che la brezza ci aveva depositato. Ora che il sole era calato, il vento marino non era più fresco sollievo, ma pungente fastidio. In qualche strano modo, quei ventosi spilli gelidi lo eccitavano: era uno strano freddo, un freddo inconfondibilmente estivo che lo stuzzicava nel profondo. Guardò attraverso i rami dei pini, rivolto verso un cielo che sovrastava fin troppe persone\ldots

Fino a che punto la sua coscienza gli avrebbe impedito di fare quello che la sua volontà gli imponeva? Aveva veramente intenzione di uccidere? In fondo, quello era il suo intento originale. Non era tanto per provare la pistola --- quello lo avrebbe fatto comunque, in un modo o nell'altro --- ma per testare se stesso. L'avventura che intendeva intraprendere imponeva la capacità di uccidere a sangue freddo: ne sarebbe stato in grado? Pensò che assassinare due innocenti non era proprio il massimo per iniziare; avrebbe preferito giustiziare per esempio un pluriomicida, o uno stupratore, o un politico. Ma, tutto sommato, quelle amebe non erano sicuramente innocenti e rappresentavano ciò che odiava. Sarebbe stato il battesimo del fuoco, la prova perfetta per temprarlo definitivamente: uccidere ciò che lui aveva deciso essere indegno di vivere. Sapeva benissimo, nel profondo, che l'unica ragione del suo odio per la gente del genere era l'invidia: loro erano ricchi e senza preoccupazioni mentre lui no. Sarebbe bastata quella pseudo colpa a giustificare a se stesso la loro esecuzione a sangue freddo? Trasse un altro respiro profondo, assaporò la tensione\ldots\ Sapeva come comportarsi, ed era consapevole delle conseguenze: decise che era tempo di agire e non di pensare.

Estratta l'arma e tolta la sicura, girò attorno alla panchina e si avvicinò alla coppia più silenziosamente che poté. La sua cautela fu inutile, dato che lo starnazzare della sua vittima femmina avrebbe impedito a chiunque di notarlo anche se si fosse messo a cantare.

«\textit{I ain't gonna eat in that shack!}» berciava la snob. Gli ammennicoli appesi alle sue braccia tintinnavano furiosamente mentre gesticolava: avrebbero sovrastato perfino il chiasso di un carro armato.

«\textit{Dear, it's a de\ldots}» provò a rispondere l'uomo, ma altri starnazzi lo chetarono.

Fabio provò un moto di compassione per quell'uomo. ``Se la faccio secca, ci sta che mi ringrazi\ldots'' pensò, in un feroce slancio di cinismo.

«\textit{I ain't gonna eat there! Go find a real place!}» proseguì lei.

Fabio era già stufo della sua voce. Si caricò di determinazione, fece un bel respiro e si manifestò con l'arma in mano.

«\textit{You ain't gonna eat any fuckin' where, if you don't shut the fuck up!}» eruppe, sfoggiando un perfetto inglese da film d'azione.

La donna strillò alla vista della pistola. L'uomo invece sgranò gli occhi e guardò con apprensione il suo aggressore, senza emettere alcun suono.

«\textit{Give me money, quick. No time to waste.}» disse Fabio, aggressivo.

Calò un silenzio assoluto, quasi imbarazzato, rotto soltanto dai mugolii spaventati della riccona avvinghiata stretta al braccio del suo cavaliere, il quale sembrava sul punto di piangere.

«\textit{Don't be a motherfuckin' cunt}» gli disse Fabio, impaziente. «\textit{You are disgustingly rich, I'm not. I've got a weapon, you don't. Can you see how things combine perfectly? Come on! Your money for my clemence.}»

Per l'effetto che ebbero le sue parole, avrebbe potuto parlare alla panchina. L'immobilismo dei due stava per fargli perdere il controllo. \textit{``Mostro di\ldots! Che ti mòvi?''} imprecò nella sua testa. Puntò la pistola dritta in mezzo agli occhi dell'uomo.

«\textit{I said: your money\ldots}» ripeté, mimando i soldi con la mano libera, «\textit{\ldots for my clemence. Don't you think it's a good deal?}»

Nessuno dei due emise alcun suono, e l'atmosfera si fece così densa che Fabio si stupì di riuscire a respirare. La collera lo stava divorando dall'interno. Perché il tizio non faceva niente? Lo stava minacciando per avere i suoi soldi, non sembrava affatto una situazione ambigua! Non riusciva proprio a capire perché le sue vittime si fossero pietrificate, invece di sbrigarsi a consegnare i soldi per aver salva la vita.

Finalmente, con tremenda lentezza, l'uomo annuì. Fabio sorrise, sollevato. Abbassò l'arma e tese pronto la mano, stampandosi in volto un falsissimo sorriso da venditore porta a porta. Successe qualcosa di incredibile: la donna trasse un enorme respiro, scoccò all'uomo uno sguardo di puro disprezzo e gli tirò uno schiaffo, cominciando a strillare furiosa.

«\textit{Are ya givin' up?! This' the way you are protectin' me?! Fight like a man, protect me!}»

Fabio ci rimase di stucco: questo proprio non se l'era aspettato. Scoppiò a ridere, completamente spiazzato dall'assurdità della scena, ma puntò con fermezza la pistola verso la signora imbizzarrita.

«\textit{I want nothing from you, you maggot!}» le disse, divertito.

Lei si chetò all'istante.

«\textit{Shut up and let your man give me some money}», proseguì. «\textit{Be happy for I'm not raping you\ldots}»

La donna si impietrì di nuovo, terrorizzata, e la situazione parve risolversi: l'uomo estrasse dei soldi dal suo costoso portafogli e li depose a terra. Fabio intimò alla coppia di stare indietro e li raccolse, soffermandosi a contarli.

Non soddisfatto, si rivolse all'uomo:

«\textit{I saw what you did, you still got some. Would you mind give me another green one? Come on, I bet you earn one of this in a day.}»

Lui obbedì, ma non proferì parola. La povera stolta, innescata dalla vista di Fabio che contava i soldi che avrebbe voluto spendere lei, ricominciò a starnazzare. Con un teatrale, grazioso movimento, percosse con la borsa il suo compagno in un improvviso slancio di follia isterica. L'uomo non reagì, anzi sospirò, probabilmente sollevato per aver avuto salva la vita. Fabio rise di nuovo. Si sentiva molto più leggero ora che avvertiva la presenza dei soldi nella tasca. Il suo proposito omicida gli sembrava un bizzarro ricordo. Perché avrebbe dovuto ucciderli? Aveva già quello che voleva.

Prima che realizzasse quanto fosse stupido intrattenersi ancora con le sue vittime, parlò:

«\textit{Leave him alone! He was just robbed. Can you please shut the fuck up?}»

L'aria si congelò di nuovo, anche se Fabio ormai aveva messo via la pistola. Cosa diavolo stava per succedere? L'uomo lo guardò eloquentemente, spaventato come mai era stato fino ad allora. La donna emise un orrendo sibilo e, senza alcun preavviso, si scagliò su Fabio, sputandogli addosso e cominciando ad insultarlo con i suoi starnazzi. Lui fece un balzo indietro, confuso dall'improvvisa aggressione, ma ben pronto a reagire.

«Come osi, stupida!», ringhiò, e le assestò un perfetto yoko-geri che la spinse indietro di parecchi metri.

Il karateka che era in lui non si sentì affatto a disagio nel colpire una donna: un'aggressione chiama sempre una difesa ed un contrattacco, indipendentemente dalle circostanze. Ma l'azione improvvisa, come un fruscio nel sottobosco per un cacciatore, aveva evocato un Fabio ben più pericoloso del karateka: il Fabio predatore.

Anche se il colpo aveva atterrato l'isterica, non la aveva certo zittita. ``Non ha proprio istinto di sopravvivenza\ldots'' pensò Fabio, mentre estraeva nuovamente la pistola.

All'improvviso, per la prima volta, l'uomo fece uso della parola: «\textit{Cecilia, quiet!}»

La sua voce, inaspettatamente ferma e dal timbro profondo, risuonò per il parco in una sorta di inquietante eco. La folle si zittì immediatamente, ma ormai era inutile: Fabio moriva dalla voglia di farlo. In uno di quei rari momenti in cui l'azione bypassa completamente l'intelletto, puntò la sua nove millimetri silenziata sulla donna, sorrise all'uomo e sparò. Non fu rumoroso: il colpo produsse solo un suono sordo, simile ad uno starnuto. Era l'arma perfetta per i suoi scopi, pensò Fabio prima di riconnettersi alla realtà: nessuno avrebbe riconosciuto quel rumore per quello che era.

La snella figura si contorceva a terra: era stata colpita sul fianco. Resosi conto di ciò che aveva fatto, Fabio fu pervaso da un panico molto strano, soprattutto inatteso. Guardò il corpo spasimare dal dolore e macchiarsi di sangue: la donna non sembrava ferita mortalmente. Non urlava, ma piangeva. Cercando di calmarsi, si leccò le labbra, sempre appiccicose di salmastro, e gli parve di assaporare le lacrime della sua vittima. Il panico si trasformò immediatamente in eccitazione, una selvaggia e crudele eccitazione. Non aveva mai provato una sensazione del genere, si sentiva come ubriaco, ma con il controllo della situazione. Non abbassò l'arma, ma si avvicinò alla donna ferita. Lei cercò di trattenere il pianto e chiuse gli occhi, tremante.

«\textit{I can't stand those like you}», le disse piano, scandendo le parole per assicurarsi che lei lo capisse. «\textit{You deserve this pain.}»

L'uomo nel frattempo si stava esibendo nella sua migliore performance di immobilità: era totalmente interdetto, sopraffatto dagli eventi.

Fabio si rivolse a lui, il volto deformato dalla malvagità, l'arma puntata verso la sua vittima:

«\textit{Should i do it?}»

Si aspettava che negasse, che lo supplicasse di lasciare in vita la sua amata. Era già pronto a ridere di lui. Invece l'uomo, celato dietro un'espressione grave, sembrava soppesare la possibilità di rispondere in disaccordo con il copione dell'ovvio. I secondi passavano, l'adrenalina stava svanendo, l'eccitazione stava lasciando il posto ad un più consono disagio. E se gli avesse chiesto di farlo? Lo avrebbe fatto davvero? Anche a sangue freddo? Lentamente, tremando, l'uomo scosse il capo.

Fabio abbassò l'arma.

«\textit{Thank you}», gracchiò egli, trattenendo a stento le lacrime.

Fabio sgranò gli occhi allibito, e per poco non gli cadde la pistola. Mai avrebbe scommesso, neanche un fagiolo contro un lingotto d'oro, che una sua vittima lo ringraziasse. Di cosa, poi?

«\textit{She's a\ldots but\ldots please, let her live\ldots}»

Una folata di vento un po' più veloce delle altre alzò un lieve turbinio di aghi di pino. La situazione aveva raggiunto un livello di assurdità ormai non più tollerabile. Fabio provò il bizzarro impulso di aiutare la donna e di scusarsi con l'uomo, ma aveva ancora abbastanza autocontrollo da soffocarlo. Ripose la pistola e, senza dire altro, andò via a passo svelto. Era estremamente turbato: il suo velo di risolutezza stava per cadere, lasciandolo faccia a faccia con la realtà di ciò che era successo.

Raggiunto un punto del litorale abbastanza lontano dal luogo del delitto, fece dei respiri profondi e si calmò. La passeggiata si stava riempiendo di gente pronta per la movida notturna, ma in spiaggia c'erano ancora alcuni temerari. Fabio li guardò, perso nei suoi pensieri. Si mise una sigaretta in bocca, ma non la accese. Prese a tormentarsi la barba\ldots

Non era andata come aveva programmato, non aveva ucciso nessuno, almeno materialmente. Però, nel momento di premere il grilletto, era diventato comunque un assassino: aveva deciso di uccidere, ed aveva posto in essere un'azione apposita per raggiungere quello scopo. Era moralmente un omicida tanto quanto lo sarebbe stato se il suo proiettile avesse colpito un punto vitale. Poteva giovarsi di questa sicurezza: era in grado di uccidere. Ma la cosa, invece di corroborarlo, lo turbava. Cosa diavolo era diventato?

E poi la reazione dell'uomo\ldots Alla fine si era lasciato andare, sicuramente disperato al pensiero di poter vedere la sua amata morire, ma per qualche istante aveva accettato l'avvenimento ed addirittura valutato la possibilità di farlo accadere. In quel momento, prima di rendersi conto dei torbidi lidi nei quali i suoi pensieri avevano vagato, il povero derubato era stato freddo, cinico, calcolatore: si era immaginato, in modo perfettamente razionale, come sarebbe potuta evolvere la sua vita senza quella palla al piede. Se fosse riuscito a resistere ai sentimenti, a restare razionale, sensato, solo per un altro minuto, avrebbe guadagnato la libertà. Ma non sarebbe potuto succedere\ldots Se l'uomo avesse posseduto quella forza, avrebbe saputo anche spezzare il suo legame con quella stupida oca senza aver bisogno di farla ammazzare. O addirittura, non avrebbe mai provato dei sentimenti per una persona così immeritevole\ldots

Ma che ne sapeva lui? Aveva troppi pregiudizi, troppa supponenza, ancora troppa empatia nei confronti dei suoi simili. La donnicciola magari non era come gli appariva, forse era lei dalla parte giusta della relazione; forse era il pover'uomo ad essere il suo peso. O forse doveva smetterla di pensare a chiunque tranne che a se stesso. Oppure, addirittura, come si trovava a dirsi fin troppe volte, doveva smettere di pensare e muoversi.

\chapter{Metamorfosi}

\begin{chapquote}{Author's name, \textit{Source of this quote}}
``This is a quote and I don't know who said this.''
\end{chapquote}

Ormai si era fatta notte, non proprio l'ora adatta per gironzolare nelle profondità de La Barçeloneta. Quella zona trasudava degrado e criminalità da ogni casa, vicolo ed angolo buio. ``È solo un'impressione'' si disse Fabio, ed in effetti lo era: tutto taceva tranquillo, come era normale che fosse in un quartiere residenziale di notte. In lontananza si udivano gli schiamazzi della movida, urla di turisti ubriachi intenti a perdere la loro dignità: sembrava tutto regolare. Non sentendosi comunque al sicuro, mise una mano sulla pistola ed affrettò il passo, scrutando nervosamente l'ambiente circostante. Si sentiva costantemente osservato, controllato dal quartiere stesso. Ogni palazzo gli sembrava uguale all'altro. Era già passato di lì?

Finalmente, dopo un lungo peregrinare, trovò l'edificio che stava cercando. Ad un primo sguardo appariva esattamente uguale agli altri, ma sul muro aveva il segno distintivo di cui il suo amico Bruno gli aveva parlato: una svastica con gli uncini al contrario, tracciata a bomboletta sopra un manifesto di propaganda per l'indipendenza catalana. Traboccando di ansia, suonò al citofono etichettato ``Urquinaona'', disse la frase codice e salì al piccolo appartamento.

Era un luogo veramente squallido, degno di un film dell'orrore. Il dottor Gambino, che sembrava un medico tanto quanto Fabio sembrava una teiera, lo ricevette senza alcun convenevole. Prese i soldi faticosamente rubati, li contò e mostrò una specie di menù con i vari ``trattamenti''. Fabio ci pensò un attimo, cercando di far entrare più modifiche possibili nel suo budget.

Alla fine decise:

« Il mento più forte, ma non tocchi la barba, se può. Le orecchie, un po' meno\ldots sì, esatto. E\ldots la gobba dal naso, la tolga, già che c'è. »

« Mento a culo,  `recchie da Legolas [1] e naso senza gobba? » riassunse il sedicente chirurgo plastico. « Molto bene. Stenditi che ti drogo. »

Fabio esitò un istante, poi chiese:

« Secondo lei, dottore, è abbastanza per rendermi irriconoscibile? Legalmente parlando, intendo. »

« Ma quale dottore, io sono perito elettronico! », ribatté lui.

« S'ha a andà bbene\ldots [2] »

« So il fatto mio, devi stare tranquillo. »

« Povero me! »

« E ricco io, che mi hai già pagato! »

Il ceffo eruppe in una beffarda risata, ma si ricompose subito. Mise una mano sulla spalla di Fabio e disse:

« Comunque, stammi a sentire: c'hai pochi soldi, lo capisco, ma non sei messo bene. Certo, puoi metterlo nel culo a qualche guardia che ti cerca con la foto. A quelli, già solo con i capelli li fai fessi. Però\ldots appena si mettono a indagare sei fritto, lo vedono che sei sempre il solito stronzo! Stai accorto e, soprattutto, io non ti ho fatto un bel niente. Ora stenditi e dammi il braccio, ti buco. »

Fabio non era per niente rassicurato. Una plastica totale del viso sarebbe stata molto meglio, ma non se la poteva proprio permettere, nemmeno rapinando altre dieci coppie di ricconi. Con un po' di fantasia, comunque, riusciva a farsi piacere gli interventi che aveva scelto: il mento forte, prominente, con carattere; le orecchie affusolate, aerodinamiche; un naso dritto, armonioso ma integerrimo. Con molta fantasia, invece, si poteva pensare che quei piccoli ritocchi fossero addirittura ben studiati: nessuno che volesse nascondere la sua identità alla società si sarebbe cambiato i connotati con caratteristiche così appariscenti. A nessun poliziotto sarebbe venuto in mente di indagare più approfonditamente sulla sua identità, a meno che non si fosse fatto coinvolgere in un grosso pasticcio. Ma quanto era alta la probabilità che succedesse? Decisamente, pericolosamente alta. Ma non importava: ormai doveva farlo, bene o male che fosse fatto. La modifica del suo aspetto sanciva in modo definitivo il suo cambiamento totale.

E il dubbio tornò, attanagliando ancora la sua coscienza come una pressa idraulica: era proprio sicuro di voler intraprendere questa strada? Pensò a quello che stava lasciando indietro. Erano cose brutte, molto brutte\ldots Non poteva sopportarle.

Si soffermò un po' più del solito su queste ultime riflessioni, sebbene non fossero poi così complesse per i suoi standard. Cercò di chiedersi il perché, ma comporre pensieri era diventato molto più difficile del solito: come pianeti in formazione da una nebulosa, l'astratto diffuso nella sua testa aveva bisogno di molto tempo per combinarsi e solidificarsi in concetti concreti. Si sentiva confuso, obnubilato, distaccato da se stesso. Il braccio gli pizzicava forte: tentò di guardarselo, ma l'impresa di volgere la testa si dimostrò ben oltre le sue momentanee capacità. Si chiese blandamente che cosa gli stesse succedendo.

« Stai buono, che la keta [3] sta facendo effetto », disse un ceffo alla sua destra.

``La keta?'', pensò. La domanda sembrò rimbombare nella sua testa, rimbalzando sulle pareti del cranio. ``Ah, la keta\ldots'', concluse. Non si ricordava più il motivo esatto per il quale era lì. In fondo era un posto piuttosto bruttino per stendersi, perché non era a casa sua? Perché? Non solo non riusciva a rispondersi, ma non ricordava nemmeno la domanda. La comprensione di quel che stava succedendo era ormai un ricordo: si arrese. La sua mente si stava avviluppando su se stessa, lasciandolo passivo in balia dei suoi sensi. Il braccio gli bruciava da morire, si sentiva invadere da uno strano fluido ardente che, inesorabilmente, sopraffaceva ogni parte del corpo che riusciva a raggiungere. Intanto, il soffitto si faceva sempre più grande ogni volta che prestava attenzione a ciò che i suoi occhi mostravano, come se la stanza si espandesse con il ridursi della sua capacità di discernimento. Gradualmente, il mondo si spense: rimase solo con se stesso, ad assistere agli artefatti sensoriali della sua mente.

\chapter{Flashback: il Fuoco}

\begin{chapquote}{Author's name, \textit{Source of this quote}}
``This is a quote and I don't know who said this.''
\end{chapquote}

Turbinii colorati, lampeggii arzigogolati. Sferzavano lo spazio, rimanendo impressi nella realtà per diversi istanti, prima di venire sovrascritti da altri guizzi colorati. Il danzare frenetico degli artefatti psichedelici era così regolare da essere ipnotizzante: tutto accadeva e smetteva di esistere armonicamente, rispettando proporzioni temporali sconosciute ma perfette. Sembrava una danza destinata ad essere eterna: tutta quella perfezione non poteva che durare per sempre. Nonostante l'eternità intrinseca, l'insieme aveva una sua progressività: ogni ciclo cromatico faceva acquisire solidità al tutto. Piano piano, senza perdere il moto perpetuo che il sistema aveva acquisito, comparve qualcosa di molto simile alla realtà concreta: c'erano tavolini bassi, da fumo, e divanetti di finta pelle scadente.

Fabio era seduto su uno di essi. Osservava i resti del bicchiere di plastica che aveva appena rotto, maledicendo la sua stretta compulsiva. Sembrava visibilmente stanco, ma più probabilmente era solo un po' brillo. Voleva prendere altro da bere, ma gli faceva tanta fatica: il divanetto era così avvolgente, la palpebra così pesante\ldots

La sua sete lo costrinse a vincere la stanchezza. Si alzò, attraversò la stanza gremita di gente e si accinse a preparare un cocktail al tavolo del buffet. ``Che gin di merda'', pensò mischiando l'alcolico scadente con rum chiaro e vodka. Mentre esternava il suo disappunto con un suono a metà fra un grugnito ed un ringhio, si accorse che mancava il triple sec.

« `io serpente maiale! » imprecò a denti stretti, mangiandosi una consonante.

Buttò via il suo abbozzo di Long Island e, esasperato, scrutò il tavolo, maledicendo chi avesse scelto le bottiglie da portare. Concluse che l'unica cosa che avrebbe potuto preparare senza scendere a compromessi disumani fosse un Vodka Tonic  ``Almeno la vodka è decente'', si convinse, mentre cercava con crescente disperazione un bicchiere più o meno pulito.

Preparato il suo misero drink, ne bevve un sorso e si guardò intorno. Il suo sguardo scorse su varie persone, senza esserne rapito; dribblò accuratamente una ragazza bionda e si soffermò infine su un tizio che somigliava ad un becchino, appoggiato al muro dalla parte opposta della stanza.

``Gazzi, quanto sei patetico'' pensò Fabio. ``Sarò disperato, cornuto, magari anch'io patetico, ma almeno non sono te''.

Il Gazzi gli restituì lo sguardo ammiccando, poi riprese ad osservare gli altri con distacco. Era un ragazzo di media statura, aveva la pelle olivastra e i capelli neri; il suo volto presentava dei tratti estremamente comuni, quasi addirittura banali. Fabio continuò a guardarlo, incapace di fermare il flusso suo di pensieri, che ormai scaturiva libero dalla sua mente non proprio lucida.

Le scarpe del Gazzi lo infastidivano parecchio. A dire il vero, odiava ogni cosa che quell'idiota aveva addosso, dalla camicia di sartoria con la cifratura sul colletto, all'orologio Panerai da diverse migliaia di euro. Però le sue scarpe\ldots le sue maledette scarpe lo mandavano proprio in bestia. Cosa diavolo aveva nella testa? Quale assurdo processo logico gli aveva permesso di discernere che indossare delle scarpe del genere in quel contesto fosse una buona idea? In fondo, c'era un motivo per cui nove persone su dieci portavano gli anfibi, e non era solo il fatto di essere ad una festa di simil-rockettari in un locale underground. La stanza era in condizioni pietose, così come metà dei suoi occupanti. Per terra c'era di tutto, dalla cenere al vomito, perfino qualche persona troppo provata per mantenere la propria dignità a livelli accettabili. Ogni passo rischiava di far perdere valore per diverse centinaia di euro a quelle scarpe. ``Gazzi io ti ammazzerò'' pensò Fabio, mentre con un furioso sorso seccava la sua bevuta, ``sei troppo stupido per vivere''.

« Fontanelli! Tutto a posto? Vuoi uccidere il buon Gazzi? »

Fabio trasalì e si voltò: un ragazzo biondo dal volto pallido e scavato apparve davanti a lui.

« Dani » prese a rispondere, un po' turbato dal fatto che l'inquietante figuro avesse indovinato i suoi pensieri, « certo che lo voglio uccidere, chi non vorrebbe? »

« Beh, zio, chi non vorrebbe non lo so, ma l'aria assassina che avevi mentre lo scrutavi non lasciava dubbi. Potevo vedere il fuoco del nazionalsocialismo avvamparti nello sguardo mentre progettavi la spedizione in Siberia di Giacomo Gazzi. Se ti avessi lasciato elucubrare un altro po' si sarebbe sentita la colonna sonora dei tuoi pensieri ».

« Cioè? » chiese Fabio, divertito. Adorava gli sproloqui totalitaristi di Daniele Brogelli, discorsi privi di qualsiasi significato ma esilaranti e terrificanti allo stesso tempo.

Lui gonfiò il petto e proclamò in tono solenne, gesticolando pomposamente:

« Un rumore\ldots anzi no, un coro, perché è musica\ldots Un coro di anfibi che marciano, calpestando facce bolsceviche; urla di prigionieri politici che, martoriati dalle fruste, poggiano le pietre per la costruzione della ferrovia transiberiana, sulla quale i leader della rivoluzione sfileranno trionfanti godendo della sofferenza che ne permise la costruzione! Allora Egli si desterà dal suo sonno, la divisa marmorea sulla fronte lambita dal vento, e marcerà trionfante sulla Piazza Rossa, acclamato da pugni alzati e saluti romani! »

La risata di Fabio eruppe.

« Dani, sei troppo sobrio », gli disse non appena riuscì a riprendere fiato, « fai discorsi fin troppo sensati. Fatti fare un beverino prima che il Leka ti senta e ti prenda sul serio. »

Fabio preparò altri due Vodka Tonic, pregando che Daniele non facesse caso al fatto che per lui aveva preso un bicchiere vuoto abbandonato sul tavolo da chissacchì.

« Insomma zio », parlò Daniele, « questo Gazzi non la vuole smettere di fare il dandy nemmeno qui al Cipher, eh? »

Fabio osservò per un attimo il volto sciupato dalla droga prima di rispondere. Daniele, con tutti i difetti che poteva avere, era affidabile. Il suo passato da tossicodipendente e le cicatrici che aveva lasciato nel suo presente lo rendevano un sicuro confessore. Se mai avesse deciso di divulgare segreti in giro, non sarebbe stato preso molto sul serio. A Fabio bastava: decise che poteva intraprendere con lui questa conversazione.

« Era proprio quello a cui stavo pensando quando lo guardavo con aria omicida », gli rispose.

« Zio, io non lo biasimo troppo. E' vuoto dentro, mi capisci vero? Ha bisogno di fare così. »

« Ma dove prende i soldi per vestirsi a quel modo? Voglio dire, io non potrei mai permettermi tutta quella roba, se mai avessi voglia di comprarmela. »

« Fa sacrifici, sicuramente. Vedi che va sempre a scrocco: non si compra mai le bombe, i beverini li gattona agli altri, cose così. Sinceramente dopo un po' scoccia anche, ma è una persona così piccola che non me la sento di dirgli mai niente. Tu lo odi, è normale che lo odi\ldots »

« Quindi, praticamente quasi non mangia per andare in giro vestito da milionario? »

« Sì, secondo me sì. Lavora e vive per i pochi momenti in cui può sentirsi davvero vivo, mi capisci? L'ho fatto anche io, so cosa significa\ldots »

Fabio pensò. Praticamente Gazzi si drogava con la sua apparenza: gli piaceva così tanto credersi migliore dei quattro stronzi che frequentava da sacrificare ogni altra cosa nella sua vita in favore di quella sensazione. Non riusciva proprio a provare pietà per lui, solo puro disgusto.

« Dai zio, basta parlare di cose brutte, fanne tre. »

Si misero a sedere e cominciarono a lavorare.

« Con la Denise come va? » chiese Fabio, mentre trafficava con il suo borsello in cerca dell'occorrente.

« Come sempre, zio. Lo sai come siamo fatti, non credo cambieremo mai. Godiamo dei nostri pregi e sopportiamo i nostri difetti. Certo, a volte sopportiamo più di quanto godiamo, ma viviamo per quando succede il contrario. Abbiamo imparato ad andarci bene così, non ci chiediamo niente di ciò che non possiamo darci. Siamo liberi. »

« Beati voi » sospirò Fabio, soppesando la mista e decidendo che la materia prima era sufficiente.

« Te e la tua bella ormai siete al capolinea? »

« Magari la situazione fosse così definita. O forse lo è, ma io non riesco a trovarne un senso. Lei ormai\ldots È palese, ma fa finta di niente. Fa sempre finta di niente\ldots Ho cominciato a farlo anch'io. E anche lui fa finta di niente, il nostro caro Gazzi! Accidenti a lui e alla sua stirpe! Si comportano come se non stesse succedendo niente, tutti quanti, ma nel frattempo tutto il mondo sa che ho le corna! »

« Zio, io non ti ho mai detto niente, ma le voci come arrivano a me arrivano anche a te, è chiaro. »

« Non è solo questo, comunque. Sta andando tutto a puttane Dani, veramente tutto: ormai non ho più un lavoro, o una qualsiasi fonte di reddito, né una famiglia che mi possa aiutare\ldots Non ho più voglia di trovare giustificazioni assurde alla merda che mi sta arrivando addosso, voglio\ldots Voglio smettere. »

« Stai pensando al suicidio? »

Il volto di Daniele appariva ancora più sciupato del solito, ma nei suoi occhi c'era un risoluto fervore. Fabio sostenne il suo sguardo.

« No » rispose amaramente. « Piuttosto mi armo fino ai denti e comincio ad ammazzare gente a caso, finché non mandano l'esercito a fermarmi. Poi erre uno, erre uno, tondo, erre due, su giù tre volte e ricomincio. Cazzate a parte, perché dovrei? Ce l'ho con gli altri, non con me. Chi si suicida\ldots È un debole. »

Ci fu un lungo istante in cui l'atmosfera si addensò, facendo precipitare sulla scena un assordante silenzio.

Infine, Daniele parlò:

« Non diresti così, amico, se avessi passato tutto quello che ho passato io. »

« Scusa » disse Fabio, vergognandosi un po' per la sua mancanza di tatto.

Il tema era molto delicato per Daniele. A causa della sua tossicodipendenza --- e dei motivi che lo avevano condotto ad essa --- aveva attraversato momenti di pura disperazione.

« Il suicidio può essere generosità, non resa », spiegò. « Può essere il sacrificio della tua esistenza in favore di quella dei tuoi cari. Se tu fossi un peso economico per la tua famiglia, un motivo di vergogna ed un pericolo per la loro incolumità, toglierti la vita sarebbe la cosa più altruista che tu potresti fare. Ma sarebbe anche la cosa più facile e più povera di amore per te stesso. »

Gli occhi di Daniele si inumidirono un po'. Fabio abbassò lo sguardo in forma di rispetto: Daniele era più grande di lui ed infinitamente più saggio. L'oscurità dalla quale era riemerso sminuiva qualunque altro tormento. Non avrebbe dovuto permettersi di apostrofarlo in quel modo.

« Dani, scusa, davvero. Sono stato indelicato. »

« No, Fontanelli, sei stato onesto. Sei fatto così, non ti interessa se quello che dici ferisce il prossimo. Ma almeno sei vero, perciò ti perdono. Di finta commiserazione ne ho avuta abbastanza. Appizza `sta fiamma, zio. »

Il fumo vorticava turbolento verso il soffitto, prima di ricadere dolcemente avvolgendo i due ceffi in una mistica nebbia. Lo stordimento penetrava deliziosamente la corazza di ostentata sobrietà di Fabio, facendo capitolare la sua patetica maschera e costringendolo ad apparire com'era veramente: ubriaco, stupefatto e stravolto dalla stanchezza. Neanche il tempo di godersi quello stato di nuda trascendenza, che un evento catturò l'attenzione di tutti i presenti. Le risse erano all'ordine del giorno in locali come quello, eppure il gruppo della festa era sempre stato piuttosto tranquillo, relativamente al tipo di gente da cui era composto.

« Ebreo di merda, t'ammazzo! » sbraitò un ragazzo enorme dai capelli rossi, gettandosi contro un tizio alto e un po' torto.

I due si aggrovigliarono in una pietosa lotta a terra. Parecchie persone accorsero per rimetterli in piedi, anche se nessuno aveva la reale intenzione di interrompere le ostilità. Fabio non si mosse nemmeno dal suo divanetto, del quale ormai faceva parte, ma osservò con vivo interesse la scena: i due combattenti erano suoi amici.

« Abbozzala » ansimò il ragazzo torto, piantando un sonoro montante nello stomaco al suo avversario, stendendolo, ma scivolando di nuovo a terra per lo sforzo.

Neanche il tempo di riprendere fiato che il rosso si rialzò inviperito, assestandogli due pedate sulle gengive, probabilmente rompendogli qualche dente.

« Che la fa'e finita! », urlò furiosa la ragazza bionda di cui Fabio tentava disperatamente di ignorare l'esistenza.

Udita la sua voce, la coppia di combattenti si congelò all'istante.

« Che c'ave'e? » continuò lei. « O andae a' ammazzavvi fòri, o l'abbozzate! Che ave'e `nteso? »

I due si lasciarono andare, non osando disubbidire alla ragazza, anche se dai loro sguardi si intuiva che avrebbero voluto combattere fino all'ultimo respiro.

« Sei un cazzo di ebreo, infame di merda! » cominciò a sbraitare il ragazzo dai capelli rossi, schifosamente ubriaco. « Che cazzo ti ho fatto?! Eh?! Sei un infame, un infame\ldots Fai schifo, speriamo tu muoia, sempre a pensare ai soldi, ebreo di merda\ldots »

« O' Anton, la fa' finita? Bada `ome tu l'ha' conciato! Oh'icché t'arà ma' fatto i' Bagonghi? » sbottò ancora la ragazza, esasperata.

Anton non sembrò minimamente intenzionato a rispondere: prese ed andò via, senza smettere di borbottare confuse maledizioni.

Il Bagonghi, senza osare rialzarsi da terra, si mise a sedere a fatica, sputò una boccata di sangue misto a denti e mugolò:

« Carabinieri\ldots La bamba\ldots Non potevo proteggerlo, mi sono tirato fuori\ldots Avrei chiuso\ldots Forse chiuderò comunque, ma ci mancava soltanto lui\ldots »

« I carabinieri hanno trovato la droga al Leka? E te? L'hai licenziato? `o cane, e ci credo l'è `ncazza'o! » lo incalzò la ragazza.

Bagonghi annuì, sputò, sillabò una bestemmia e poi rispose:

« Lavinia\ldots Che potevo fare\ldots Non li ho mai corrotti i carabinieri, non ho avuto scelta, non potevo impedirgli di entrare e di frugare dove volevano, c'era il mandato\ldots »

Fabio, vinta la repulsione che aveva ad avvicinarsi a Lavinia, si scollò dal divanetto e, insieme al Brogelli, alzò il suo amico da terra.

Dopo essersi rimesso in piedi, egli continuò:

« Quell'idiota del Leka crede che li abbia chiamati io i caramba, l'altra settimana abbiamo litigato\ldots Si fa troppo i cazzi suoi, mi ha fatto degli errori che mi hanno fatto perdere diversi soldi\ldots Crede che volessi una scusa per licenziarlo e che non mi importasse di rovinargli la vita\ldots »

« Non posso credere che tu abbia fatto una cosa del genere! » esclamò Daniele.

« No che non l'ho fatta! » replicò brusco il Bagonghi, sputacchiando sangue e saliva ovunque. « L'ho solo rimproverato perché mi aveva fatto perdere dei soldi, non lo voglio far schedare! Non so come cazzo hanno fatto i caramba ad avere un mandato\ldots »

« Certo è proprio stupido » intervenne Fabio, incapace di trattenersi. « Cosa ti porti a fare la coca al lavoro? Una bottarella in pausa caffè, così la giornata prende tutta un'altra piega? »

« Credo la spacciasse a qualche mio dipendente », disse il Bagonghi, ammiccando in una certa direzione. « Ma non me ne frega niente, se non l'avessero beccato non lo avrei licenziato, che cazzo me ne frega della droga! »

« Già, che cazzo te ne frega della droga\ldots » ripeté meccanicamente Fabio, assorto nell'osservare qualcuno a cui, nella più completa ipocrisia, della droga fregava eccome, soprattutto quando poteva ostentare il fatto di potersela permettere. Perché non riusciva a smettere di tirarlo in ballo in qualsiasi suo pensiero?

Le cose stavano cominciando ad assumere nuovamente la loro forma eterea ed i suoni si stavano ovattando sempre di più. Ma la scena non svanì abbastanza in fretta: Lavinia si accorse che Fabio stava guardando il Gazzi, probabilmente sfoggiando il suo sguardo trucidatore. Fece per avvicinarsi, forse stava per parlare, ma Fabio la bloccò.

« Lascia stare, lascia stare\ldots », le sussurrò.

Il suo sguardo incrociò quello della ragazza: amaramente, ci sprofondò dentro. Come una dolce morfina che neutralizza il dolore al malato, l'ambiente psichedelico riapparve prepotente, cancellando ogni barlume di realtà.

\chapter{Flashback: il Fumo}

\begin{chapquote}{Author's name, \textit{Source of this quote}}
``This is a quote and I don't know who said this.''
\end{chapquote}

% troppi punti di sospensione, levane un po'

Com'era bello vagare alla deriva fra i meandri della mente! Colori, suoni, brividi\ldots Faceva tutto parte di un unico, vaporoso maelstrom psichedelico. Lentamente, quel mondo bizzarro si addensò in una sorta di etere colorato, che si ridusse improvvisamente in foschia. Non una foschia mistica, artefatto della mente: una densa, puzzolente nebbia concreta.

La stanza che la nube nascondeva era così piccola che una sola sigaretta la aveva ridotta ad una camera a gas. Fabio odiava il puzzo del tabacco: aprì una finestra e la nebbia sparì. Comparve una ragazza fulva, seduta ad un tavolo, intenta a girare la manovella di una specie di macinino. Il suo volto era confuso, gonfio, stanco, ma i suoi occhi verdi erano vispi.

« Denise » disse Fabio, sedendosi, « Che ho fatto di male? »

« Mah, un sacco di cose direi\ldots » rispose vaga lei, mentre rovesciava il contenuto del dispositivo su una specie di tagliere di ceramica. « Bevi, fumi, ti droghi, appena hai l'occasione rubi, se ne avessi motivo uccideresti\ldots Fai tante cose di male, Fabio. »

« Quindi mi merito tutto questo? »

« No, non l'ho detto. Mi hai chiesto cosa hai fatto di male, e io\ldots Tutto qui. »

Fabio sospirò. Appoggiò la vita alla finestra, si sporse leggermente e guardò fuori. La luna si rifletteva debolmente in uno specchio d'acqua, come nelle migliori cartoline suggestive, totalmente incurante di riflettersi in una pozza rimasta su un piazzale di una zona industriale. Questo contrasto, la bellezza della luce lunare e il grezzo mondo del lavoro, faceva spaziare l'atmosfera dal romantico al sinistro. ``Le zone industriali'' pensò Fabio, incapace di non dare un giudizio a quel luogo, ``sono strane per viverci. Questo condominio\ldots di giorno, accerchiato dal trambusto del lavoro\ldots di notte, immerso nell'abbandono\ldots Una casa come questa, piccola, in alto, sembra quasi un rifugio, un nido\ldots''

Una voce lo distolse dai suoi pensieri. Si era addirittura dimenticato di essere in compagnia.

« Che guardi? »

« È tetro, fuori » disse Fabio, senza voler significare granché.

« Già », rispose Denise, « è sempre tetro fuori. Ma rintanarsi non serve, quando il tetro è dentro. »

Fabio sospirò ed annuì. Era troppo stanco per distinguere i discorsi sensati dall'aria fritta. Abbandonò il suo avamposto alla finestra e si lasciò andare, inerme, sulla sedia.

« Se si fa questa, a casa non ci torni » disse Denise, mostrando la sua creazione.

Fabio sorrise amaramente, prima di rispondere.

« Non voglio tornare a casa. »

Si frugò in tasca in cerca di un accendino. Denise gliene indicò uno da tavolo, poi lo guardò negli occhi, e disse:

« Si capisce. Ho sentito tante cose, Fabio\ldots Mi dispiacerebbe dover credere anche solo alla metà. Ma non lavori domani? »

« No. Non ha senso lavorare. Non se non mi pagano, almeno. »

« Allora è proprio definitivo? L'avevo sentito dire dal Bagonghi, ma credevo fosse una cazzata. »

« No. Si chiude. La crisi, il costo del lavoro\ldots le solite cose di tutti. Era ovvio toccasse anche a noi, prima o poi.»

Denise storse la bocca in una smorfia. Sembrava sinceramente preoccupata.

« Bruno non può trovarti qualcos'altro? », chiese. « Così, tanto per andare avanti\ldots »

Fabio scosse la testa e sospirò.

« Gli ho già chiesto tanti favori\ldots E anche lui è nei guai\ldots Non posso mettermici anche io. »

L'accendino cliccò, ed apparve il fuoco.

« Comprensibile\ldots », rispose Denise in uno sbuffo di fumo.

Afferrato un posacenere, la ragazza si spostò sul divano. Fabio la seguì, accasciandosi dalla parte opposta.

« Scusa se non sto composto. Spero di non scalfire l'immagine dignitosa che hai di me » scherzò, più per abitudine che per sentimento.

Denise rise, porgendogli il magico artefatto.

« Idiota\ldots Negli anni d'oro, ti ho visto vomitare nel cappello del Pacelli, cagare in un sacchetto della spazzatura, masticare cenere e chissà quante altre cose che non mi ricordo! »

« In ogni modo, tutti comportamenti assolutamente appropriati. Piuttosto, senti una cosa\ldots »

Fabio esitò. Doveva chiederglielo, anche se la risposta era ovvia, ma si sentiva un po' in imbarazzo.

« \ldots posso rimanere qui? Tanto tra un po' ribalto per terra e svengo, non ti do noia. »

Prima di rispondere, Denise gli si avvicinò. Mise una gamba sopra le sue, usandolo a mo' di poggiapiedi, e gli fece fare un tiro di mano sua.

« Fabio, te rimani qui e dormi nel letto, non per terra. Dani rimane fuori tutta la notte, c'è tutto il posto che vuoi! La Lavinia è a giro? »

« Sì. Credo. Sinceramente, non ne ho idea. Ma non è per lei, né per Dani. Tanto lui sverrà da qualche parte, come sempre » disse Fabio, senza riuscire a trattenersi.

Lei lo guardò interrogativa. Fabio proseguì:

« E' già un paio di volte che lo trovo appoggiato al muro del piscio alle cinque. Una volta quella fava del Gazzi gli ha anche pisciato addosso. »

« Ma che\ldots? »

« Tranquilla, appena mi sono accorto che lo stava facendo davvero gli ho fiondato la testa sul muro! »

Denise passò la gioia, perplessa.

« Credevo che, almeno, andasse a puttane. Mi aveva detto che aveva smesso\ldots »

Fabio si pentì, con qualche frase di ritardo, di aver toccato quel tasto.

« Mi dispiace » disse, desideroso di cambiare discorso il prima possibile.

« Di cosa, di avermi detto la verità? »

« Sì, esattamente. Odio essere l'ambasciatore di queste cose. »

« Tranquillo, non dirò a Dani che te lo sei lasciato sfuggire » sibilò lei, maliziosa.

Fabio tirò un sospiro di sollievo, senza nemmeno premurarsi di nasconderlo. Denise ridacchiò e lo abbracciò.

« Tu sì che sei un ubriacone responsabile. Sai, avrei preferito cento volte te a Daniele » gli disse, mordicchiandogli un orecchio.

« Maledetta » sibilò Fabio, liberandosi stizzito dalla presa. « Infili il dito nella piaga? »

« Tu lo hai infilato nella mia » rispose lei, senza smettere di sorridere.

« Ma non l'ho fatto apposta! »

« Certo che l'hai fatto apposta. Ricordi? »

Fabio fissò stupidamente il vuoto. Parlare con Denise in quelle condizioni era oltremodo faticoso: discorso fine a sé stesso o cosa sensata? L'intuizione lo colpì come un fulmine, ed un sorriso esasperato gli si dipinse sul volto.

« Che arguta allusione, Denise. Così scontata e banale da nascondere sicuramente un messaggio subliminale. Sappilo: non è il caso. »

Neanche questa risposta le spense il sorriso, anzi, la fece ridacchiare.

« E' sempre il caso. Siete schiavi del vostro istinto, voi maschi. Basta solo sapervi accendere e poi\ldots fate il vostro », sentenziò.

La sigaretta di droga esaurì il suo potere, e finì accartocciata senza pietà nel posacenere. Denise si alzò a fatica dal divano, gonfia come una zampogna ma sorprendentemente ben ferma sulle gambe. Con confusa lentezza, riempì due calici con del vino bianco.

« Vermentino » disse, porgendone uno a Fabio, il quale borbottò un grazie con voce funerea.

Lei rimase a fissarlo per un po', in silenzio. Poi sbottò:

« Che hai? Ti è morto il gatto? »

Effettivamente, la faccia di Fabio era più adatta ad un cimitero che ad un after party a due. Tracannò mezzo bicchiere in pochi sorsi. Poi rispose:

« Io sono morto, non il gatto. »

Denise assunse un'espressione dura, come se stesse impiegando ogni suo grammo di forza per evitare di alzare gli occhi al cielo. Piombò accanto a Fabio e lo costrinse a bere il resto del suo bicchiere, per poi riempirglielo di nuovo. Lui, intontito dall'alcool e dalla droga, non riusciva a fermare il suo flusso di pensieri. ``Altro che schiavi dell'istinto'' pensò, ``siamo talmente stupidi da sprecare la vita per una persona, la mettiamo al centro del mondo, lavoriamo sodo per costruire un futuro insieme\ldots e poi\ldots Ma gli altri no, non sono come me'' disse a sé stesso, in una disperata presa di coscienza. Il fumo continuava a fluttuare nella stanza, e lui lo osservava. Gli pareva di vedere ciò che pensava in quelle grigie spire. ``Gli altri maschi non sono così\ldots Sono io quello stupido. Dovrei andare a caccia, invece di fare il nido. Dovrei essere un predatore\ldots Dovrei farlo?``

« Ooh! Fabio! Ce la fai?! E' dieci minuti che parlo da sola! » sbottò Denise, strappando via Fabio dalla sua mente riflessa nel fumo.

« Eh? Oh\ldots Scusa\ldots Mi sento un po'\ldots stupefatto\ldots » rispose, incespicando nelle parole.

« Bene no? »

« Credo di sì\ldots Perché non dovrebbe essere bene? »

Ma non era bene, neanche un po'. La droga lo aveva reso debole, fragile, incapace di proteggersi dal giudizio di sé stesso. Quella situazione, quell'intimità\ldots Non aveva del fracasso in cui nascondersi, una rissa con cui distrarsi o un Gazzi da odiare. Era nudo al cospetto della sua anima. Bevve un ennesimo, generoso sorso dal bicchiere che teneva in mano. Non riusciva a smettere di pensare, di riflettere su tutto ciò che lo aveva reso così miserabile, di tormentarsi per la sua incapacità di impedire il degenerare degli eventi\ldots

« Se almeno avessi ancora qualcosa a cui aggrapparmi\ldots qualcuno\ldots » pensò Fabio ad alta voce.

Denise gli scoccò uno sguardo perplesso.

« Stai pensando a lei? » chiese, dopo una lunga pausa.

« Si\ldots è lei, cazzo, il tassello mancante! Ho investito tutta la mia giovinezza su di lei, per cosa? Per ritrovarmi ad affrontare questa merda da solo? Sarei riuscito a sopportare tutto, ad affrontare tutto! Ma così\ldots è come se mi mancasse la terra sotto i piedi\ldots »

Aveva detto troppo. Non doveva parlare quando era stordito, se l'era ripetuto infinite volte in tutta la sua vita. Ma a che servono gli ammonimenti quando, in preda alla confusione indotta dalle sostanze, si riesce a malapena ad esistere?

Denise gli tolse di mano il bicchiere piuttosto bruscamente.

``Eccoci, ora si è offesa!'' pensò Fabio, con una punta di panico. ``Non potevo stare zitto? Perché non riesco a stare zitto? Sono proprio un coglione\ldots''

Ma non si era offesa. Stava trafficando goffamente intorno ad un mobile a vetro, dal quale riuscì finalmente a tirare fuori una bottiglia di liquore ambrato. Neanche il tempo di tirare un sospiro di sollievo, che Fabio si ritrovò in mano un bicchiere colmo di profumato amaretto.

« Amaretto di Saronno » disse lei, dolcemente. « Non è la cosa più elegante e pregiata del mondo, ma se hai \textsc{\char0}ste turbe per i\textsc{\char0} capo ci vuole qualcosa di più forte del vino. »

Fabio ne bevve un sorso. Il buon sapore di amaretto, quel mix di dolce e mandorla amara, gli disegnò sul volto una strana espressione.

« Non ti garba? », chiese Denise.

« Lo devono ancora inventare lo spirito che non mi garba », sospirò sconsolato.

Bevve un altro, generoso sorso.

« Sono proprio un alcolizzato di merda » pensò ad alta voce, fissando per qualche istante il vuoto.

Dopo quelli che gli parvero parecchi minuti, Denise gli parlò all'orecchio:

« Sì che lo sei. Ma l'alcool e la droga non ti bastano, vero? Ne vuoi sempre di più. Non ne hai mai abbastanza per calmare le tue angosce\ldots »

Aveva una voce stranamente diversa dal solito, dolce ma pungente; lo intrigava. Lei gli sfiorò il collo con le labbra. Fabio sentì qualcosa muoversi in zone oscure. Possibile che\ldots?

« Sì » la assecondò Fabio, imitando il suo tono suadente senza neanche rendersene conto. « Non ne ho mai abbastanza\ldots Ne faccio un'altra. »

Non sapeva perché stava rifiutando così stupidamente le avances di Denise, nonostante ai fini pratici le apprezzasse. Era veramente ancora così legato a Lavinia, dopo tutta l'indifferenza e l'incertezza che aveva sopportato? O forse era fuso dalle varie droghe che aveva assunto, legali e non, e quindi semplicemente non capiva più niente? Perplesso da sé stesso, cominciò a produrre. Alzò lo sguardo dalla sua attività solo a lavoro ultimato, accorgendosi che Denise lo stava osservando smarrita. In uno slancio di acidità, Fabio suppose che con tutta probabilità nessuno prima di lui aveva mai rifiutato le attenzioni della bella Denise con così tanta indifferenza. Si sentì immediatamente stupido, ma non poteva farci niente. Ogni boccata di fumo o sorso di liquore lo rendeva sempre più triste ed incapace di controllare sia la sua mente che il suo comportamento.

Come se avesse indovinato i suoi pensieri, Denise ruppe l'imbarazzante silenzio che era sceso nella stanza.

« Capisco. Io\ldots se vuoi ti ascolto ».

Non voleva riversare se stesso su di lei, ma non poteva impedirselo. Prima che potesse pensare qualsiasi cosa, cominciò a parlare.

« Cosa vuoi che ti dica? Sai già più o meno tutto, no? Forse ne sai addirittura più di me\ldots L'ho sempre data per scontata, sai? Siamo insieme da millenni, ormai. Senza di lei, tutto mi crolla addosso. Credevo che mi amasse e che tenesse a me. Credevo che ci sarebbe sempre stata. Nonostante tutto\ldots »

Fece una pausa e inspirò. Tracannò disperatamente l'amaretto che gli rimaneva nel bicchiere, quasi strozzandosi. Denise gli si avvicinò per battergli una forte pacca sulla schiena, ma non gli tolse il braccio di dosso quando lui riprese a respirare normalmente. Si accese la sigaretta corretta e, come posseduto, continuò.

« Il lavoro mi ha sempre fatto schifo. Volevo riprendere a studiare, volevo prendere la situazione in mano, volevo un sacco di cose\ldots Ora che assisto impotente al crollo di tutto quello che abbiamo costruito in questi anni, schifo o non schifo, mi sarei persino messo a pregare pur di continuare ad avere una sicurezza economica. Al cesso i sogni e le aspettative\ldots Tanto ci sarebbe stata lei. Avrebbe sostenuto tutte le mie scelte, mi avrebbe dato quella sicurezza che mi serve per affrontare il mondo\ldots »

Fece una pausa per fare un tiro. Denise ne approfittò subito:

« Era tutto rose e fiori? Davvero? »

Le sue mani in qualche modo erano finite sul petto di Fabio e lo stavano accarezzando dolcemente. Lui le ignorò e proseguì:

« Col cazzo. Litigavamo, si\ldots Eccome se litigavamo. Ma poi bastava guardarci per capire che ci si stava scannando per cazzate, e che quello che ci legava era molto più forte di quello che ci stava dividendo. Ma ora è tutto diverso. Ora è tutto freddo, statico, indifferente\ldots Preferirei che ci prendessimo a pedate, piuttosto. »

« Si capisce » rispose Denise con dolcezza. « Dai, vieni a stenderti, sei un pezzo in là. Continuo ad ascoltarti, se vuoi\ldots »

Ma Fabio non si mosse né parlò. Il pensiero proibito, quello che tanto a lungo aveva provato a nascondere a sé stesso, prese forma. Perché sì, era come aveva detto, lei era fredda, statica, indifferente; ma al puzzle della realtà mancavano ancora diversi tasselli, uno dei quali era che Lavinia, la sua Lavinia, si comportava in quel modo solo con lui. Mentre la sua coscienza si spegneva, corrosa fino all'osso dalla combinazione di sostanze che aveva assunto, la sua immaginazione gli dipinse con crudele realismo l'incubo che lo tormentava ormai da tempo: la sua amata Lavinia e la sua nemesi Gazzi, insieme. Tese le mani avanti, cercando di afferrare i due spettri per dividerli, per riappropriarsi di ciò che gli apparteneva, per strozzare chi aveva osato tanto. Si sentì pervaso da un tremendo senso di impotenza\ldots

Udì una voce in lontananza:

« Che fatica, \textsc{\char0}sti òmini disperati\ldots »

Si accorse di essere trascinato, incapace di opporsi a qualunque evento stesse accadendo. Accettando l'oblio come unica salvezza, si arrese: tutto sparì, inghiottito da un profondo abisso oscuro.

\chapter{Flashback: la Cenere}

\begin{chapquote}{Author's name, \textit{Source of this quote}}
``This is a quote and I don't know who said this.''
\end{chapquote}


Grigio. Niente più forme eteree, ma solo plumbeo, monolitico grigio. Fabio attese paziente, sicuro che qualcosa sarebbe accaduto. Forse una rivelazione? Magari una premonizione? Qualcosa di così bizzarro da non poter essere neanche immaginato? No, niente di tutto questo. L'operazione era già conclusa da un pezzo, la droga aveva quasi esaurito il suo potere. Restò deluso quando cominciò a capire che cosa lo stesse aspettando: l'ennesimo, banale flashback.

Il vento gelido soffiava impietoso per le viuzze di Prato. Raramente il centro storico era stato così deserto di domenica pomeriggio. Niente ragazzi a bighellonare in San Francesco, niente vecchietti a lamentarsi in Piazza del Comune: in giro c'erano solo i temerari imbacuccati che tiravano dritto per la loro strada, oltre a qualche tossico in disperata ricerca di eroina. Fabio non aveva proprio idea del perché si trovasse lì invece che in un qualsiasi luogo chiuso.

« S'ha a fa' festa? » urlò al suo compagno di disavventura, sovrastando il rumore del vento.

Quello annuì, ben lungi da aprire bocca. Ma perché diavolo erano usciti?

Il sollievo li pervase non appena varcarono la porta di casa Bagonghi. Il tepore, il silenzio, il divano: che piacere! Svestito ed accomodato, Fabio offrì una sigaretta al padrone di casa. Lui gentilmente rifiutò, preferendo un suo sigaro.

« Ora sì, che dianzi no! » sentenziò Bruno, seminascosto in una nuvola di fumo profumato.

« C'arebb'a correre\ldots» confermò Fabio.

Aveva le mani congelate. Resse la sigaretta fra le labbra e mise a scaldare le sue estremità sotto le cosce.

« Prato, città dello sgomento! », sentenziò a mezza bocca.

« Se \textsc{\char0}un piove e \textsc{\char0}un sòna ammorto, tira vento! Attento, ti sta cadendo la cenere addosso », gli rispose Bruno.

Fabio si scrollò e riprese la sigaretta fra le dita. Guardò il suo amico attraverso la foschia: la lotta della sera precedente aveva lasciato molti segni.

« Bada come sei conciato » gli disse con amarezza, « non sei ancora andato a rendergliele? »

Bruno non rispose, ma era chiaro che non provava rancore per il suo aggressore. Difficilmente il buon Bruno Bagonghi si lasciava mettere i piedi in testa, ma era facile al perdono.

« Hai avuto tempo di ascoltare quel disco? » incalzò Fabio, incapace di tacere.

« Book of Souls? »

« Sì, il libraccio delle anime. Cosa significa, poi? »

Bruno sbuffò assente un'altra nuvola di fumo. Sembrava altrove con il pensiero, ma rispose:

« Boh, cose di sciamani, credo. Comunque, ti dico la verità, non mi sta piacendo granché. È fin troppo\ldots »

« Prolisso? »

« Già. I pezzi sarebbero potuti durare la metà, senza perdere nessun significato. »

« Sono d'accordo, ma nonostante questo a me piace. »

Bruno sospirò. Pareva non essere dell'umore adatto per discutere, ma l'argomento era certo di suo interesse. Non si sarebbe sottratto al confronto di idee. Fece un altro tiro dal sigaro e disse:

« Ti dirò che non mi ha deluso. È perfettamente in linea con i lavori precedenti. Sono sincero, non è che avessi chissà quale aspettativa da un lavoro dei Maiden degli ultimi decenni. »

« Dai, Brave New World è oggettivamente un ottimo disco! Non puoi\ldots »

Improvvisamente, suonarono alla porta. Senza la minima esitazione, Bruno abbandonò il sigaro nel posacenere e si alzò.

« Anche di domenica, berva di'C\ldots! » esclamò.

« Dobbiamo proprio? » mugolò Fabio mentre si imbacuccava per uscire di nuovo.

« Sì, meglio di sì, a meno che tu non voglia vedermi lavorare anche oggi. Era scontata prima o poi una loro visita, venerdì avevo i Carabinieri nel magazzino. Ma ora proprio non mi va di parlare di queste faccende. »

Uscirono dalla porta sul retro, e si ritrovarono di nuovo investiti dal vento gelido. Imbestialito per il freddo, Fabio si prodigò in una sfilza di maledizioni verso la razza asiatica, rea di costringerlo a sottrarsi al caldo per evitare la mafia cinese. Ogni sillaba che pronunciava gli faceva entrare in bocca sorsi di aria ghiacciata, alimentando la sua ira.

« E basta, che sarà mai! », urlò Bruno, stufo del continuo borbottare del suo compagno di sventura.« È inutile, non sei credibile, non pensi quello che dici! Il razzismo non è cosa per te. Lascialo a chi ne può fare buon uso, ai neofascisti e agli ignoranti »

« Tipo quel neofascista ignorante che ha trasformato la tua faccia in un campo minato? » lo stuzzicò Fabio.

« Tipo lui, sì. »

« In effetti ti ha chiamato --- com'era? --- ``ebreo di merda'', o qualcosa del genere. Ti sei sentito perseguitato per la tua razza? No, perché sai, checché ne dica lui, tu non sei ebreo! »

« Stai continuando, smettila, non starò a questo gioco. Odio le discriminazioni di ogni sorta. »

« Che palle che fai, per due anatemi di estinzione! Che vuoi che sia il mondo senza i musi gialli? »

« Un mondo senza un quarto della sua popolazione. Personalmente, ritengo i cinesi ottime persone. »

Fabio sogghignò. Avrebbe riso, se non avesse temuto di inghiottire per disgrazia un sorso d'aria ghiacciata. A furia di parlare, aveva l'ugola gelata.

« Così ottime che non vuoi farle entrare in casa la domenica, ma preferisci parlarci lunedì, in ufficio, con un fucile a pompa sotto la scrivania! » disse a denti stretti, beffardo.

« Questi sono cinesi\ldots un po' più cattivi degli altri », replicò Bruno senza scomporsi.

« Maledetti loro, la loro stirpe e il giallo della loro pelle! Cerchiamo un posto al chiuso, piuttosto\ldots uno qualsiasi\ldots questo gelo mi ha già fatto venire il mal di gola! »

« Se tu avessi tenuto sigillata quella cazzo di bocca, invece di maledire gente\ldots Oh, ma che te lo dico a fare? »

Camminarono e camminarono, sfidando il vento impervio. Si era quasi fatto buio quando un pub che faceva angolo, finalmente, aprì. Ci si fiondarono dentro ed ordinarono la roba più calda che trovarono sul menù.

« Senti una cosa », eruppe Bruno da dietro la sua coppa di caffè irlandese, « ieri sei stato con Denise? »

Fabio non rispose, ma a Bruno sembrava bastare il suo silenzio per capire.

Proseguì:

« Avrei piacere che tu non cascassi fra le braccia della prima troia che capita. Metaforicamente, intendo. Cosa intendi fare con la Lavinia? Certe persone stanno dicendo cose\ldots La situazione non mi è chiara. »

« Non è chiara neanche a me, non so proprio che cosa dirti » rispose Fabio, controvoglia.

« Tranquillo, non devi spiegarmi niente. Volevo solo sapere se hai un'idea di cosa stai facendo, oppure agisci del tutto casualmente. »

« Agisco a caso. Contento? »

« Le mie scuse, non volevo essere inopportuno. »

« Ma che inopportuno! Madonna, quanto cazzo sei pomposo! Mi dai un fastidio bestia! »

« Allora va'a' caare, mi voleo fa \textsc{\char0}azzi tua! Contento? »

Fabio sorrise controvoglia. L'argomento lo aveva fatto piombare in un pessimo umore. Non aveva voglia di rimestare ancora nel calderone dei suoi problemi, ma solo di svagarsi insieme a uno dei pochi veri amici che gli erano rimasti. Bruno lo aveva di certo capito, ma riprese comunque l'argomento:

« So che ti do noia a parlare di queste cose. Solo che\ldots beh, sta andando tutto ai maiali, no? Non solo a te, a tutti quanti! A me per primo! Ora ho anche le forze dell'ordine che gironzolano per il magazzino in cerca di droga! Ci mancava solo questa. Non mi resta altro che sperare che trovino solo i soldi del nero dietro la cimatrice, che sono la cosa meno illegale che tengo lì dentro! Io mi rifiuto di credere a cose ridicole come la sfortuna, ma di questo passo non saprò più cosa pensare! »

Fabio sospirò.

« Dove vuoi arrivare? » chiese senza il minimo entusiasmo.

« D'accordo, te la faccio breve. Io non ho più nessuno. E anche te mi pare di capire che non sei messo tanto bene. Va bene, c'è il Brogelli\ldots ma mi è parso di capire che gli trombi la donna. C'è l'Arrighi, c'è il Gazzi\ldots »

 « No, il Gazzi proprio non c'è!  »

« Invece c'è, qualunque cosa questo comporti. Ma il mio discorso è un altro, Fabio\ldots A questo punto, dopo tutto ciò che ci sta accadendo, nelle nostre vite chi rimarrà? Te lo dico io\ldots rimarrò io, e spero proprio che rimarrai te! Quindi, in estrema sintesi\ldots se hai bisogno di qualcosa, qualsiasi cosa, cercherò di fartela avere. »

Fabio rimase perplesso. Che offerta era mai quella?

« Grazie per\ldots beh, il pensiero e tutto quanto » rispose con cautela. « Mi servirebbero tante cose, molte delle quali non so neanche che cosa siano. Non credo che tu possa darmi qualcosa che non sai cos'è. E, comunque, visto che alla fine sei un po' ebreo per davvero\ldots che cosa vuoi in cambio? »

« Non ti sto offrendo qualcosa perché voglio altro in cambio! » ribatté Bruno, teatralmente offeso. « Ti sto offrendo il mio aiuto incondizionato, così che quando avrò bisogno di te, tu ci sarai! Putacaso, avrei proprio qualcosa da chiederti\ldots ma è una mera coincidenza! »

Fabio sospirò.

« Dimmi cosa devo fare e lo farò. Anche io ho soltanto te, credi forse che senza una contropartita non ti farei un favore? »

« Dimmi prima tu. »

« Te l'ho detto, non saprei\ldots L'unica cosa che so di desiderare, tu non puoi darmela. »

« Su, parla. »

« Beh, voglio una nuova vita. Voglio lasciarmi tutto alle spalle, voglio imparare dai miei errori. »

Bruno contrasse il volto, probabilmente per trattenere una risata di scherno.

« Cosa ti impedisce di farlo? », bofonchiò.

Fabio apprezzò lo sforzo che il suo amico stava facendo per non deriderlo, ma si irritò lo stesso.

« Non hai capito\ldots » gli disse, grave. « Voglio sparire. Vorrei vivere una vita completamente diversa, vorrei diventare un\ldots un selvaggio! La caverna, la clava e tutto quanto! Non ne posso più di farmi il culo per poi vedere tutto andare a puttane! Voglio smettere di voler bene\ldots Voglio uccidere per mangiare, non avere altra preoccupazione se non quella di procacciarmi il cibo! Voglio vedere le persone come risorse o come minacce, non le voglio più mettere sul mio stesso piano emotivo! Se solo potessi\ldots »

Bruno alzò garbatamente una mano, sorridendo, e Fabio tacque.

« So che non ti piace quando te lo dico, ma a volte io e te facciamo pensieri molto simili » disse, affabile.

Fabio sbuffò.

« Consideralo già fatto », riprese Bruno. «Complimenti, adesso hai una nuova vita! Potrai uccidere per mangiare, e tutte quelle diavolerie che hai detto. Certo, dovrai aspettare qualche mese prima di cominciare a viverla, spero tu non pretenda che organizzi tutto per domani! »

« Non ci sto cascando\ldots »

« Non fare così! Torniamo verso casa, ti spiego due cosette. Ormai i cinesi se ne saranno andati. »

Bruno fece per alzarsi, ma Fabio lo bloccò.

« Facciamo finta per un momento che tu non mi stia prendendo per il culo. Cosa vorresti in cambio? »

« Ah, giusto! » esclamò, ostentando una finta sorpresa.

Fabio inarcò le sopracciglia e  non disse niente. Si fidava ciecamente del Bagonghi, non avrebbe esitato un attimo a mettere la sua intera vita nelle mani del suo amico. Ma giocare a fare il sospettoso alimentava la sua curiosità: cosa poteva mai volere Bruno da lui? Aveva molto poco da offrirgli, purtroppo.

« Alla luce di quello che hai detto, ciò che voglio chiederti diventa ancora più facile per te! »

« Stiamo a sentire\ldots »

« Ecco, bravo, stai a sentire. Tra qualche tempo, non so di preciso quando, diciamo appena ti sarai annoiato di giocare all'uomo di Neanderthal, mi piacerebbe che tu tornassi qui, a Prato, anche solo per qualche giorno. Sarà una bella sorpresa per te, non sai quanto pagherei per assistere alla tua reazione! »

« Facciamo che non ho voglia di contraddirti, perché o non hai capito che cosa ti ho chiesto, o non ho capito io cosa mi stai chiedendo te. »

« Abbiamo capito entrambi, ne sono certo. »

« Ho capito che dovrò tornare a Prato tra qualche tempo. Vuole dire che hai un modo per farmi sparire dalla circolazione? Ti avverto, se mi dai in pasto a qualche tuo amico del cazzo e finisco a lavorare su una piattaforma nel mezzo al Pacifico, o in Uzbekistan a difendere i pozzi di petrolio dai terroristi, o qualunque altra cazzata del genere, quando torno ti strozzo! »

« Uhm\ldots hai mandato in frantumi una buona metà delle soluzioni che stavo per proporti. Ma non preoccuparti, ho tante frecce al mio arco! »

« Mi inquieti, non posso mai stare tranquillo quando sono nelle tue mani. Ma facciamo che, fra tutte le stronzate che puoi inventare, tu abbia una buona idea. Vuoi solo che prima o poi torni a Prato? Davvero? »

« A dire la verità, no. So che è una cosa bizzarra, ma ti prego, tienilo a mente: qualunque cosa mi sia successa quando tornerai\ldots vienimi a trovare. »

« Eh\ldots tutto qui? »

« Sì. Vienimi a trovare e fuma un sigaro con me. Occhio, un toscano dei miei, non uno qualsiasi! A qualunque costo, non importa cosa tu debba fare per riuscirci. Lo farai? »

Fabio sbatté le palpebre, perplesso. Non capiva proprio dove Bruno voleva arrivare.

« Ma che cazzo di favore è? », sbottò.

« Per favore, dimmi che lo farai », incalzò lui. « A qualunque costo! »

« Sì che lo farò, vai tranquillo! Ma senti questo\ldots Io l'ho sempre detto, te sei tutto matto\ldots »

Ma era Fabio ad essere matto. Tutto ad un tratto, Bruno lo stava strangolando con delle luci da albero di Natale. Nemmeno il tempo di stupirsi che tutto cambiò. Le cose si fecero veramente strane, il senso degli avvenimenti era proprio un mistero. In preda al terrore, Fabio gridò senza emettere alcun suono. Qualcosa nella sua testa gli diceva che era tutto finto, tutta un'allucinazione, ma la paura era così reale\ldots

Senza speranza, attese il risveglio.

\chapter{Rinascita}

\begin{chapquote}{Author's name, \textit{Source of this quote}}
``This is a quote and I don't know who said this.''
\end{chapquote}

% sezione in cui descrive il volto scorre poco, migliorare
% ripetzioni: momento, ora, rinascita

Il sole era sorto da pochi minuti e Fabio già non lo sopportava più.

« Abbozzala  di splendere, coso giallo! » inveì, la voce ancora impastata dal sonno.

Schiuse con cautela gli occhi, tentando di abituarsi alla luce che inondava quella che adesso era casa sua.

A chiamarla casa ci voleva un bel coraggio: era un vecchio, osceno appartamento dove caos, sporcizia e trascuratezza regnavano sovrani. Il pavimento era costellato da un inquietante misto di cianfrusaglie e rifiuti; poteva far invidia ad un accumulatore seriale come ribrezzo ad una persona più o meno sana. L'arredo principale era un vecchio tavolo traballante, che occupava da solo quasi metà dell'unica stanza; un bell'oggetto, finemente decorato con scatole di medicine, buste d'erba vuote e riviste ormai imparate a memoria.

Fabio richiuse gli occhi, immaginando di trovarsi su un lettino in riva al mare. Il trucco non funzionò: rinunciò all'idea di riaddormentarsi e si costrinse ad accettare di essere sveglio.

Sconsolato, si stiracchiò lentamente e fece per sedersi, ma imprecò dal dolore e si alzò di scatto: si era involontariamente grattato l'orecchio, togliendosi la crosta del taglio. Corse subito verso lo specchio per controllarsi la ferita, ma inciampò nei suoi pantaloni, abbandonati per terra ormai da qualche giorno.

« Mmmhrrraaaw! »

Ringhiando furioso, si rimise in piedi. Ma la repentina ascesa infierì sulla sua pressione ancora bassa: barcollò goffamente, prima di piombare di nuovo sul pavimento.

« Mapporcodd\ldots! »

Dopo essersi sfogato offendendo varie divinità alle quali non credeva, si calmò. ``Non è proprio giornata'' pensò, rialzandosi a fatica e raggiungendo finalmente la parete opposta a quella del suo letto; lì aveva appeso un minuscolo calendario promozionale di una farmacia. Prese un pennarello da una mensola piuttosto sudicia e tracciò la nona, microscopica croce.

Rimase ad osservare stupidamente le caselle segnate per qualche momento, poi, all'improvviso, si sentì pervaso da una potente eccitazione: la convalescenza era finita!

Erano stati nove giorni terribili, costellati da dolore, fastidio, buio, cibo in scatola, puzza, antibiotici e soprattutto dalla più intensa sensazione di noia che avesse mai provato. Certo, i postumi dell'operazione erano stati molto più leggeri di quanto si aspettava, ma restare chiuso in un sudicio appartamento per nove giornate, senza potersi lavare e riuscendo a malapena a dormire era stato più duro del previsto. Si era organizzato: aveva fatto scorta di erba e di riviste da leggere per ammazzare il tempo. Ma la droga, forse per l'interazione con gli antibiotici di profilassi o forse per la solitudine, non lo aveva divertito come avrebbe dovuto. E le riviste, beh, ad un lettore digitale incallito come Fabio erano durate sì e no mezza giornata.

Sospirò. Ormai era finita. Almeno ora, con le ferite ormai del tutto rimarginate, avrebbe potuto concedersi una doccia.

Di umore non proprio terribile per la prima volta dopo tanto tempo, corse nel bagno e si specchiò. Il suo riflesso avrebbe spaventato chiunque, ma lui aveva imparato a direzionare lo sguardo con chirurgica precisione, ignorando la visione d'insieme. Vide che le ferite alle orecchie ormai si erano cicatrizzate. Sorrise debolmente, senza smettere di osservarsi. La garza sul naso non sanguinava ormai da qualche giorno ed il mento era già a posto da un pezzo, nonostante la sensazione di rigidezza che si irradiava a tutta la mandibola. Si tastò delicatamente nel punto in cui qualcosa di molto simile ad uno scalpello aveva plasmato l'osso: vide le stelle dal dolore, ma a parte quello sembrava abbastanza solido, per quanto ne poteva capire lui. ``È davvero finita'', pensò radioso. Si tolse definitivamente la garza e si concesse finalmente un momento di narcisismo.

Un'incudine si formò nel suo stomaco. Trattenne a malapena una smorfia di disgusto, mentre un'amara delusione lo avviluppava. Non riusciva a credere che quello specchio potesse riflettere una tale bruttura senza rompersi per protesta. Nove giorni senza curare il suo aspetto lo avevano reso assolutamente inguardabile. Inoltre, il finto chirurgo aveva fatto veramente un pessimo lavoro con il naso: sembrava aver subito una bella spianata con una lima da ferro, tanto era evidente la fresca cicatrice della rimozione della gobba. Il mento sembrava davvero aver ricevuto un trattamento a suon di martello e scalpello, ma tutto sommato non era sgradevole alla vista; se solo non avesse provocato così tanto dolore ad ogni minimo contatto, avrebbe potuto essere un lavoro quasi accettabile. Le orecchie invece erano qualcosa di inquietante: senza le croste, il taglio della cartilagine non si notava, ma si percepiva comunque un qualcosa di artificiale in quella forma, quasi fosse un bizzarro esperimento di body modification.

Senza pietà, un'antica angoscia colpì Fabio. Prese a tormentarsi la barba, incapace di fermare il flusso di pensieri che lo investiva. Il suo bel volto, molto equilibrato e tutto sommato di bell'aspetto, era stato vituperato da un tizio inquietante per quattrocento euro. Non era tanto il risultato, che alla fine era all'altezza delle aspettative e più che sufficiente a garantirgli quel minimo di camouflage di cui aveva bisogno, ma il fatto di essere stato costretto a sacrificare la sua immagine per continuate a vivere. Aveva ventiquattro anni, era nel fiore della gioventù, nel momento di massimo splendore; aveva sacrificato senza esitazione ciò di cui sarebbe andato fiero da vecchio, il ricordo della bellezza giovanile! Ma in nome di cosa?

La verità di quello che aveva fatto lo colpi come una mazzata sulle gengive. Si appoggiò a peso morto alla parete, lasciandosi sfuggire un penoso lamento. Perché non era stato in grado di reagire alla situazione? Perché era dovuto scappare? Serrò la bocca e respirò profondamente, ma non riuscì a trattenersi: scoppiò a piangere. Guardò il proprio riflesso, sprofondando nei suoi occhi umidi riflessi. Si era fatto mutilare, non era possibile che quella gli fosse sembrata un'idea sensata! Era questo che i suoi cari gli avevano insegnato? Cosa avrebbero detto se l'avessero visto in queste condizioni?

Lo specchio, come un perverso quadro raffigurante il più inquietante dei deformi, rifletteva impassibile il Fabio straziato. ``Perché non ti rompi, specchio?'', pensò lui. ``Perché non protesti per la mia disperazione?''. Fiumi di lacrime gli scorrevano sulle guance, scavando solchi argentei che scintillavano alla luce del sole. Era come su un palcoscenico, esposto al giudizio dell'unico spettatore presente. Quella visione, densa di significato, innescò la bomba accesa ormai da troppo tempo. Un barlume rosso fremette nei suoi occhi\ldots

Sferrò un pugno al suo riflesso, deciso a cancellare dalla storia la sua immagine piangente. Il suo dolore non aveva più ragione di esistere. La sua educazione, la sua visione del mondo, le sue esperienze: tutto cancellato. Era per questo che se ne era andato, che aveva fatto strappare via pezzi di sé stesso. Lui non esisteva più. Lui era la sua volontà, non la sua storia. Sorrise follemente a ciò che rimaneva del suo riflesso. Fino ad allora, senza averlo scelto coscientemente, aveva vissuto osservando rispetto per il prossimo, provando solidarietà per i bisognosi, riponendo fiducia nell'amore. Esattamente come tutti gli altri, come tutti gli stupidi\ldots Ma lui non era stupido. Era arrivato il momento di ammettere l'errore, di voltare pagina, di abbandonare la sua coscienza, la sua moralità! Folle debolezza, gratuito punto debole! Da quel momento, sarebbe diventato un Dorian Gray, un'entità dedita solo al proprio piacere e tornaconto. Un perfetto egoista dawkinsiano, ma in pura chiave edonistica. Da quel momento, tutto sarebbe stato diverso.

Riprese il controllo di sé e si infilò sotto la doccia. L'acqua che gli scorreva addosso sembrò purificarlo, in una sorta di bizzarro battesimo che consacrava la sua rinascita. Si lavò con estrema cura, togliendo ogni traccia della sua convalescenza, e quando ebbe finito si ravversò la barba, cresciuta incolta fino a quel momento, riducendola ad un dignitoso pizzo. Si riflesse in ciò che era rimasto dello specchio, osservando il suo nuovo volto. D'ora in avanti avrebbe mostrato questa faccia al mondo: ne prese atto senza il minimo rimpianto. Si vestì ed uscì, pronto a compiere l'ultimo gesto necessario per la sua rinascita.

Uscire fuori dopo tutta quella prigionia fu un'esperienza fantastica: ogni aspetto del mondo esterno destava in Fabio molto più interesse di quanto non avrebbe fatto in condizioni normali. Il sole che bucava i suoi poveri occhi avvezzi al buio, la perenne brezza salmastra che gli rendeva barba e capelli stopposi, la gente da scansare sui marciapiedi: tutti i piccoli aspetti di Barcellona che aveva odiato appena arrivato, adesso gli apparivano meravigliosi. Sorrise ad un passante e quello gli restituì la cortesia. Era fin troppo felice, pensò. Prese ad aggeggiarsi la barba, ma non compulsivamente come aveva fatto appena un'ora prima: ora accarezzava i peli per il verso, cercando di farli convergere tutti in un punto con movimenti lenti e fluidi. Preso dalla sua nuova forma mentis, avvistò ciò che stava cercando senza quasi rendersene conto.

Era una casottino per le fototessere. ``Al fianco di un palazzo sul lungomare all'angolo di Selva de Mar. Deve essere questo!'' pensò Fabio, con tutta la strana calma che lo contraddistingueva. Cercò nel suo borsello una moneta ottenuta molto tempo prima, simile ai due euro ma costruita con un materiale molto più pesante. Avrebbe funzionato? L'amico del Bagonghi era sembrato affidabile, nonostante fosse un mafioso.

Entrò nella cabina e manca poco svenne per l'aria soffocante che vi era dentro. Era talmente stretta che quasi non riusciva a sedersi sullo sgabellino. Si trovò davanti ad uno schermo in stand-by. Curiosamente, le istruzioni per ottenere le fototessere non erano scritte in catalano e spagnolo, ma in inglese, arabo e cinese. Sempre più fiducioso che quella non fosse una normale macchinetta, Fabio inserì la moneta nera nella fessura per gli spiccioli.

Non successe niente. ``\ldots nna maiala!'' pensò furioso, prendendo a calci e pugni ogni cosa che riusciva a raggiungere. ``Ora t'accomodo a pedate se un tu parti!''

Forse stuzzicato dai colpi, qualcosa dietro lo schermo si mosse. Si sentì un rumore metallico e, con immenso sollievo di Fabio, sulla schermata comparve un conto alla rovescia. Un flash, un rumore di stampante anni novanta ed eccolo: un documento di identità catalano con la sua foto appiccicata e timbrata.

Si chiamava Jorge Pedrosa.

\chapter{Lo Sketch}

\begin{chapquote}{Author's name, \textit{Source of this quote}}
``This is a quote and I don't know who said this.''
\end{chapquote}

% pulled
% tex syntax
% rough editing

Forte della sua nuova identità, Fabio era finalmente libero. Avrebbe cominciato una vita completamente slegata dal suo passato. Si sentiva come se avesse preso in mano uno spartito vuoto, dopo mesi passati ad arrangiare un pezzo che proprio non voleva tornare: che gioia abbozzare finalmente idee nuove, fresche, vive! Tutte le possibilità che gli erano precluse nel pezzo cominciato, già rigido nella struttura e chiuso a stravolgenti nuove idee, finalmente tornavano in gioco. Ma da dove avrebbe dovuto cominciare? In quale tonalità avrebbe scritto la sua nuova composizione?

Decise che come prima cosa avrebbe reso vivibile il suo appartamento. Non era sicuramente la casa dei sogni, ma l'affitto era bassissimo e Fabio non intendeva vivere nel lercio. Era un obiettivo modesto, ma realizzabile: poteva partire da lì.

Guardando ad un orizzonte temporale più lontano, prima o poi sarebbe stata ora di trovarsi un impiego. Certo, con una rapina al mese sarebbe campato dignitosamente senza lavorare, ma era fin troppo rischioso; la sua unica esperienza in quel campo non era stata proprio piacevole, non smaniava dalla voglia di ripeterla. Eppure, più pensava all'occupazione da intraprendere, più notava che non aveva intenzione di lavorare onestamente. Il suo desiderio di rivalsa, spettro di una vita passata che doveva in tutti i modi dimenticare, lo avrebbe portato a fare il trafficante, il truffatore o addirittura l'assassino? La prospettiva, invece di preoccuparlo, lo incuriosiva.

La sua prima giornata da Jorge Pedrosa trascorse in un lampo. Il tempo vola quando si hanno fiducia nel futuro e tonnellate di spazzatura da buttare. Decise di andare a godersi quel che restava del giorno sul lungomare. Ormai il sole stava per calare, eppure Fabio sentiva che quel tramonto, in realtà, era la sua alba. Si lisciò la barba ed inspirò profondamente l'aria di mare, prima di inquinarla con un tiro di sigaretta. Stava per piombare in un profondo flusso di pensieri quando la sua attenzione fu catturata da una scena piuttosto bizzarra.

Una coppia distinta, forse sulla sessantina, stava discutendo animatamente con un pachistano, mentre una ragazza riprendeva la scena con lo smartphone. Contrariamente a quanto certi pregiudizi razziali potessero fargli pensare, dal punto di vista di Fabio sembrava proprio che fosse la coppia ad incalzare un'accesa aggressione verbale. Sebbene l'istinto lo supplicasse di farsi gli affari propri, la curiosità lo spinse ad avvicinarsi per scoprire cosa stava effettivamente succedendo.

Quatto quatto, si avvicinò con gattesca destrezza al parapiglia, senza farsi notare. Fu solo un inutile sfoggio di abilità: avrebbe potuto anche avvicinarsi a bordo di una portaerei e nessuno si sarebbe accorto di niente. Ebbe un deja-vu, ma lo ignorò.

Le urla del gruppo si sovrapponevano, creando un frastuono tale da non permettere a Fabio di capire che cosa stesse succedendo. Tuttavia, il suo fenomenale intuito, aiutato dal fatto che il pachistano stava incautamente sventolando una busta piena di roba verde, gli fece pensare che forse la situazione era ai limiti della tollerabilità in quanto a bizzarria.

Uno strano sorriso sbarazzino si dipinse sul volto di Fabio. Non riusciva a credere a quello che stava pensando. Era forse rincretinito? Si lisciò la barba e osservò il tramonto. Quella luce, quei colori\ldots\ era tutto bellissimo. Si volse di nuovo verso il parapiglia, che nel frattempo si era fatto ancora più intenso, e rise. Non sarebbe riuscito a trattenersi: si arrese al pensiero di fare una cosa stupida. Guardò intorno, giusto per scrupolo: non c'erano poliziotti o altre autorità a fare da deterrente alla sua smania di giocare. Buttò via la sigaretta e si avviò a passo svelto verso il gruppo, spaventato dalle conseguenze di quello che stava per fare, ma incapace di starsene nel suo.

«Woh, woh, woh, stop that weed waving, man!» gridò al pachistano appena fu a portata di voce. «That is weed, isn't it?»

Il presunto spacciatore lo degnò di un solo sguardo, prima di riprendere a litigare con la coppia. Era il perfetto stereotipo dello spacciatore a Barcellona: tratti e carnagione dell'est Europa, pantaloni lunghi, infradito, volantini di nightclub nei taschini della camicia e lattine di birra appoggiate per terra vicino a lui.

«Se no vuole tu va via! No urla, no arrabbia!» sbraitava quello, senza smettere di sventolare la busta con la droga.

«Ah ma siete italiani?» lo interruppe Fabio, sorpreso, rivolgendosi alla coppia. Davano proprio l'impressione di essere due vecchi rompipalle: palandrana per entrambi nonostante la piena estate, musi contriti dalla rabbia e qualche accenno di capelli bianchi.

«Impara a farti gli affari tuoi, giovane!» sbottò l'uomo, senza scollare gli occhi dallo spacciatore. «Che qui ci gira la feccia della società!».

«Spacciare in mezzo alla strada!» incalzò la donna, la voce traboccante di disprezzo. «Dovrebbe essere rinchiuso!»

«Scusi l'impertinenza, signora» cominciò Fabio con un tono falsamente sofisticato, «ma spacciare significa vendere la droga alla gente, e gran parte della gente è in giro sul viale. Cosa dovrebbe fare, aprire un negozio in centro? Venderla porta a porta?»

La ragazza eruppe in una risata isterica, ma la vetusta coppia non gradì.

«Giovane, non fare il simpatico!» ruggì l'uomo, infuriato come non mai. «Ora chiamo le guardie! Questa merda farà la fine che si merita: marcirà in galera!»

«Marcirà in galera!» gli fece eco la donna. «Vendere la droga ad una famiglia in vacanza, per di più con la bambina! Ma cosa ha nel cervello!»

Il pachistano, ammutolito ormai da un pezzo, scoccò un occhiata divertita prima alla ragazza, poi a Fabio, che gli sorrise. La bambina, come la moglie dell'obsoleto pisquano l'aveva definita, era proprio una bella figliola, alta, magra e bionda; forse era davvero troppo giovane per farci certi pensieri, ma certo non era una bambina. Non c'era da sorprendersi che lo spacciatore l'avesse avvicinata. Comunque, a lei la situazione non pareva affatto dispiacere, visto che stava piangendo dalle risate mentre riprendeva tutto con il suo smartphone. Fabio le lanciò un ammicco con le sopracciglia e declamò solennemente, imitando i modi dell'antipatico duo:

«Signor spacciatore, la signora ha ragione. Ma cosa le salta in mente di proporre la droga alla bambina! Una dolce, immacolata bambina! La sua candida innocenza dovrebbe intimidirla a tal punto da farle regalare la sua merce. Come osa chiederle dei soldi in cambio! Mi dia retta, se vuole vendere, lei sta sbagliando target commerciale.»

La ragazza fu scossa da risate così violente che quasi lasciò cadere lo smartphone a terra. La coppia invece sembrava aver visto un mostro: tacquero entrambi, pietrificati.

«Guardi me» proseguì Fabio, ormai deciso a sbigottire quei tizi fino in fondo. «Sono un ragazzo losco, con i lineamenti a teppista, do del lei agli spacciatori --- che sono la feccia della società! Tutti elementi che indicano inequivocabilmente che ho bisogno di droga! E che sarei disposto a pagarla profumatamente!»

La situazione improvvisamente degenerò: uomo e donna cominciarono a sbraitare all'unisono, agitandosi come dei folli e cominciando ad attirare l'attenzione dei passanti, rimasti fino ad allora indifferenti al moderato parapiglia. Il pachistano spacciatore fece un breve cenno di saluto a Fabio e, riposta finalmente la busta di droga in tasca, si dileguò veloce come il vento.

Nonostante la scena offerta dalla coppia, che tentava disperatamente di persuadere chiunque gli capitasse a tiro a chiamare le forse dell'ordine, Fabio decise che era arrivato il momento di levarsi di torno. Colto da un improvviso timore, si avvicinò alla ragazza.

«Scusa, ma non vengo bene in video» le disse in fretta, mentre le strappava il telefono di mano.

Lei squittì sorpresa, ma non reagì.

«Dai, non fare così!» le gridò Fabio, mentre si allontanava dalla scena a passi svelti. «Se ti ritrovo in giro te lo rendo, promesso!»

Ma la ragazza non sembrava proprio voler cedere il suo gingillo senza dare battaglia: si scrollò di dosso la madre, che nel frattempo la aveva agguantata per un braccio, forse per mostrare ai passanti la corruzione che aveva subito dal malvagio spacciatore, e trotterellò verso Fabio, decisa a riprendere il maltolto. Con la feroce ragazzina alle calcagna, lui si diresse verso il centro della città, deciso a mettere quanti più metri possibili fra lui e i vecchi chiassosi.

Non pensò neanche un istante a quanto aveva e stava rischiando: si stava divertendo tantissimo.

La preda

%pulled

Scusate il ritardo, ma ogni tanto vado in vacanza pure io. Questo é peso, godetevelo.

- Simone

« Dammi lo smartphone! »

« No. »

Mai in vita sua Fabio aveva faticato così tanto per trattenere i suoi istinti violenti. Quella ragazza lo irritava nel profondo.

« Figa, dammi il mio smartphone! » ripeté lei, per la sedicesima volta.

« Senti me » le disse Fabio mentre accelerava il passo, nel disperato tentativo di seminare la scocciatrice, « perché non te ne torni dai tuoi? Non vorrai farli preoccupare, vero? »

« Cazzoméne [1], sono dei vecchi rompiballe! Sai che robe se gli dico che mi hai rubato l'iPhone! Dammelo, dammelo subito! »

Fabio sogghignò e disse, beffardo:

« Sei fin troppo disinibita per essere una bambina, con buona pace di tua madre. Quanti anni hai? »

« Diciannove » rispose lei, senza dar cenno di aver colto il doppio senso.

« Porco didd...! E t'hai vent'anni! Bimba una sega! [2]»

« Eh? Voglio il mio smartphone! Cancellati il tuo video di merda e dammelo! »

« Non ci casco, come minimo hai iCloud [3] automatico. Di dove sei? » le chiese automaticamente, senza un reale interesse per la risposta.

« Milano. Figa, non si sénte? » disse, unendo le mani ed agitandole leggermente.

« Non mi intendo di dialetti sabaudi » ansimò Fabio. A furia di andare così svelto aveva quasi il fiatone. "Maledette sigarette" pensò, mentre si sforzava di tenere il passo spedito che aveva preso.

Per qualche metro regnò il silenzio: la ragazza aveva smesso di parlare. Fabio la osservò con la coda dell'occhio: guardava in giro, ostentando una moderata curiosità. Forse si stava chiedendo se sarebbe riuscita ad orientarsi da sola in Barcellona, una volta recuperata la refurtiva. Pur sprezzante, Fabio non poté fare a meno di essere colpito dall'intraprendenza della sua inseguitrice; era testarda e piena di energia, non aveva nessuna speranza di liberarsene camminando.

Avrebbe potuto renderle lo smartphone in cambio di soldi, che tra l'altro gli avrebbero proprio fatto comodo, ma come fare per avere la certezza di non apparire sul web in un video virale? Con tutti i servizi di archiviazione online automatica che esistevano, cancellare un file da un dispositivo non bastava più. Se solo avesse avuto due minuti di pace, sarebbe sicuramente riuscito a togliere ogni traccia della pericolosa ripresa. Ma aveva bisogno di stare fermo, se non voleva prendere in faccia qualche lampione. Doveva liberarsi della ragazza, o comunque riuscire a farla stare zitta e ferma. Senza neanche formulare il pensiero, capì che cosa doveva essere fatto e cominciò ad agire di conseguenza.

« Sono stato una volta sola a Milano » disse, prima che il disco rotto ripartisse, « all'esposizione internazionale del motociclo. Non mi è piaciuto più di tanto, l'ambiente è troppo... artificiale, non so se mi spiego. »

« Ma quale Milano, le fiere le fanno a Rho! » rispose lei, punta nel vivo. « Io sono di Milanomilano! I miei vecchi hanno il loft in via Montenapoleone, eh! [4] Non ci andiamo quasi mai, è sempre affittato... ma ce l'abbiamo! »

« Figa, una milanese d.o.c.! Allora senti una roba, mollami! » gli fece il verso Fabio, mentre prendeva una decisione cruciale. Risoluto, svoltò a destra, verso un posto che si ricordava di aver visto da qualche parte in Bogatell [5].

Continuò a parlare, sperando che la scocciatrice non facesse caso a dove la stava portando.

« Senti, palle, s'ha a fare uno scambio culturale? C'ho una fame che pàian sei, te tu mi fai un quintale di riso a i'salto e io ti ridò i'ttelefono. Mi c'andrebbe un monte d'altra roba, ma dé, tusse' milanese, più che riso e cotoletta un si può pretendere! [6] »

« Il riso al salto? Non se lo incula più nessuno, ormai è roba da Cracco [7] » rispose lei, divertita.

« E allora dillo che un'se'bona a una sega! Ni' culo a' sandali di C... [8]»

« Ascolta, fiorentino! Parla italiano, che non capisco cosa dici. »

« Fiorentino? No davvero. Ci sarò andato tre volte a Firenze. »

« Cazzodici [9], adesso cosa, la spari che sei di Napoli? Il dialetto si sente! »

« Sapete una sega, voi milanesi, del dialetto toscano! Il fiorentino è diverso dal pratese, che è diverso dal pistoiese, che è diverso dal senese e da quello del val d'Arno! Sulla costa poi, lasciamo fare! »

« Si, va beh, ciao. Il tuo che roba è? »

« Mmh, vediamo... Un po' Firenze, sì, un minimo Livorno e parecchio, parecchio Prato. »

« A me sembra comunque f... !!! »

Fabio si fermò improvvisamente, facendo inciampare la ragazza su di lui. Dove diavolo erano finiti? Voleva distrarre la ragazza con le chiacchiere, ma alla fine si era distratto pure lui.

Prese a lisciarsi il pizzo, pensando e guardandosi intorno alla ricerca di un punto di riferimento. Avrebbe dovuto esserci un vicolo cieco alla loro destra, invece c'era un minimarket. Maledisse mentalmente la Catalogna e la stirpe di ogni ingegnere civile che aveva lavorato a quelle strade, ma era colpa sua: aveva smarrito del tutto l'orientamento.

« Ti sei perso? » lo canzonò la ragazza.

« Sì » rispose lui, nervoso ma sincero. « Dove cazzo è la fermata della metro? »

« Laggiù, c'è la emme rossa. Mi restituisci lo smartphone, per favore? »

« No. Quindi se la fermata è lì... e se la guardo le torri mi rimangono alle spalle... io devo andare... di là! »

« Vai dove ti pare, » lo schernì la ragazza, « ma io non ti mollo! »

« Brava, seguimi » rispose lui, ancora intento a pensare alla strada da prendere.

Aveva proprio difficoltà ad orientarsi fra i vari luoghi di Barcellona; qualcosa di quella città gli rimaneva ostile, anche se non sapeva dire esattamente che cosa.

Tremendamente inopportuno, il pensiero corse con nostalgia alla sua Prato. Gli mancava proprio quella vecchia, sudicia città. Ogni aspetto, ogni contraddizione, tutte le cose che aveva detestato adesso gli comparivano in mente come piacevoli ricordi. Quasi si commosse nel ripercorrere col pensiero le stradine del centro storico; lui poteva essere lontano quanto voleva, ma quei luoghi avrebbero continuato a significare casa per lui. E non importa chi tu sia, non importa quanto lungo sia stato il viaggio, prima o poi tutti vogliono tornare a casa. Sì, sarebbe tornato a Prato, chissà quando e chissà come, ma ci sarebbe tornato. Se non altro, per il favore che doveva al Bagonghi.

« A che pensi? », lo interruppe una voce femminile.

Fabio sospirò ed emerse dai suoi pensieri, costringendosi a tornare nel mondo reale. La nostalgia sparì, lasciando spazio ad un fremito di eccitazione. Erano vicini alla meta; lì avrebbe potuto fare alla sua inseguitrice tutto quello che gli pareva. Sperando che continuasse ad andargli dietro senza rendersi conto del pericolo che stava correndo, le rispose:

« Pensavo alla mia città. E a Barcellona. Ti piace Barcellona? »

« Mah... sinceramente, ho visto di meglio » sentenziò la ragazza con fare snob.

Fu solo per via di un colpo di tosse che Fabio evitò di farle il verso.

«Sei in vacanza anche tu? » proseguì lei.

« Diciamo - cough, che cazzo! - diciamo che sono... in trasferta. Questa città sembra un po' Viareggio, solo più moderna e sei volte più grande. Sei d'accordo? »

« Viareggio è quel posto vicino al Forte [10]? Non ci sono mai stata. Come ti chiami? »

"Tsk! Come ti chiami? Sul serio?" pensò Fabio, divertito. Ci mancava solo che le dicesse come si chiamava!

« Sì, è quello. Non ti perdi niente. » le rispose, evasivo.

« Oh! Come ti chiami ho detto! »

A pensarci bene, lui un nome ce lo aveva, per brutto che fosse. Perché non cominciare ad usarlo?

« Eh... Jorge » gracchiò, per nulla convinto di quello che stava dicendo.

« Bum! Dai come ti chiami? »

« Jorge! » rispose Fabio, teatralmente offeso.

« Uno di Firenze non si chiama Jorge! »

« Io invece sì. Ho... mio padre che... è fan di Jorge Lorenzo [11]. Che ti importa? Non puoi prendere per buono che mi chiami Jorge? »

« I nomi sono importanti! » tubò lei con veemenza.

« Non per me » tagliò corto Fabio.

« Nel tuo nome c'è scritto chi sei! Io sono Vittoria, e vincerò tutte le sfide che affronterò nella vita. »

Fabio non riuscì a trattenersi.

« Questa cosa è stupida », sillabò velenoso. «Il nome è solo una parola che la gente dice per chiamarti. »

« Assolutamente no! Se non mi chiamassi Vittoria, non sarei la persona che sono! »

« Quante cazzate... Se ti chiamassi Patata, saresti la stessa ragazzina petulante che sei adesso, non un tubero. Per quanto mi riguarda, ci potremmo chiamare tutti con dei numeri, o dei codici, o anche non chiamarci affatto. »

« Che persona triste che sei. Oh, ma guarda dove vai, di qui non si passa! »

Erano arrivati. Il cuore di Fabio prese a battere forte. Lì poteva abusare di lei in qualunque modo gli venisse in mente. Non voleva andare troppo oltre, gli sarebbe bastato terrorizzarla per farla stare zitta e ferma per un po', mentre lui faceva sparire ogni traccia del video, oppure farlo fare a lei sotto minaccia.

Fingendo di essersi perso di nuovo, si aggirò distrattamente per il vicolo, mettendo la sprovveduta fra sé e il muro. Era in trappola: nessuno avrebbe potuto vederli o sentirli, a parte forse i topi nei cassonetti. Si parò davanti a lei e, per la prima volta, la osservò con attenzione. Era veramente una bella ragazza: aveva i lineamenti morbidi, ed era ben proporzionata. Fabio si sentì fremere, scosso da istinti che venivano dal basso. Assaporò il momento, lasciando che l'istinto trasportasse la sua immaginazione.

Fu lei a rompere il silenzio:

« Facciam quelli che vanno [12]? Questi cosi fanno caldo! » si lamentò, indicando le unità esterne dei climatizzatori degli uffici.

Ma era tardi, gli uffici erano chiusi e l'aria condizionata spenta. La ragazza avvertiva il pericolo, voleva andarsene da lì. Il suo volto angelico e disinvolto cominciava a tradire un po' di nervosismo. Forse, finalmente, si sentiva a disagio in quel vicolo buio con uno sconosciuto. Forse aveva addirittura paura.

« Vittoria, hai detto? Sì, era Vittoria... » le disse Fabio a voce bassa.

« Jorge, hai detto? No, ma chi ci crede... » gli fece il verso lei.

Fabio ridacchiò. Si mise le mani in tasca e parlò al suolo:

« Il tuo autocontrollo è incredibile, Vittoria. Ad averlo io, mi farebbe proprio comodo. »

Alzò lo sguardo. La ragazza si era irrigidita: non appariva tranquilla.

« Che vuoi dire? Ce ne andiamo, per favore? » disse a voce alta, ma incerta.

« Ce ne andiamo? » la schernì lui, avvicinandosi. « Sei te che mi hai voluto seguire. Quello che accadrà sarà soltanto colpa tua. »

Finalmente, la ragazza gettò la maschera. Indietreggiò verso la parete del vicolo, e cominciò a parlare molto velocemente:

« Sai cosa? Puoi tenertelo l'iPhone, non lo voglio più! Contento? Ora vado, salutami Firenze, ciao! Fammi passare! Non... »

Fabio le mise una mano sulla spalla e lei si ammutolì di colpo. Sentire quella pelle liscia sotto le dita lo scosse. Assaporò un attimo quella sensazione, poi si costrinse a controllare i suoi istinti; doveva assicurarsi la distruzione del video, il resto non era importante.

« Ascoltami bene » le disse, parlando sinceramente. « Questa situazione è molto pericolosa. Sia per te che per me. Non vorrei proprio ritrovarmi a fare cose che non mi conviene fare. Capisci, vero? »

« Per favore, voglio andarmene... » mugolò lei, ad un passo dal pianto.

Fabio provò un moto di insofferenza verso di lei. Si era cacciata da sola in quel pasticcio, cosa le sarebbe costato - a parte uno smartphone costoso - lasciarlo andare via in santa pace?

« Ubbidiscimi » rispose lui. « e non ti farò niente. »

Lei annuì, tremante.

« Tieni il tuo telefono » le disse, porgendoglielo.

Appena ci mise le mani sopra, si bloccò un attimo, come se avesse avuto l'idea di scappare all'improvviso.

Fabio fece scivolare la mano che aveva sulla sua spalla su fino al suo collo.

« Non ci siamo capiti » le disse amareggiato.

« Non ho fatto niente, sono qui ferma! » squittì lei, riportando con cautela la mano di Fabio sulla spalla.

Il panico della ragazza lo faceva sentire strano. In fondo non voleva farle del male, ma adorava terrorizzarla, soprattutto ora che poteva farlo semplicemente essendo sincero. Inebriato da quelle sensazioni, Fabio non diede importanza ai segnali di movimento che avrebbe potuto percepire.

« Voglio che tu capisca in che mani sei » le sussurrò, la voce intrisa di malvagità.

« Che vuoi dire? »

« Voglio dire che metterò tutte le carte sul tavolo. Non ha senso bluffare, ho io la mano più forte e tu non puoi tirarti indietro. Quando la vedrai, accetterai di aver perso. E sarà tutto più facile! »

« Non capisco... Ti prego, smettila! »

« Le metafore non sono il tuo forte, eh? Va bene, proviamo così: tu, da brava persona scaltra, stai pensando a come uscire da questa situazione con il minimo danno. Ma se non conosci le mie vere intenzioni - o i miei limiti, per fregarmi! - allora rischi di trascurare elementi utili per valutare la situazione e, alla fine, uscirne peggio di come non ne saresti uscita dandomi retta. Hai capito che cosa intendo? »

Lei appariva sconvolta. Una lacrima le scese da un occhio, e le labbra cominciarono a tremarle.

« Credo di sì » disse con voce malferma « ma perché mi dici questo? Tieniti il telefono e lasciami andare, per favore! »

« Non credo che tu sia in una posizione tale da poter avanzare richieste, sai? Ora io ti mostrerò una cosa. Ti renderai finalmente conto che seguirmi è stata una pessima idea. Consideralo... un favore personale che ti faccio. »

Senza scollare lo sguardo dal suo volto, Fabio si sganciò la cintura. Lo sguardo di lei guizzò, intriso di terrore. Appena si accorse della fibbia ciondolante, tirò sorprendentemente un sospiro di sollievo.

Inspirò forte col naso e mugolò, visibilmente sollevata:

« Oddio, pensavo... Non lo so, un coltello, una pistola... Bastava dirlo... non è un problema così grande... se poi mi lasci... ti posso fare... »

Fabio sghignazzò.

« In effetti, sto proprio tirando fuori una pistola » le disse, puntandole l'arma addosso. Lei si terrorizzò all'istante e indietreggiò improvvisamente, andando a sbattere la schiena contro il muro del vicolo.

« No » gemette disperata. « Non mi sparare! »

« Vedi, quest'arma è proprio uno dei motivi di cui ti parlavo prima » sibilò Fabio, traboccante di malvagità.

« Ce ne sono ovviamente altri, ma non credo siano rilevanti in confronto a questo, no? Ora, da brava, cancella il video. »

Lei, con mani tremanti, armeggiò per qualche istante sul suo cellulare. Fabio assaporò il silenzio, godendo della situazione di potere che aveva su di lei. Una volta finito di giocare con la tipa, avrebbe potuto lasciarla in pace e tornare alla sua vita. Forse avrebbe comprato un po' di erba da qualche spacciatore, giusto per fare qualcosa prima di addormentarsi. Ma l'istinto maschile non lo aveva affatto abbandonato. Doveva evitare in tutti i modi di prendersi altri rischi, e una violenza sessuale si poteva considerare come estremamente rischiosa. Ma come diavolo avrebbe fatto a non stuprarla? Se mi lasci... ti posso fare... Lei si era addirittura offerta di sollazzarlo per aver salva la vita. Davanti a delle pulsioni così profonde, rimanere razionali era molto difficile. I balbettii disperati di Vittoria lo costrinsero a tornare alla realtà.

« Non... n-non si può... è pending... non ho il 3G... s-siamo all'estero...»

« Fammi indovinare: il telefono sta cercando di mandare il file ad iCloud nonostante tu non abbia internet, e non ti ci fa fare niente nel mentre? » tentò di tradurre Fabio.

« Sì! Esatto! Proprio quello! Non è colpa mia! Non posso fare niente! Lo.. lo cancello in albergo! Lì ho la connessione wi-fi! Giuro! »

Fabio sbuffò.

« Lascia stare. Dallo a me, e che si fotta la Apple e le sue diavolerie. Vedrai che dal fondo del mare non caricherà proprio nulla su iCloud! »

Si rimise in tasca lo smartphone della ragazza. Era il momento, se voleva davvero abusare di lei. Il volto della povera vittima era deformato dall'insieme di emozioni che aveva provato. Le fece una carezza al volto, senza pensare. Quel gesto fece accadere qualcosa nella testa di Fabio: con suo immenso sollievo, la pietà prevalse sulla fame.

« Visto? Fatto! Sei libera! » le disse, mettendo via la pistola.

« Dai, non fare quella faccia » continuò, tentando di consolarla. « So che questi cosi costano uno stipendio. Ne avevo uno anche io prima di... beh, prima. Se mi dici dove posso trovarti, nei prossimi giorni ti riporto i soldi del suo valore. »

« No, non è per quello... i miei sono vecchi bacucchi pallosi, ma hanno tantissimi soldi! È solo che... che.. »

Scoppiò improvvisamente a piangere.

« Mi hai fatto veramente tanta paura! Credevo di morire! »

« Stupida ragazzina... » mormorò Fabio. « Vedila così: ti servirà da lezione. Cosa cazzo ti salta in mente di seguire un tipo come me, tutta sola? Ti è andata bene, sei viva e non ti ho nemmeno stuprato, ma parliamoci chiaro, non ne trovi a dozzine di malviventi che ti fanno questi garbi. Sei giovane, indifesa, ingenua e terribilmente attraente! »

Lei sorrise debolmente, prendendo evidentemente quello di Fabio come un complimento. Lui non ci fece caso e riprese:

« Ma se fossi stato una carogna che se ne frega, e che ammazza senza pensarci due volte? È stata una mossa stupida, totalmente stupida. Così tanto da farmi quasi cambiare idea. Sono sempre in tempo per abusare di te, sai? »

« Te l'ho già detto, » biascicò Vittoria tra i singhiozzi, « quello è il minore dei problemi! Sono maggiorenne e vaccinata, posso farti quello che vuoi! Voglio dire... tu non sei mica... UN NERO COL COLTELLO! »

« Ah, siamo razzisti, eh? » la canzonò Fabio, ma Vittoria lo ignorò, sopraffatta da una nuova paura.

Stava indicando qualcosa alle spalle di Fabio. Lui si girò e quasi perse i capelli dallo spavento.

Un individuo li stava fissando a pochi metri di distanza, e sembrava davvero armato di coltello. Da dove diavolo era spuntato? Neanche il tempo di rendersi conto della situazione che l'uomo si avvicinò a passo svelto e si scagliò su Fabio. Il suo cuore fece un doppio salto mortale. Totalmente impreparato ad un attacco, non era in grado di reggere il confronto fisico con l'aggressore.

Schivò a pelo un affondo di lama e mandò a segno un perfetto hiza geri [13] all'addome, ma nel giro di un istante subì una serie di colpi pesanti come cannonate al fianco e alla testa, e si ritrovò a terra rantolante.

Confuso e senza fiato, sentì la ragazza gridare ed il tintinnare della lama che cadeva a terra. Si rialzò in piedi, appoggiandosi al muro, cercando di non vomitare. Quella montagna di violenza era addosso alla ragazza, che si dimenava con tutte le sue forze. Fabio non pensò a nient'altro che a lei. Estrasse in tutta fretta la pistola e gridò:

« Qui! Sono qui, figlio di troia! »

Non ottenne la sua attenzione. Vittoria ormai aveva smesso di gridare, completamente immobilizzata. Non poteva sparare, rischiava di colpire anche lei!

Raccogliendo tutto il fiato ed il coraggio che poteva, in uno slancio prese il coltello dell'uomo da terra e glielo piantò nella schiena.

« Toh! Senti nella groppa, sai! [14]» gridò Fabio, in preda all'adrenalina.

Quello urlò, un ululato simile ad una bestia, un verso animalesco di sofferenza. Si alzò in fretta dalla povera ragazza, ma ricadde subito sulle ginocchia, tentando freneticamente di togliersi la lama dalla schiena. Senza la minima esitazione, Fabio approfittò del momento e prese la mira.

Una scintilla, un rumore sordo e quello cadde, immobile.

NOTE DELL'AUTORE

[1] : (gergo giovanile lombardo) Che mi importa?

[2] : (dialetto toscano) Accidenti! Hai vent'anni! Non sei più una bambina!

[3] : servizio di archiviazione online offerto da Apple.

[4] : importante strada di Milano, piena di negozi, appartamenti dai prezzi assurdi ed attività commerciali di ogni genere.

[5] : zona di Barcellona vicina ad una delle spiagge principali della città.

[6] : (dialetto toscano) Ascolta, facciamo uno scambio culturale? Ho tanta fame, se mi prepari il riso al salto io ti restituisco il telefono. Mi piacerebbe mangiare altre cose, ma sei milanese e non posso pretendere altro che riso e cotoletta!

[7] : Carlo Cracco, famoso chef e personaggio televisivo.

[8] : (dialetto toscano) Allora dillo che non sei buona a nulla! Maledizione!

[9] : (gergo giovanile lombardo) Ma che dici, non ti credo.

[10]: Forte dei Marmi, località turistica dell'alta costa toscana.

[11]: Jorge Lorenzo, famoso pilota di moto originario di Maiorca; ha vinto cinque titoli iridati nel Campionato Mondiale di Velocità.

[12]: (gergo lombardo) Andiamo via?

[13]: (giapponese) colpo inferto con il ginocchio.

[14]: (dialetto toscano: livornese) Senti che male ti faccio alla schiena!

Il prossimo, meno cruento capitolo sarà pubblicato il 31 agosto 2017, sempre che non me ne scordi anche questa volta.

- Simone




\chapter{Compagnia}

\begin{chapquote}{Author's name, \textit{Source of this quote}}
``This is a quote and I don't know who said this.''
\end{chapquote}


% pulled

Fabio si precipitò verso la ragazza.

« Ti ha ferita? »

Lei non rispose. Tremava fortissimo ed era scossa da brevi singhiozzi; sanguinava da una gamba. Fabio cercò di rassicurarla come poteva.

« Su, su... non è poi chissà cosa... » le disse con un filo di voce.

Era ancora zuppo d'adrenalina, stava tremando anche lui. Istintivamente, cercò di tamponare il taglio sulla coscia della ragazza come meglio poteva, ma per poco non svenne. Detestava il sangue. Si costrinse a restare presente e lo esaminò: non era profondo, per fortuna, ma era molto esteso e si stava sicuramente infettando.

« Quello... quello... » rantolò la poveretta, indicando a caso nel buio.

« Stai tranquilla » le disse Fabio, cercando di suonare rassicurante. « Ora è tutto finito. La gamba non è grave, è tutto tranquillo. »

In realtà, Fabio non pensava proprio che potesse esistere una situazione meno tranquilla di quella. La ragazza era ferita e sotto shock, avevano accanto uno stupratore con un coltello nella schiena ed il cervello sull'asfalto e, per di più, Fabio stesso era un clandestino armato.

« Quello... è morto...? » mugolò lei, tremando violentemente.

« Ora è tutto finito » ripeté Fabio. Non sapeva proprio che altro dirle, temeva che la verità potesse sconvolgere quella povera creatura. Con tutta probabilità, quella sera aveva provato la paura più intensa di tutta la sua vita.

« Cazzo, quello è morto o no? » chiese lei con inaspettato vigore.

« Stai calma, sei ferita! Sì, quello è morto. Mi ha quasi staccato la testa a cazzotti, ma ora è morto. »

L'energia della ragazza fece riscuotere Fabio. Lo shock passò e si ritrovò di nuovo freddo e calcolatore. Il pericolo immediato era finito, doveva decidere in fretta che cosa fare. La cosa più sensata gli sembrò quella di levarsi di torno il prima possibile, chiamando un'ambulanza per soccorrere la poveretta, poi far sparire finalmente il quel maledetto smartphone gettandolo in mare. No, non poteva farlo: i soccorsi avrebbero ritrovato la ragazza insieme ad un cadavere con un buco in testa, lei avrebbe parlato, lo avrebbe descritto e lui sarebbe stato ricercato. Per un attimo fece l'orribile pensiero di sbarazzarsi di lei, ma ne fu così disgustato da spaventarsi. L'istinto gli aveva detto di salvarla e così aveva fatto, senza pensare alle conseguenze. Ormai quella strada era imboccata.

« Ce la fai a camminare? » le disse dolcemente, passandole un braccio dietro le spalle.

« Certo! » rispose Vittoria infastidita, allontanando il braccio di Fabio. Si rimise in piedi con sorprendente rapidità, ma gemette di dolore non appena tentò di muovere un passo.

« Sono caduta di culo, » piagnucolò con voce tremante, « mi ha dato proprio una bella spinta quella merda! »

« Fosse il male di quello, ti ha aperto una gamba! » sbottò Fabio. « Forza, aggrappati, non fare la stupida. So che è uno dei tuoi migliori talenti, ma per favore smetti di servirtene almeno per un po', ok?  »

Con tre gambe, i due si incamminarono fuori dal vicolo.

« Ma dove vado così , sono mezza nuda... » fece Vittoria dopo qualche metro.

« Non ti guardo né ti tocco, giuro. Forza, dobbiamo levarci di torno! »

« Sono mezza nuda! »

« Allora toh, prendi la mia maglia e legatela in vita! Così sono io mezzo nudo... »

La marcia riprese.

« Grazie. Sei stato... un eroe. Io credevo... »

« Tutto quello che vuoi, ma adesso, nel nome del Cristo, ci vogliamo togliere di torno? »

Riuscirono a trascinarsi fino alla fermata della metropolitana, lontano dalla scena del delitto. Ripresero fiato. Fabio si stirò lo schiena ed imprecò: i colpi ricevuti lo avevano ammaccato non poco. Per fortuna la ragazza era leggera!

« Dove mi porti? Hai una casa? Non voglio tornare in albergo, sono quasi morta, i miei vecchi mi uccideranno appena vedranno come sono conciata... » si lagnò Vittoria.

« Non ci penso nemmeno a portarti dai tuoi » disse seccamente Fabio. « Ora ti porto da me e ti sistemo la gamba, poi penserò a qualcosa. »

« Sei nei guai? »

« Fai un po' te... »

« Mi dispiace... »

Le dispiaceva! Dopo quello che le era capitato, lei provava empatia per il responsabile. Fabio ne fu sbalordito. Da quando aveva intrapreso il suo viaggio verso la sua nuova esistenza, già due volte il comportamento delle persone lo aveva stupito. Perché la gente non capiva l'ovvio? Possibile che fosse lui a non vedere le situazioni nella loro interezza? Non poteva permettersi simili errori, ma non poteva nemmeno diventare un esperto di etologia umana da autodidatta in quell'istante. Si promise di tornare a quel pensiero in seguito, semmai gli fosse capitata l'occasione. Dopotutto, due casi isolati potevano tranquillamente essere una coincidenza.

Il silenzio regnò sull'improvvisata coppia, fino a che i due non arrivarono alla dimora di Fabio. Pareva passato un secolo da quando aveva rotto lo specchio a pugni, quella stessa mattina.

Provò un moto di vergogna: ora che erano in due, quel buco sembrava ancora più piccolo. Mai avrebbe pensato di avere ospiti! L'interruttore scattò e, dopo qualche secondo, la luce di una vecchia lampadina ad incandescenza scese tetra sulle orribili suppellettili dell'appartamento, esaltando al massimo il degrado di quel luogo.

Imbarazzato, Fabio fece stendere Vittoria sul letto e prese gli antibiotici che gli erano avanzati.

« Tieni, butta giù » le disse, porgendole una pasticca e dell'acqua. « Non è certo la roba più indicata per infezioni cutanee, ma è l'unica cosa che posso darti adesso. Domani cercherò di procurarmi un po' di acqua ossigenata, o roba così... Scusami, ma non so proprio che altro fare, sono medico tanto quanto il tizio che mi ha fatto gli interventi... »

« Grazie » rispose lei. Ingoiò la medicina e tacque per qualche istante. Fabio aprì la finestra, si accasciò su una sedia e si accese una sigaretta. Era esausto e dolorante. La ragazza invece era ancora vispa: sembrava avere una gran voglia di parlare.

« Per interventi intendi quelle cicatrici sulla faccia? » esordì.

« Accidenti a me! Fai finta che non abbia detto niente. »

« Pensavo avessi tipo la psoriasi o una malattia del genere. Ti sei fatto togliere dei porri, o roba simile? O hai fatto un lifting? No, non credo... A proposito, quanti anni hai? »

Fabio si nascose in una nuvola di fumo, evitando di rispondere. Non aveva la forza di pensare a cosa inventarsi.

« Perché sei nei guai? » chiese ancora Vittoria.

« Per un botto di cose. Non riesci ad immaginartelo? » sbottò lui, soffocando uno sbadiglio.

«No. Dai, ti prego, dimmelo. Te l'ho detto che mi dispiace... »

« Tanto per dire la più grave, io in pratica ti ho rapita. Ti staranno cercando per tutta Barcellona, ed era l'ultima cosa che volevo. Ma perché non mi sono fatto i cazzi miei, sul lungomare? »

« Cazzodici [1], è stata la scena più bella che abbia mai visto! » esclamò lei, illuminandosi di gioia. « Prima mia madre in para [2] e mio padre che sgrida il pakistano, poi sei arrivato tu e giuro, non sapevo più come fare, morivo dalle risa! »

Fabio sospirò e scosse la testa.

« Cara la tua risata... Ora sono troppo stanco per preoccuparmene, ma la mia situazione - e, di riflesso, la tua - è gravissima. Se non mi viene in mente un modo per assicurarmi di essere estraneo alla tua scomparsa, credo proprio che non potrò lasciarti andare. »

« Lasciarmi andare? No! Ti prego, ti scongiuro, non voglio tornare dai miei genitori! »

Fabio la guardò intensamente. In un certo senso, la capiva: a meno che sua madre non fosse già morta di crepacuore e a suo padre non fosse scoppiata una coronaria a forza di urlare per la disperazione, rientrare nei rigidi ranghi familiari non sarebbe stato indolore. Era comprensibile che la ragazza volesse rimandare il più possibile l'ora della cinghia. Certo, preferire la compagnia di uno sconosciuto armato a quella della propria famiglia la diceva lunga su quanto quella figliola avesse la testa sulle spalle. Non che il Fabio ragazzino fosse stato un perfetto esempio di responsabilità, ma sicuramente non era un viziatello ingenuo e tronfio. Lei invece, altro che! Una sicurezza ostentata con così tanta veemenza poteva voler dire insicurezza, oppure semplicemente un'errata percezione delle proprie abilità; comunque fosse, quella tipa non era assolutamente in grado di badare a sé stessa.

Fabio sorrise debolmente. Una vaga ombra di pensiero si affacciò sua alla mente... Lei era molto bella e tutto sommato simpatica, anche se milanese, mentre lui era tanto solo. L'idea di avere un po' di compagnia nella sua crociata contro il nulla gli appariva irresistibile. Era una follia, certo che lo era! Ma anche intervenire in un litigio fra uno spacciatore e due estranei lo era stata. Anche rubare il telefono ad una giovane ragazza, e poi condurla in un vicolo cieco per terrorizzarla ed abusare di lei. Anche uccidere un aggressore che non stava più mostrando interesse per lui, per salvare quella che fino a pochi istanti prima era stata una sua vittima.

« Perché mi fissi? » chiese Vittoria.

« Nessun motivo in particolare » rispose Fabio, scuotendo rapidamente la testa, come per rimescolare i pensieri che vorticavano al suo interno.

« Sì, certo... sei forse timido? »

« Ti sono sembrato timido, prima? »

« No, proprio no! Solo che mi fissi e non dici niente... Pensavo che volessi prendere la tua ricompensa. »

Fabio sorrise. Ricompensa? Quella parola, per la prima volta da mesi, lo fece sentire in qualche modo apprezzato.

« Cosa ridi? Non vuoi prendere ciò che ti spetta? »

« Ti ho rapita, Vittoria. Ho una pistola, e anche se quel negro mi ha fatto venire due o tre emorragie interne, ho comunque più forza di te. Potrei avere da te quello che mi pare. »

« Come sei antipatico! » sbottò lei. « Non hai proprio sensibilità. Hai mai letto qualche libro, visto qualche film... no, eh? La storia del guerriero che combatte, che uccide... e poi si prende il suo trofeo... »

Il sorriso di Fabio si allargò. Se la ragazza aveva così tanta voglia di avventura, tenerla con sé sarebbe stato fin troppo facile.

***

Un raggio di sole penetrò molesto nella stanza e Vittoria si svegliò. Che ore erano? Si alzò di scatto, ma la gamba ferita le diede una fitta; si ributtò sul misero letto sul quale aveva dormito.

Cosa diavolo ci faceva in un posto così brutto? Confusa e smarrita, cercò di ricostruire mentalmente la sera precedente. Ricordava tutto: il bel ragazzo divertente, lo smartphone, poi il vicolo ed infine l'aggressione. Istintivamente, sorrise. Era stata proprio una bella avventura, senza ombra di dubbio la più eccitante di tutta la sua vita. Certo, in dei momenti c'era mancato poco che se la facesse addosso... Quando quella schifosa bestia le si era avventata addosso, aveva addirittura accettato il peggio... Ma il ragazzo la aveva salvata! Proprio come in un film, dove lo stronzo di turno alla fine si converte, ed usa il suo potere per salvare la ragazza indifesa dal vero cattivo!

A proposito del ragazzo, Jorge, o qualunque fosse il suo vero nome... dove era andato? La sua compagnia le piaceva molto: era sarcastico, intelligente e a modo suo premuroso. Se non altro, le aveva offerto un posto dove nascondersi dalla furia dei suoi genitori.

Una sgradevole sensazione la attraversò. Non voleva proprio pensare a quello che sarebbe successo quando, prima o poi, la sua fuga dalla normalità fosse giunta al termine. Non avrebbe mai voluto che succedesse: odiava la sua famiglia e l'oppressione bigotta che essa esercitava sulla sua vita. Lei era edonista e libertina, ed adorava il brivido del rischio. Sapeva come divertirsi, a differenza dei suoi vecchi. A che serviva essere responsabile, se ciò portava ad una vita noiosa ed insignificante? Piuttosto, avrebbe voluto vivere ogni giorno esperienze come quella della sera scorsa. Certo, magari stando un po' più attenta, ma non sacrificando tutta l'avventura nel nome della sicurezza! Sbuffò. Doveva proprio tornare alla sua vecchia vita?

Chissà per quanto tempo Jorge le avrebbe concesso di stare da lui...


NOTE DELL'AUTORE:

[1]: (dialetto lombardo) Ma che dici!

[2]: (dialetto lombardo) confusa dalla paranoia.


Il prossimo, sorprendente capitolo verrà pubblicato il 30 settembre 2017.

- Simone




sorpresa

% pulled

 Fabio era quasi senza fiato.

« Hombre... hombre! [1] » chiamava, mentre inseguiva un catalano dall'aria arcigna. 

Da quando l'uomo si era accorto che Fabio lo stava seguendo, i due avevano camminato a passo svelto per almeno un chilometro, mantenendo più o meno la stessa distanza.

« Hombre, andiamo... Una pregunta [2], una soltanto! » ansimò. 

Ricevette solo fugaci occhiate spaventate, senza essere degnato di risposta. Erano così ovvie le sue reali intenzioni?

Lo strano inseguimento proseguì fino alla fine del lungomare, poi l'inseguito prese le scale che portavano alla zona del Porto Olimpico. Fabio soffocò una bestemmia: era un luogo troppo affollato per poter tentare una rapina. scocciato e sfinito, rinunciò al suo obiettivo e si abbandonò alla ringhiera della passeggiata. Si accese una sigaretta. Scrutò il mare, cercando nell'orizzonte risposte ai suoi dilemmi.

I quaranta euro che aveva in tasca, rubati ad un turista qualche ora prima, non sarebbero bastati a comprare quello di cui aveva bisogno. Che sudata aveva fatto per farseli consegnare! Non ne poteva più di inseguire la gente, sperando ogni volta di trovare un'occasione. Gli serviva un colpo grosso, da diverse migliaia di euro; così sarebbe stato a posto per un po'. Ma chi poteva mai portare simili somme in tasca? Non aveva certo la tempra per mettersi ad assaltare banche o negozi...

La risposta gli apparì davanti agli occhi, sotto forma di spacciatore.

« Hello man, wanna party tonight? [3] » gli fece quello, con il finto entusiasmo tipico del mestiere.

Fabio prese la palla al balzo:

« No, thanks man. Do you sell any drug? [4] »

Quello si smarrì un attimo; evidentemente non si aspettava una richiesta così diretta. Si guardò intorno, fece dei gesti ad un ceffo poco distante e si rivolse di nuovo a Fabio:

« Deutsche? Française? [5] Italiano? »

« Italiano. »

« Capito, amico. That guy can help you. [6] Prego, aspetta... he'll be on you in a moment [7]. Ciao. »

Gli ingranaggi del cervello di Fabio presero a girare velocemente. Ora che aveva trovato lo spacciatore, doveva trovare il modo di attirarlo in un posto isolato, così da poterlo rapinare senza problemi. Non fece in tempo a pensare niente che l'uomo della droga arrivò.

« Zio, su cosa andiamo? Oggi il bottino è grosso, ho della roba che non si vedeva da... »

La mandibola di Fabio quasi cadde dallo stupore.

« Daniele...!?» esclamò.

« No! Zio, che coincidenza! Ma che ci fai a Barcellona, amico? »

Fabio lo abbracciò con calore, ma il suo sangue era gelato. L'ultima cosa di cui aveva bisogno era imbattersi in qualcuno che conosceva.

''Merda...'', pensò.

« Sono... in vacanza. » rispose, sperando con tutte le sue energie che il Brogelli non volesse indagare oltre.

« Sì... anche io, zio! Cielo, ma che hai fatto al viso? »

Già, che aveva fatto? ''Merda! Merda, merda, merda!'', si disperò mentalmente. ''Che cazzo gli dico? Sù, cervello, inventa... Sono ferito... no, ho delle cicatrici! Sì, cicatrici. Perché... mi sono operato...oh, merda... eh...''

« ...dei porri! Ma me li hanno tolti, non era niente di grave, davvero, non ci pensare! »

''Bravo Fabio!'', si disse. ''Ora contrattacca!''

« Ma dimmi di te, non mi aspettavo di trovarti a Barcellona! Da quanto sei qui? Quando riparti? Eh? »

Daniele esitò; non era mai stato bravo a mentire al volo. Sembrava proprio che Fabio non fosse il solo ad avere qualcosa da nascondere.

« Vai tranquillo » gli disse, prima che l'altro riuscisse ad inventare qualcosa. « Facciamo così: non facciamoci domande e non raccontiamoci cazzate. Ok? »

« Meno male l'hai detto, zio! Stavo per entrare in crisi, paranoia pura, cosa potevo dirti? »

''Avresti potuto dire la verità, per quanto me ne frega'', pensò Fabio con stizza, ma in realtà era curioso. Perché il Brogelli si trovava lì? A meno che non avesse vinto alla lotteria, non poteva permettersi di far vacanze. Forse era scappato dall'Italia per sfuggire a certi amici, quelli ai quali si chiedono i soldi quando si è al verde. Era già successo almeno altre due o tre volte, Fabio ne era sicuro. Chissà da quanto tempo era via da Prato, e quando mai sarebbe tornato.

Fra tutte le persone che conosceva, Daniele era sicuramente il meno pericoloso per la segretezza della sua posizione; Bagonghi a parte, ovviamente. La sua passione per le droghe pesanti era ben nota a chi lo conosceva, pochi avrebbero creduto che avesse trovato Fabio a Barcellona per puro caso. No, averlo incontrato non era motivo di preoccuparsi.

« Dai zio, sono contento di averti rivisto. » disse lui, costringendo il cervello di Fabio a riconnettersi al resto della situazione. « Quanto stai? Uno di questi giorni ci becchiamo per un aperitivo, ti va? »

Ci mancava solo di andare a bere insieme! ''Niente panico, fai il vago'', si disse.

« Dovrei ripartire presto, tipo... fra qualche giorno, direi. Un momento per una bevuta lo troviamo, stai tranquillo. »

« Certo, amico! Però, lo sai, prima il dovere... Fumo, coca, emmedì [8], cosa ti serve? »

« Cosa mi serve? Eh, sapessi... mi serve... »

Non gli serviva proprio niente. Se avesse avuto un po' più soldi in tasca avrebbe preso volentieri dell'erba, anche solo per fargli un piacere e toglierselo di torno il più alla svelta possibile. Ma aveva solo quaranta euro, non poteva proprio farlo; quei soldi gli servivano per cose un pelo più importanti, tipo per comprare del cibo e un disinfettante per la gamba di Vittoria. Niente droga, non se ne parlava.

Certo, la sua idea era quella di rapinare uno spacciatore e, Brogelli o no, ne aveva uno proprio di fronte. Avrebbe potuto proseguire con il suo piano ed attirare Daniele in un luogo isolato, ma con che faccia lo avrebbe avrebbe poi minacciato per prendergli dei soldi? Con la stessa con cui gli aveva chiesto consiglio e consolazione? O con quella con la quale aveva condiviso momenti spensierati, malgrado l'oscurità che avvolgeva entrambi? Niente da fare: il Brogelli era un amico. Non era stato certamente l'amico su cui puoi sempre contare nel momento del bisogno, anzi, era stato più che altro l'amico di cui un po' di vergogni, ma tutto sommato gli voleva bene. Non poteva fargli del male. Doveva levarselo dalle scatole al più presto e trovare un'altra vittima. Come poteva fare?

Improvvisamente, gli venne un'ispirazione.

« Senti... sto cercando roba pesa. Ho un amico che è in una brutta situazione, gli sto facendo un favore. Sì, insomma, lo sto nascondendo perché non può proprio farsi vedere in giro. Non posso dirti di più, cerca di capire... »

« Tranquillissimo, zio. Ti puoi fidare. »

Fabio sospirò e si guardò attorno, simulando disagio. Storse la bocca e, sfoggiando una delle sue migliori performances recitative, fece con voce amara:

« Beh, arrivo al sodo... ha bisogno di eroina. »

« Cielo... »

« È un periodo intenso per lui, si sta nascondendo da certe persone... La dipendenza che ha non è fortissima, ma non riesce proprio a farne a meno. »

« Cielo, cielo... aiutalo Fabio, ti prego. Non lasciare che la roba lo divori, chiunque egli sia. Nessuno merita quel destino. »

Fabio trattenne un sorriso; aveva fatto centro. Rincarò la dose:

« Faccio quello che posso. La dipendenza non è troppo avanzata, ma ora non può proprio cominciare il percorso giusto per smettere. Te l'ho detto, si sta nascondendo. »

« Capisco... Senti, io non ce l'ho. Non posso proprio tenerla, non voglio neanche più vederla. Però so chi la tiene. »

« Puoi mandarmi dall'uomo giusto? »

« Sì, posso farlo. Segnati questo numero... »

Fabio estrasse lo smartphone di Vittoria. Un lampo lo colse: si era completamente dimenticato di disfarsene! Sperando con tutto il cuore di non togliere per errore la modalità offline, lo usò per segnarsi il numero di telefono e tutte le informazioni necessarie per comunicare con questo spacciatore di eroina.

La mente di Fabio lavorava frenetica. Ovviamente, avrebbe dovuto organizzarsi: non poteva certo azzardarsi a scrivere o telefonare con il numero di Vittoria; doveva procurarsi un altro apparecchio, con un altro numero, in modo da poter contattare chi gli pareva. Fatto questo, se fosse riuscito a fissare un appuntamento con lo spacciatore, avrebbe potuto finalmente rapinarlo. O magari conoscerlo, entrare nel giro e spacciare a sua volta. Ma spacciare era un lavoro vero e proprio, con dei capi e degli obiettivi, e Fabio non voleva più niente del genere nella sua vita. No, non c'era proprio questione sul da farsi: avrebbe contattato lo spacciatore, lo avrebbe incontrato e derubato della sua eroina, di tutti i contanti e degli oggetti di valore che portava con sé. Non aveva proprio bisogno di entrare in un business quasi onesto come lo spaccio di droga.

« Zio, io ti ho fatto un favore » disse il Brogelli.

Fabio riemerse dai suoi pensieri e lo guardò interrogativo.

« Avrei bisogno che me ne facessi uno anche te. »

« Va bene, dimmi pure. »

« Comprami qualcosa. Se torno a mani vuote anche oggi il señor [9] mi ammazza. »

"Eccoci, e ora che si fa?", piagnucolò Fabio nei suoi pensieri. "Come faccio a dirgli di no?''

« Guarda, lo sai che ti aiuterei volentieri, ma... ho speso tutto in questa vacanza, mi sono rimasti solo quaranta euro! »

« Dai, perfetto! Ho della roba zio, della roba... »

« Davvero Dani, mi servono quei quaranta euro, non posso spenderli in droga! »

« Zio, puoi permetterti un iPhone, non venirmi a fare storie per due spiccioli! »

« A dire il vero questo telefono non... niente, non importa. Davvero, ho solo quei contanti e niente altro. Non ho carte o altre amenità del genere con me. »

« Zio, ti scongiuro... Puoi sempre rivenderla se proprio ti vuoi fare l'ultimo drink, o scambiarla per altra roba, ma ti assicuro che dell'acido in boccette non l'avevi mai visto prima! »

« Dell'acido... in boccette? Acido lisergico? LSD [10]? » chiese Fabio, sinceramente interessato. "Non comprare niente!" gli intimò il suo cervello, ma la risoluzione di non spendere cominciava già a vacillare.

« Davvero! È da intenditori, non lo trovi in discoteca o nei parcheggi. Sapessi che fatica ho fatto per averne qualche dose... Non è che lo puoi prendere così, occhio! » disse Daniele, ficcandogli in mano un oggetto che assomigliava ad un flacone di smalto.

« C'è il contagocce, vedi... non ora, mettilo in tasca! » proseguì. « Una goccia, o due se proprio te la senti. Lo puoi mettere nell'acqua, oppure alla vecchia maniera sul cartoncino. Sono tanti, tanti trip! »

« Dani... una boccetta così saranno cento dosi su cartone. »

« Stai scherzando? Zio, sono cinquanta millilitri al venti per cento! Volendo puoi farci migliaia di dosi! »

Sul volto di Fabio era dipinto un bizzarro miscuglio di desiderio e tristezza. Non poteva, non doveva, non voleva comprare quella droga; eppure la desiderava.

« Capisco, beh, in effetti è proprio della bella roba », dichiarò incerto. « Dani, è fantastico, davvero, ma... il problema è che non posso permettermi di comprarlo. Ho solo quaranta euro. »

***

Nonostante la fermezza di Fabio nel non voler spendere i suoi soldi, qualche minuto dopo si ritrovò a passeggiare sul lungomare, senza un euro e con un quintale di droga in tasca.

Cosa diavolo ne avrebbe fatto? Di usarla su di sé non se ne parlava; non aveva mai provato allucinogeni così potenti e non poteva permettersi di rischiare. Nella sua mente c'erano troppe cose che stavano molto bene nel profondo degli abissi in cui le aveva sepolte. Certo che rinunciare ad un'occasione così ghiotta... Che aveva da perdere? La paura di un bad trip non era essa stessa un bad trip? Se avesse affrontato l'esperienza col terrore di star male, sarebbe stato male di certo. Ma anche se fosse riuscito ad affrontarla senza paura, cosa gli garantiva che la sua mente non si sarebbe soffocata fra le sue stesse spire? Non poteva controllare l'effetto della sostanza in alcun modo. Nella sua vecchia vita non avrebbe mai osato un tale azzardo, sicuro, ma adesso...

Maledì il cielo. Avrebbe fatto molto meglio a non comprarla, almeno non ci sarebbero stati dubbi sul da farsi. Nondimeno, ormai l'aveva in tasca e non poteva tornare indietro. Sbuffò. Decise che ci avrebbe pensato in seguito. La priorità in quel momento era raccattare qualche spicciolo per fare la spesa... di nuovo. Maledetto il Brogelli e la sua persuasività!


NOTE DELL'AUTORE:

[1] : (spagnolo) uomo.

[2] : (spagnolo) domanda.

[3] : (inglese) Ehi ragazzo, vuoi divertirti stanotte?

[4] : (inglese) no, grazie. Vendi qualche droga?

[5] : Tedesco? Francese?

[6] :(inglese) quel tizio può aiutarti.

[7] :(inglese) sarà da te fra un momento.

[8] : (gergo underground) hashish, cocaina, ecstasy. 

[9] : (spagnolo) signore, inteso come grossista dello spaccio di droga.

[10]: LSD, potente sostanza stupefacente dalle proprietà psichedeliche ed allucinogene, molto in voga negli anni '60.


Il prossimo, nostalgico capitolo verrà pubblicato il 31 ottobre 2017.

-Simone




\chapter{Casa (dolce?) casa}

\begin{chapquote}{Author's name, \textit{Source of this quote}}
``This is a quote and I don't know who said this.''
\end{chapquote}


% pulled

A Prato il tempo non era mai stato così bizzarro. Aveva piovuto per buona parte della mattina, ma sull'ora di pranzo il sole era tornato prepotente, asciugando tutto in un baleno. Col proseguire della giornata, si era formata una cappa di nubi alte e grigie, dalle quali ogni tanto osavano cadere dei radi goccioloni; l'aria era così rovente e umida da essere irrespirabile.

Giacomo Gazzi stava bollendo. Si sbottonò il colletto della camicia, sperando che il sudore non lo avesse macchiato. Non era solo il caldo a farlo sudare: era molto teso. Aveva sentito voci strane, voci che non gli piacevano affatto. Doveva saperne di più a tutti i costi, ma non poteva dare l'impressione che gli importasse più di tanto. Era in gioco una parte troppo grossa del suo mondo: doveva essere impeccabile per scongiurare il crollo di tutto quello che si era costruito.

Il Bagonghi aveva capito, e già questo di per sé era un enorme problema. Ma non aveva ancora aperto bocca, o almeno così pensava Giacomo. Se l'avesse fatto, le sue parole avrebbero avuto un peso enorme. Insomma, l'amico più intimo di Fabio rende pubblica la sua verità, ma nessuno gli dà credito? Impossibile. Ogni singola sfaccettatura della storia che Giacomo era riuscito a spargere in giro era pura fantasia, senza fonti né referenze. Nel pensiero comune dei pettegolezzi, non avrebbe assolutamente retto il confronto con le confidenze del migliore amico dello scomparso Fontanelli. Il Bagonghi non poteva aver parlato con nessuno, non c'erano dubbi. Però, ora che in giro c'erano delle nuove, scottanti dicerie, il rischio che lui finalmente rivelasse al mondo la reale sorte di Fabio era assai più alto di prima. Una sua parola avrebbe potuto rovinare definitivamente ogni cosa: doveva trovare un modo di metterlo a tacere.

Il cielo si esibì in un fragoroso tuono. I pensieri di Giacomo accelerarono: e se il Brogelli avesse scritto la verità, sulla chat di gruppo? Se Fabio fosse davvero a Barcellona con lui, e stesse sul serio per tornare? Soffocò un'imprecazione. Gli mancava così poco... Ancora qualche settimana e Lavinia avrebbe accettato la scomparsa dell'ormai ex compagno e sarebbe stata pronta a farsi consolare da lui. Il suo sogno non era mai stato così vicino all'essere realizzato. Non poteva permettere alle circostanze di impedirgli di ottenere la felicità. Ma come fare? Si era esposto troppo negli ultimi tempi, non poteva riparare in alcun modo ad un eventuale ritorno del Fontanelli. Non dopo aver giurato di aver avuto contatti con lui. Non dopo averlo dato per tossico, pazzo e pericoloso. Non dopo aver spifferato al mondo che aveva perso la testa per una russa con chissà quante malattie veneree. Non avrebbe avuto modo di salvare la faccia ad una smentita collettiva di tutto ciò che aveva raccontato su di lui, sarebbe dovuto come minimo emigrare dall'altra parte del mondo. La prospettiva era inaccettabile: non lo avrebbe permesso.

Cominciò finalmente a piovere. Violenti goccioloni si infrangevano fragorosi contro l'asfalto. Non c'era ombrello in grado di dare riparo da un simile nubifragio: l'unica soluzione era stare sotto ad un tetto. Lavinia chiuse la tenda, tirando un sospiro. Non ci sarebbe stato bisogno di chiamare il Bagonghi per disdire l'appuntamento: era palese che con quel temporale non sarebbero andati proprio da nessuna parte. Prese lo stesso il telefono, decisa a vederlo; lo avrebbe invitato a casa, o sarebbe andata lei da lui. Doveva incontrarlo ad ogni costo, parlarci faccia a faccia. Doveva vedere la sua reazione all'ultima notizia su Fabio. Sarebbe stato sorpreso? Avrebbe finto di esserlo? Oppure avrebbe bollato l'ultima nuova come semplice pettegolezzo, scaturito dalla fantasia di un povero tossico, come diceva Giacomo? In ogni caso, qualunque reazione avesse avuto Bruno, sarebbe stato un altro prezioso tassello da aggiungere al puzzle generale, alla cui soluzione le sarebbe stata chiara la fine del suo Fabio. Perché era ovvio che il Bagonghi sapesse tutto, tanto ovvio quanto il fatto che il Gazzi non sapesse niente di niente, checché ne dicessero loro. Erano entrambi così scarsi a mentire che inducevano quasi a chiedersi se si stessero davvero impegnando. Per lei, ragazza smaliziata, le espressioni sul volto di quelle due parodie di uomini erano come pagine di libri. Ah, se solo avesse saputo come fare per consultarli a piacimento...

E il pensiero le corse a Fabio, alla persona che era diventato nei mesi precedenti alla sua scomparsa: una mera ombra di quello che un tempo era l'uomo che amava. Magari avesse saputo intravedere anche solo un barlume di emozione sincera dal suo volto! Un messaggio sentito dalle sue parole! Una velata richiesta d'aiuto dal suo linguaggio del corpo, qualsiasi cosa! Era questo che la tormentava più della solitudine, più della mancanza del suo amato, più di ogni altra cosa: e se non fosse stata in grado di capire che il suo Fabio aveva bisogno di lei? Non avrebbe potuto perdonarsi. Ma andava sempre tutto bene: mai un discorso strano, un comportamento insolito, anche solo una smorfia. L'umore di Fabio non era mai stato così... stabile come negli ultimi tempi. Sembrava aver imparato, finalmente, a gestire tutto lo stress e le difficoltà di una vita lavorativa tutt'altro che serena. Le sue reazioni erano sempre proporzionate a quello che gli accadeva, mai una volta sopra le righe, mai esuberante, sempre coerente. Cazzate. Per quanto potesse sperarlo, per quanto potesse piacerle l'idea, il Fabio con cui si era messa insieme anni prima, il suo Fabio, non era una persona normale. Quello doveva essere il segnale che qualcosa non stava andando per il verso giusto. Imprevedibile, incontrollabile, incoerente, inopportuno: così era il ragazzo con cui si era fidanzata. Si era detta tante cose, che stava crescendo, che stava maturando... No, col senno di poi era chiaro che stava impazzendo. E lei non se ne era accorta.

« Pronto, Bagonghi. » disse la voce all'altro capo del telefono, ripescandola dal torbido dei suoi pensieri.

« Bruno, sono io. Piove. »

« Ciao Lavinia. Altro che pioggia, questo è un bombardamento! Ma lo sai com'è... »

« Eh, lo so, lo so... Com'era? Prato, 'io bove? »

« Esatto, esatto! Se 'un tira vento e 'un sona a morto... »

« ...e' piove! »

« Bravissima. »

« Mi fai buttare via! Comunque... che si fa? »

« Io sarei per fare passo, non credo che con un meteo del genere si possa... »

« Un bicchiere a casa mia? »

« Grazie per l'invito Lavinia, ma credo di preferire un sigaro sul divano e poi dritto a letto. »

« Per favore, anche solo mezz'ora! »

« Grazie lo stesso, ma sono proprio stanco. Ci sentiamo presto, devo lasciarti. »

« Aspetta... »

La voce di Lavinia fu silenziata alla pressione di un tasto. Bruno Bagonghi mise stancamente il telefono in tasca. Per quanto avrebbe ancora dovuto evitare la sua amica? Non se la sentiva di bere qualcosa da solo con lei, non era bravo a mentire ai suoi amici tanto quanto lo era a recitare le sue battute davanti alla mafia o alle forze dell'ordine. Per riuscirci, doveva entrare in uno stato mentale totalmente incompatibile con lo stare in compagnia allegramente. Non poteva proprio farlo, non avrebbe voluto a nessun costo mischiare le varie formae mentis richieste dal suo lavoro con quelle del suo svago. Era il Fontanelli quello forte nel modificare al volo il suo comportamento, non lui. Fabio non avrebbe avuto problemi a spacciarsi per cattolico di fronte al Papa in persona. Fabio era sempre vero, anche quando faceva finta. Lui, Bruno, invece, aveva bisogno di schemi, di rigide strutture alle quali uniformarsi per spacciarsi credibilmente per ciò che non era. A Fabio bastava il puro, sfacciato talento!

Quel pensiero lo stava irritando. Lo allontanò dalla mente con stizza. Guardò fuori dalla finestra, giusto per distrarsi: pioveva proprio forte. Un lampo spettacolare squarciò il cielo. Ah, la forza della natura! Incredibile, inarrestabile. Per quanti sotterfugi potesse architettare, niente avrebbe potuto alterare il naturale corso degli eventi. Non credeva nel destino, il Bagonghi, ma lo scorrere del tempo era una forza della natura, come il fulmine, come la pioggia. Come avrebbe potuto un patetico umano controllare gli eventi? Voltò le spalle al temporale e cercò un sigaro nella tasca della giacca. Un senso di impotenza lo avvolse. Permise ad una lacrima di bagnargli la guancia. Giorno dopo giorno, la sua decisione sembrava sempre più incastrarsi alla perfezione nel naturale corso degli eventi.

Un altro lampo saettò nel cielo plumbeo. Un altro ancora. Una vera e propria tempesta di fulmini! Il rumore dei tuoni era assordante, le vibrazioni incredibili. Tutto vibrava, scosso dall'enorme forza della tempesta. Anton Leka chiuse le persiane di casa sua, nel vano tentativo di isolarsi dall'armageddon che si stava consumando sopra di lui. Non che avesse chissà cosa da proteggere: forse l'unica cosa di valore che gli restava erano proprio le persiane. Chiusa fuori la poca luce naturale che riusciva a filtrare attraverso le nuvole, il buio scese impietoso sulla stanza. Anton si fece luce con l'accendino e brancolò verso il tavolo, in cerca della candela. L'elettricità era un lusso che, da disoccupato, non si poteva più permettere. Un paio di click ed una luce timida balenò, mostrando tetra la miseria rimasta su quel tavolo. Chiunque sarebbe rabbrividito al pensiero che quello, solo quello, era tutto ciò che possedeva. Non ci avrebbe potuto pagare neanche mezz'ora di affitto, figurarsi un mese! Le chiavi della macchina scintillarono alla luce della candela. Ormai l'auto non gli serviva più, era schedato come tossicodipendente, non avrebbe rivisto la patente per decenni. La sua povera Fiat Punto di terza mano avrebbe potuto fruttare cinque o seicento euro, non senza un folle ottimismo. Probabilmente non valeva neanche la pena di provarci. No, non c'era proprio modo di mantenersi un tetto sulla testa.

Ma al Leka non importava un granché. Era apatico, impassibile di fronte alla sua miseria. Aveva accettato il suo destino, non aveva nessun impulso a fare niente per risollevarsi dal suo disperato pantano. Solo un pensiero riusciva ad animarlo: vendetta, tremenda vendetta contro il responsabile di tutto questo.

NOTA DELL'AUTORE:

Mi sono rotto le palle di mettere le note a tutto. Come disse il mio professore di progettazione e disegno meccanico all'ITI, siete vostri!

Se non capite qualcosa, beh, avete internet: arrangiatevi!


Il prossimo, estatico capitolo verrà pubblicato il 30 novembre 2017.

-Simone




l'estasi

% pulled

Vittoria si passò una mano sul volto. Sudava dal caldo e si annoiava a morte.

« Facciamo qualcosa? » cantilenò, rivolgendosi al ragazzo che stava trafficando intorno al tavolo.

« Eh... cosa? » rispose distrattamente Fabio.

« Non lo so, qualsiasi cosa! »

« Ah... eh, sì, aspetta che... Oh Cristo di un Dio, stai fermo, fermo! Così... »

« Dai, figa che palle! Cioè, veramente, che palle! »

« Ho quasi finito... Solo un altro... Se 'un volasse via d'ogni bene avre' bell'e finito, Madonna... Chiudi un secondo la finestra, per favore, prima che rompa il tavolo a testate! Che stress... »

« Ma fa caldo! » piagnucolò lei.

Fabio la ignorò: era totalmente assorbito dalla sua occupazione.

« E io mi annoio! » aggiunse lei al piagnisteo. « Voglio andare fuori! A cercare un po' di vita, un po' di avventura! Sto già bene, la gamba non mi fa più male, vedi? »

Scattò in piedi e si picchettò il taglio con la mano. Fu difficile mascherare le sforfie di dolore con un sorriso, ma credette di riuscirci.

« Non puoi andartene » le ricordò Fabio, senza staccare gli occhi dal suo lavoro. « Visto che sei in piedi, mi scorri giù la pagina? Ho le mani impegnate. »

« Ma io voglio uscire! Voglio vedere il mondo, non stare chiusa in gabbia! Voglio fare esperienze, sentirmi viva! Guarda che sennò me ne torno a casa eh! Almeno lì c'è la wi-fi... e il mio iPhone non me lo tocca nessuno! »

« Il tuo iPhone? Tsk! Ormai ce l'ha il tizio del seminterrato in Poblenou. »

Ci volle qualche istante prima che Vittoria registrasse quell'informazione.

« Cosa... che cosa!? » fece incredula.

« Già. In cambio ho avuto cento euro, questa cinesata qui e una scheda sim con cinquanta euro di credito. Mi aiuti, per favore? »

La ragazza non si mosse.

« Ma... le mie robe! Le foto, le chat! »

« Tutto pulito. Un mago questo tipo, davvero. Non ho capito di preciso come ha fatto, mi ha detto che c'era della roba da sbloccare, ma ha resettato tutto. Tolta la sim, formattato e rimesso il sistema operativo. Del tutto indistinguibile da uno nuovo di scatola. »

Un grave silenzio riempì la stanza; più denso del salmastro, più soffocante dell'afa. Era insopportabile. La voce di Vittoria lo ruppe:

« Ah, sì? Va bene. No, ma va bene! » urlò. « Rovina pure tutta la mia vita digitale, cancella pure i miei ricordi! A chi importa!? Mi dici che vuoi la mia compagnia, e poi mi lasci qui, prigioniera! Mi dici che mi farai fare esperienze, che mi farai vedere il mondo, e poi cancelli i miei ricordi! Ma tanto a chi importa? A chi!? Non a te, certo che no! Tu mi hai rapita e con me ci fai quello che ti pare! »

Le si ruppe la voce. Fabio si voltò e la guardò: aveva gli occhi umidi e arrossati, ma il volto era contorto in un'espressione agguerrita.

« Ne avevamo già parlato » le disse freddamente, costringendosi a non perdere la pazienza.

Vittoria abbassò lo sguardo, forse pentita di aver urlato. Tirò su col naso e rispose:

« Sì. »

« Lo sapevi che lo avrei fatto. »

« Sì. »

« Mi dispiace per le foto e per il resto. Non c'era proprio altro modo. Sai la mia situazione, vero? »

« Sì »

« Dammi retta e vedrai più mondo di quello che riuscirai a fotografare. »

« Va bene. »

« Ora, di grazia, mi scorri questa pagina in giù? Non voglio farlo di nuovo col naso. »

Vittoria non rispose, ma obbedì. La guida per confezionare dosi di LSD che Fabio stava seguendo sullo smartphone mostrava ora il suo ultimo passaggio. Lui lo lesse e disse:

« Ah, beh, grazie... »

« Hai finito? » gli chiese Vittoria.

« Sì, in pratica sì » rispose lui. « Metto a posto ogni cosa e poi si può andare a spacciare. »

« Con la droga in lungomare, andiamo a spacciare » canticchiò lei, molto più allegra.

« Pluralis majestatis. » la gelò immediatamente Fabio. « Ho detto si può, ma intendevo posso. »

L'umore della ragazza precipitò nuovamente.

« Dai! Cioè, ma che due coglioni! Davvero! Sto morendo, morendo dalla noia! »

Fabio sospirò. Tacque per un momento, pensieroso. Si annoiava, eh? La avrebbe sistemata lui.

« Oh! » lo incalzò lei. « Hai perso la lingua? »

Lui ancora non rispose. La guardò intensamente; lei sostenne il suo sguardo e sorrise.

« Vuoi fare... roba? », tentò.

Fabio non rispose: era completamente assorbito dal suo flusso di pensieri. Sembrava che stesse ponderando una decisione difficile. Il silenzio regnò per diversi istanti, poi finalmente Fabio parlò.

« Quanta voglia di avventura hai? », chiese a Vittoria.

« Tutta quella che vuoi! MI. STO. ANNOIANDO. Te capì? »

« Non prenderla alla leggera », la ammonì Fabio. « E' da quando ce l'ho che ci penso. Da solo non mi azzarderei mai. In due, invece... si possono fare... cose. »

« La fai tanto lunga, neanche fosse la prima volta » disse lei, un po' delusa. « Abbiamo già fatto... cose. Ma così, senza atmosfera, non ha senso! Io voglio avventura! Emozioni! Come quando ti ho incontrato! Come quando mi hai salvata! »

Fabio continuò a studiarla per qualche istante, poi sorrise.

« Sia. Meno parlare, più fare! » esclamò eccitato.

Riempì l'unico bicchiere a sua disposizione con una piccola porzione dell'acido diluito che aveva usato per confezionare le dosi da spacciare. Ne bevve metà, poi lo porse a Vittoria esclamando:

« Te la do io l'atmosfera, te la do io l'emozione! Giuro che non sarà noioso. Butta giù. E speriamo bene! »

***

Vittoria fissava fuori dalla finestra, lasciando che le prime luci dell'alba le riempissero la vista. Si sentiva come se fosse appena nata: niente parole, niente pensieri, solo primitive emozioni. Il percepire le manifestazioni del mondo era fonte inesauribile di meraviglia. Si nutriva di quella sensazione come un assetato beve dell'acqua. Contemplava la luce, sorgente di bene, di vita; riusciva a sentire il rumore che faceva propagandosi, trasmettendo energia positiva a tutto ciò che toccava; riusciva a sentirne l'odore, una fragranza appagante come quella del pane appena sfornato. Non era mai stata così bene, così connessa con il mondo che la circondava.

Un pensiero la investì, potente, sbaragliante, come un lampo. Quel ragazzo... lui non sentiva quello che sentiva lei. Non godeva degli stimoli che il mondo gentilmente concedeva loro, non percepiva l'immensa grazia che la luce e la brezza rappresentavano. O, almeno, non come riusciva a farlo lei. Durante l'amplesso lo aveva inglobato, lo aveva sommerso con la sua ricettività: la loro estasi era stata unica, condivisa. Ma dopo si era chiuso, intrappolato nei meandri della sua mente. Niente di quello che lei aveva provato a fare era riuscito a liberarlo, solo la fine dell'effetto della sostanza lo aveva fatto.

Cosa significava questo? Perché a lei quella droga spalancava le porte della percezione, mentre a Jorge rinchiudeva l'animo nelle segrete della sua psiche? Forse, lei era... destinata a godere del mondo? Forse il mondo stesso la aveva creata perché lei si beasse a pieno di ogni sua caratteristica. Chiuse gli occhi e fece un respiro profondo. Sì, era senza dubbio così: perfino l'aria nei suoi polmoni glielo diceva. Lei era del mondo, ed il mondo era il suo harem. Di fronte a questa verità, ogni altra questione terrena perdeva senso.

***

Fabio non aveva mai provato niente del genere. Un'abbondante decina d'anni di esperienza su sesso e sostanze stupefacenti era andata in fumo in una giornata scarsa. Lui era rimasto lui, eppure era anche la ragazza. Era dentro di lei, ed era lei. Era tutto intero, tutto insieme, ma era anche ogni parte, ogni nervo, ogni anfratto dei loro corpi. Indescrivibile, impossibile, incredibile. Mai nella vita aveva sentito certe sensazioni, forse lontanamente paragonabili a quelle che si provano a fare sesso dopo aver fumato marijuana, ma dieci, cento, mille volte più potenti, più coinvolgenti.

Ma tanto erano stati lo stupore ed il godere, che tanto era ferma la sua sicurezza di non voler mai più ripetere un'esperienza del genere. Dopo il sesso, la grande epifania estatica, era arrivato impietoso l'abisso. Tutti i suoi errori, tutti gli spettri di un passato che non voleva starsene al suo posto, tutti i ricordi di tutte le brutte sensazioni che aveva mai provato in vita sua, si erano riversati su di lui senza il minimo riguardo per la sua povera testa. Aveva temuto di impazzire, aveva sperato di morire... e poi, quasi all'improvviso, era finito tutto. Nessun sollievo, solo terrore che ricominciasse.

Era stato insostenibile. Non avrebbe più preso quell'acido, non importa quanta goduria sarebbe stato in grado di trarne. Non avrebbe rischiato per nessun motivo al mondo di avere di nuovo un bad trip, c'erano troppe cose nella sua mente che dovevano restare al loro posto. Era stato bello, magnifico, ma anche tremendo. Per lui era abbastanza: aveva chiuso.

La ragazza invece aveva scoperto la manna. Per lei era stata estasi pura, senza nessuna ripercussione negativa. Appena una settimana dopo quella prima esperienza, le dosi confezionate erano ancora quasi tutte da spacciare, ma la bottiglia con la droga diluita era vuota. Vittoria la aveva consumata tutta da sola, assumendo la soluzione ad intervalli sempre più ravvicinati. Si beava di quel liquido come fosse l'unica sua ragione di essere, poi cercava sesso, o musica; a volte sedeva semplicemente vicino all'unica finestra del misero appartamento dove era confinata ed osservava fuori. Qualunque cosa stesse succedendo nella sua testa in quei momenti, sembrava che avesse trovato il suo nirvana.

Fabio l'aveva lasciata fare, se non altro per tenerla occupata durante le sue assenze. Girovagava spesso per Barcellona alla disperata ricerca di fondi per partire. Ormai lo spaccio non era più un tabù, ma la sua principale fonte di reddito; l'eroina che era riuscito a rubare allo spacciatore indicatogli da Daniele gli era fruttata un bel po' di soldi, ed era riuscito persino a vendere qualcuno dei cartoncini bagnati di acido a degli occasionali neo-hippie in Barceloneta, così si era deciso a chiedere a qualche collega dei contatti per avere dell'erba. Ora ne aveva a volontà e nonostante un tempo fosse la sua droga preferita, ne fumava pochissima, la vendeva quasi tutta. Con quel ritmo, entro poche settimane poteva metter su una piccola fortuna.

Ma non anelava al lusso, la vita semplice gli andava più che bene; mica era il Gazzi! Il vero scopo di accumulare soldi era quello di avere un tesoretto per scappare, per cominciare un'altra volta da zero. Si era già compromesso troppo per poter rimanere a Barcellona: aveva spacciato, rapinato, rapito e ucciso; ogni persona che incontrava, ogni parola che diceva, ogni istante che finiva nel campo visivo di un agente delle forze dell'ordine, ognuna di queste cose rischiava di essere quella che lo avrebbe messo in grossi guai. Aveva addirittura incontrato Daniele! No, non c'era proprio verso: ogni giorno che passava, aumentava la probabilità di essere raggiunto dal suo passato o, peggio, di lasciarci la libertà o l'osso del collo.

E poi c'era Vittoria. Lei a Barcellona non poteva proprio restare, né lo voleva. Non c'erano dubbi: sarebbe andata con Fabio; gli avrebbe proprio fatto comodo un po' di compagnia là dove doveva andare.

Il prossimo, fatale capitolo verrà pubblicato il 1 gennaio 2018. Godetevi l'ultimo mese dell'anno, bischeri!

- Simone




\chapter{Caronte}

\begin{chapquote}{Author's name, \textit{Source of this quote}}
``This is a quote and I don't know who said this.''
\end{chapquote}

% pulled

Il tempo volava per Fabio. Non sapeva con precisione da quanto si trovasse a Barcellona; aveva gettato il suo microscopico calendario secoli fa, ma poteva essere sicuro di essere lì ormai da qualche mese. Lo capiva dal clima, che non era più estivo: faceva quasi freddo, tanto che la brezza marina era diventata tagliente come un rasoio, e le foglie degli alberi inglobati dalla città erano già quasi tutte sui marciapiedi.

L'ultima volta che aveva messo gli occhi su un giornale, la data segnava il primo di settembre. Lo aveva comprato, quel giornale; era un corriere locale, di una decina di pagine scarse. Un euro e cinquanta per quei fogliaccio, che furto era stato! Ma aveva dovuto prenderlo, perché sulla prima pagina spiccava un titolo irresistibile. Faceva riferimento alla vicenda di una turista italiana scomparsa da qualche settimana, annunciando finalmente il ritrovamento del suo cadavere. Fabio non aveva la minima idea di chi potesse essere quella poveretta trovata morta, ma era certo che il Mossos non fosse stato a farsi troppe domande: una ragazza era scomparsa, un cadavere era comparso. Per quanto quella soluzione semplice fosse invitante, era sicuramente falsa. Un grossolano errore? Un cadavere provvidenziale, per farla finita con un caso scomodo? E chi poteva dirlo... L'unico fatto sicuro dell'intera vicenda era che, dal punto di vista di Fabio, la vera turista in questione era tutt'altro che dispersa.

Comunque, ne era passato di tempo da quella vicenda. Doveva essere circa ottobre, se non addirittura novembre. Non aveva importanza; quale che fosse il mese corrente, Fabio non aveva intenzione di passare l'inverno in Catalogna. Aveva lavorato duramente per preparare la sua migrazione, rastrellando Barcellona in lungo e in largo, raschiando fino all'ultimo centesimo che poteva da ogni attività. Si era dato da fare con spaccio, furti e rapine. Aveva persino fatto da guida turistica ad una coppia di giapponesi per qualche decina di euro. I suoi sforzi erano stati fruttuosi: aveva messo da parte diverse migliaia di sudati quattrini. Gli sarebbero serviti, ovunque fosse riuscito ad andare. Già, appariva buffo perfino a lui: aveva già organizzato quasi tutto, ma non era riuscito a decidere la meta.

Che poi, dicendo la verità, organizzare è un verbo completamente inadeguato a quella che era situazione; la destinazione ignota era solo la ciliegina su una torta fatta di incognite e di speranze. Il piano di Fabio per andarsene dalla Catalogna era fragile, incompleto e totalmente inadeguato per fronteggiare i vari intoppi che si sarebbero potuti presentare. D'altronde, quando la tua agenzia di viaggi in realtà è uno spacciatore libanese che conosce uno scaricatore del porto industriale, non è che puoi avere chissà quali pretese. Ma a Fabio non importava granché: si sarebbe accontentato di sbarcare almeno nel continente giusto, casomai ne avesse scelto uno.

A dire il vero gli importava, eccome! L'ansia di andarsene manovrava Fabio come un burattino, facendolo scalpitare dalla fretta di muoversi; sembrava non potesse restare a Barcellona neanche un'ora in più. Probabilmente, nel profondo della sua mente, giaceva pericolosa un'antica inquietudine: ancora una volta, si trovava a dover fuggire vigliaccamente dal poco che aveva costruito, preso dal bisogno di ricominciare tutto da capo. Stava forse diventando un vizio? In altri tempi si sarebbe psicoanalizzato, cercando di estrarre dalla sua mente il motivo del suo continuo desiderio di fuga, ma ormai non era più il caso. Doveva correre! Veloce a pensare, ad organizzare, ad agire! Una paura alimentava la sua follia: se si fosse fermato a riflettere, avrebbe sicuramente capito qualcosa di terribile.

Ma la frenesia da sola non era in grado di seppellire la verità. Per quanto si sforzasse, Fabio temeva che un giorno avrebbe finalmente preso coscienza di ciò che il suo subconscio aveva già realizzato da tempo. Mai e poi mai avrebbe ammesso apertamente a sé stesso che, in fondo, non sarebbe stato poi così male restarsene a Prato e tentare di rimettere insieme i cocci della sua vita, invece di buttare via tutto. Alla fine qualcuno che gli voleva bene lo aveva, anche se non tutto stava andando come aveva sempre sperato. E cosa aveva trovato invece, nella sua nuova avventura, a parte solitudine e guai? Fabio era terrorizzato da questa verità: aveva sacrificato tutto quanto, fuggendo, per trovare soltanto un'altra situazione da cui fuggire. E allora via, di corsa, a razionalizzare ciò che la sua inquietudine gli metteva in testa: si era già compromesso troppo, aveva fatto cose che non doveva fare ed incontrato persone che mai avrebbe dovuto incontrare. Era chiaro, ovvio che dovesse scappare da Barcellona. Per l'amor del cielo, aveva una ragazza rapita in casa sua! Cambiare emisfero era proprio il minimo che doveva fare prima di mettersi a costruire, finalmente, la sua tanto agognata nuova vita.

Anche Vittoria non vedeva l'ora di andarsene da Barcellona. Beh, in realtà lei non vedeva l'ora di fare qualunque cosa, se aveva con sé la sua boccetta di droga. Era diventata completamente un'altra persona rispetto alla ragazzina viziata e petulante che Fabio aveva derubato, terrorizzato e poi salvato qualche mese prima. Non erano solo i necessari cambiamenti estetici a renderla diversa - che diavolo voleva che fosse tagliarsi i capelli, era proprio il minimo, era ricercata! - la droga le aveva storpiato a fondo il carattere. Sotto l'effetto della sostanza appariva un po' stordita, ma era perfettamente in grado di fare le cose con il solito dinamismo; era sorprendente e Fabio per questo aspetto la invidiava molto, visto che a lui bastava un po' di erba per essere fuori gioco un paio d'ore. Da sobria, invece, si comportava da ragazza matura e faceva discorsi profondi, anche se conservava le insensate fobie tipiche delle donne - per l'amor del cielo, sono solo capelli!

Il tempo trascorreva tiranno, capelli o non capelli. Fu così che, una gelida mattina di quello che pareva essere dicembre, la strana coppia chiuse per l'ultima volta la porta dello squallido appartamento che era stato la loro tana per diversi mesi. Sarà stata l'aria fredda, sarà stato il cielo grigio, ma una strana malinconia avvolse Fabio mentre, imbacuccato, si incamminava verso il porto con la sua compagna al seguito. Non si era mai affiatato con Barcellona, ma la prospettiva di lasciare per sempre il suo continente natio lo intristiva. Alla fine quella città non era poi così' lontana da casa sua, anche se poteva considerarsi a distanza di sicurezza dagli spettri del suo passato. Cambiare continente però, era tutta un'altra storia. Scappare, sempre scappare... La fuga si era confermata una costante della sua vita. Prima dal suo paese natale, uno sperduto villaggio di campagna con sì e no cento abitanti, per raggiungere un paese più grande, dove i suoi genitori avevano il lavoro. Poi la città, Prato, per poter lavorare lui. Infine Barcellona, per fuggire dai suoi demoni. Ora neanche quella andava più bene... E se neanche il Brasile, o l'Argentina, o qualunque altro posto sulla Terra fosse andato bene? Dove si sarebbe rifugiato? Sulla luna?

Si sentì salire un groppo in gola. Cercò di non pensare al fatto che ogni marciapiede, ogni albero, ogni vicolo che attraversava lo avrebbe visto per l'ultima volta.

"Basta!", si impose.

Asciugò la lacrimuccia che gli scivolava sulla guancia e rivolse un sorriso a Vittoria. A quella ragazza non sarebbe importato un bel niente di quello che Fabio avrebbe deciso di fare. Forse, se proprio avesse voluto salvare qualcosa dell'esperienza barcellonese dalla definizione di fiasco totale, sarebbe stato proprio l'incontro con lei. Aveva trovato una persona a cui piaceva stare con lui, tanto gli bastava per darsi la carica. Ovunque fosse dovuto andare, lei lo avrebbe seguito. Per voglia di avventura, forse per affetto, non importava il motivo: non era da solo. Già per questo, si sentiva di dover volerle bene. Non c'erano sentimenti amorosi fra di loro, per lo meno non da parte di Fabio; si sentiva talmente saturo che forse non sarebbe stato mai più in grado di amare. Ma l'amore non gli interessava più, per lo meno consciamente. E neanche a lei sembrava interessare. A lui piaceva giocare con lei, e a lei piaceva stare al gioco. Se lo avessero voluto, sarebbero potuti essere compagni di giochi per il resto della vita. Sarebbe stato bene ad entrambi, Fabio ne era convinto.

Tra un pensiero e l'altro, arrivarono nei pressi del porto in un baleno. La bolla di malinconia che racchiudeva Fabio scoppiò, facendolo tornare nell'altrettanto tremendo mondo reale. Non sarebbe stato facile imbarcarsi senza farsi scoprire. Appoggiò su una panchina il fagotto che conteneva tutti i suoi averi e si fermò un attimo per riflettere.

« Cara mia, ora comincia il bello », disse a Vittoria. « Devo trovare Ahmed e sentire com'è la situazione. Non ti drogare nel frattempo, dobbiamo essere scaltri e svegli! »

« Tranquillo » disse Vittoria con fin troppa condiscendenza.

« Che te lo dico a fare, tanto sei già fatta... » sibilò Fabio, avendo cura di non farsi sentire.

Rivolse la sua attenzione all'ambiente circostante, cercando di capire se si trovava nel posto giusto. Non aveva mai avuto a che fare con un porto industriale prima di allora, non sapeva proprio da che parte rifarsi. Sembrava tutto squallido ed abbandonato; ogni cosa in quel luogo era di colore spento, opaco, sbiadito dalla continua esposizione al salmastro. Una piccola baracca a fianco di una sbarra mobile rappresentava l'ingresso in una bolgia di darsene e container. Fabio si incupì più che mai: la bruttezza di quel luogo lo aveva contagiato.

Stette per qualche istante ad osservare le persone che passavano di lì, poi sentì una mano sulla spalla e si voltò; il suo contatto Ahmed gli sorrideva. Era un giovane marocchino di bell'aspetto, addobbato di tutto punto con una tuta da lavoro.

« Tu ce l'ha, Jorge? » gli chiese.

Fabio fece un cenno con la testa verso Vittoria. Lei si frugò nelle mutande e tirò fuori un rotolo di banconote. Ahmed le prese e le annusò avido.

« Nel nome del Cristo, che schifo! » sbottò Fabio.

« Ma che ne sa... tu culattoni, io già detto! » gli rispose quello. Parlava un italiano passabile, ma aveva un buffo accento nordafricano.

« Tutto come concordato, allora? » domandò Fabio ansioso.

« No, casino. Oggi vero casino... Ma nave per São Paulo salpa oggi, se voi vuole lasciare Europa, non altra scelta. »

« Casino? Che casino? »

« Ieri carico, ma di notte nessuno. Carico è lì, su nave che tu vuoi andare... Qualcuno lo prende oggi, prima che salpa! »

« E a noi che ce ne importa, scusa? »

Il giovane scaricatore sbuffò ed alzò gli occhi al cielo.

« Bela ragazza, di qualcosa a tuo uomo! Lui stupido quanto te sei bella! »

Vittoria sorrise assente e rispose sognante:

« Grazie, sei gentile. Lui va in para, ma non è stupido. Il suo pensiero è... aggressivo. Non riflette mentre pensa, va dritto al punto e a volte... pensa troppo veloce, senza... pensare. »

Ahmed strabuzzò gli occhi. Era sicuramente un ragazzo sveglio, ma dava tutta l'impressione di trovarsi più a suo agio con i pensieri concreti; simili filosofie astratte non facevano certo per lui.

Infatti, chiese titubante:

« Cosa... cosa è quel che tu ha detto? »

Vittoria cercò di spiegare:

« Come quando parli senza pensare...ma un livello... sopra... capisci? »

Il bel marocchino guardò Fabio preoccupato.

« Lascia perdere, è la droga » disse lui, sbrigativo. La situazione lo stava facendo sentire un po' a disagio.

Un silenzio imbarazzante scese sul trio. Vittoria guardò Fabio intensamente per qualche istante, poi disse:

« Jorge ha capito, vero? Lui capisce quello che dico, perché è intelligente. Aggressivo, ma intelligente. »

Fabio sospirò. Non potevano perdere tempo a parlare de massimi sistemi, doveva tagliare corto.

« Sì, ho capito. » ammise. « Purtroppo per me, ho capito benissimo. Per favore, possiamo muoverci? »

Ahmed scosse la testa.

« Tu capisce cazzate di ragazza e non quello che io dico. C'è carico di droga su tua nave, mafia potente, prima che tolgono tu non prova a salire! Tu sei clandestino per guardie, ma se va su poi sei clandestino anche per mafia! »

« Questo si che è buffo» civettò Vittoria. « Somiglia a quella cosa che mi hai fatto leggere l'altro giorno sui computer! »

« Che?! » fecero in coro Fabio ed Ahmed.

« Sì... la clandestinità è... ricorsiva - era così la parola? - figa, non lo vedi? Non ci credo... Come quella sigla... Noi siamo clandestini in un contesto clandestino rispetto ad un altro contesto... »

Fabio era senza parole. Era un collegamento arguto, era vero, i livelli di clandestinità nei quali si stavano cacciando si potevano immaginare in quel modo, ma... cosa ci incastrava con il contesto? Da quando Vittoria era entrata in simbiosi con l'acido simili uscite erano all'ordine del giorno, eppure Fabio non si era ancora abituato alla strana sensazione che provava quando riusciva a cogliere il senso della riflessione senza avere idea del perché essa avesse un significato specifico.

La concretezza dello scaricatore nordafricano ruppe audacemente il silenzio che si era venuto a creare:

« Quando voi è Argentina, o Brasil, o dove va dopo che scende, voi va da psi... da pi... da dottore della testa! Voi pensa cose male! Voi non fa... »

Non seppero mai che cosa non avrebbero fatto. Ahmed si bloccò di colpo, lo sguardo fisso in un punto alle spalle di Fabio. Lui fece per voltarsi a guardare che cosa aveva pietrificato il suo contatto, ma quello gli strinse forte una spalla.

« Guai » disse Ahmed, preoccupato. « Non muove, non parla. Loro rivali di scarico, altra mafia, ma io conosce... »

Fabio annuì, ma non seppe resistere all'impulso di voltarsi; vide una dozzina di figure tutt'altro che amichevoli avvicinarsi verso di loro. Si sentì prendere la mano da Vittoria. Avrebbe preferito restare con le mani libere, ma non la scacciò: anche lui aveva paura.

« Amics? » urlò Ahmed, alzando le mani.

« Això depèn », rispose uno di loro, sguainando una pistola. La stretta di Vittoria sulla mano di Fabio si fece più forte.

« Buona... » le sussurrò lui.

« Avui dia és una porqueria, els altres han d'acabar » proseguì Ahmed.

Fabio non aveva idea di cosa il suo contatto stesse dicendo, ma non poté fare a meno di notare che in qualsiasi lingua parlasse, manteneva il suo buffo accento.

« Seva? » gracchiò uno dei figuri, indicando Fabio e Vittoria.

« Ambdós han d'emprendre. »

« Pagar! »

« No ha problema, que poden pagar. »

Fabio aveva intuito abbastanza del dialogo da capire che doveva metter mano al portafoglio. Sapeva che sarebbe potuto succedere, si era preparato; tuttavia non aveva proprio pensato all'evenienza di dover pagare ben due cricche mafiose per essere lasciato in pace. La sua agitazione galoppò al pensiero di vedersi estorcere tutta la sua piccola fortuna dalla malavita catalana.

Tuttavia, quello che sembrava essere un grosso problema diventò totalmente trascurabile nel giro di pochi istanti. Successero molte cose contemporaneamente: si sentì uno sparo, delle urla e, come evocate dal nulla, apparvero altre figure inquietanti intorno al gruppetto formato dai mafiosi, Ahmed, Fabio e Vittoria. C'erano più pistole in quei dieci metri quadrati che in un poligono di tiro.

Spaventato, Fabio alzò le mani in segno di resa, ma nessuno sembrava avere interesse per lui. Ci furono altre urla, poi un'accesa discussione in catalano scoppiò fra i membri delle diverse cricche mafiose. Urla, sputi e spintoni: se non fosse stato a tiro di arma da fuoco, Fabio avrebbe riso di quell'impietoso sfoggio di maleducazione. Si limitò ad osservare in silenzio quelle stupide scimmie che si esprimevano come potevano, cercando di non far notare il suo divertimento. Rivolse uno sguardo a Vittoria e strizzò un occhio; lei ricambiò, ma la sua espressione era tutt'altro che divertita. Evidentemente, l'enorme pericolo che stavano correndo le impediva di notare quanto quella scena in realtà fosse buffa. Fabio continuò ad osservare il parapiglia, che nel frattempo si era fatto sempre più violento. Le due fazioni mafiose non sembravano proprio trovare pace. Un elemento in particolare era decisamente arrembante: si era aggrappato al collo di un altro e gli urlava nell'orecchio: "Ha d'escoltar! Ha d'escoltar! Dividir els diner!".

Dopo qualche istante che lo osservava divertito, Fabio trasalì. Come aveva fatto a non riconoscerlo subito?! Sciupato, scheletrico e pelato, il partecipante più animoso della rissa era proprio Daniele Brogelli. Le parole di stupore gli uscirono di bocca senza che potesse farci niente:

« Daniele! Cristo santo, ma che ci fai con la mafia! »

Fabio avrebbe decisamente dovuto tenere il becco chiuso. La rissa si fermò di schianto e l'aria si congelò; tutti gli sguardi erano puntati su di lui. Il silenzio assoluto che scese sulla scena non presagiva niente di buono.

Daniele era stupefatto: sembrava avesse visto una strega.

« No... non può essere... Fabio... » ansimò disperato.

Il mondo esplose. Fabio non capì molto di ciò che accadde: era totalmente in balia degli eventi, intorpidito da una trance di incredulità.

« Espia! »

« Espia de merda! »

« Traïdor! »

Alle urla seguirono spinte, sputi e pugni. Qualcuno sparò un colpo. L'escalation della rissa verso la guerriglia fu brutalmente veloce; in pochi istanti il gruppo si disperse ed i vari guerriglieri occuparono ripari strategici, esibendosi in uno scontro a fuoco in piena regola.

Fabio non era certamente in grado di fronteggiare un evento del genere. Lo sbalordimento per aver perso il controllo della situazione lo aveva paralizzato. Per fortuna ci fu Ahmed, che spinse prontamente sia lui che Vittoria dietro un container, momentaneamente fuori dal pericolo.

Lei strillò. Ahmed le si era accasciato addosso e non accennava a togliersi. Vittoria non era abbastanza forte per reggere il peso del marocchino: lo lasciò cadere. Una scia rossa si dipinse sul vestito della ragazza.

« È s - è sangue! » balbettò lei in preda al panico.

Fabio non si stava ancora rendendo contro di cosa stesse succedendo. Tremava forte, gli fischiavano le orecchie ed elaborava a scatti le immagini che vedeva; era sopraffatto da ciò che si stava consumando intorno a lui. Cercò stupidamente di aiutare Ahmed ad alzarsi, ma lui gli sputò del sangue addosso, rantolò qualcosa e poi fece silenzio.

Provò la sensazione peggiore del mondo quando capi che in quel corpo non si celava più alcuna vita. Finalmente si rese conto di quello che stava capitando: delle persone si stavano ammazzando per colpa della sua stupidità. Fu sconvolgente, come svegliarsi da un gradevole torpore a colpi di bombarda. Il suo cervello cambiò marcia, passando dal sonno ad un pericoloso overdrive. Due soli pensieri frullavano nella sua testa: Ahmed e Daniele.

L'urlo di rifiuto sgorgò violento:

« No! NO! »

Non poteva essere, non voleva crederci: strinse forte un braccio a Vittoria, come per aggrapparsi alla vita. Lei lo guardò attonita e fece come per dire qualcosa, ma era senza parole. Fabio doveva sapere: lasciò la presa e si sporse dal suo riparo. La terra sotto i suoi piedi sembrò scomparire quando vide il corpo di Daniele Brogelli disteso a faccia in giù. Neanche si accorse che qualcuno aveva preso a sparare verso di lui.

Si lascio cadere sulle ginocchia, incapace di sostenere la responsabilità di ciò che era appena successo. Si sentì trascinare indietro, al sicuro; Vittoria lo aveva tolto di peso dal pericolo. Fabio la guardò: il suo volto era una maschera di paura. Provo a dirle qualcosa, ma dalla sua bocca non uscì alcun suono.

« Jorge... » tento di dire lei con un filo di voce. Il fragore dei colpi di pistola coprì le sue parole; la battaglia stava ancora infuriando.

Fabio non rispose. Era catatonico, gli occhi sbarrati dall'orrore, incapaci di nasconderlo dalla visione del corpo Daniele, ormai impressa nella sua mente.

« Jorge, che facciamo? Che facciamo? » chiese Vittoria con più vigore.

Fabio continuò a tacere, la bocca sigillata dall'angoscia. In che guaio la aveva cacciata? Era colpa sua, solo e soltanto colpa sua. Qualsiasi cosa sarebbe successo a quella povera figliola, lui ne era responsabile. Cercò di dirle qualcosa per rassicurarla, ma il fiato di Fabio era un grande assente in tutta quella faccenda. La situazione lo stava riempiendo di emozioni troppo intense per lui. Rivolse un'occhiata riluttante al corpo di Ahmed. Non era un suo amico, lo conosceva a malapena, eppure il suo ultimo gesto era stato quello di spingere lui e Vittoria lontano dal pericolo. Anche Ahmed era morto a causa sua.

La misura di Fabio era colma: traboccò. Un folle impeto lo pervase. Non sarebbe rimasto lì, nascosto, in attesa che tutto finisse. Che diavolo, era tutto sbagliato! La sua fuga dall'Europa non doveva andare così, nessuno avrebbe dovuto farsi male, né estranei né amici. Ormai era successo l'irrecuperabile, ma non poteva essere una scusa per evitare di agire. Cacciò un urlo per caricarsi ed estrasse la pistola. Non avrebbe permesso che accadesse qualcosa a Vittoria, la cui unica colpa era stata quella di fidarsi di lui! No, non se ne parlava: se doveva morire, lo avrebbe fatto cercando di difendere almeno lei.

« Jorge... JORGE, FERMO, CHE FAI?! » gridò lei, vedendo il suo compagno animarsi all'improvviso.

« Scappa! » le intimò Fabio. « Io cerco di fare pulito, te corri più lontano che puoi! »

Lei improvvisamente scoppiò a piangere.

« No... no! » singhiozzò.

« Fallo, dammi retta! È tutta colpa mia, te non c'entri niente in questo inferno! Vai, scappa! »

« No, non ti lascio! NON TI LASCIO! »

Fabio le tirò uno schiaffo. Si sentì morire dentro, ma non sapeva come altro farsi ascoltare.

« Devi darmi retta! FAI. COME. TI. DICO. Non voglio che ti succeda niente, hai capito? Scappa, vai verso l'entrata di questo cazzo di porto! Poi vai in un posto affollato, almeno sarai al sicuro... »

Una raffica di colpi si infranse sul container che li celava, facendo un rumore metallico assordante.

« Presto! MUOVITI! » abbaiò Fabio, allarmato.

« Ma muoviti dove?! Dove vado?! La gente spara! »

Altri proiettili colpirono la loro barriera.

« La gente sparerà a me! Vai, cazzo! VAI! »

« E l'America? Il viaggio, la nave! Voglio partire con te! »

« Questo è l'ultimo dei nostri problemi, adesso! Facciamo così: ci vediamo a mezzogiorno davanti al Corte Inglés. D'accordo? »

Lei non sembrava affatto rassicurata. Aveva capito quello che voleva fare Fabio: ingaggiare quei tizi per darle il tempo di fuggire, a costo di rimetterci la vita. Lo abbracciò fulmineamente, si alzò in punta dei piedi e lo baciò.

« Ti prego, non rendere tutto più difficile! » disse Fabio, svincolandosi dalla presa che lo avviluppava.

« E se non vieni? E se... »

« Se non vengo, sei tua. Sei forte, ce la farai. Ora vai! VAI! »

Vittoria scoppiò in lacrime e, finalmente, corse via. Fabio la guardò per un ultimo istante, prima di scattare allo scoperto. Sapeva quello che doveva fare: rapido, prima che potesse tradire la paura, cercò dei bersagli con lo sguardo e sparò.

Il suo unico pensiero, mentre affrontava quell'inferno, era Vittoria.

Il prossimo capitolo sarà pubblicato il 31 gennaio 2018. Buon anno, bifolchi in affanno!

-Simone




\chapter{Rendez-Vous}

\begin{chapquote}{Author's name, \textit{Source of this quote}}
``This is a quote and I don't know who said this.''
\end{chapquote}

% pulled

Plaça de Catalunya era piuttosto affollata. Vittoria fissava una vetrina di El Corte Inglès, osservando il riflesso del suo volto. La sua mente era altrove.

"Sei tua", le aveva detto. Vittoria sapeva badare a sé stessa, ma quella frase la aveva angosciata molto, ancor più di tutto il resto della situazione. Essere propri... voleva forse dire essere gli unici responsabili della propria sorte? "Se non vengo, sei tua". Quindi, fino ad allora era stata di Jorge? Lui si sentiva veramente responsabile per lei? Teneva a lei fino a quel punto?

Non lo sapeva. Jorge era un mistero per Vittoria; nei brevi mesi in cui erano stati insieme, non era riuscita a farsi un'idea della reale persona che aveva di fronte. C'era poco da fare: non lo capiva, non importava quanto si sforzasse. Le parole del ragazzo spesso contraddicevano le sue azioni, ed altrettanto spesso succedeva viceversa. Non contava quale stimolo potesse ricevere, niente di vero trapelava da lui.

Vittoria aveva meditato a lungo sul quel misterioso uomo; lo aveva messo al centro di molte sue estasi mistiche, usandolo come perno sul quale far ruotare tutta la sua mente durante quelle ore di intossicazione. Non era servito proprio a niente; a dire la verità, non si era nemmeno aspettata che funzionasse. Nessuna illuminazione le era giunta. Jorge le era rimasto impenetrabile, il suo misterioso fascino intatto. Era una maschera troppo efficace per essere frutto del caso: quel bel ragazzo si ostinava di proposito a nascondere il vero sé.

Che fosse segretamente innamorato di lei? Vittoria lo era di lui. Nemmeno tanto segretamente, a dire il vero.

Voltò le spalle alla vetrina e scrutò la piazza con malcelata disperazione. Non poteva finire tutto così, non era giusto. Lui doveva venire. Lei voleva che venisse. Quella mattina era stato tutto così intenso: non poteva non esserci un lieto fine. Costrinse la sua attenzione ad occuparsi dell'ambiente circostante, ma nella piazza non c'era niente di interessante, ed il centro commerciale non sembrava esercitare più alcuna attrattiva su di lei. Anche se, a ripensarci...

Un accenno di sorriso comparve sul suo volto. El Corte Inglés era stato uno dei principali motivi per cui aveva voluto farsi portare in vacanza a Barcellona. Beh, qualunque posto sarebbe stato migliore dei soliti paesini liguri che i suoi vecchi sembravano adorare. Rapallo, Camogli, le Cinque Terre: non c'era niente, assolutamente niente. Almeno Barcellona era una città, con dei locali da giovani ed una bella collezione di negozi interessanti! Al diavolo calamite, cartoline e varie reliquie di una vacanza all'insegna della noia più totale. Si sarebbe fatta comprare l'intero stand della Mac dai suoi genitori, ed al ritorno avrebbe fatto morire di invidia le sue amiche.

Ma ormai non gli importava più granché di tutto quello. Nella testa aveva soltanto lui, Jorge: la sua espressione tagliente, il suo cinismo, il suo coraggio...

Non poteva continuare a pensare al ragazzo, doveva distrarsi. Nessuno dei passanti sembrava far caso a lei: si frugò nel reggiseno e ne estrasse un foglietto. Lo osservò per un lungo istante, poi lo ripose. Nemmeno tutta la droga del mondo la avrebbe calmata in quella situazione. Anzi, probabilmente avrebbe rotto i freni coscienti che aveva posto alla sua angoscia. Si costrinse a restare presente, sforzandosi per non scoppiare a piangere dalla disperazione.

All'improvviso, un tocco lieve sulla spalla.

« Sono vivo. »

Il suo fragile autocontrollo crollò, ridotto in briciole da un'improvvisa valanga di Oh-Mio-Dio-Stai-Bene. Si gettò fra le braccia di Jorge e frignò tutte le lacrime che aveva.

« Cre-credevo... avevo paura... » balbettò dopo un po', tentando di controllarsi.

« Anche io avevo paura » mormorò lui, « ma ce l'ho fatta. »

Vittoria si asciugò gli occhi e lo guardò.

« S-stai... stai bene, v-vero? » gli chiese, intimorita.

Sul volto di Jorge era scolpita un'espressione così terribile da farla preoccupare. Sembrava che avesse passato cento anni di dolore.

« Sì » rispose semplicemente lui.

La ragazza lo osservò per un istante. Aveva vestiti diversi rispetto a quella stessa mattina, sembrava essersi appena fatto una doccia. Non aveva più con sé il fagotto con i suoi averi.

« Che... che è successo? S-sei ferito? », tentò.

Jorge non rispose. Si voltò e fece cenno a Vitoria di seguirla; i due cominciarono a camminare verso il centro della piazza. Il silenzio durò solo qualche passo.

« Oh, tutto bene? » insisté lei con vigore.

« Sì, va tutto bene. Perché sei nuda? Fa freddo. »

« E' una canottiera, non sono nuda! Ero tutta sporca di sangue, che dovevo fare, andarmene in giro in quel modo? »

Jorge si limitò a ridacchiare sottovoce. In qualche modo, quello fu troppo: la sospensione di incredulità creata dalla forte emozione andò in frantumi. Vittoria si fermò di colpo. Quello che stava succedendo non era genuino, doveva fare qualcosa.

« Ascolta, non puoi cavartela così! » disse bruscamente.

Prese Jorge per mano e lo strattonò verso la panchina più vicina, facendocelo sedere a forza. Lui non oppose resistenza.

« Lo vedo che sei sconvolto, sai? » proseguì. « Riesco a capire la situazione, sono una persona, mica una bestia! Ho quella cosa che si chiama empatia, posso sentire quando non stai bene! E se stai male tu, sto male anche io. Quindi ora mi parli dei tuoi problemi, io ti ascolto, tu stai meglio e io ti vedo felice. Te capì? »

Jorge sorrise un sorriso amaro, ma non aprì ancora bocca.

« Su! » esortò lei. « E' per quel ragazzo che è morto? E' perché hai visto morire tutta quella gente? E' perché il viaggio è saltato? Non fare il sostenuto, hai una faccia che è un pomodoro! Non devi essere di ghiaccio con me, non è che conti meno se mi fai vedere un po' di emotività, eh? »

La risposta fu glaciale:

« Sì, sì e no. Impara a non fare troppe domande in una frase sola. »

« Eh no figa, eh! Enne ci esse, proprio! Non ci siamo. E-mo-ti-vi-tà, ce la fai? Riproviamo. »

« Emotività » ripeté cupo Jorge, la voce quasi un sussurro. « Fa rima con stupidità. »

« Vedi che va meglio così? Parliamo. Ti sei convinto dei legami fra le parole? Perché l'altro ieri ti ho detto qualcosa del genere, ma non mi ricordo bene come è andata a finire. »

Jorge sospirò, facendo schiudere leggermente l'aria da funerale che lo avvolgeva. Parlò:

« E' andata a finire come tutte le altre sere: te che fissavi il soffitto, strafatta, ed io che mi chiedevo perché certi pensieri erano nella mia testa. »

Vittoria proseguì, sforzandosi per infondere tutto il calore che poteva nella sua voce:

« No perché, ora che non prendo la roba da qualche ora, non è che abbia più di tanto senso eh! Cioè, quasi tutti i nomi di cose astratte finiscono con la a accentata! emotività, stupidità, bontà... ehm... calamità... »

« Acidità, aridità, virilità, validità, meschinità. Potrei dirtene altre venti senza mai prendere fiato. »

« Ah. Non me ne venivano in mente altri. Come sei intelligente... »

Jorge soffocò una risata beffarda, senza allegria. Si passò una mano sul viso, poi prese a tormentarsi il pizzetto. Parole sofferte gli uscirono di bocca:

« Intelligente? Ti ringrazio per il complimento, ma i fatti mostrano il contrario. Una persona intelligente avrebbe sputtanato un amico alla mafia e fatto scoppiare un conflitto armato? Lo avrebbe fatto per il solo motivo di aver seguito l'impulso della sorpresa senza pensare alle conseguenze? »

Vittoria gli scoccò uno sguardo intenso.

« Non significa che sei stupido » gli disse dolcemente. « Significa che sei umano. Gli umani sbagliano e... stanno male, poi. Va bene così, non siamo macchine. »

« Beh, allora non voglio più essere umano » proseguì amaramente Jorge. Una dolorosa scintilla di risolutezza apparve nei suoi occhi.

« Non fare così... » mormorò lei.

I toni si alzarono.

« Tu non sai... non puoi capire! Ahmed era umano. Mi ha aiutato, e guarda dove è finito! Per terra in una pozza di sangue! »

« Guarda che l'hai pagato, eh! Mica gliel'ha detto il dottore di aiutarti! »

« Daniele era un caro amico. » tirò dritto lui, ignorando l'interruzione. « Un amico vero e sincero. Pieno di difetti, tutto quello che vuoi, ma ci volevamo bene per davvero. E ora è li per terra pure lui! Ammazzato come una bestia, senza esitazione o ritegno. Lui era umano! A chi è importato della sua vita? A i'ccane didd.., ecco a chi! Sono bastate due parole sbagliate al momento sbagliato per distruggerlo! Questa è la verità, questa è la realtà! E allora, sai cosa? Se queste sono le cose che succedono agli umani, io mi chiamo fuori! Se il prezzo per essere umano è far succedere queste cose, io non voglio più esserlo! »

Un pizzico di peli di barba cadde a terra, strappato con violenza sul finire dell'ultima frase. Vittoria rimase senza parole, ma Jorge continuò come un fiume in piena: la sua voce non era più un lamento animoso e tremante, ma chiara e fredda. Una mano decisa ricompose il suo pizzo torturato e prese a lisciarlo delicatamente. Il suo sguardo era fisso su qualcosa che solo lui poteva vedere.

« Ho perso il conto di tutte le volte in cui ho indugiato in questi pensieri. Daniele era come me, in questo senso. Era messo peggio, ma condividevamo bene o male lo stesso abisso. Sai qual era la sua soluzione, quando era giù? Cercava di ammazzarsi, quel cretino. Non lo vedevi più in giro per qualche giorno, poi magicamente tornava, pieno di belle speranze pronte ad infrangersi alla prossima occasione. Capisci quanto era stupido? Non imparava mai.

« Ma io non sono come lui. Non mi è mai passato per la mente di uccidermi. Invece di fuggire dalla vita, sono scappato solo dalle situazioni che mi facevano stare male. Credevo di potercela fare... Confesso che ci speravo davvero. Ho mollato tutto per tentare questa via. Eppure sto sbagliando. Le cose sono due: o non è la strada giusta, o non l'ho percorsa tutta. Ormai è imboccata, non mi resta che stringere i denti ed arrivare in fondo. Se l'unico modo per sopravvivere è diventare un mostro, allora lo diventerò. »

E Vittoria finalmente capì. La luce del timido sole di novembre sembrò illuminare Jorge più di quanto non avrebbe fatto un riflettore. Vide per la prima volta l'uomo che aveva vicino, quell'uomo che non aveva ancora compreso: un'anima sola, fragile, che non accettava la propria debolezza.

« No, no! », negò con forza. « Non è così che funziona! Non puoi... Cioè, non puoi proprio dire... che non sopporti di stare male, allora... diventi cattivo e poi... e poi non provi più nulla! Non riesce... non funziona, non va bene! »

« Sei sicura? » sibilò lui, dopo un'intensa pausa.

Vittoria si stava agitando. Il suo Jorge non avrebbe dovuto fare così.

« Certo! » gli rispose, tentanto di tenere la sua crescente disperazione lontana dalla voce. « È come - è una di quelle cose così'ovvie che sono difficili da spiegare, ma - ma proprio perché sono ovvie! Non puoi diventare cattivo, o lo sei o non lo sei! E poi, anche se sei cattivo non diventi automaticamente uno psicopatico che non sente niente! »

Jorge mostrò un orribile ghigno senza allegria. Parlò duramente:

« Sai a cosa pensavo stamani, quando ti ho fatta scappare? A un bel niente. Ero pilotato totalmente dall'impulso di proteggerti. Avevo visto un conoscente ed un amico morire nel giro di trenta secondi, ero schiavo dell'impeto di protezione. Sopraffatto dall'errore, cercavo disperatamente di limitare i danni. Mi sarei sacrificato pur di non vedere anche te lì per terra, in una pozza di sangue. Ma ad un certo punto... »

Il suo volto si deformò nell'espressione più terribile che Vittoria avesse mai visto sulla faccia di una persona: l'arricciatura degli angoli delle labbra era semplicemente malvagia, e gli occhi sembravano brillare di una luce perversa, immonda.

« ...ad un certo punto ho capito. Mentre danzavo da un nascondiglio all'altro come uno scemo, cercando più di non farmi sparare che di combattere, è stato allora che ho compreso che stavo sbagliando tutto. Cosa diavolo mi era saltato in testa? Non dovevo cercare di proteggerti, ma semplicemente smettere di provare quell'impulso alla protezione. Smettere di avere dei cari da proteggere. Rendermi indifferente al resto del mondo. È per quello che sono qui, a Barcellona, invece che a casa mia. Invece eccomi lì, a pregare di non morire per impedire che ti venisse fatto del male. Mi sono sentito così stupido - »

Vittoria non poteva sopportare di sentire quelle cose. Lo interruppe:

« Così sei scappato via... »

Parte del male dipinto su quel volto si dissipò: Jorge rise.

« Scappato via? E dove? La gente mi sparava addosso. No, sono stato furbo: mi sono accasciato e ho fatto il morto. Col senno di poi, era la cosa ovvia da fare fin dall'inizio: ero già sporco di sangue e avevo una bella pozza su cui buttarmi. Sono stato un cadavere perfetto, una magnifica interpretazione. Per fortuna nessuno è venuto a controllare se respiravo, ma avevo pensato a qualche carta da giocare anche in quel caso. Tempo neanche dieci minuti la battaglia si è spostata un po' più in là: ho ripreso i miei soldi dal corpo di Ahmed, ho rubato qualcos'altro e me ne sono andato. Il resto lo puoi anche immaginare. »

« Allora » tentò timidamente la ragazza, « tutto è bene quel che finisce bene, no? »

Lo sguardo della follia tornò, gli occhi dei due si allacciarono in uno sguardo. La temperatura attorno a quella panchina sembrò precipitare.

« No » disse semplicemente Jorge. « Non è finita bene, ma ormai non mi importa più. Non hai capito, Vittoria? Io non baderò più a te, non avrei mai e poi mai dovuto cominciare a farlo. Devi andartene, se ci teni a te stessa: le cose d'ora in avanti saranno sempre più pericolose. »

La lacrime a lungo trattenute sgorgarono dagli occhi della ragazza:

« Io ti amo » singhiozzò lei.

Se quell'ammissione lo colpì, Jorge non lo diede a vedere.

« Questo, cara mia, è affar tuo », le disse deciso. « Io sono un assassino: ho ucciso e ucciderò ancora. Se verrai con me, non è escluso che ad un certo punto tu muoia per mano mia. »

« Non mi importa! Io voglio scappare con te... Rifarmi una vita insieme a te, come avevamo detto... »

« La mia vita temo che sarà piuttosto strana, d'ora in poi. La strada che ho scelto di percorrere non porta in nessun luogo bello. Sono cosciente di quello che faccio, e so bene che andrà a finire terribilmente male. Se resterai con me, o morirai o diventerai qualcosa che non saresti voluta diventare. »

« Non mi importa... » ripeté lei, avvicinandosi.

Lo baciò.

Lui non si scansò.

***

Un volantino, ignorato da tutti, svolazzava ormai da qualche giorno nella ventosa Barcellona di novembre.

"OFERTA ESPECIAL - VIARGE INAUGURAL DE NOU CREUER", diceva da un lato; "SPECIAL OFFER - FIRST JOURNEY OF THE NEW CRUISE SHIP", mostrava l'altro.

"Take a trip on the brand new GESUCCA cruise ship!" , proseguiva il foglio pubblicitario. "Don't let this SPECIAL OFFER slip away! With only € 299.90 you will coast Mediterranean Sea from Barcelona to Leghorn and be right back in just 2 days!"

Quel volantino, al termine del suo vagare, un bel giorno finì proprio in grembo ad un certo Fabio Fontanelli, mentre egli se ne stava seduto su una panchina a farsi i fatti suoi assieme ad una ragazza.

Un sorriso sincero si dipinse sul volto del giovane uomo quando i suoi occhi scorsero le microscopiche parole:

"Reservations open until 22/11, ship sails from Barcelona on 01/12 at 9:00. This is a sea-only cruise, no ID or passport is required to attend. Passengers are not allowed to get off the ship for any reasons during docking."

Forse qualcosa stava cominciando a girare per il verso giusto.


NOTA DELL'AUTORE:

Il prossimo, vendicativo capitolo verrà pubblicato il 2 marzo 2018, e tante grazie a me che ho trovato il tempo di sistemarlo. Avete rischiato grosso, io ve lo dico!

- Simone

EDIT: piccolo cambio di programma: ve lo metto domenica 4 marzo, ad urne chiuse, mentre guardo la maratona di Mentana.




\chapter{Vendetta}

\begin{chapquote}{Author's name, \textit{Source of this quote}}
``This is a quote and I don't know who said this.''
\end{chapquote}

% pulled

« Ma chi ti si incula, Gazzi! Sei come lui, sei una merda! »

Anton Leka era furioso. Come osava il Gazzi presentarsi di fronte a lui, dopo quello che era successo? Credeva forse che gli avrebbe permesso di farsi beffe di lui, della sua condizione di vita?

« Anton, per favore, monta in macchina... »

Giacomo Gazzi non appariva affatto a suo agio nell'affrontare il suo ex amico, e ne aveva ben donde.

« Per favore? PER FAVORE?! », lo aggredì verbalmente Anton. « Leccami le palle, PER FAVORE! Sei una merda, non mi devi parlare! Levati dai coglioni! »

« Calmati, Anton... devi ascoltarmi. »

Leka gli avrebbe volentieri chiuso la testa nello sportello del suo SUV, ma si limitò a proseguire l'invettiva: 

« Ascoltarti? ASCOLTARTI? L'ultima volta che hai aperto bocca mi hai fatto licenziare! "Dai, un po' di coca, portami la coca", porcodd... la coca! Io non spaccio, te l'avevo detto, ma te no, dovevi avere quella cazzo di droga da ricchi! L'unico rischio che mi sono preso in tutta la mia cazzo di vita... Per te! Capito, infame schifoso? Per te, uomo di merda! SONO NELLA MERDA PER COLPA TUA! »

Gazzi esitò un istante, prima di azzardarsi nuovamente a dire qualcosa.

« Anton... per favore... » fece, con un filo di voce.

« Allora parla, forza! » lo aggredì di nuovo il Leka. « Sentiamo cosa ha da dire Giacomino, invece di andare a prendere il posto alla mensa! »

La voce di Giacomo riacquistò improvvisamente vigore.

« Alla... oh, andiamo, ti porto a cena se è questo il problema! »

« Ti... ti porto a - A CENA? »

Dopo tutto quello che era successo, sapendo ciò che stava passando, Giacomo Gazzi osava offrirgli una cena?

« Ti prego... », lo implorò l'impomatato cicisbeo. « Bisogna che tu mi ascolti... voglio aiutarti! »

Aiutarlo? Se Anton fosse stato un vulcano, avrebbe eruttato tutte viscere della Terra.

« TU - TU CHE?! Ma sei rincoglionito? HAI CAPITO CHE COSA MI HAI FATTO?! »

« Davvero, Anton, devi ascoltarmi... » piagnucolò il Gazzi.

Uscì dall'auto e tentò di mettere una mano sulla spalla del Leka, ma questi la scansò brusco, senza dire una parola.

« Per favore! » proseguì deciso. « Ho scoperto delle cose! Cose importanti su quello che è successo! »

Anton gli scoccò il più intenso sguardo intriso di disgusto che riuscì. Si dovette sforzare per tradurre la sua furia in parole acide.

« Oh, il Gazzi ci ha pensato e si è inventato la storiella! E dimmi, è bella? Finisce bene? Perché non mi pare proprio che per me sia finita bene! »

***

Già, era stata proprio una bella storia. Ripensando alla fatica fatta per trattenersi dall'aggredire il suo interlocutore, Anton quasi non credeva che fosse stato possibile farsi convincere ad andare a cena con lui. Era sicuro che la fame avesse giocato un ruolo da protagonista nella faccenda, ma non riusciva comunque a perdonarsi di aver dato retta a quell'imbecille.

Era stato in imbarazzo per tutta la sera. Il Gazzi lo aveva portato in un noto ristorante nel centro storico, uno dei posti più costosi di tutta Prato; un posto che lui stesso conosceva molto bene. Prima di venire licenziato, frequentava spesso quel locale: ci portava le ragazze che sperava di rimorchiare, spendendo montagne di denaro in piatti e vini costosi per impressionarle. Aveva addirittura confidenza col proprietario! Presentarsi in quello stato, conciato come un barbone, senza l'ombra di un quattrino e totalmente dipendente dalla carità del suo detestato amico, era stata una grande umiliazione. Inoltre, la presenza stessa del Gazzi non lo aveva certo messo a suo agio: quel bastardo inamidato se ne stava lì, tronfio, a decantare vini e piatti, avvolto dal suo completo di alta sartoria, con la camicia cifrata e le scarpe di marca tanto lucide da potercisi specchiare. Il contrasto fra loro due non avrebbe potuto essere più netto.

Ma aveva sopportato. Era da tanto che Anton non mangiava e beveva così bene. Anzi, ad essere onesti, era quasi un giorno che non mangiava affatto. Con il bere si era potuto arrangiare; per quel che ne sapeva lui, gli alcolici scadenti da due soldi erano altrettanto efficaci a stregare i sensi quanto quelli più pregiati. Ma investendo nell'oblio dell'alcool tutto il poco denaro che riusciva a raccattare, non poteva far altro che affidarsi alla mensa dei poveri per ottenere del cibo. Il senso di vergogna che provava era così insopportabile che spesso preferiva saltare un pasto piuttosto che abbassarsi ad elemosinare. Ogni volta che si metteva in fila, che chiedeva una scodella di zuppa e un cantuccio di pane, ogni singola volta che ringraziava per la carità che gli era stata fatta, la verità si ergeva impietosa davanti a lui: Anton Leka non era più nessuno, solo un patetico barbone.

Il Gazzi si era professato disposto ad aiutarlo. Aveva detto che gli avrebbe dato dei soldi, che gli avrebbe trovato un lavoro, che lo avrebbe strappato a tutti i costi dalla spirale di disperazione che lo avviluppava. Non poteva vederlo in quello stato, non avrebbe permesso che un suo amico si riducesse così. Giacomo Gazzi aveva detto tante cose quella sera. Di tutto quel blaterare, niente lo aveva toccato. Solo la pronuncia di nome era riuscita a catturare l'attenzione di Anton: Bruno Bagonghi.

E così erano finiti a parlare di lui, il suo acerrimo nemico, colui che lo aveva condannato all'umiliazione togliendogli il lavoro. Pareva che l'impresa del Bagonghi navigasse in cattive acque: quello non era affatto un segreto, eppure Gazzi si era comportato come se stesse confessando i più terribili misteri mentre raccontava arzigogolate storie di contese con la mafia cinese, attività illegali e sotterfugi degni del peggior telefilm giallo. In un'altra situazione, ad Anton questi pettegolezzi non sarebbero interessati granché. Avrebbe bollato tutto come le ennesime storielle di Giacomo Gazzi, pizzichi di realtà mischiate a fantasia di scarso interesse. Tuttavia, il tremendo risentimento che provava verso Bruno la pensava diversamente: era stato licenziato in tronco, "per giusta causa" diceva la lettera, senza alcuna possibilità di fiatare in sua difesa.

Fu quello l'amo che il Gazzi predispose, ed al quale lui abboccò. Anton Leka si abbandonò all'odio, senza preoccuparsi di filtrare quello che gli veniva detto. Credette tutto quanto: che Bruno avesse premeditato il suo licenziamento; che avesse complottato abilmente contro di lui, ricattando Giacomo affinché chiedesse della cocaina. Credette che avesse chiamato lui stesso i carabinieri per inscenare una perfetta giusta causa di licenziamento in tronco. Tornava tutto, nella mente stritolata dal rancore di Anton Leka. Il Bagonghi voleva licenziarlo, aveva sempre voluto farlo, ma era pieno di debiti e non poteva permettersi la sua liquidazione. Era ovvio che fosse andata in quel modo, che fosse o meno Giacomo Gazzi a dirlo.

Forte della sua nuova convinzione e corroborato dal vino, Anton Leka si recò nella notte alle porte del Lanificio Bagonghi, portando con sé un malcelato intento omicida. Sarebbe stato facile, aveva detto Gazzi, entrare nello stabile passando dal parcheggio, aprire la porta a vetro con la chiave nascosta nel sottovaso ed assassinare Bruno quella sera stessa... ma erano solo parole di sfogo, non diceva sul serio, aveva addirittura supplicato Anton di non farsi venire strane idee.

Fu così che Leka arrivò indisturbato al cospetto della porta dell'ufficio del suo nemico: l'unica fra le tante di quell'orribile e polveroso corridoio ad avere una parvenza di eleganza. Il cuore gli martellava prepotente contro le costole, segnalando il rischio; ignorarlo era difficile, ma doveva imporsi la calma ad ogni costo. Delicatamente, premette un orecchio contro il legno massiccio che lo separava dal Bagonghi.

Sentì una voce: era lui! Almeno a proposito di questo, Gazzi aveva detto il vero. Ma stava davvero lavorando, alle undici di venerdì sera? Anton ascoltò: la voce di Bruno era l'unica a risuonare in quella stanza. Forse parlava al telefono?

« ...avevamo già detto. No, niente neanche lui: solo tu e l'ingegnere. Non è che non mi fido, è che... bravo, hai capito. Certo, infatti... »

Sì, parlava proprio a telefono.

« ...già firmati. No, post-datati. Domani. Ti prego, segui il piano alla lettera. No, non è per quello. No, no, assolutamente no! Non gli permetterò di mettere le zampe su tutto quello che ho costruito! Sì, scusa. Giusto, ma ho già pensato a tutto. Esatto. Bravo: fatta la magia, il patrimonio della società sarà immune a qualsiasi pretesa. Tu sarai a posto, mia sorella pure e quel cane sarà servito. Che ti importa di me? Ti sei forse affezionato? Dai, stavo scherzando. No, non passerò da Gambino... »

Di cosa diavolo stava blaterando il Bagonghi? Con chi ce l'aveva? Anton era così concentrato sull'ascolto che non si accorse affatto di un ceffo alto e snello che si era materializzato nel corridoio.

Fece un salto di quasi un metro quando sentì una mano gentile posarsi sulla sua spalla.

« Lei è Leka, giusto? Il signor Bagonghi è in ufficio, ma temo proprio non possa riceverla. Non può aspettare fino a lunedì? »

Anton rimase senza fiato dalla paura. Era stato l'ingegner Govidi a spaventarlo: un ragazzo tanto distinto quanto spilungone, tanto pacato quanto fastidioso. L'appellativo ingegnere, a quanto ne sapeva Anton, non aveva niente a che vedere con nessun titolo di studio, ma era solo un nomignolo simpatico per descrivere l'aura da grigio accademico che emanava la sua figura. Si occupava lui delle scartoffie del lanificio. Era stato lui a consegnargli la lettera di licenziamento senza scomporsi di una virgola, la sua maschera di cortese distacco intatta.

Un pugno rabbioso, dritto in faccia: l'ingegnere cadde a terra, e lì rimase.

Fiondò nuovamente l'orecchio sulla porta. Bagonghi aveva smesso di parlare: non c'era più alcun rumore in quella stanza, nessuno che Anton riuscisse a captare. Era probabile che il siparietto con Govidi fosse stato notato. Doveva agire, prima che fosse troppo tardi! Si frugò nelle mutande con mani tremanti e ne estrasse un fagotto di stoffa. Il coltello da carne che aveva trafugato dal ristorante stava per squarciare il maiale più grande della sua carriera. Fece un bel respiro, si armò dell'odio più nero che riuscì a trovare nel suo animo e spalancò la porta con un calcio.

Bruno Bagonghi non si scompose nemmeno di un millimetro: lo attendeva appoggiato alla scrivania, sigaro in bocca e fucile in mano.

« Che c'è? » chiese brusco. « Ti aspettavi che sobbalzassi? Che urlassi di paura? Povero te... »

Fece un sospiro e si stampò in faccia un sorriso senza allegria. Il fucile non puntava verso l'intruso ma era rivolto verso l'alto, retto da una mano sola.

Leka rimase senza parole, interdetto dalla reazione della sua nemesi. Il suo sguardo rimbalzava intermittente fra il volto del Bagonghi e la sua arma da fuoco.

« Caro Anton » proseguì quello, la voce ora fastidiosamente affabile, « dovresti essere più silenzioso quando accoppi i miei dipendenti. L'ingegner Govidi, sebbene sia snello, ha fatto un bel tonfo cadendo a terra! »

L'odio eruppe di nuovo in Anton, potente più che mai. Non avrebbe sopportato le beffe del suo nemico, nemmeno se gli avesse puntato contro un pezzo d'artiglieria.

« Come - come osi - NON TI PROVARE A PRENDERMI PER IL CULO! Chiudi subito quella fogna, maiale ebreo! »

« Oh, fammi un favore », sbottò lui, « smettila con queste cazzate da razzista. Intanto, dare dell'ebreo a qualcuno è considerabile un'offesa solo in qualche barbaro gergo, usato da pochi ancor più barbari figuri, non certo da me. Poi, il fatto che la maggioranza degli ebrei sia stata storicamente più ricca del ceto sociale più incline a credere alle fake news del periodo - e ti prego di notare la lunga perifrasi che mi tocca fare per non darti direttamente dello stupido - non basta certo a trarre l'implicazione contraria. »

Ci fu silenzio per qualche istante.

« Non ci hai capito molto, vero? Comprensibile. Lo dirò in parole povere, e se ancora non capirai saranno cazzi tuoi: non sono ebreo, sono solo ricco. »

La lama di Anton sferzò l'aria, carica di rabbia.

« RICCO UN CAZZO! », urlò. « Sei nella merda, pensi che non lo sappia? Non ci hai pensato nemmeno un secondo prima di buttarmelo in culo per salvare te stesso! »

Bagonghi si accigliò.

« Penso che tu sappia solo ciò che ti hanno detto », disse senza emozione. «Il che non è molto di concreto, in realtà. Ma d'altronde, da uno che crede all'autenticità dei Protocolli di Sion non mi aspettavo certo di meglio. »

« Per l'ultima volta - NON OSARE PRENDERMI PER IL CULO! Credi che sia proprio scemo, vero? Un povero imbecille, che se le beve tutte! Lo so benissimo che il Gazzi è un cazzaro patentato, se credi che dia retta a tutte le sue stronzate non mi conosci affatto! »

Bruno sorrise, ma il suo sguardo si indurì.

«Il Gazzi, eh? Ottimo. Potevo anche indovinarlo, ma tanto valeva controllare. E così, tra una cazzata e l'altra, ti sei fatto manipolare da Giacomino per venire fin qui a compiere la sua opera? Lasciatelo dire, sei veramente un coglione: aggraziato come un pachiderma, armato con uno stupido coltellino da tavola, che speravi di fare? Non ti smentisci mai, sei un disastro ambulante, ora come sempre da quando ti conosco. D'altro canto... lui mi spiazza. »

Clack-clack. Il fucile a pompa si rivolse verso Anton.

« Perché il Gazzi ti ha mandato al macello in questo modo? Non poteva sperare veramente che bastassi tu per uccidermi. Che cosa intendeva ottenere, mandandoti qui a farmi perdere tempo? Sei un diversivo? Parla! »   

Anton si senti rimpicciolire. Non poteva lasciargli dire quello che voleva, avrebbe dovuto infilzarlo, sgozzarlo, fargli più male possibile, anche a costo della vita. Eppure, un po' per la minaccia armata, un po' per uno strana sensazione che lo stava invadendo, rispose:

« Non so un cazzo di quello che voleva o non voleva fare il Gazzi. Sono qui per conto mio, per chiuderti per sempre quella fogna sputasentenze! Credi che quel cretino mi possa intortare come gli pare e piace? »

« Sì », rispose semplicemente il Bagonghi.

« BEH, TI SBAGLI! Te ne stai lì, dietro quel cazzo di ferro, credi di essere al sicuro, di tenermi sotto tiro! Tu non sai niente, NIENTE di me! »

Bagonghi sospirò. Si tolse il fucile di mano, appoggiandolo sulla scrivania.

« Molto bene », fece senza entusiasmo. « Non avevo voglia di interpretare questa parte, ma se le cattive maniere non funzionano... »

Assunse un'espressione di sufficienza e declamò con tono monotono:

« Anton Leka, sei nato a Scutari il 30 gennaio del '91, abbandonato dopo il parto da una sconosciuta montenegrina. Hai vissuto in un orfanotrofio fino ai nove anni, quando sei stato adottato da una vecchia coppia di Durazzo. Sei scappato di casa a diciassette anni ed emigrato in Italia da solo; i tuoi genitori adottivi non si sono mai dati la pena di venirti a cercare. Hai conseguito un attestato di frequenza biennale alle scuole serali, e lavori per me da quando avevi diciannove anni. Non hai mai avuto una relazione stabile, sei un tabagista ed un alcolista. Non credo di sia altro da sapere su di te: non sei niente di più che un mucchietto di dati su una delle mie schede. »

Anton rimase completamente senza parole. Bruno Bagonghi proseguì:

« Ascoltami. Sarò meno sottile di quanto non sia di solito: il Gazzi sta tramando contro di me, ed io di riflesso mi sto difendendo. Vedilo pure come un gioco fra noi due: tu, in questo gioco, sei soltanto una pedina. E non sono stato certo io il primo a muoverti. »

La strana sensazione aveva ormai avvolto completamente Anton. Si sentiva gabbato, truffato, ingannato. Ma non riusciva ancora razionalizzare quello sgradevole presagio. Dove stava andando a parare il Bagonghi, con tutti i suoi discorsi? Non era sicuro di volerlo sapere. Scosse forte la testa, raccolse tutto lo sdegno che riuscì a trovare in sé e disse:

« I tuoi giochi a me non interessano. Io voglio vendetta. Tu mi hai licenziato, hai rovinato la mia vita! Osi negarlo? »

Il suo nemico emise un sofferto sospiro.

« Le metafore non sono il tuo forte, eh? », si lamentò. « Va bene, Anton. Proverò a spiegarmi di nuovo, con un discorso meno allusivo. Mi segui? Bene. Giacomo-Gazzi-vuole-la-mia-azienda. Il suo intento è così chiaro che sono sinceramente sorpreso dal fatto che non lo avessi capito persino tu. Se io muoio, la potrà avere; non sto a spiegarti come o perché. Ti conosce, sei rancoroso e cedi facilmente alla rabbia: lui avrà sicuramente pensato che se tu ti credi rovinato per causa mia... Andiamo, non guardarmi così! Te la sto mettendo più semplice che posso. »

Anton ringhiò. Sviare la colpa, si trattava solo di quello? Non se la sarebbe cavata così facilmente, neanche se avesse avuto ragione sul serio. Ululò la sua ira:

« Pensi che non lo sappia? Quando avrò finito con te, ammazzerò anche lui! Ammazzerò tutti! Finirò in galera, non mi importa niente, sono finito ormai! Sono un cazzo di barbone, non conto più niente per nessuno e nessuno conta più per me, non importa che cazzo mi possiate raccontare! »

Bagonghi ridacchiò.

«Ti sbagli », fece tranquillo.

« Ah si? Mi dirai anche te che non mi lascerai a morire per strada, perché sono tuo amico eccetera? Ho già sentito questa storia, proprio stasera! »

Bagonghi continuò a ridacchiare.

« No, no! Ti sbagli se credi di finire in galera. Sei una provvidenza per me, Leka: sto imbastendo delle cose per le quali mi potresti servire e tu capiti proprio al momento giusto. »

Il mondo di Anton si fermò.

« Mi... mi stai offrendo un lavoro? », chiese guardingo.

Bagonghi rise a lungo.

« No, non direi proprio », bofonchiò non appena riprese abbastanza fiato. « Non ho la più flebile intenzione di pagarti per i tuoi servigi, quindi non oserei mai chiamarlo lavoro... »

Anton capì.

« Vuoi che uccida il Gazzi per te », disse senza emozione.

Sul viso del Bagonghi si dipinse un'espressione indecifrabile.

Era passata da poco la mezzanotte quando, nella zona industriale di Prato, una vita si spense.


NOTA DELL'AUTORE

Il prossimo capitolo... eh, ragazzi miei, ve lo metto il 23 di aprile.

Mi mancano due esami per laurearmi, in ufficio sono sommerso di lavoro ed ogni tanto vorrei anche vedere la luce del sole. Ho già scritto tutto il resto del libro: la sostanza c'è, ma la forma è impresentabile. Concedetemi un mese in più e vedrete che non resterete delusi.

In cambio della vostra pazienza, vi concedo un piccolo spoiler del prossimo capitolo: si intitolerà IL REMO.

Mentre aspettate fiduciosi... potete sempre rileggere tutto e tentare di indovinare cosa succederà, no?

A presto,

Simone (o Gesucca, o Gegiùcca, chi lo sa...)




\chapter{Introductory Chapter}

\begin{chapquote}{Author's name, \textit{Source of this quote}}
``This is a quote and I don't know who said this.''
\end{chapquote}

% pulled

Una nave da crociera solcava pigramente il Mar Mediterraneo. Era una bellissima giornata, fredda ma soleggiata. Tutti i passeggeri erano intenti a godersi il viaggio, osservando la costa, il mare o dilettandosi in qualche attività sul largo ponte scoperto; tutti, tranne uno.

Fabio fissava pensieroso un minuscolo taccuino. Il vento spiegazzava incurante quelle piccole pagine, gli schizzi delle onde ogni tanto le bagnavano; lui ignorava tutto, immerso nelle parole confuse che aveva scritto.

Regola 27: le buone maniere sono gratis, abusane più che puoi.

Qualcosa non gli tornava. Pensò ancora qualche istante, poi tracciò altri segni confusi.

Regola 28: i tuoi interessi invece sono cari, molto cari; non sacrificarli in nome della cortesia.

Le cose che non gli tornavano continuarono a non tornargli, anzi, si ingarbugliarono ancora di più. L'istinto gli diceva di fondere quelle due regole in una sola, eppure le varie combinazioni che gli venivano in mente somigliavano troppo ad una citazione che aveva letto da qualche parte. Avrebbe quasi voluto riportarla per intero, ma non ricordava la formulazione esatta, tanto meno sapeva a chi attribuirla. Forse sarebbe stato meglio lasciare le due regole divise, rendendo però la ventotto un comma della ventisette\ldots

Ponderò relativamente a lungo la faccenda, poi decise di lasciar perdere. Non c'era bisogno di impazzire per questioni del genere, non quando stava già impazzendo senza fare alcun ulteriore sforzo. Non stava stilando la costituzione mondiale: non avrebbe dovuto esitare nemmeno un secondo nell'andare avanti. Gli venne un'intuizione e prese a scribacchiare.

Regola 29: esitare è facile, ma raramente è utile.

Quella non era decisamente farina del suo sacco, ma non poteva costringersi a ricordare dove la aveva letta. Obbedendo alla regola, non esitò e passò oltre. Gli venne subito un'altra idea.

Regola 30: finché qualcuno non reclama una proprietà intellettuale, è tua.

« Che scrivi adesso? » flautò una voce da un altro mondo.

La bolla di follia in cui Fabio si era isolato scoppiò, rivelando i dintorni. Comparvero una ragazza magra e bionda, il ponte di una nave ed il mare.

« Le stesse cose di stamani » sospirò Fabio, sconsolato.

Strappò l'ultima pagina del suo taccuino e se la cacciò in bocca.

« Ancora le tue regole? Non stai facendo molti progressi, vero? » tubò Vittoria.

Fabio non si disturbò a rispondere: era troppo intento a masticare i suoi scritti.

« Io non me ne intendo », mise le mani avanti lei, « ma\ldots non faresti meglio a lasciar perdere? Va bene pensare, ma figa a una certa anche basta. »

Fabio interruppe la masticazione e la fissò per qualche istante.

« No, non scriverlo, ti prego! » si affrettò ad aggiungere lei. « Dai, molla quel quadernino. È da stamattina che sei in sbatti, oggi non ti ho visto sorridere nemmeno una volta. »

La replica arrivò fin troppo in fretta:

« Vonvollio vovviee\ldots puh! Non voglio sorridere! »

« E invece sorriderai, perché io ti voglio vedere contento! Facciamo qualcosa insieme\ldots è soleggiato, il mare è bello, non fa troppo freddo\ldots facciamoci mezza goccia in due! Ci passa prima ancora di arrivare! »

Fabio rifletté per un istante. Senza pensarci, strappò un'altra pagina di scritti e fece per sbranarla; Vittoria lo bloccò repentinamente. Lui non protestò, ma rispose:

« Tu fai pure, io non credo proprio di volermi buttare in mare dalla disperazione. Non ho dimenticato quanto ho sofferto l'ultima volta. »

Vittoria si incupì.

« Però hai dimenticato quanto siamo stati bene prima. »

Fabio sospirò. Esitò per dei lunghi istanti prima di rispondere.

« Ti sbagli, non l'ho affatto dimenticato. Ma su una cosa hai ragione, devo proprio prendermi una pausa dai miei pensieri: andiamo a prendere una cosa da bere e godiamoci il viaggio. »

Il bar era ben fornito e la crociera poteva rivelarsi molto piacevole, ma Fabio non riusciva proprio a stare sereno. C'era qualcosa che non tornava nel loro viaggio; anche se non riusciva a formulare concretamente il suo dubbio, avvertiva distintamente la sensazione di stare commettendo un grave errore.

Eppure sembrava andare tutto per il meglio: all'imbarco nessuno aveva chiesto documenti, avevano preso il biglietto \- caro asserpentato, non aveva mancato di commentare Fabio - e via. Riguardo al loro intento di usare la crociera per emigrare da clandestini, il piano era così semplice da non poter andare storto: quando la nave avrebbe fatto porto per i rifornimenti, loro sarebbero scesi di soppiatto. Se qualcuno li avesse beccati, sarebbe stato messo in condizione di non parlare. Non era certo quello che lo preoccupava; aveva alle spalle esperienze così intense da rendere lo sfuggire alla sorveglianza di una nave veramente una bazzecola.

Ma questa faccenda di Leghorn semplicemente non tornava. C'era un'immagine che continuava a comparire nella testa di Fabio, un ricordo di un cartello bianco con scritto proprio quel nome. Che fosse dannato se fosse riuscito a ricordarsi dove diavolo lo aveva visto. Avrebbe giurato di non aver mai incontrato il nome di quella città in tutta la sua vita. Che poi, Leghorn\ldots Gamba-corno? Che sarebbe dovuto significare?

Il comandante della nave aveva parlato varie volte dagli altoparlanti, illustrando le varie caratteristiche delle località che stavano costeggiando, ma Fabio non aveva capito molto. Sarà stato l'inglese strascicato dell'arcigno catalano, o il rimbombo metallico che accompagnava ogni comunicazione, fatto sta che in media Fabio era riuscito ad intendere una parola su dieci. E quelle poche che era riuscito ad udire correttamente, non gli tornavano per niente.

« Brr - Bzz - Mar -sei - krrrrr »

« hhhhgggg - Naiss - ssssssrrrrr »

Marsiglia? Nizza? Che diavolo ci facevano sulle coste della Francia? Si sarebbe aspettato che gli venisse indicato Valencia, Gibilterra e poi qualche posto in Portogallo, prima che la nave facesse finalmente porto a Leghorn. Avrebbe pensato di aver capito male, da quell'impianto venivano fuori così tanti ronzii che non si sentiva effettivamente proprio niente, però, stando sulla prua della piccola nave, la costa rimaneva a sinistra. Se due indizi facevano una prova, stavano proprio costeggiando la Francia, diretti verso\ldots

Nonostante Fabio cercasse di ignorare tutto questo con tutte le sue forze, l'illuminazione si innescò quando la voce metallica - tra una robotica gracchiata e l'altra - annunciò abbastanza chiaramente che stavano costeggiando il Principato di Monaco. Dovette accettarlo: non stavano andando in Inghilterra. Ma allora, in nome del cielo, dove stavano andando? Che c'era dopo la Francia? Cosa avrebbe potuto chiamarsi Leghorn?

Era appoggiato sul bordo di prua quando finalmente lo capì, e per poco non cadde in mare. Quel cartello nella sua testa\ldots Era un ricordo vecchio di quasi vent'anni. In automobile, al porto di Olbia, lui giocava con il Game Boy, non riusciva a sconfiggere la settima palestra, suo padre che bestemmiava in cerca dell'imbarco corretto...

Non stavano affatto andando in Inghilterra. Quanto era stato stupido, era ovvio, ovvio! Non era proprio possibile andare in Inghilterra da Barcellona in così poco tempo! Come aveva fatto a convincersi del contrario, come era stato possibile ingannarsi così?

« Uuuh, guardaaaa! Io lì ci sono stata! » civettò Vittoria, anche lei abbandonata sul bordo di prua.

« Tra poco saremo in posti dove sono stato anche io » mormorò Fabio con voce funerea.

Lei lo ignorò, completamente presa dal paesaggio.

« Lì c'era l'albergo dove eravamo! Abbiamo visto il Gran Premo dalla terrazza di camera! »

Fabio bestemmiò. L'unica volta che aveva assistito dal vivo ad un gran premio di Formula 1, aveva speso una cifra che per lui era un patrimonio per stare in un prato fangoso, facendo a cazzotti per intravedere mezza curva da una rete.

« Dai, non fare così » gli disse la ragazza, appoggiandosi a lui. « Lo stiamo facendo, stiamo andando in un posto nuovo a rifarci una vita! Sei sempre\ldots triste? »

Fabio scrutò l'orizzonte. La sua espressione era così dura che sembrava scolpita nella roccia. Dopo una lunga pausa meditabonda, rispose:

« Mi dispiace fare il bastian contrario, ma ogni cosa che hai detto è sbagliata. Non stiamo andando in un posto nuovo. Non ci rifaremo una vita e no, non sono triste. »

Lei lo abbracciò. Lui non ricambiò l'abbraccio, ma neanche lo respinse.

« A me invece non dispiace, e ti dico che non mi importa niente di quello che hai detto. Tranne che non sei più triste. »

Fabio ignorò il moto di insofferenza che lo investì. Replicò secco:

« Non sono neanche felice, se ti può interessare. »

Lei lo strinse a sé ancora di più.

« Vedrai che fra poco starai meglio. Saremo in una città nuova, nessuno ci darà fastidio, saremo solo io e te. Mi prenderò cura di te così bene ti farò passare la voglia di essere cattivo. »

Fabio provò l'impulso di fingere un conato di vomito, ma lo soppresse. Riluttante, ricambiò l'abbraccio e tacque. Lo aveva scritto qualche ora prima, in uno degli innumerevoli foglietti finiti in mare o nel suo stomaco: non c'era alcun vantaggio nel ferire gratuitamente i sentimenti della ragazza. Presto avrebbe capito da sola l'errore che avevano commesso.

***

Le ore trascorsero; il sole sprofondò nel mare, lasciando spazio ad un magnifico cielo stellato. Quasi tutti i passeggeri si erano ormai rifugiati sottocoperta, l'aria era pungente. La voce metallica del comandante gracchiò:

« kkkaaa- Leee - ghorn - drrrggoff the ship for any rrreaaonzzz - brrr »

Enio Filippi dormiva sodo negli alloggi riservati all'equipaggio. Un collega lo svegliò senza tante cerimonie: toccava a lui fare rifornimento. Enio maledisse il clero, il cielo e varie divinità, ma tutto sommato si alzò di buon umore: era nella sua Livorno, e fare porto a casa dava sempre una bella sensazione.

Caricare i rifornimenti era sempre un letale mix di fatica e noia: lo sforzo fisico impediva di fare quattro chiacchiere coi colleghi, la noia costringeva la mente a vagare in cerca di qualcosa con cui intrattenerla. Quella sera poi, con l'equipaggio ridotto al minimo, le sue braccia erano le uniche su cui poteva contare per tirar dentro le casse lasciate lì sula banchina dal fornitore. Ma Enio era un uomo di sostanza, e non si sarebbe certo tirato indietro di fronte ad un compito sì ingrato, ma tutto sommato banale. Così, cuffie nelle orecchie e maniche arricciate nonostante il freddo, il suo corpo si muoveva automaticamente, come un meccanismo ben oliato che scorreva nelle guide scavate da una ferrea memoria muscolare. Immerso in quel trance lavorativo, per Enio sarebbe stato molto facile perdersi qualche dettaglio di ciò che succedeva in quella stiva buia.

Ma una bella biondina che sbirciava oltre una pila di bancali non poteva cerco passare inosservata. Trattenendo a stento un sorriso, mollò sul posto la cassa di bibite che aveva fra le braccia e le andò incontro.

« Bimba, 'sa ci fai vi? Devi tornà di sopra coll'altri! » fece Enio, tirandosi via un auricolare senza tante cerimonie.

Lei lo guardò con un'espressione indecifrabile.

« Sei sporco » gli disse sognante. « Hai un odore strano e parli in modo buffo. Sei interessante. Ora però, foeura di ball!»

Enio rimase completamente senza parole.

Mentre pensava come replicare a delle affermazioni del genere, sentì un tocco sulla spalla che lo fece trasalire; si voltò di scatto e vide una figura maschile, seminascosta dall'ombra. Enio odiava essere spaventato. Sbottò:

« Dé, ma se' scemo? Mi fai veni' un coccolone, te e quer tegame di tu ma'! »

Il figuro rise.

« Bel tentativo, Vittoria. Ma devi parlare nella sua lingua per farti capire. »

... « 'canzati, o piglio 'l remo », ordinò.

...

« Molto bene, io ti avevo avvertito. Ora piglio 'l remo. »

Punto di cista di due mozzi aggrediti.

Ma chi é sto boia?

Alba sul terrazzo mascagni, edicola, bagonghi morto.

Decidono di tornare a casa


\chapter{Uno e Nessuno}

\begin{chapquote}{Author's name, \textit{Source of this quote}}
``This is a quote and I don't know who said this.''
\end{chapquote}

% pulled

Una piccola nave era ormeggiata al porto di Livorno. La sua stiva era fredda, umida e buia. Quel poco che si vedeva era illuminato solo dalla fioca luce che penetrava dalla banchina. C'era rumore e odore di salmastro, ma i sensi più acuti avrebbero potuto individuare dei singhiozzi e un caratteristico sentore metallico. Erano in tre in quella stiva: un elfo, un nano... ah, no, quella è un'altra storia. [1] In questa c'erano un ragazzo alto, una bella biondina ed un uomo adulto che piangeva come un bambino, rannicchiato a terra nel suo sangue; se quello fosse stato l'inizio di una barzelletta, nessuno avrebbe potuto dire di averla già sentita.

Il ragazzo alto si avvicinò all'uomo, esaminando divertito la ferita da arma da fuoco che gli aveva causato.

« Dai, dai... », commentò con fare canzonatorio, « non fare così, ti ho appena regalato qualche giorno di malattia, no? »

Il giovane ridacchiò, divertito dalla sua stessa battuta.

« Farò il signore », proseguì, « ed eviterò di dirti che ti avevo avvertito. Se non mi sbaglio, ora dovresti essere disposto a prendermi sul serio. Possiamo dire quindi che siamo una squadra? Che lavoriamo tutti e tre per raggiungere il mio obiettivo? »

L'uomo non rispose; gemeva e piangeva, in preda ad un dolore terribile. Il ragazzo si rivolse allora alla sua compagna:

« Diglielo tu, Vittoria: siamo una squadra? »

Lei non sembrava a proprio agio in quella situazione.

« Sì », tentò, « ma non sarebbe meglio andarcene, tipo, non so... adesso? »

Il ragazzo rise. Una risata fredda, precisa, senza la minima parvenza di allegria. Sul suo volto si dipinse un sorriso malvagio.

« Ma come? Vuoi andare via proprio adesso? La festa è appena cominciata! »

Pestò con forza la gamba ferita dell'uomo a terra, che cacciò uno straziante urlo di dolore. Il sorriso sul suo volto vacillò appena.

« Ssshhh! » fece beffardo. « Piano, piano. Non vogliamo che ci senta qualcuno, vero? Non vogliamo altra compagnia, giusto? »

L'uomo forse intravide una speranza in quella provocazione. Prese ad urlare con tutto il fiato che aveva, dimenandosi per cercare di liberare la gamba che veniva torturata. Fu tutto inutile: in uno scatto fulmineo il ragazzo gli fu addosso, la pistola puntata in mezzo agli occhi, e lui si arrese.

« No, no! », gli fece animoso. « Questo non è carino. Credevo che noi fossimo una squadra, che non ci servisse l'aiuto di nessuno! »

Nessuno osò ribattere niente. Sulla stiva scese un silenzio totale, perturbato solo dal respiro irregolare dell'uomo a terra.

Il ragazzo alto si alzò, la sua arma sempre fissa sul bersaglio.

« Sono un tipo orgoglioso, sai? », disse a voce molto bassa. « Orgoglioso e possessivo. Immaginami come una fidanzata appiccicosa. Come reagirei se ti sentissi chiamare a gran voce una tua amica? Oh, reagirei male, molto male. Mi metterei ad urlare, ti prenderei a schiaffi. Sarei così fuori di me che potrei fare qualche pazzia. Ma non succederà, perché tu non vuoi che succeda. Noi non siamo due fidanzatini, siamo adulti. Siamo più furbi, vero? Noi lavoriamo insieme, non ci facciamo questi dispetti. »

Il silenzio tornò per qualche istante, poi la voce della ragazza bionda lo ruppe. Parlò fermamente, ma non riuscì ad occultare del tutto il profondo disagio che provava.

« Jorge, te lo devo proprio dire. C'è proprio bisogno di... ehm... perdere tempo così? Lo so che non lo faresti se non ci fosse un motivo, ma cioè, non è meglio semplicemente andarcene via? »

Il ragazzo le rivolse uno sguardo intenso, pensando a chissà cosa.

« Hai ragione » le disse finalmente. « Non posso controbattere, hai perfettamente ragione. Però è un vero peccato, non credi? Si stava creando un bel rapporto fra me e il nostro marinaio. »

Il suo sguardo si perse in lontananza, nella direzione generica dell'uomo a terra. Un qualcosa di lui sembrò improvvisamente cambiare.

« Lo sai che non posso lasciarti vivere, vero? » disse a voce estremamente bassa.

Non arrivò nessuna risposta, ma il giovane non sembrava aspettarsene una.

« Vittoria, lo sai dove siamo? » chiese alla ragazza, lo sguardo sempre fisso sul vuoto.

« In un posto dove non dovremmo essere. » fece pronta lei. « Per favore, dobbiamo andare via! »

« Sempre concreta » replicò il ragazzo, per niente turbato dal senso di urgenza della sua compagna. « Sempre presente, sempre coerente. Sei capace di astrarti, di vedere oltre, anche troppo oltre a volte, ma rimani sempre ancorata nel tuo centro. Come fai, Vittoria? È davvero la droga che ti aiuta a rimanere te stessa, a non perderti dietro le maschere che indossi? »

« Non... non lo so », rispose lei.

« Non lo so neanche io. »

Per dei lunghi istanti nessuno fiatò. Poi, all'improvviso, il ragazzo si riscosse:

« Basta, leviamoci di torno. Marinaio, tu vuoi vivere o morire? »

L'uomo a terra cercò di mettersi seduto, ma il meglio che riuscì a fare fu sorreggere il busto con le braccia.

« Bah, lascia perdere, non importa » fece sbrigativo il ragazzo alto. « È chiaro che vuoi vivere, a quanto pare tutti vogliono vivere. Beh, sei fortunato, oggi hai vinto il resto della tua vita. »

L'uomo si lasciò cadere pesantemente a terra. Prese a singhiozzare e pronunciò delle parole di ringraziamento.

« Sì, certo, prego! » sbottò il ragazzo. « Quello che mi dice grazie dopo aver preso una pallottola mi mancava. Giuro, un giorno o l'altro capirò che diavolo avete in testa voialtri! Ma ormai non mi stupite più, e forse in realtà nemmeno me ne frega un cazzo. Forza Vittoria, su! O non avevi tanta fretta? »

Senza tante altre cerimonie, i due giovani scomparvero a passo svelto nella notte.

***

Qualche ora più tardi, un arrabbiato barista in cerca di caffè scese nella stiva, trovandovi Enio Filippi incosciente e ferito. Il marinaio fu soccorso ed ebbe fortunatamente salva la vita, pur avendo perso molto sangue. Non seppe dire niente riguardo al suo aggressore; raccontò solo di aver avuto strane allucinazioni, forse causate dal dolore.

Da quella notte in poi, Enio prese a provare una strana diffidenza nei confronti dei ragazzi alti e delle giovani, attraenti ragazze bionde. Sua moglie ne fu più che felice.

NOTA DELL'AUTORE
[1] : scusatemi, non ho potuto resistere!

Il prossimo, mostruoso capitolo sarà pubblicato il 30 giugno 2018.

- Simone




\chapter{Mostri}

\begin{chapquote}{Author's name, \textit{Source of this quote}}
``This is a quote and I don't know who said this.''
\end{chapquote}

% pulled

Vittoria seguiva il suo compagno Jorge a passo spedito nella notte. I due si stavano lasciando il porto alle spalle, inoltrandosi rapidamente verso il cuore della città. Non stavano seguendo un percorso regolare, anzi, tutt'altro: era come se cercassero di disorientare eventuali pedinatori, cambiando direzione senza un'apparente senso logico.

In ogni caso, ormai Vittoria non poteva più ignorare il fatto che erano tornati in Italia. Strade e stradine, ora larghe ora fitte senza il minimo senso, serpeggiavano attraverso palazzoni costruiti con sfacciata incuria architettonica per poi sfociare in piazze piuttosto ampie, tutte quante vegliate dall'immancabile marmo raffigurante questo o quel tizio. Tutto quel meraviglioso caos, però, falliva nell'oscurare la diffusa sensazione di bellezza che quella bizzarra accozzaglia di vecchiume e dubbia antichità trasmetteva a chi vi si immergeva. Non era proprio possibile che quella fosse una cittadina anglosassone: l'Inghilterra che aveva visitato Vittoria era completamente diversa. Inoltre, le scritte in italiano sulle insegne di bar e negozi non lasciavano spazio a dubbi, rendendo superflua qualunque arguta deduzione tratta dal paesaggio urbano.

Vittoria però non aveva idea di quale potesse essere la misteriosa città nella quale si trovava. Faceva schifo in geografia, ma non era scema: dalla nave aveva riconosciuto Monaco e poi la Liguria; sotto doveva esserci la Toscana, o il Lazio, o giù di lì. In ogni caso, era abbastanza lontana dalle sue parti per poter stare tranquilla. Jorge, d'altro canto, sembrava essere molto vicino a casa sua. I suoi movimenti spediti, il modo in cui si guardava in giro in cerca di punti di riferimento, la sua strana espressione facciale quando esitava qualche attimo per orientarsi; tutto dava a Vittoria l'impressione che quei luoghi non fossero proprio sconosciuti al suo compagno. Eppure, lui aveva detto di venire da qualche posto vicino a Firenze, ma Firenze mica c'era il mare... Vittoria si inquietò nel fare questi pensieri; cercò di scacciare via la sua preoccupazione osservando distrattamente l'ambiente circostante.

Si erano appena lasciati alle spalle i loggiati di un'ampia via con molti negozi per lei interessanti - ovviamente chiusi, data l'ora - per addentrarsi nuovamente negli stretti viottoli fra palazzo e palazzo. Lo sguardo le cadde su quello che diversi giorni prima era un ratto e lo stomaco le si rivoltò, perciò cercò rapidamente di distrarsi leggendo qualunque cosa le capitasse a portata d'occhio.

« Torta... pizzeria Napoli... pizza e torta... ancora torta? », lesse ad alta voce, interrogando la sua guida. « Jorge, perché ci sono così tanti negozi di torte in questo posto? »

Lui rise a lungo; quando finalmente le rispose, non si voltò nemmeno a guardarla.

« Benvenuta a Livorno, cara mia », fece giocondo. «Se non fossero qui i tortai, dove avrebbero a essere? »

Si fermò di colpo e si girò verso di lei.

« Vuoi un cinque e cinque? », le chiese sorridente. « Non manca tanto all'alba, forse qualcuno ha già aperto. »

« Un che? », fece lei, spaventata.

« Un panino fatto con cinque lire di pane e cinque lire di torta di ceci » spiegò lui. « O comunque, era così qualcosa come un secolo fa. Ora costano di più e ci mettono anche le melanzane. »

« Torta - di ceci!? Con le melanzane!? » fece lei, disgustata. « Ma scherzi? Ma che è? »

Jorge ridacchiò di nuovo e riprese a camminare.

« Sai, la pensavo proprio come te fino a qualche anno fa. » disse, dandole le spalle. « Quando venne a trovarci quel ragazzo, lo presi per il culo tutta la sera, poverino. E invece una volta, dopo una nottata a Cecina col Bagonghi un po'... rocambolesca, diciamo così, non si riuscì a trovare niente di aperto per fare colazione tranne un maledetto tortaio. A lui gl'importava una sega, io avevo una fame bestia. Insomma, ci si fece coraggio e... boia, in tutti quegli anni quanto torto ho avuto! Comunque, non ci pensare, ho detto una cazzata. A Livorno nessuno ha voglia di lavorare, figurati se i tortai aprono all'alba! »

Vittoria rimase muta. Quel breve racconto di vita vissuta, per simpatico che fosse, aveva rinnovato le sue inquietudini, facendole sentire il suo Jorge molto lontano. Era palese che lui avesse nostalgia della sua vita precedente, qualunque essa fosse. Vittoria invece non ne aveva neanche un po', proprio per niente. E se Jorge a un certo punto avesse deciso di riabbracciare il suo passato, abbandonando lei al suo destino? La prospettiva era terrorizzante. Avrebbe dovuto investire ogni grammo della propria energia nello sforzo per tenere Jorge lontano da Firenze o da dove caspita veniva, eppure non riusciva a costringersi a farlo; un eco lontano nella sua testa gridava un terribile avvertimento che rischiava di restare inascoltato.

« Manca molto a dove vuoi andare? », chiese al suo compagno. « E, tra parentesi, dove stiamo andando? Sono tipo venti minuti che giriamo in tondo. »

« Voglio andare alla stazione, chiaramente, ma che sia dannato se mi ricordo dov'è.  Ora siamo... eh, intorno i fossi, quindi vedrai la Fortezza Nuova non deve essere tanto lontana. Insomma, c'è ancora da camminare un bel po'. Sei stanca? »

« No, era solo per sapere se abbiamo tempo prima che tu faccia del male a qualche altro sconosciuto. »

Jorge rise ancora.

«Tranquilla, voglio solo arrivare con discrezione alla stazione. Non credo che assassinerò nessuno nel tragitto, credo che sarebbe poco discreto. »

« Quindi posso prendere mezza goccia o no? »

« Puoi fare tutto quello che vuoi, sei tua. »

Stettero in silenzio per un po', ciascuno immerso nei propri pensieri, continuando a gironzolare sempre più freneticamente nel cuore di Livorno.

***

Il cielo albeggiava; con snervante lentezza, la città cominciava ad animarsi. Un ragazzo alto e una ragazza bionda erano finalmente seduti su una panchina nei pressi della stazione, aspettando pazientemente l'apertura di un'edicola; erano due fuorilegge, ma se volevano prendere il treno tanto valeva comprare i biglietti. Il ragazzo alto si lisciava distrattamente la barba, macchinando chissà cosa in quel suo diabolico cervello; la ragazza bionda fissava semplicemente il vuoto.

Gli effetti della droga erano ormai svaniti da un bel pezzo, ma la mente di Vittoria ancora viaggiava a rilento, intorpidita dagli strascichi lasciati dalla sostanza. Le immagini di ciò che aveva visto quella notte nella stiva della nave la tormentavano; non si poteva proprio definire una ragazza da scrupoli morali, ma il comportamento di Jorge le appariva indubbiamente oltre la sua zona di comfort. Non era stata tanto la crudezza della scena a turbarla, ma quanto il fatto che il suo compagno non si era semplicemente liberato di un ostacolo; lo aveva torturato, costretto in uno stato di totale prostrazione.

Le terribili sensazioni che aveva provato in quella stiva le si presentavano prepotenti e intermittenti all'attenzione: l'arto dilaniato dal colpo di pistola, la pozza di sangue che lentamente si allargava sul pavimento, quel disgustoso ma familiare odore ferroso, le urla dell'uomo, i suoi singhiozzi... Era stata una cosa terribile alla quale assistere.

Tuttavia, ciò che spaventava Vittoria più di ogni altro aspetto della vicenda non era l'inutile crudeltà con cui Jorge aveva trattato quella persona, ma la maestria con cui quel marinaio era stata manipolato nella totale obbedienza. Era precisamente quell'aspetto che suscitava in lei una pericolosa sensazione di disagio. La minaccia, le parole argute, l'illusione di salvezza immediatamente infranta, il tutto sapientemente mescolato per creare quell'orribile sensazione di sottomissione che porta ad accettare passivamente ogni sopruso per inseguire la vana speranza di aver salva la vita. Non era solo un gioco di dominanza o una banale perversione, ma una violenza psicologica disumana, messa in atto con manipolazione a regola d'arte degna di un sociopatico coi fiocchi. Secondo la sensibilità di Vittoria, forse ancora un po' amplificata dalla droga, si trattava di un crimine assai peggiore dell'assassinio: era come negare l'umanità di una persona, renderla meno di una bestia, meno di una pianta, smontarla pezzo per pezzo e ridarle una parvenza di esistenza secondo il proprio comodo e diletto. Un individuo in grado di godere di questo non poteva essere umano. Quello non era il suo Jorge, ma un mostro.

Cosa gli era successo? Perché doveva comportarsi così? La voce interiore che rispose a Vittoria prese il timbro e la cadenza del suo compagno:

"Te lo avevo detto cosa sarei diventato. L'uomo di cui ti sei innamorata non c'è più."

Non aveva senso, non poteva essere. Appena qualche mese prima quel bel ragazzo era coraggioso e intrigante; possibile che un trauma lo avesse cambiato al punto da fargli tormentare un marinaio per il solo gusto di farlo?

Una sensazione sgradevole la pervase. Il suo inconscio aveva formulato una terribile risposta, una verità così orrenda da spaventarla. Non poteva esprimerla con delle parole, non voleva rendersi conto della realtà: cercò di resistere il più possibile, ma alla fine il ricordo che cercava di ignorare irruppe prepotente nello spettro della sua attenzione cosciente. La voce di Jorge tuonò chiara e nitida nella sua memoria:

"Ascoltami bene. Questa situazione è molto pericolosa. Sia per te che per me. Non vorrei proprio ritrovarmi a fare cose che non mi conviene fare..."

"Voglio che tu capisca in che mani sei ..."

"Ora io ti mostrerò una cosa. Ti renderai finalmente conto che seguirmi è stata una pessima idea. Consideralo un favore personale che ti faccio..."

"Ti è andata bene, sei viva e non ti ho nemmeno stuprato ..."

La forza della rivelazione la investì con dirompente violenza. Usò fino all'ultimo grammo della sua forza di volontà per controllarsi; riuscì a mantenere apparentemente la calma, o per lo meno a impedirsi di urlare, ma non poté trattenersi dallo scoppiare silenziosamente a piangere.

Jorge non aveva torturato il marinaio perché era cambiato. Lo aveva torturato perché semplicemente quello era il genere di cose che Jorge faceva. Quell'uomo fragile ma risoluto non era diventato un mostro dopo averlo deciso l'altro giorno, lo era stato fin dal primo momento in cui lo aveva conosciuto. Vittoria stessa aveva subito il suo gioco: attirata in trappola, minacciata, sottomessa ed infine sedotta.

Non voleva crederci, ma nient'altro aveva senso. Lei lo amava davvero, profondamente, sinceramente: quel sentimento le sembrava genuino, non indotto. La realizzazione della verità aveva scosso brutalmente le radici della sua affezione, ma non la aveva fatta immediatamente scoppiare come una bolla di sapone. Il buono che aveva visto nel suo amato non era stata solo un'illusione, era reale; offuscato, soppresso da impulsi nocivi, ma comunque presente. Il suo Jorge... non era sicuramente Jorge. Ma chiunque egli fosse, era sempre una persona, non ancora corrotta al punto di diventare solo un demone. Era un ragazzo debole che combatteva contro la sua malattia mentale. Per quando stare con lui potesse essere pericoloso, la strada di Vittoria era segnata: lei lo amava, lo doveva aiutare.

Ma lui voleva essere aiutato?

Fabio si lisciava la barba. Strani pensieri lo affliggevano; man mano che la sua inquietudine cresceva, il suo povero pizzetto veniva attorcigliato con sempre più veemenza. Avrebbe dovuto essere finalmente libero, ma non lo era. Nonostante tutte le sue risoluzioni, tutte le regole che si era dato, si sentiva esattamente lo stesso di prima, con gli stessi fardelli da portare sulla coscienza. Solo una cosa era cambiata: solo più scosso che mai.

Non erano tanto le esperienze che aveva vissuto a tormentarlo, quanto il fatto che si trovava ad un centinaio di chilometri dalla sua vecchia vita. Non poteva impedirsi di pensare a quello che aveva lasciato alle sue spalle. Aveva avuto tanta bramosia di scappare, ma adesso, in cuor suo, tutto ciò da cui era fuggito gli mancava terribilmente. Chissà cosa era successo a Prato mentre era stato via... Come se la stava passando Bruno? Forse era riuscito a trovare altri clienti... non poteva essersi arreso, non era da lui. Denise aveva scoperto cosa era successo a Daniele? No, come avrebbe potuto? Avrebbe dovuto trovarla e dirglielo lui, era l'unico a sapere. E Lavinia... lei no, non le mancava per niente. Era finalmente caduta nelle grinfie del Gazzi? E soprattutto... aveva pensato a Fabio almeno un po'?

Sospirò in preda all'angoscia e abbandonò la testa fra le mani, lasciando perdere la barba. Chi voleva prendere in giro? Solo sé stesso, a quanto pareva; ma non ne era in grado. Non lo avrebbe ammesso nemmeno sotto tortura, ma a lui Lavinia in realtà mancava moltissimo. Non era certo solo, anzi, Vittoria era una gran bella compagnia femminile, ma... era diverso. Lavinia lo capiva, gli teneva testa, o comunque ci provava come poteva. La giovane milanese, invece, era sua. La aveva rigirata come un calzino, annichilita e posseduta come un qualunque balocco. Non la aveva mai considerata al suo pari, come una volta invece considerava la sua fidanzata.

Fabio si arrese. Avrebbe dato letteralmente qualunque cosa per strappare questi i sentimenti dal suo essere, abbandonando finalmente la sua umanità per imboccare qualunque altra strada fosse disponibile. Ci aveva provato. Lo aveva detto a Vittoria, se lo era raccontato per sé parecchie volte; in realtà, aveva sempre e solo recitato una parte. Gli veniva bene, per carità, riusciva a immaginare ed emulare molto bene una sua versione psicopatica. Ma per quanto si sforzasse, per quanto si immedesimasse, la maschera che abeva infossato era rimasta pur sempre una maschera. Forse aveva convinto Vittoria, ma non aveva ingannato se stesso.

***

L'edicola che il pensieroso duo stava fissando ormai da parecchio tempo, finalmente, aprì.

« Andiamo a comprare 'sti biglietti? », fece Vittoria con voce fin troppo vivace, dopo aver tirato su con il naso.

Fabio non rispose. Il suo volto era deformato da un'emozione indecifrabile, lo sguardo era fisso sulle locandine che l'edicolante aveva appena messo fuori. Si alzò lentamente e si diresse verso l'edicola in uno strano trance, facendosi quasi investire da un marocchino in bicicletta. Entrò dentro l'edicola e, senza dire una sola parola, afferrò una copia de Il Tirreno di Prato.

OMICIDIO BAGONGHI, SCOMPARSO EX DIPENDENTE SOSPETTATO

Custodia cautelativa per l'ex dipendente albanese, che però risulta irreperibile. Mistero sull'autopsia della salma di Bagonghi, disposta e poi revocata.

I fatti che riguardano l'assassinio di Bruno Bagonghi, l'imprenditore proprietario dello storico Lanificio Bagonghi trovato senza vita la settimana scorsa, sprofondano sempre più nel mistero. Il Pubblico Ministero ha finalmente disposto la custodia cautelativa per Anton Leka, l'uomo di nazionalità albanese sospettato fin da subito per l'omicidio, ma egli risulta irreperibile: si teme che abbia già messo in atto una fuga. Ci si chiede come mai non sia stato immediatamente emesso un mandato di cattura appena verificata l'irreperibilita di Leka, ma questa è solo l'ennesima domanda da aggiungere alla...

Solo una voce stizzita riuscì a riscuotere Fabio dallo shock:

« Dé, ma lo 'ompri 'sto giornale o no? Un so' mi'a una biblioteca! »

Il prossimo capitolo, sarà pubblicato il 31 luglio 2018. Sarà corto, fatevelo bastare perché poi vo a i'mmare!

- Simone




\chapter{La Città dello Sconforto}

\begin{chapquote}{Author's name, \textit{Source of this quote}}
``This is a quote and I don't know who said this.''
\end{chapquote}

% pulled

Hai scagliato te stesso così in alto, – ma ogni pietra scagliata deve cadere! Condannato a te stesso, alla lapidazione di te stesso: o Zarathustra, è vero: tu scagliasti la pietra lontano, – ma essa ricadrà su di te!

Friedrich Nietzsche - Così parlò Zarathustra


L'aria era densa, permea d'acqua. Molte persone silenziose erano riunite attorno ad una bara chiusa. L'atmosfera rispecchiava il diffuso sentimento di tristezza: perfino il tempo piangeva il buon Bruno Bagonghi.

Una coppia di merli osservava discreta la cerimonia funebre, forse in attesa che il piccolo cimitero di Chiesanuova si svuotasse. Una ragazza li notò. La irritavano, aspettavano che tutto finisse per poter cacciare i lombrichi sulla terra smossa, non gli importava niente di Bruno. Fece come per scacciarli, ma poi si bloccò: a che sarebbe servito? Erano merli, anche loro avevano bisogno di mangiare. Riprese a fissare la bara, immersa nella tristezza più profonda di tutta la sua vita.

A pochi chilometri di distanza, due loschi figuri fissavano un cartello che recitava 'PRATO'. Come la coppia di merli del cimitero, anche loro sembravano attendere qualcosa.

Cominciò finalmente a piovere. Radi, pesanti goccioloni presero a precipitare dal plumbeo cielo. 

« Eccoci, Vittoria », annunciò il figuro più alto. « Il cerchio si chiude... »

« Jorge... non mi prendere per matta, forse è la pioggia che mi fa ansia, ma... avverto una sensazione strana », disse l'altra figura. « Sei sempre sicuro di volerci tornare? »

Non si udì risposta. Incurante dell'acqua che cadeva dal cielo, l'uomo che un tempo era Fabio Fontanelli si tolse il cappuccio. Sul suo volto era scolpita una maschera di dura determinazione.

« Bruno... », sussurrò al vento.

Aveva fatto una promessa. Doveva mantenerla, a qualunque costo.

« Prato, città dello sconforto! », gridò al cartello.

Fece un passo e mise finalmente piede sul suolo pratese.

Senza alcun evidente nesso di causa ed effetto, quel passo innescò un'incredibile serie di eventi. L'intensità della pioggia aumentò vigorosamente: una fragorosa bufera prese a infrangersi su tutta la città. Nel piccolo cimitero dove si commemorava il Bagonghi, la folla si disperse. Anche la coppia di merli cercò riparo, rimandando il suo banchetto di lombrichi a dopo la pioggia. 

Solo una ragazza rimase impietrita dove era, come paralizzata dalla pressione dell'aria. Prima ancora che avvenisse, avvertì quello che stava per succedere.

Un enorme lampo squarciò il cielo plumbeo e si infranse sulla campana in vetta al campanile della cappella. Il suono che produsse fu spaventoso: pareva il ruggito della morte stessa. Un vento formidabile sferzò l'aria, mulinando l'acqua che cadeva martellante. La ragazza sorrise; un sorriso folle, disperato. La sensazione che la aveva paralizzata ora la possedeva: aprì le braccia e scoppiò a ridere. Forse fu il vento che fischiava, o forse lo scrosciare dell'acqua per terra, ma nella sua testa la sentì chiaramente, un timbro vocale che mai avrebbe dimenticato ma che non sentiva da tanto tempo: "Prato, città dello sconforto...".

Corroborata da una strana sensazione, convinta che tutto quello che stava succedendo non fosse casuale, completò la formula che si stava realizzando tutta insieme, urlandola al cielo:

« O piove, o tira vento, o sòna a morto! »

E per diverse ore piovve e tirò vento, come per lavar via una folle speranza durata un istante.


Il prossimo capitolo ve lo beccherete... eh, prima o poi. Non temete: non resterete a bocca asciutta per molto. Buone vacanze a chi le fa!

- Simone




\chapter{Domande}

% pulled

Una ragazza si allontanava dal cimitero a grandi passi, maledicendo gli dei e la pioggia. Non c'erano dubbi, Denise avrebbe proprio preferito starsene a casa. D'altronde, del povero Bagonghi in realtà a lei importava il giusto e quanto a lui, beh, tanto era morto, che lei ci fosse stata o meno non si sarebbe certo lamentato.

Si rimproverò mentalmente. Lei non era una persona religiosa - aveva la certezza che nessuno la avrebbe mai giudicata per i suoi pensieri, fintanto che fossero rimasti nella sua testa - ma fare un discorso del genere nei riguardi di un suo amico la faceva sentire comunque un po' a disagio. Nonostante questo, non poteva farci niente: provava una discreta quantità di ingiustificata insofferenza verso il Bagonghi, anche se quel povero cristo non le aveva fatto niente di male. Era qualche settimana che si sentiva apatica verso il mondo, e non si era stupita quando non aveva provato granché nell'apprendere la notizia della morte di un suo amico.

Aveva sperato che il funerale le avrebbe fatto rendere conto della situazione: disprezzava la religione e si definiva atea, ma capiva e temeva l'aspetto mistico della ritualità, cristiana o occulta che fosse; conosceva la presa che semplici gesti e parole potevano avere sull'anima, sapeva cogliere il valore di quello che agli occhi di un cinico poteva sembrare solo un ridicolo teatrino. Nonostante credesse questo, Denise era rimasta completamente indifferente tutto il rituale, finendo per prendersi una messa per niente: le parole dette nell'omelia erano state solo ciance di un prete, dell'offerta di fiori sul feretro aveva visto soltanto della gente che poggiava cose costose su una scatola di legno, ed alla sepoltura nemmeno si era data pena di assistere. La potenza del rito funebre, semplicemente, non era bastata a perforare la sua ferrea barriera di apprensione.

In realtà, non si aspettava davvero che uno stupido funerale le restituisse la capacità di accorgersi del mondo. L'unico, vero motivo che la aveva spinta a presentarsi lì era stato, ovviamente, il desiderio di ritrovare Daniele. Ormai era passato un bel pezzo dall'ultima volta che lo aveva visto, ma non era preoccupata. Non era raro per lei vederlo assentarsi dalla sua vita addirittura per dei mesi. In fondo, la loro relazione si reggeva sull'assoluta libertà; chi era lei per impedirgli di andare dove più gli aggradava? Tuttavia, ci avrebbe tenuto molto ad incontrarlo perché, semplicemente, le mancava più di quanto non le mancasse chiunque altro. Fabio era scomparso e Lavinia si era isolata, Daniele si era reso irrintracciabile, il Cipher aveva chiuso ed il vecchio gruppo di metallari si era disperso, ora pareva che il Leka avesse addirittura ammazzato il Bagonghi... nel giro di tre mesi, aveva visto dissolversi gran parte di quella che era la sua vita; era rimasto solo il Gazzi, che rompeva ancora le scatole come se nulla fosse successo. Di tutto questo, forse le importava mezzo accidente di Fabio e l'altro mezzo sicuramente di Daniele. 

Così era finita lì, a cercare il suo ragazzo a un funerale. Lui aveva un rapporto del tutto personale con il culto dei morti, era probabile che si sarebbe presentato a dare l'ultimo saluto ad un amico. Eppure non c'era.

Immersa in questi pensieri e frastornata dal rumore della pioggia che batteva incessante sul suo ombrello, Denise non si accorse che due individui la stavano seguendo. Improvvisamente, uno di loro le si parò davanti con un balzo. Era alto, snello e totalmente avvolto in un impermeabile scuro. Il suo volto era nascosto dal cappuccio, ma aveva una mano in bella mostra ed esibiva una pistola.

Subito lo spavento la avvolse:

« Non ho niente, ti giuro! Non ho niente! », esclamò agitando il braccio, come per evidenziare la mancanza di una borsa con degli averi.

Quello non si scompose.

« Dei tuoi giuramenti non ho mai saputo che farne. Voglio risposte », disse con una voce maschile piuttosto familiare.

Denise rimase completamente spiazzata.

« Non so nulla! », fece istintivamente.

« Già », concordò la figura alta. « Non hai mai saputo nulla. Eppure ascolti, eppure a volte parli. »

Senza aspettare replica, l'aggressore si girò verso l'altro ceffo incappucciato.

« Credo che Govidi sia quello con l'ombrello argento. Vai », disse seccamente.

Rimasero in due. Denise non ci stava capendo niente. Quella non sembrava proprio una rapina, e poi... aveva detto Govidi?

« Chi sei? », tentò.

Lui rise per un lungo momento.

« Non hai bisogno di chiedermelo », rispose finalmente. « Puoi arrivarci da sola. Non sei stupida, Denise. »

Lei capì subito. Quel ceffo conosceva il Govidi, sapeva il suo nome ed era troppo alto per essere Daniele.

« Fabio... » mormorò lei. Il suono di quel nome si perse nel fracasso della pioggia.

Una viva speranza si accese prepotente nell'animo di Denise.

« Fabio! » esclamò con vigore.

Dal figuro incappucciato non arrivò nessuna risposta. Con deliberata lentezza, quello le si avvicinò e le puntò la pistola dritta in mezzo agli occhi. Lei rimase immobile come paralizzata, ma non indietreggiò di un millimetro. Non si era mai tirata indietro nelle situazioni intense, e se quello era davvero il Fontanelli, qualunque fosse il gioco a cui stava giocando, non le avrebbe fatto niente; ma soprattutto, forse le avrebbe saputo dire che fine aveva fatto il suo Daniele.

« Sei tornato per Bruno », gli disse.

« Preferisco fare io le domande », replicò lui.

« La mia non era una domanda », ribatté lei, secca e decisa.

Lui ridacchiò, ma non disse altro.

« Se vuoi una domanda per avere l'ultima parola », lo incalzò, « te la servo: dove hai preso questo affare? Me lo levi dal viso, per favore? »

Ci fu un istante di pericoloso silenzio. Fu il misterioso individuo a romperlo:

« Sei sempre stata così, Denise. Sottile, arguta quanto basta, ma soprattutto dura e testarda. Non hai paura che ti uccida? »

Lei lanciò il suo sguardo più intenso dentro a quell'oscuro cappuccio.

« Noi siamo amici », gli disse duramente. « Puoi dire, pensare o fare quello che ti pare - e che mi venga un colpo se so che cazzo stai facendo - ma se sei il Fontanelli che conosco non farai mai del male a una tua amica. »

Un rabbioso sibilo filtrò dal cappuccio:

« Fabio Fontanelli è morto! »

La replica di Denise fu tagliente:

« Eppure è proprio qui, davanti a me. »

Con la stessa deliberata lentezza con cui il figuro incappucciato le aveva puntato contro la pistola, Denise abbassò la mano armata del suo aggressore e gli tolse il cappuccio.

Sorrise nel guardare chi aveva di fronte. Certo un po' diverso, forse sciupato, con quelle cicatrici sul naso e sulle orecchie; più oscuro, in un certo senso. In ogni caso, non c'erano dubbi che dietro quello sguardo glaciale ci fosse lo stesso ragazzo che conosceva fin dai tempi della scuola, nella stessa versione angosciata che aveva visto una certa sera di diverso tempo fa.

I due vecchi amici si guardarono per qualche istante in silenzio, pensando forse a quanto le loro anime fossero simili, ma opposte, speculari.

Per Denise, Fabio era sempre stato una sorta di riflesso che negli anni le aveva mostrato che cosa sarebbe potuta diventare se avesse avuto più talento, ma anche quanto in basso avrebbe potuto scendere se avesse costantemente seguito l'impulso ad essere sé stessa. Lui le piaceva, ovviamente, ma l'idea che loro due potessero formare una coppia romantica non l'aveva mai sfiorata. Denise era una persona sì ingombrante, ma con dei confini; il Fontanelli invece non aveva proprio limiti. Lui stesso, in una qualche nottata di stupidaggini, aveva entusiasmato tutti annunciando di non essere solido, ma gassoso: ed aveva senso, perché Fabio sentiva la pressione dei suoi impulsi e, per abbassarla, si espandeva fin quanto poteva, invadendo ogni centimetro di spazio metaforico che riusciva a raggiungere. Le era chiaro come il sole che loro due avrebbero lottato in eterno per la supremazia, così come le era cristallino che lei avrebbe sempre perso. E siccome non avrebbe mai accettato di fare da comprimaria nelle dinamiche di coppia, era ovvio che la cosa non avrebbe potuto semplicemente funzionare. Comunque, non aveva mai soppresso il desiderio irrazionale di avere quel ragazzo vicino il più possibile, per bearsi dell'aura di puro potere che emanava il suo semplice esistere.

Se c'era una cosa che Denise non aveva mai capito, era chi o cosa fosse riuscito a rinchiudere il suo amico nella gabbia di depressione in cui lo aveva visto prigioniero prima che partisse, e nella quale probabilmente si trovava ancora. Proprio lei, che era stata vicina al suo Daniele nei momenti peggiori e ormai chiamava per nome i vari mostri che potevano piegare l'anima di una personcina semplice come Daniele, non riusciva a pensare a niente che avrebbe potuto anche solo scalfire Fabio Fontanelli. Per tutto il bene che potesse volere a Lavinia - poco, in realtà - non poteva proprio esistere che fosse stata lei la causa del sua profonda tristezza di Fabio. Proprio lei, una ragazzetta fra le più semplici nel mondo, non poteva aver pesato così tanto nel bilancio del'umore di Fabio Fontanelli. Proprio lui, che con un distratto colpo d'arguzia avrebbe potuto spazzar via ogni pensiero negativo, che con un distratto balzo di ironia avrebbe potuto scavalcare tutte le situazioni avverse. Denise ne era sicura: non poteva essere stato solo un amore andato male a sconfiggerlo.

Se quelli erano i pensieri che affollavano la mente di Denise, Fabio invece non pensava proprio niente. Abbandonandosi al suo istinto, una verità gli uscì di bocca con tutta la sua spaventosa gravità:

« Daniele è morto. »

Uno scenografico fulmine squarciò il cielo, come a fare da cornice alla rivelazione. Se non ci fosse stato il fragore del tuono, si sarebbe distintamente udito il rumore di una persona che andava in frantumi.

« È stata colpa mia », confessò Fabio.

Denise non parve reagire. Rimase immobile, il volto contratto in un'espressione indecifrabile. Fabio non parve attendere risposta:

« Mi rendo conto che non è un buon momento, ma devo farti delle domande. E mi dovrai rispondere. »

« Sei qui per Bruno, si capisce », squittì lei con un filo di voce.

Fabio annuì. Senza il minimo pensiero, si avvicinò a Denise e la abbracciò.

« Andiamo da qualche parte », le sussurrò mentre lei versava un fiume di lacrime sulla sua spalla. « Abbiamo tante cose da dirci. »

\chapter{Risposte}

% wip

Vittoria si faceva largo incappucciata in una piccola foresta di ombrelli che sciamava verso il parcheggio del piccolo cimitero di Chiesanuova.

« Signore! Signoreee! », chiamò a gran voce, cercando di sovrastare il rumore della pioggia.

« Quell'uomo con l'ombrello grigio chiaro! Dico a lei, signore! SIGN - figa, però, eh! Mica posso urlare! »

Se solo si fosse ricordata come Jorge aveva detto che si chiamava, lo avrebbe chiamto per nome. Si fece largo fra la gente con poca grazia e agguantò per una spalla l'uomo per cui si era sgolata.

« Scusi, eh! » gli fece con un tono di voce che poco si addiceva a delle scuse. « Sono due ore che la chiamo! »

Quello non si scompose affatto. Era un uomo molto alto e snello, dall'età indecifrabile. Vittoria si sentì un po' a disagio al suo cospetto.

« Le chiedo chiedo scusa », disse lui a voce alta ma tranquilla. « Fra le persone e la pioggia, non l'ho proprio sentita. Ah, ma non ha l'ombrello, è completamente bagnata! Venga, venga sotto il mio, ci stiamo in due, la accompagno alla sua automobile. Ci conosciamo? »

« Ehm... diciamo che la conosco di parola », fece Vittoria, mentre un po' in imbarazzo cominciava a farsi largo in lei.

« Mi conosce di parola... » le fece eco quell'uomo, squadrandola con evidente sospetto.

Passò qualche istante di imbarazzante silenzio, durante il quale Vittoria si accorse che approcciare uno sconosciuto in quel modo forse non era stata poi una grande idea e decise che, se fosse sopravvissuta alla situazione, avrebbe preso Jorge e gli avrebbe ficcato l'ombrello di quel tizio dritto nel...

« No, mi arrendo », sospirò finalmente il figuro. « Così non riesco proprio ad indovinare. Mi dia qualche indizio, la prego. »

Il disagio di Vittoria diminuì appena.

« Facciamo così », tentò. « Io le dico tutto quello che so, ma lei non mi uccide. »

L'uomo rise in un modo assai strano; si poteva capire che era una risata sincera, ma la compostezza con cui rideva era così impeccabile da far pensare che facesse finta.

« Oh, cielo », fece non appena riprese fiato. « Mi scusi, ma la sua battuta mi ha fatto proprio ridere. »

Vittoria si sciolse un po'.

« Mi scusi lei, ma visto che è un amico di Jorge, pensavo che la tendenza fosse questa »

L'uomo smise di camminare di colpo.

« Mi dica subito chi diamine è lei », fece secco, ma tranquillo.

Il cuore di Vittoriale saltò in gola.

« Io... Ecco, cioè, lei non mi conosce, che... che le devo dire? Mi manda Jorge, dice che è un suo amico, vuole sapere delle cose, ma... ma io che ne so! »

L'uomo giardò intensamente attraverso il suo cappuccio, incrociando direttamente il suo sguardo. Sembrava che stesse soppesando una decisione.

« Mi descriva il suo mandante, per favore. », disse dopo un po'.

« E'... è alto, non come lei ma comunque alto, e... »

« E' Fontanelli. » fece deciso a se stesso, interrompendo Vittoria e lasciandola interdetta.

« Lei non ha un mezzo, vero? » proseguì cordiale. « Venga con me, così mi racconta tutto. Non la uccido, stia tranquilla, quando avremo finito di informarci a vicenda la lascerò andare viva e vegeta. » rise sottovoce.

Vittoria acconsentì, più a disagio che mai. Cercò comunque di non perdersi di spirito.

« Può prestarmi il suo ombrello, dopo? Avrei da farci una certa cosa »

\chapter{eeh}

% pulled

Forse il destino, forse solo un curioso guizzo del caso, forse un programma ben preciso di un qualche narratore onniscente, fatto sta che i fili dell'ordito di una brutta vicenda stavano finalmente intrecciandosi con la trama della triste città di Prato.

Lo spazio era bizzarro, un cosmico pulviscolo oscuro dentro il quale Fabio fluttuava senza alcun controllo sui suoi movimenti. Ne sentiva l'odore, così terribile a al tempo stesso così seducente, ne avvertiva la presenza al tatto, come se quei colori freddi e sinistri lo potessero toccare. Fabio lo sapeva di aver perso il cervello, ma la conoscenza non aveva potere quel luogo misterioso: lui era quello che provava, e provava ciò che era diventato.

Senza preavviso, priva di qualunque motivo per essersi manifestata, una voce lo raggiunse: era tagliente, acuta e gelida, ma aveva un timbro estremamente familiare.

...lo designerà come suo eguale, ma egli avrà un potere a lui sconosciuto...
e l'uno dovrà morire per mano dell'altro, perché nessuno dei due può vivere se l'altro sopravvive...

La voce di Lavinia tacque, ma gli echi di quelle rimbombanti frasi rimbalzarono a lungo fra la gabbia di follia in cui Fabio era rinchiuso. Era chiaro, fin troppo chiaro a chi si riferisse quella profezia. Un senso di determinazione lo pervase, un vivo incendio che alimentava la sua ira; totalmente in balia di quell'ardente sensazione, si sentì prendere letteralmente fuoco, fino a diventare tutt'uno con le fiamme.

Appena il fuoco svanì, Fabio lo vide. Più bianco di un teschio, con grandi, lividi, occhi rossi, il naso piatto come quello di un serpente, due fessure per narici: privato dalle sue maschere, quello era il vero aspetto della sua nemesi. 

« Sei stato uno sciocco a venire qui stanotte, Giacomo », gli dsse, ostentando una tranquillità che non aveva. « Gli Auror stanno per arrivare. »

Egli parlò, la voce acuta, distorta, ma inconfondibile:

« Per allora me ne sarò già andato, e tu sarai morto »



g: voglio combattere! io devo vendicare il mio amico Junior

G: andatevene subito! è me che vuole, non voi! Se restate qui farete la stessa fine!

K: non possiamo farlo
g: io rimango, papà!

F: sei un illuso, che cosa ti fa pensare che li lascerò andare via? Io li voglio eliminare tutti quanti!

g: oh no, krilin

F: sei così ridicolo, moscerino

Dart Fener: Sei battuto. È inutile resistere. Non lasciarti distruggere come fece Obi-Wan. Non hai scampo. Non lasciare che ti distrugga. Luke, tu non ti rendi ancora conto della tua importanza. Hai solo cominciato a scoprire il tuo potere. Vieni con me e io completerò il tuo addestramento. Unendo le nostre forze possiamo mettere fine a questo conflitto distruttivo e riportare l'ordine nella galassia.
Luke Skywalker: Non verrò mai con te!
Dart Fener: Se tu solo conoscessi il potere del lato oscuro. Obi-Wan non ti ha mai detto cosa accadde a tuo padre!
Luke: Mi ha detto abbastanza: che sei stato tu ad ucciderlo!
Dart Fener: No, io sono tuo padre![1]
Luke: No! Non è vero! Non è possibile!!
Dart Fener: Cerca dentro di te! Tu sai che è vero!
Luke: Nooo!! Noo!!
Dart Fener: Luke, tu puoi distruggere l'Imperatore. Lui lo ha previsto. Questo è il tuo destino. Unisciti a me e insieme potremo governare la galassia, come padre e figlio. Vieni con me. È l'unica strada.
Lo spazio era bizzarro, un cosmico pulviscolo oscuro dentro il quale Fabio fluttuava senza alcun controllo sui suoi movimenti. Sentiva l'odore di quel nulla, così terribile a al tempo stesso così seducente; ne avvertiva la presenza al tatto, come se quei colori freddi e sinistri lo potessero toccare. Fabio lo sapeva di aver perso il cervello, ma la conoscenza non aveva potere quel luogo misterioso: lui era quello che provava, e provava ciò che era diventato.

Senza preavviso, priva di qualunque motivo per essersi manifestata, una voce lo raggiunse: era tagliente, acuta e gelida, ma aveva un timbro estremamente familiare.

...lo designerà come suo eguale, ma egli avrà un potere a lui sconosciuto...
e l'uno dovrà morire per mano dell'altro, perché nessuno dei due può vivere se l'altro sopravvive...

La voce di Lavinia tacque, ma gli echi di quelle rimbombanti frasi rimbalzarono a lungo fra la gabbia di follia in cui Fabio era rinchiuso. Era chiaro, fin troppo chiaro a chi si riferisse quella profezia. Un senso di determinazione lo pervase, un vivo incendio alimentava la sua ira; totalmente in balia di quell'ardente sensazione, si sentì prendere letteralmente fuoco, fino a diventare tutt'uno con le fiamme.

Appena il fuoco svanì, Fabio lo vide. Più bianco di un teschio, con grandi, lividi, occhi rossi, il naso piatto come quello di un serpente, due fessure per narici: privato dalle sue maschere, quello era il vero aspetto della sua nemesi.

« Sei stato uno sciocco a venire qui stanotte, Giacomo », gli disse, ostentando una tranquillità che non aveva.

Egli parlò, la voce acuta, distorta, ma inconfondibile:

« Per allora me ne sarò già andato, e tu sarai morto. »

Successe qualcosa di terribilmente strano, eppure Fabio non ne fu stupito: in qualche senso, se lo aspettava. Lo scheletrico Giacomo Gazzi pronunciò delle parole arcane, e una forza oscura esplose verso di lui. Fabio schivò con sdegnosa eleganza il pericoloso raggio verde, poi, estratto repentinamente un lungo e nodoso bastoncino dalla tasca, rispose al fuoco nemico con un ardente lampo di colore rosso acceso.

« Non vuoi uccidermi, Fontanelli? » gli gridò Gazzi, gli occhi scarlatti socchiusi. « Sei superiore a tanta brutalità, vero? »

« Sappiamo entrambi che ci sono altri modi per distruggere un uomo, Giacomo » replicò Fabio con amarezza. La sua maschera di perfetta tranquillità fu incrinata appena da una violenta ondata di rancore. « Ammetto che non mi darebbe soddisfazione toglierti soltanto la vita… »

All'improvviso, un raggio color cremisi balenò nell'aria, e il suo nemico cadde pesantemente a terra, colpito. Questo, Fabio proprio non se lo era aspettato. Si voltò di scatto terrorizzato e, sopra una roccia che prima non c'era, vide qualcosa di completamente folle: un uomo basso, totalmente nudo e senza sesso, la pelle perlacea e liscia, con la testa e le spalle di uno sgradevole colore. Bruno Bagonghi non aveva mai avuto quell'aspetto, ma quella orribile figura era inconfondibilmente una sua incarnazione.

Qualcuno accanto a lui parlò:

« Voglio combattere! Devo vendicare il mio amico Giacomo! »

Era Daniele. Aveva un vestito strano, era più basso e aveva un ridicolo raglio di capelli, ma non c'erano dubbi che fosse lui. Lo stava guardando con aria sconvolta, ma determinata.

Fabio provò una travolgente sensazione di rabbia. Rivolse lo sguardo verso quella disgustosa cosa che era Bruno, e con tutta la grinta che riuscì a trovare, ordinò al suo amico:

« Vattene subito! È me che vuole, non te! Se resti qui, farai la sua stessa fine! »

La pallida imitazione di Bruno Bagonghi parlò con la voce melliflua, leggera, ma intrisa di malvagità:

« Sei un illuso! Che cosa ti fa pensare che lo lascerò andare via? »

Un gelido sorriso, una risata e un semplice gesto con un dito: Daniele fu sollevato da terra, totalmente in balia della volontà del suo aggressore.

« Sei così ridicolo, moscerino! »

Fabio non poté fare niente: il suo amico urlò e si contorse, poi, semplicemente, esplose.

Fabio rimase immobile, totalmente privo della forza necessaria per comprendere quello che stava accadendo. Perché proprio Daniele? Era indifeso e debole, non era una minaccia da eliminare! Di tutto il male che il pallido Bagonghi aveva fatto, di tutte le cose meschine e crudeli di cui si era reso responsabile, quella proprio Fabio non poteva comprenderla. Una terribile rabbia si impadronì di lui, una furia devastante dalla sua schiena fluiva in ogni anfratto del suo corpo. Bagonghi non la avrebbe passata liscia.

Fabio estrasse nuovamente il suo bastoncino - che per qualche ragione era diventato un cilindretto metallico - e si scagliò verso il Bagonghi, che aveva abbandonato il suo pallore in favore di una forma mascherata e ammantata di nero. Carico di determinazione, Fabio si trovò a brandire una verde spada luminosa contro quello che era sempre stato il suo perfetto opposto, il suo acerrimo nemico, tutto quello che non sarebbe mai diventato fintanto che si fosse voluto bene. Egli rispose all'attacco con un'arma simile alla sua, ma di colore rosso; le due spade lucenti collisero in una tempesta di violente scintille, e i due si trovarono a combattere un duello senza esclusione di colpi su una stretta passerella sospesa sul vuoto.

Per quanto Fabio si potesse impegnare, non c'era niente da fare: il suo nemico era più forte. Bastò un attimo di incertezza, uno spiraglio aperto nella sua guardia, e un fendente gli tagliò di netto una mano. Un dolore terribile lo straziò: cadde a terra urlando, aspettando la sua fine.

La nera figura parlò:

« Sei battuto. È inutile resistere », sentenziò, la voce offuscata dalla maschera che indossava. « Non hai scampo. Non lasciarti distruggere come ha fatto Daniele. Fabio, tu non ti rendi ancora conto della tua importanza! Hai solo cominciato a scoprire il tuo potere. Vieni con me e io completerò il tuo addestramento. Unendo le nostre forze possiamo mettere fine a questo conflitto distruttivo e riportare l'ordine a Prato! »

Fabio fu colpito da quelle parole, ma non esitò un istante a strillare la giusta risposta:

« Non verrò mai con te! »

La figura oscura non si scompose.

« Se tu solo conoscessi il potere del lato oscuro... » disse, lasciando che una certa dolcezza filtrasse attraverso la sua  maschera. « Giacomo non ti ha mai detto cosa accadde a suo padre? »

Fabio rimase un attimo interdetto. Qualcosa era andato in frantumi.

« A... suo padre? » chiese confuso. « A mio padre, semmai! »

Fabio non capiva. Per qualche ragione, credeva di sapere già dove doveva andare a parare quella conversazione, ma quell'ultima battuta era sbagliata. Eppure, sembrava esserci tutto: la passerella sul vuoto, le spade laser, il tizio mascherato di nero...

« Scusa - scusa una cosa... » tentò, guardingo. « Non è che mi puoi ripetere l'ultima cosa che hai detto? »

Dopo un lungo istante di silenzio, dalla maschera uscirono delle parole:

« Non lasciare che ti distrugga. »

Fabio si sentiva sempre più confuso, Si grattò la testa con la mano destra, ma poi si ricordò che la aveva persa pochi momenti prima. Qualcosa decisamente non tornava.

« Non lasciare che ti distrugga come ha fatto Giacomo », continuò imperterrita la figura ammantata.

« No, non quella cosa. Ascolta, hai detto qualcosa sul padre di Giacomo, ma...  »

« Non lasciarti distruggere come ha fatto Anton. »

« No, no! » fece Fabio, frustrato. « Non questa, quella sul padr - aspetta, hai detto Anton? Hai distrutto Anton? Che significa? »

« Giacomo non ti ha mai detto cosa accadde a suo padre? »

« Ooh! » esclamò trionfante Fabio. « Quello volevo risentire! Ascolta, mi sa che hai confuso un po' il copione, dovresti alludere al fatto che tu sei mio padre, no? »

Con snervante lentezza, il figuro oscuro si tolse la sua maschera; il volto di Bruno Bagonghi era invecchiato e impallidito, ma perfettamente riconoscibile.

Esibì uno strano sorriso condiscendente, e parlò:

« Tuo padre? Guarda che abbiamo più o meno la stessa età, coglione! »

In quel preciso istante, Fabio seppe di essere pazzo. La sua sospensione di incredulità perse finalmente ogni appiglio sul suo essere, e lui capì di non capire proprio più niente. Scoppiò a ridere, senza freni, senza inibizioni, senza il minimo ritegno.

Delle parole arrivarono dal Bagonghi:

« Cerca dentro di te; tu sai che è vero. »

Fabio rise ancora più forte. Si accasciò a terra, impotente, e si sentì improvvisamente cadere. Qualcosa di terribile gli stava accadendo. Serrò istintivamente gli occhi, ma era impossibile proteggersi dalle forme dai bizzarri colori che gli danzavano intorno.

Ebbe la fortuna di riuscire a utilizzare con successo il briciolo di senno che gli era rimasto: capì che cosa gli stava succedendo, e si arrese. Impotente e volutamente passivo, si abbandonò al vorticare del maelstrom di follia in cui si trovava. Passarono quelli che a Fabio sembrarono giorni: successero ancora cose e comparvero altre persone ma, alla fine, quella bolla di finzione solipsistica si dissolse, e la realtà coninciò lentamente a riacquistare la sua usuale solidità.

\chapter{Trama e Ordito}

% pulled

Forse il destino, forse solo un curioso guizzo del caso, forse un programma ben preciso di un qualche narratore onniscente, fatto sta che i fili dell'ordito di una brutta vicenda stavano finalmente intrecciandosi con la trama della triste città di Prato.

Parte 1

Manca poco all'alba, ma il cielo è sempre nero
Fabio si accorge di essere cosciente, ma è rincoglionito
Caracolla verso la tomba del Bagonghi e riflette: non gli sembra che l'esperienza psichedelica lo abbia aiutato a migliorare il rapporto che ha con il dolore
Si accorge che ci sono dei sigari e si ricorda della promessa al bagonghi
Gli manca da accendere, provvidenzialmente arriva un tipo incappucciato; Fabio crede sia il guardiano notturno
I due fumano un sigaro insieme, e hanno una conversazione molto vaga
Il becchino fa leva sui sentimenti negativi di Fabio per aizzare il suo odio verso il Gazzi e Lavinia, colpevoli di averlo fatto stare male
Il becchino se ne va, Vittoria e Denise si riprendono; è quasi l'alba, primi chiarori
I tre parlano di come si fa ad essere sé stessi, e di come alla fine, l'unica via per essere felici è di accettare il dolore e fare ciò che è giusto, a prescindere dal costo


Parte 2

Finalmente il sole sorge; il cimitero apre, ed arriva Lavinia
Fabio e Lavinia si fronteggiano, molta tensione, conversazione criptica credendo di capirsi a vicenda
Arriva Gazzi, preoccupato per Lavinia
Denise impedisce a Fabio di ucciderlo, suggerendogli di accettare la realtà e andare avanti
Lavinia si arrabbia
Fabio minaccia Gazzi, che vuota il sacco: ha sparso voci per mettere zizzania fra Fabio e Lavinia ed ha manipolato Anton per fargli uccidere Bagonghi, temendo che Fabio tornasse prima che lui conquistasse Lavinia
Lavinia lo deride, Fabio ammette di avere bisogno di aiuto e di non avere il diritto di giustiziarlo: lo risparmia
Fabio e Lavinia si riconciliano, se pur con amarezza


Parte 3

Ormai il sole è alto nel cielo, ma è nascosto dalle nuvole
Denise si propone per consegnare Gazzi alla polizia
Vittoria invece propone di aspettare, e racconta l'aiuto che le aveva promesso Govidi
Lavinia si insospettisce e fa notare che il cimitero è ancora deserto nonostante sia mattina inoltrata

Non appena Fabio capisce chi era il tizio incappucciato, egli irrompe: è Bagonghi, armato ed infuriato perché Fabio non ha ucciso Gazzi






\chapter{cap 27} 

(offuscamento?)

«Allora?» incalzò la figura. Era un uomo alto, dalla folta barba scura, parte del volto nascosta dal cappuccio di un enorme impermeabile.

«Boh, penso di sì», fece Fabio spassionatamente, parlando con fatica; la droga e la nottata all'umido avevano reso la sua voce era roca ed impastata. «Anche se, a essere sincero, non ne sono più così sicuro quanto prima. Vuoi essere annoiato con qualche sega mentale sul concetto stesso di realtà? Là c'è una ragazza che ti può aiutare.»

La figura rise. «Credici o no, ho già avuto notizia delle tue compagnie. Sembra tu sia beato tra le donne, almeno stasera.»

Fabio sbuffò sarcastico. «La beatitudine non è esattamente il mio stato d'animo.» Fissò per qualche momento il cielo terso, poi scoccò uno sguardo penetrante all'uomo incapucciato. «Non mi hai ancora risposto. Sei reale?»

«Certo che lo sono», rispose egli.

«Reale come una sequenza di correnti nel mio cervello, o reale come una persona?»

«Tu cosa sospetti che sia?»

Fabio sospirò. «Sospetto che tu ti stia divertendo a prendermi per il culo e credimi, questo gioco mi sta stancando.»

Il cappuccio di quel tizio lasciò intravedere un sorriso. «Non hai paura di me?»

Stavolta fu Fabio a ridere. «E perché mai dovrei averne? Se sei un'allucinazione, mettiti pure là con le altre. Vuoi farmi credere di essere un fantasma, o uno spirito? Sono pazzo, mica scemo. E se sei davvero una persona\ldots{} beh, se tu avessi voluto farmi del male l'avresti fatto e basta, non mi avresti offerto del fuoco --- che, tra l'altro, sto ancora aspetttando.»

Senza dire niente, l'uomo incappucciato gli lanciò un'accendino.

«Nero come l'anima», fece distrattamente Fabio, mentre si accendeva il sigaro. «Sai, c'era una certa persona che comprava sempre e solo accendini neri. Non aveva un vero motivo, non gli importava davvero del colore; però, quando qualcuno glielo chiedeva, lui diceva sempre così. Di che altro colore avrebbe dovuto prenderli?»

L'uomo misterioso ridacchiò. Un po' goffamente, si mise a sedere per terra accanto a Fabio. «Sembra un tipo niente male, questa persona».

«Già, niente male davvero», sussurrò Fabio, assente. «Povero diavolo\ldots{}»

«Che fine ha fatto?» domandò l'altro, accendendo il sigaro a sua volta.

«Siamo seduti sulla sua tomba», fece Fabio, amaramente. «Fa un po' cliché, vero?»

L'uomo non rispose, e i due stettero in silenzio per un po'. 

Fabio prese una copiosa boccata di fumo e provò a sbuffarlo in dei cerchi concentrici, fallendo miseramente. Non era mai riuscito a farlo; né lui né Bruno si erano mai presi la briga di imparare quel genere di cose. Amava le cose utili, il buon Bruno Bagonghi.

«Sai», disse Fabio dopo un po', «sto fumando questo sigaro sulla tomba del mio amico perché lui mi ha chiesto di farlo. Riesci a crederci?»

«Sì», rispose semplicemente l'altro. «So molto di te, caro il mio Fabio Fontanelli.»

Fabio ridacchiò. «Sai solo leggere i giornali e fare due più due, caro il mio custode del cimitero.»

«Anche tu sembri saper fare due più due. Forza, spiega.»

«Sono un tizio strano che di notte se ne sta seduto sulla tomba dell'imprenditore morto l'altro giorno. Chi potrei mai essere, se non il suo amico scomparso da diversi mesi?»

L'uomo fece un tiro profondo, nascondendosi in una nuvola di fumo profumato. «No, quello è ovvio», fece sbrigativo. «Spiega come mai pensi che io sia il custode del cimitero.»

Fabio rimase un po' interdetto. «Non sei un darkettone e tra un po' è mattina, quindi devi aprire questo posto. Non è ovvio anche questo?»

Non ci fu risposta, ed il silenzio tornò fra i due per dei lunghi istanti.

«Dimmi, Fabio», parlò il presunto custode, «perdona la mia curiosità: perché sei scomparso? Era intuibile che non ti fosse successo niente, che te ne fossi andato di tua volontà --- allora dimmi, perché hai voluto andartene?»

Fabio fu secco: «Non mi va di parlarne.»

«Capisco», fece l'uomo con l'aria di chi la sa lunga. «Allora forse ti va di parlare di questo: perché sei tornato?»

Fabio produsse un sospiro pesante. «Te l'ho detto, perché dovevo fumare questo sigaro con il mio amico a qualunque costo.»

Il custode lo incalzò: «Una lealtà ammirevole la tua, ma mi cheido se in realtà ci sia un altro motivo. Capirei se in realtà tu fossi tornato a vendicarlo --- o addirittura a vendicare te stesso.»

Fabio si voltò a guardare l'uomo dritto nell'oscurità del suo cappuccio. «Sai dove posso trovare il Leka?»

«Oh, lui? Sì\ldots{}» fece quello, la voce carica di un'emozione indecifrabile. «Sì, so proprio dove trovarlo --- ah, ed è più vicino a noi di quanto tu possa pensare!»

I muscoli di Fabio si irrigidirono di scatto, ma durò solo un momento; con lo sgaurdo perso nel vuoto, si abbandonò di nuovo contro la lapide. 

«Non mi interessa», disse stancamente. «A che servirebbe? Non mi ridarebbe il Bagonghi, né metterebbe a posto la mia vita. Non servirebbe a niente se non a farmi diventare sempre più scemo, sempre più diviso, sempre meno\ldots{} beh, sempre meno me. Non voglio più uccidere.»

«Ti manca la fermezza per farlo?» chiese l'uomo, la voce tagliente e severa. «Hai paura di non esserne in grado, nemmeno con qualcuno che merita tutta la tua ira?»

«Oh, ci riuscirei, eccome. Anzi, credo che uccidere sia una delle cose meno peggio che ho fatto  ultimamente.»

La voce dell'uomo si ammorbidì. «Non puoi dirmi una cosa del genere e sperare che me la beva così. Ora voglio sapere che diavolo hai combinato di peggio.»

«Oh, un sacco di cose. Me ne sono accorto solo stanotte, sai? Ho ammazzato, questo non lo nego. Questa notte ho rivissuto per tante volte i miei omicidi, lì ho\ldots{} diciamo che li ho sognati ---»

L'uomo rise forte. «Stai confessando di essere un assassino, ma ti premuri di nascondere che tu e le tue amichette vi siete drogati stanotte?»

Anche Fabio rise. «Sì, scusa, forse è l'abitudine --- oh, ma lascia perdere. Sai qual'è la cosa interessante di tutto questo? Che ogni volta che ho ucciso, l'ho fatto per proteggere qualcuno. Cristo santo, stasera ogni cosa che dico è un cliché, che diavolo mi succede? Insomma, dicevo\ldots{} quando invece avrei potuto uccidere ma non l'ho fatto\ldots{} quando ho assaporato quel potere, quando ho goduto nello sbandierare la mia possibilità di concedere la vita invece della morte\ldots{} è stato allora che ho toccato il mio fondo.»

L'uomo stette in silenzio per un po', come se stesse pensando velocemente a qualcosa. Quando finalmente parlò, lo fece sussurrando: «Tu mi parli del potere di uccidere? Ingenuo. Il vero potere è quando abbiamo ogni giustificazione per uccidere, ma non lo facciamo».

«Sì, l'ho visto anch'io quel film», rispose Fabio. «Anche se credo che la morale lì fosse un pelo diversa dalla mia.»

Ci fu un altro lungo silenzio fra i due.

«Mi hai detto», disse lentamente l'uomo, «che uccidere Leka non ti ridarebbe la tua vecchia vita. Hai ragione. Ma quando parlavo di vendicarti, in realtà non mi riferivo a lui.»

«Vai avanti.»

L'uomo esitò un attimo, poi riprese: «Sai, non ho molto da fare, mi piacciono i pettegolezzi sui fatti di cronaca. Ho seguito con interesse il tuo caso, sulla stampa e parlando con qualcuno che conosce una
certa\ldots{} com'è che si chiama? Lavinia Gori, mi sembra.» 

Il respiro di Fabio si interruppe.

«Ah, vedo che ricordi ancora il nome della tua fidanzata», proseguì l'uomo. «Insomma, mi sono stati raccontati vari retroscena che non sono stati scritti sui giornali. Pare che lei se la facesse con un certo Giacomo Gazzi, il quale ha messo in moto diversi eventi per impossessarsi di varie cose che non gli appartenevano.»

Fabio continuò ad ascoltare, pietrificato.

L'uomo fece una pausa, prendendo solennemente una boccata di fumo. «Si dice che dietro la morte del tuo amico imprenditore ci sia proprio questo Gazzi. Oh, non mi guardare così, in fondo ha senso: lavorava per Bagonghi, magari ha intravisto un modo per fare carriera liberandosi di lui; e l'esecutore materiale era un suo collega, è probabile che fossero d'accordo, o che Gazzi avesse esercitato una qualche leva su di lui. Ho letto molto su Anton Leka, molto più di quanto valesse la pena di scrivere riguardo a uno come lui; pare che avesse mille motivi per odiare il suo datore di lavoro, e non puoi che concordare che fosse una persona facilmente manipolabile da qualcuno che avesse l'interesse a sfruttare la sua rabbia.»

Il tempo si era come fermato.

Finalmente, tutto aveva un senso. Come poteva non esserci il Gazzi dietro tutto quello che era successo al povero Bruno? Gazzi gli aveva portato via il suo amico. Gazzi gli avva portato via la donna. E se due più due faceva quattro, anche Daniele era morto per colpa di Gazzi, in qualche strano e distorto modo. Aveva perfettamente, pericolosamente senso.

Una strana visione riaffiorò nella mente di Fabio, un alto e pallido Gazzi,bizzarramente incarnato in un simulacro dell'arcinemesi della saga di Harry Potter, il terribile Lord Voldemort; il male puro e semplice, senza sfumature, senza ambiguità. Il significato di quell'immagine era fin troppo chiaro: Giacomo Gazzi era l'opposto di Fabio Fontanelli, il nemico giurato con cui non ci sarebbe mai potuta essere pace.

\emph{Nessuno dei due può vivere se l'altro sopravvive\ldots{}}

La voce dell'uomo nel cimitero penetrò nei pensieri di Fabio, rimbombante come quella della profezia che aveva ricordato. «Quindi», annunciò con tono definitivo, «se mai decidessi che in fondo non c'è niente di male ad avere la tua meritata vendetta\ldots{} beh, non sono affari miei, ma credo che dovresti dirigerla verso di questo Gazzi.»

Fabio annuì assente, perso nei meandri della sua mente. Per dei lunghi istanti fissò il nulla davanti a sé, riflettendo sulla metafora che le visioni lisergiche gli avevano donato, mentre un Gazzi-Voldemort danzava balordo nella sua immaginazione.

«Tutto bene?» disse l'uomo dopo un po'.

«No», piagnucolò Fabio, tornando alla realtà. «Perché penso queste cose? Non ho mai neanche letto Harry Potter!»

Il tizio tossì la sua boccata di fumo, come se fosse sbiottito. «Prego?»

«Niene, lascia perdere. Credevo avessi capito che sono matto da legare.»

L'uomo rise. «Beh, chi non lo è?»

Fra i due calò nuovamente il silenzio, e nessuno lo ruppe per un po'.

Dopo qualche istante, o forse qualche ora, lo strano uomo sbuffò un ultima nuvola di fumo profumato e posò pesantemente una mano sulla spalla di Fabio, spingendosi in piedi.

«Caro mio, è stato un piacere fumare con te. I toscanelli al caffé sono i miei sigari preferiti», disse. «Purtroppo, il tempo è tiranno e sono sicuro che abbiamo entrambi del lavoro da fare.»

Fabio lo seguì con lo sgaurdo mentre si allontanava a passo lento.

«Non piangere troppo il tuo amico Bagonghi», aggiunse ad alta voce, senza voltarsi. «Lui ha solo scelto di seguire il suo destino. E se anche te seguirai il tuo\ldots{} beh, se lo facessi, lui sarebbe fiero di te, ne sono certo.»

Fabio non disse niente. Non aveva proprio niente da dire.

\chapter{Titoli di Coda, parte 2}

Casa Circondariale di Prato "La Dogaia", un altro giorno a caso.

Un uomo barbuto, alto ma un po' torto, rimirava pensieroso il prodotto di quello che, dall'inizio della prigionia, era stato il suo unico passatempo.

La Repubblica, prima pagina, taglio medio:
TORNA DOPO MESI E FA ARRESTARE UN MORTO
LA FIDANZATA: "CHE CAZZO SUCCEDE?"

La Nazione, prima pagina, titolo principale:
RAGAZZO SCOMPARSO DENUNCIA IMPRENDITORE MORTO: 
CARABINIERI ARRESTANO TUTTI E DUE

Il Tirreno, prima pagina, articolo di fondo:
IMPRENDITORE CREDUTO MORTO È VIVO: ARRESTATO
SOSPETTATA PISTA MAFIOSA NELLA VENDITA DEL SUO LANIFICIO

Notizie di Prato, articolo online:
BAGONGHI È VIVO, LEKA E' MORTO E FONTANELLI E' TORNATO
SCONCERTANTE LA VERITA' DIETRO LA SURREALE VICENDA

La svolta nella vicenda che ha coinvolto l'ormai tristemente famoso Lanificio Bagonghi, a dir poco, rasenta il ridicolo: Fabio Fontanelli, la cui scomparsa era stata denunciata ormai da mesi, si è presentato ieri pomeriggio alla stazione dei carabinieri di Poggio a Caiano in compagnia di Bruno Bagonghi, il noto rampollo della famiglia Bagonghi che aveva ceduto lo scorso mese il suo storico, omonimo lanificio ad una holding cinese, appena qualche giorno prima di darsi per morto con tanto di finto funerale. Fontanelli ha poi consegnato una pistola, dei documenti falsi ed una certa quantità di sostanze stupefacenti, ed ha inoltre confessato una lunga serie di crimini. I due sono stati immediatamente arrestati, e [...]

Con molta calma, l'uomo staccò gli occhi quei frammenti di articoli, concedendo uno sguardo alla copia - intera - de La Repubblica che aveva appena ricevuto: il titolo dell'articolo di fondo era per lui di estremo interesse.

SCOMPARE, RITORNA CON UN MORTO E SCOMPARE DI NUOVO
LA (EX?) FIDANZATA: "ORA MI SONO ROTTA IL CAZZO"

L'uomo barbuto lasciò perdere per un attimo la sua rassegna stampa, e rivolse un'occhiata assente al poco che vedeva del mondo esterno. Si lasciò tangere il volto dalla timida luce del sole che filtrava attraverso le sbarre, quasi come se temesse di passare per quella minuscola finestra. 

Sospirò.

"Fabio... questa te la faccio ricacare", pensò per l'ennesima volta, lasciandosi corrompere dal rancore. "Non so come e non so quando, ma giuro che questa te la faccio proprio ricacare".

Lo sguardo gli corse nuovamente al giornale. Sorrise: la partita non era ancora persa, era solo cambiato il suo avversario.

\chapter{cap 29}

// separare fabio e lavinia punti di vista

Il cuore di Fabio si era fermato. Sulla soglia del grosso, rugginoso cancello, inondata dallo scintillante sole mattutino, si stagliava un'inconfondibile figura. Occhi gonfi e sguardo basso, passo lento e rassegnato, un crisantemo fresco in mano: Lavinia Gori si trascinava sul vialetto del cimitero con l'aria di chi ha subito qualche colpo di troppo dalla vita.

Né Denise né Vittoria osarono muoversi o fiatare. Ne avevano parlato la sera precedente, dopo essere scampate al finimondo per un pelo, quando per puro miracolo Lavinia e Fabio non si erano notati. Le due ragazze erano d'accordo: sarebbe potuto andare disastrosamente male, ma se Fabio voleva avere anche solo una minima speranza di tornare in sé, doveva affrontare apertamente quello spettro del passato.

Senza parlare, senza respirare o anche solo sbattere le palpebre, Fabio lentamente si mosse; la testa alta, il passo incredulo ma fremente, prese il vialetto e andò incontro al proprio destino.

(«Scommettiamo che la uccide?», bisbigliò Vittoria alla sua compagna, la quale la zittì con una violenta gomitata nelle costole.)

Non appena i due furono a opchi passi di distanza, Lavinia alzò distrattamente lo sguardo da terra e si fermò. «Sì?» chiese distrattamnete al tizio che le si era parato davanti.

Fabio non rispose subito. Vedere la sua ex compagna di vita in quello stato gli causava una strana sensazione. Portava il vestito buono, il più elegante che aveva; tuttavia era spiegazzato, non propriamente sporco ma comunque si vedeva che non era fresco di armadio. Lo avevano comprato insieme, in un caldissimo sabato pomeriggio di luglio, in una delle rarissime volte in cui Fabio aveva acconsentito a fare un uscita mirata per fare shopping a buon mercato. Un'increspatura quasi apparve sulle sue labbra al ricordo: Fabio odiava quelle spedizioni, ma quasi sempre tornava a casa con un bottino migliore della sua compagna. L'increspatura scomparve subito non appena realizzò l'ovvia implicazione: si era messa quel vestito perché era l'unico vestito nero che possedeva. Non amava vestire di nero, Lavinia Gori; non si vestiva di nero quasi mai, né nelle serate metal al Cipher, né al concerto stesso degli Iron Maiden; nemmeno alle serate eleganti per le quali aveva appositamente comprato quel capo d'abbigliamento. Solo la morte di un caro amico la aveva fatta vestire di nero.

Fabio si sentiva molto strano. Molti ricordi di vita quotidina con
Lavinia gli stavano balenando in mente, scuotendo intensamente le sue
fondamenta; era come se la memoria gli stesse tornando dopo tanto tempo,
ma Fabio non credevadi aver mai dimenticato niente. Aveva evitato quei
ricordi perché gli facevano male, ed anche adesso che gli si
presentavano all'attenzione contro la sua volontà non era sicuro di che
cosa provava.

La voce di Lavinia lo riscosse. , fece, la voce piccola e distratta,
«avrei da pass - »

I due sguardi si incontrarono, e il mondo intero rabbrividì di nuovo.

Fabio si sentì morire dentro nel rivedere quegli occhi. Luminosi come il
sole, chiari come il cielo. Gonfi, stanchi, specchio di un'anima ferita
e disperata.

, fece Fabio con un filo di voce.

Lavinia era come pietrificata.

Fabio proseguì: «Dimmi, Lavinia, se una luce ne oscura altre è
sempre luce?»

?? Una nuvola passeggera coprì coreograficamente il sole.

Lavinia non parve far caso a quelle parole. La sua attenzione era fissa
su Fabio, il suo sgurado lo esplorava avidamente, come se volesse capire
se era davvero lui. «Come stai?» gli chiese con
le lacrime che stavano cominciando a sgorgare dai suoi occhi.

«Se un bene fa del male, è davvero bene?» domandò
di nuovo Fabio, gli occhi fissi sulla sua ex compagna di vita.
«Se per amore menti, o rubi, o uccidi, sei comunque una
persona buona?»

Un barlume rosso attraversò brevemente gli occhi di Fabio. Sul suo volto
si dipinse un'espressione dura, risoluta, e i suoi occhi divennero duri,
insostenibili, come se stesse cercando di incendiare la sua vecchia
compagna di vita con il solo sguardo.

(«Dai, venti euro su lei morta, niente morti si
patta?», bisbigliò Vittoria alla sua compagna, la quale
rispose nuovamente con un altro violento colpo di gomito.)

Fabio tirò dritto col suo discorso. «Se ogni tanto sbirci
nell'abisso solo per vedere l'effetto che fa, e ogni tanto ti concedi
qualche piccola deviazione dal sentiero giusto, per prenderti quella
piccola soddisfazione che ti fa andare avanti, sei sempre
buono?»

La sua voce si indurì ancora di più. Una terribile furia si era come
impadronita di ui.

«Se invece a un certo non ne puoi semplicemente più e dici
basta, e cominci a cercare di vivere peggio che puoi, tuffandoti in un
abisso oscuro senza la minima esitazione ogni volta che ne hai
l'occasione con la folle, disperata speranza di perderti, di non essere
più lo stesso quando torni su. Ma sono solo momenti, e quando passano ti
rendi conto che in realtà sei sempre il solito coglione di sempre,
soffocato dallo stesso dolore, condannato a patire per sempre senza
avere altro che insignificanti vendette contro il mondo ogni volta in
cui fai del male a qualcuno! Dimmi, Lavinia - se questo è il tuo
destino, sei una persona buona?»

Lavinia parve aver ascoltato attentamente di ciò che Fabio aveva detto.
I loro sguardi si incrociarono di nuovo, e nel suo cominciarono a
comparire delle lacrime. «Questo --- questo che c'entra? E'
così che stai?» chiese, la voce piccola e già rotta dal
pianto.

La faccia di Fabio si raggrinzì su sé stessa, come se si fosse
improvvisamente rotta. Tutto il suo dolore parve mostrarsi su di essa. ,
disse in un patitico squitìo.

Durò solo un'istante.

L'ardore ritornò potente nelle sue parole quando proseguì. Lo sguardo
che Fabio le restituì avrebbe potuto ucciderla all'istante.
«Scusami, non è importante. Sto esattamente come l'ultima
volta che mi hai visto. Non ti interessava allora, non vedo perché
dovrebbe interessarti adesso.»

Qualcosa dentro Lavinia parve rompersi; quasi inciampò sui suoi piedi
pur essendo ferma, ed emise un penoso lamento interrogativo.

Fabio non le dette tempo di aggiungere altro. «Lasciamo
perdere, parliamo un momento di te. Se una ragazza non ha né arte né
parte in tutto il mondo, e vuole solo raccogliere dalla vita il meglio
che trova, magari può trovarsi un coglione qualunque che la desidera per
viverci insieme e dividersi le fatiche della vita. Se poi ne trova
addirittura un altro, e se in fondo gli sta più che bene perché alla
fine le attenzioni gli piacciono, può succedere che alla fine lei si
inganni per addolcire la realtà, per mascherarsi il fatto che in realtà
è una stronza opportunista che mantiene una relazione solo per il comodo
di dividere gli oneri della vita, magari in attesa del momento opportuno
per fare il ribaltone e andare con l'altro tizio. E se questa ragazza
non si accorge - o addirittura non gli importa dell'effetto che il suo
essere troia fino al midollo fa alle persone che gli stanno intorno e
che magari le vogliono bene, questa ragazza è o non è una
puttana?»

Fabio aveva le fauci asciutte. Ribolliva di rabbia, il suo discorso
aveva seguito uncrescendo di itnensità tale dall'aver ringhiato le
ultime parole.

Lavinia aveva ascoltato rapita ogni cosa che aveva detto Fabio. Era
pallida, sudata, addolorata dal sentir dire quelle parole dalla persona
che amava; eppur enon disse ancora niente. Le sue mani stringevano così
forte il crisantemo che aveva preso per il Bagonghi da avern quasi
stritolato il gambo.

«Comunque, non ha più importanza», concluse
Fabio, ora calmo, una nota di rassegnazione nella voce. «Sono
solo cose che mi sono accorto che ti avrei voluto dire, ma non abbiamo
più avuto occasioni di parlare con franchezza da tanto, tanto
tempo.»

Fabio si voltò con una certa, bizzarra solennità. «Offri pure
quel fiore al povero Bruno, dagli l'ultimo saluto come meglio credi, poi
vattene.»

Singhiozzò.

«Continua ad essere felice, te che riesci a farlo. Non hai
bisogno di Bruno\ldots{} e nemmeno di me.»

Se ne andò con passo lento verso le due ragazze, che Lavinia aveva
totalmene ignorato fino a quel momento.

Una sola parola, piatta, incredula, uscì dalla bocca della ragazza
mentre guardava la figura del suo compagno di vita allontanarsi da lei:
.

Incrociò lo sguardo di Denise, la quale si limitò a guardarla e ad
alzare leggermente le mani in segno di rassegnazione.

, ripeté con più vigore.

Il suo ragazzo era scomparso. Un loro amico aveva scritto a tutti di
averlo visto, poi era scomparso anche lui. Un altro loro amico era stato
ammazzato dall'ennesimo amico, che era a sua volta sparito.

, ripeté con ancora più vigore.

Ora, il suo ragazzo si era presentato dopo \emph{mesi}, concio come le
bestie, per dire cose a caso e prenderla a parole.

«Cosa \emph{cazzo} - »

Lavinia esplose.

Buttò via il crisantemo ormai stritolato, si tolse una scarpa e la
lanciò dietro a Fabio con tutta la forza che aveva.

 urlò, completamente fuori di sé. «FERMO --- FERMO, \emph{DIO
B ---}» lanciò l'altra scarpa.

Fabio si fermò, ma non si voltò.

«Ora te\ldots{} sì, *ora te `ttu vieni qui\emph{»
ringhiò, la voce vibrante per la rabbia. «}Ora 'ttu pigli, tu
'tti metti fermino\emph{, sì, e }'ttu* mi spieghi un po' CHE CAZZO STA
SUCCEDENDO!»

L'ultimo grido riecheggiò sinistramente in tutto il cimitero.

(Poco lontano, Vittoria bisbigliò: «Ultima offerta, dieci euro
morto lui!» Denise non si disturbò a rimproverarla)

Lavinia continuò, schiumante di ira: «*E' da quande t'ha'
preso e ttu se' spari'o che gl'è tutto un succede' di roba che nemmeno a
i' telegiornale! Ho chiamato e' harabinieri, la polizia, la guardia
nazionale, i' porco d'Idd --- nulla! Poi gl'è spari'o anche i' Brogelli!
Ora i' Bagonghi s'è fatto ammazzare da i'Leka, e tu'sentissi icché mi
vien'a 'ddire quell'attro rintrona'o d'i Gazzi, tu gli
stiaccerest'iccapo!»

Fabio si voltò di scatto a guardare la sua vecchia compagna di vita, il
volto solcato dal dolore e da un'improvvisa rabbia. , sbottò a denti
stretti, «il tuo amichetto Giacomo farà presto la fine che
---»

«CHETO!», ruggì ferocemente Lavinia,
interrompendolo e quasi spettinandolo. La ragazza fece un aggressivo
passo in avanti; Fabio non indietreggiò, ma non poté fare a meno di
sussultare di fronte a quella furia.

Lavinia sputò le parole come fossero veleno. «Amichetto?
Quella serpe!? *Se `ttu sentissi la metà d'icché m'ha racconta'o* --- te
un tu'nnha' un'idea! Ogni volta che ne succedeva una, lui l'era sempre a
dire male di quello e di quell'altro, se uno mi diceva coppe lui mi
veniva a dire che era picche! Io sta'o `n pena, e lui gl'era lì a dì
male di te, di me e di tutti quelli che m'hanno ma' detto una parola di
conforto!»

Il mondo di Fabio si inceppò. I colori, ancora fin troppo vividi a causa
della droga, si spensero di colpo, ed il suo occhio sinistro si
contrasse di scatto per alcune volte.

«Cosa», disse piano, senza emozione.

«Cosa?», gli fece eco Lavinia.

Fabio prese un bel respiro. , scandì cautamente «che
\emph{sai} che il Gazzi è una serpe. Lo \emph{sai} che per lui mettere
zizzania è come respirare, e lo \emph{sai} che ti ha sempre raccontato
un sacco di cazzate. Stai --- stai dicendo questo?»

«Eh!» fece Lavinia, insofferente. «Che
mi se' diventa'o grullo? Tu lo sa'`nche te come fa! Gl'ha scritto su i'
gruppo che t'ha' fatto degl'affari che neanche la mi nonna quand'e
guarda Quarto Grado! Prima t'ave'i la russa, poi la cinese, poi e'
debiti di gioco* - » La voce della ragazza fu rotta da un
singhiozzo. «\emph{Te un tu'nn'ha' un'idea\ldots{}} Io\ldots{}
Mesi e mesi a sentirmi dire queste cose su di te\ldots{} Senza potergli
dire di chiudere quella fogna e di andare a nascondersi, senza potergli
andare n'i'vviso a digli --- lascia fare, guarda!»

Il cerchio si chiuse, e Fabio andò finalmente a sbattere alla velocità
della luce contro il granitico muro della realtà.

Il suo più intimo terrore era lì, di fronte a lui. Non aveva via
d'uscita: lo disse ad alta voce. «Pensavo che tu non mi
volessi più. Che non vedessi l'ora di stare con lui.»

Lavinia spalancò la bocca per lo stupore e si bloccò per alcuni secondi.
«EH?» urlò finalmente, non appena riuscì a dar
voce alla sua incredulità.

(«Ecco, ora si rimettono insieme\ldots{}» si
lagnò Vittoria, prima di essere violentemente zittita per la terza
volta.)

Lavinia sembrava non trovare le parole adatte per esprimere quello che
provava. «Fabio\ldots{}» tentò, ma il suo
tentativo si estinse subito.

Fabio non disse niente; non aveva proprio altro da dire.

Il silenzio si prolungò e i due se ne stettero lì, immobili, per diversi
istanti, entrambi traditi dalla loro eloquenza.

Una voce rassicurante ruppe lo stallo. «Siamo tutti un po'
troppo agitati», disse piano ma con decisione Denise, che
nel frattempo si era avvicinata a Fabio e Lavinia. «Agitati
tutti quanti, noi tre addirittura scompigliati e infreddoliti. Che ne
dite di fare un bel respiro, sospendere per un attimo tutto quello che
stava succedendo e andare a raccattare i nostri cocci? Il Bagonghi tanto
non scappa, torneremo a trovarlo quando ci saremo presi cura di
noi.»

Il buonsenso di quelle parole permeò subito tutti quanti.

«Sì» fece Fabio. «Io ho freddo. Cazzo se ho freddo -- non mi ero
accorto fino a ora di avere freddo.»

«E io sono scalza», fece Lavinia.

«Io ho dei capelli terribili» pigolò Vittoria. «Devo farmi la doccia, e poi la piastra ---
avete una piastra e una spazzola da prestarmi? Ah, e anche una
doccia.»

«Te chi sei?» chiese debolmente Lavinia, ma
Denise la bloccò subito: «Prima le docce e le piastre, poi le
presentazioni. Abbiamo tante cose da raccontarcdi l'un l'altro, si
capisce.»

Senza dire altro, la strana comitiva che si era appena formata si
apprestò a uscire dal cimitero, con la speranza di lasciarsi alle spalle
non soltanto quel luogo maledetto, ma anche tutta la confusione, i dubbi
e, soprattutto, il freddo che ognuno di loro aveva patito.

\chapter{cap 30}

Il destino, si sa, a volte è simpatico. Altre volte invece è dispettoso, oppure addirittura inopportuno. Certe volte, però, può essere semplicemente una\ldots{}

«Merda!», esplose Denise, non appena si accorse di chi stava trottando in tutta fretta su quel dannato vialetto di quel doppiamente dannato cimitero, dritto e rapido verso di loro come un traballante treno merci stracolmo di guai.

«Lav!» latrò quell'uomo, la voce insopportabilmente untuosa. «Cucciola, ti ho cercata dappertutto!»

Denise scattò in avanti verso di lui, decisa a fermare o per lo meno a rallentare questa ennesima catastrofe. «Gazzi, \emph{sta' bòno}», cominciò aggressiva, «guarda, non è proprio aria. Abbiamo già abbastanza cose per il capo, ci mancheresti soltanto te. Capisci? Dagli pace un secondino alla Lavinia, sù!»

Il tono di Giacomo Gazzi cambiò repentinamente; era come se le sue parole contenessero un ringhio di sottofondo, un chiarissimo presagio di malcelata ostilità. «Ero solo preoccupato, ok? Levati di torno, ora ci penso io a lei.»

«No, guarda, non hai proprio capito», gli tenne testa Denise. «Ora ci lasci un attimo in pace, tutti quanti, poi fai quello che ti pare. Te lo ridico: ora ci mancavi soltanto te.»

«Senti, ma che cazzo vuoi, eh!?» sbottò il Gazzi, sfoderando la sua aggressività senza filtri. «Lav sa badare a sé stessa! Non ha bisogno di te! Se non mi vuole vedere sarà lei a dirmelo!»

«Gazzi ma che stai dicendo? Certo che sa ---»

«E poi te, chi cazzo ti credi di essere? Non te ne è mai importato niente di lei, con che coraggio ora ti metti ---»

«Ma che c'entra! Non è questo il punto! Ti ho solo detto di lasciarci un attimo in pace ---»

«Lei ha bisogno di me! Non ferisci me, ferisci lei se cerchi di tenermela lontana!»

«Ma chi ti vuole ferire!? Ma chi ti caga!? Io non --- senti, basta, lasciamo perdere.»

Denise  per un attimo considerò seriamente di portarlo via di peso prima che gli altri raggiungessero il loro parapiglia sul vialetto. 

Si rassegnò a compiere un ultimo, diplomatico tentativo: «Io te lo dico, Gazzi: qui siamo in un bel casino. Se non ti vuoi fare ammazzare, vai via più veloce della luce. Poi ti spiego, ti chiamo io; dopo pranzo, o dopo cena, o quando sarà il caso, promesso. Ora però per piacere, davvero, \emph{vai via}.»

«Non esiste. Devo vederla e parlarle, \emph{ora}.»

Denise si arrese. «Fai come ti pare», gli disse con la voce intrisa di tutta la sua tristezza. «Per me erano successe abbastanza disgrazie, ma se proprio ci tieni a farne succedere altre, fai pure.»

Il Gazzi grugnì soddisfatto. «Fatti da parte. Non sei mai stata una protagonista, rimettiti al tuo posto.»

Denise gli rivolse un ultimo sguardo interrogativo, ma poi scosse la testa e si fece da parte. Il Gazzi si avviò tronfio e inesorabile verso quello che sarebbe stato l'ennesimo disastro di quella settimana.

\begin{center}
***
\end{center}

Fortunatamente, Fabio pareva non essersi ancora accorto dell'intrusione del suo acerrimo nemico.

Non aveva occhi che per Lavinia: se la guardava e riguardava come se non l'avesse mai vista, come se ogni increspatura ed ogni ombra che il dolore aveva causato su quel bel volto liscio e pulito potesse raccontargli un'incredibile storia in cui lui, ovviamente, era l'assoluto antagonista. 

C'era rimorso nei suoi occhi, c'erano sensi di colpa e tanto, tanto dispiacere al pensiero di che cosa era riuscito a perdersi. Eppure, nonostante tutta questa negatività, nonostante tutto quello che aveva fatto succedere a causa della sua stupidità, Fabio non si sentiva triste: il poter guardare Lavinia, il solo fatto che lei sopportasse ancora la sua presenza, che lui potesse posarle lo sguardo addosso, tutto questo poteva significare soltanto che il mondo non si era ancora guastato del tutto. Per quanto male poteva averle fatto, lei era comunque lì, esisteva, viveva, era rimasta sé stessa anche dopo la tempesta di pura e semplice merda che Fabio le aveva stupidamente fatto pivoere addosso.

Niente sarebbe stato più come prima con lei. Questo, Fabio lo sapeva.

Non avrebbe mai osato sperare in un perdono delle sue malefatte, ed in cuor suo neanche lo voleva. Gli bastava che lei fosse lì, che non rifiutasse il confronto con lui e che fosse pronta a giudicarlo per le sue colpe, se mai lui avesse deciso di confessarle.

Perso in questi pensieri, Fabio vedeva e ascoltava un solo aspetto dell'ambiente a lui circostante: Lavinia.

Quando il suo ritrovato angelo parlò, la voce melliflua ma dritta, penetrante, il suo vernacolo bello come la più aulica delle lingue, Fabio non capì del tutto cosa stesse dicendo, e soprattutto a chi.

«\emph{Oh Gia'omino, ma che mi de'i venì dietro dietro 'om un cane!}»

«Mi hai fatto preoccupare!» qualcuno le rispose con una familiare voce unta e lamentosa.

«Te l'avevo scritto ieri sera che stamani avevo da fare! Che passavo dal cimitero e poi andavo a fa' la spesa, e che semmai si sentiva dopo desinare! Preoccupare di che?»

«Cucciolina, ma non te ne devi andare in giro da sola, lo sai che su di me ci puoi contare per ---»

Il tizio dalla voce unta e lamentosa non voleva che la sua Cucciolina andasse in giro da sola. Fabio emerse bruscamente dal suo mondo interiore, ed un brutale fuoco lo avvampò non appena cominciò a rendersi conto di chi si era manifestato al suo cospetto.

«Intanto non sono da sola, poi --- lascia fare. Senti, ma te non c'hai proprio null'altro da fare? Dai, fa' per bene, è una giornatina niente male anche questa, lasciami un po' respirare, ci si risente dopo desinare.»

Gazzi fece per replicare, ma un sommesso sghignazzio gli gelò il sangue prima che potesse aprire bocca.

Fu Fabio a rompere il breve silenzio che seguì, la voce così bassa che sarebbe potuta passare tranquillamente per un borbottio del cielo, nuovamente, improvvisamente e piuttosto coreograficamente coperto; bassa, ma carica di una chiara, chiarissima sfumatura omicida, una sfumatura gelida e affilata come la lama di un coltellaccio appena uscito dall'arrotino.

«Giacomo Gazzi», disse. Solo questo.

Il Gazzi guardò stupidamente l'uomo che lo aveva nominato. «In persona» rispose, piuttosto a disagio. «Tu chi sei?»

Fabio riprese a sghignazzare. Anche l'osservatore meno attento si sarebbe potuto accorgere che qualcosa non andava in lui: la bocca gli tremava e si deformava come se stesse cercandod i trattenersi, il suo sguardo era marmoreo, fisso come quello di un predatore sull'uomo che odiava. Incapace di controllarsi, la sua risata crebbe fino a diventare un terribile ululato, un animalesco annuncio di cattive intenzioni.

Denies si stava letteralmente cacando sotto. Sapeva quello che stava per succedere, ma sapeva anche di essere totalmente impotente. La situazione sarebbe degenerata, Fabio avrebbe ucciso il Gazzi e tutto sarebbe andato ai maiali. Niente di quello che lei poteva fare avrebbe potuto cambiare il corso di questi eventi.

Ci provò comunque.

Si avvicinò a Fabio lentamente, gli prese una mano fra le sue e gli sussurrò all'orecchio «Fai per bene. non so che altro dirti, se non ti prego, ti scongiuro, fai per bene»

La folle risata di Fabio si trasformò in una ringhiata, folle risposta. «Oh sì, vecchia mia, farò proprio per bene»

Fu solo questione di un momento. Gazzi si ritrovò una pistola puntata dritta in mezzo agli occhi.

Vittoria urlò e LAvinia imprecò. Denise si sedette semplicemente per terra, sconfitta. 

«Giacomo Gazzi» ripeté fabio, col tono solenne di chi stava pronunciando una sentenza. «Il bagonghi è morto. il Brogelli è morto. tu sei ancora qui, e questo non va bene.» 

GAzzi realizzò finalmente chi aveva davanti.

«non è possibile», si lagnò pietosamente. «Sei, sei davvero te?»

Fabio esibì un sorriso così ampio che avrebbe mostrato tranquillamente una quarantina di denti, se li avesse avuti.

«In persona» ghignò.

Il Gazzi lo stava guardando come se fosse il Cristo in persona, sceso dal cielo e rivelatosi a lui con il solo scopo di ucciderlo brutalmente. « Fabio » mugolò balbettando, «io - io sono - d - dove - siamo amici noi, lo siamo!»

«Amicissimi» fece Vittoria, avvicinandosi con decisione a Fabio e appoggiando delicatamente una mano sul braccio con cui egli reggeva la pistola.

«Fabio la degnò di un solo, breve sguardo, prima di rimettersi a fissare il Gazzi col suo migliore sguardo inceneritore. «Non intrometterti» le sibilò. «Questo verme non avrà salva la vita»

«Salva la vita?» ripeté lei. «Cazzomene, ma chi lo conosce. Volevo dire, fai quello che ti pare, ma veloce, che qui c'è gente che ha freddo e che deve darsi una sistemata»

Un denso silenzio scese su tutto il gruppo, fino a che Denise non scoppiò a ridere. «Scusate» fece non appena si fu ricomposta. «questa ragazza mi spezza, ha queste uscite che... boh. Comunque ha ragione, eh. Fabio, sbrigati con questa vendetta che una bella doccia calda non mi farebbe proprio scomodo»

Lavinia intervenne. «Vendetta? Che gli ha fatto il Gazzi?»

«Ti ha portata via da lui, si capisce. O comunque, questo è quello che pensa Fabio.» 

Lavinia aprì e chiuse la bocca varie votle, incapace di formulare una frase.

Fabio, senza accennare a voler abbassare la pistola, dichiarò al mondo intero: «Non temete, presto sarà tutto finito e ci faremo tutte le docce calde che vogliamo. Gazzi: sei responsabile - direttamente o indirettamente, non lo so, non mi interessa - della morte del Bagonghi, del Brogelli, del Leka, di aver tramato alel mie spalle per portarmi via l'affetto più importante della mia vita e di aver reso quest'ultima un inferno tramite l'influenza della tua mera esistenza.» Tirò indietro il cane della sua 9mm silenziata, sebbene fosse un modello semiautomatico e non ce ne fosse alcun bisogno. «Neghi di non essere colpevole di tutto questo?»

«No!» strillò il Gazzi. «No, no! Non mi ammazzare!»

«Perfetto» sentenziò Fabio. «Se tu avessi contato le negazioni nella mia domanda, forse non avresti confessato. Non che questo di avrebbe risparmiato, parliamoci chiaro. Ebbene, Giacomo Gazzi, se avessi imparato il passo di Ezechiele che dice il nero di Pulp Fiction, adesso te lo reciterei. Sappi comunque che questa è la mia vendetta, e che mi macchierò del tuo assassinio per vendicare i torti subiti da me, dalla Denise e dalla Lavinia!»

Un ordine venuto dal cielo fermò la sommaria amministrazione di giustizia. «Abbozzala subito»

Era stata Lavinia a parlare. Con un rapido guizzo si mise fra il Gazzi e Fabio, e con u nmovimento repentino si impossessò della pistola di quest'ultimo.

«AscoltaZ» disse, fumante di rabbia come non lo era mai stata prima di allora. «ascoltami molto, molto bene, Fabio Fontanelli. Fai pure quello che vuoi della tua vita, diventa pure la canaglia che aspiri a diventare, ma non provare neanche per un momento a tirarmi in ballo in tutto questo. Per troppo tempo ho pensato che fosse colpa mia se sei impazzito, che fosse colpa mia se non ho fatto miracoli per impedire che avvenisse. Sbagliavo, e non sbaglier; pi#. Ammazza pure questo scemo, fai veramente che cazzo ti pare. Ma non azzardarti a darmi anche un puttanesimo di responsabilit' per quello che sei diventato!»

Si fece da parte e gli porse la pistola. «Prego, continua pure per la tua strada, per tutto il bene che può farti»

Fabio aveva ascoltato tutto qusto a bocca aperta. «Io non ti sto dando proprio niente», le disse debolmente. «Voglio vendicarmi, e visto che ci sono volevo farlo anche per conto tuo»

// dove sta scritto che debba tramare una sola persona per volta?

con gazzi:

«Per conto mio!?» fece Lavinia, minacciosa. «Io non la voglio di certo la tua carità! Ci penso da sola a vendicarmi --- e tra l'altro non mi voglio vendicare, e di certo non di lui!»

Si rivolse al Gazzi con uno sguardo intriso del più puro disprezzo. «Che cazzo c'è da vendicarsi di quest'ometto! Lo so che persona sei, stronzo, viscido e marcio fino al budello! Ti piace, ti ecciti proprio a ficcare il dito nelle piaghe degli altri, a cercare di smerdare chiunque sia in difficoltà per sembrare più bello, più grande, più importante! Mi fai schifo, e per ogni cattiveria che hai sputato su di me e chi mi sta caro te ne farò ingoiare tre appena ne avrò l'occasione! §Io son bona e cara, ma quande m incazzo vu lo sapete, le cose le dico male: vai in culo, Giacomino, te e quella zoccola di to ma!»

Fece un bel respirone come per calmarsi, ma in realtà stava soltanto riprendendo fiato.

«Lui l'è sistemato» fece con l'aria di chi ha appena spuntato dalla lista delle cose da fare un lavoro faticoso e pesante. «Ma \emph{te}...»

Fabio fece un patetico salto all'indietro, come a volersi scansare dall'incandescente dito accusatorio che Lavinia gli aveva puntato contro.

«Con \emph{te} non ho ancora finito. Certo, nini, s'ha da fare i conti, \emph{io e te}» ringhiò al suo probabilmente ex compagno di vita.

Dopoq quest'ultiam dichiarazione, sul gruppo calò un pesante silenzio.

Lavinia sbuffò, questa volta cercando di calmarsi davvero. <Due volte> dissse, la voce sottile per il poco fiato rimastole. <Fabietto, in neanche un'ora mi hai fatto incazzare due volte. Non ti strozzo solo perché sei alto e al collo ci arrivo male.>

// vittoria chiama la polizia col telefono di Fabiomi hai rubato il telefono
e tu mi hai rubato l'iphone, credi forse che l'abbia dimenticato?
\chapter{cap 31}

Alto e imponente, il volto nascosto da un ampio cappuccio, uno strano
figuro dalla barba folta si avvicinava alla sfortunata combriccola.


arriva bagonghi un casino guarda

\chapter{Titoli di Coda, parte 1}

Presidio ospedaliero di Prato "Misericordia e Dolce", un giorno a caso di un mese a caso.

Una ragazza riccia e una sua collega più anziana discutevano svogliatamente del più e del meno, sorseggiando di tanto in tanto un pessimo caffè. Il fumo delle loro sigarette non riusciva a uscire del tutto dal minuscolo abbaino che, aperto, costituiva l'unica fonte di luce naturale di cui godeva quel corridoio. L'aria ingrigita di quell'ambiente era perfettamente in linea con il loro umore: dopo anni di servizio in quel manicomio, una profonda apatia era l'unica difesa possibile contro il peso schiacciante della loro sanità mentale, che lì era più unica che rara.

Una risata sguaiata eruppe in quel corridoio, sovrastando il sommesso chiacchiericcio delle due infermiere. Senza dare il minimo segno di sorpresa, la più anziana resse la sigaretta in bocca e preparò la mano chiusa a pugno. La più giovane la imitò, e le due cominciarono il loro consueto gioco di morra cinese che aveva in palio il resto della pausa.

La giovane perse, ma provò a rilanciare.

«Un cicchino che è il Fontanelli », fece spassionata, mentre spengeva rassegnata quella che era la penultima sigaretta che aveva.

L'altra infermiera sorrise appena, scuotendo il capo.

La ragazza più giovane si avviò allora verso la fonte di quel rumore, che nel frattempo non aveva dato alcun cenno di cessare. 

Chiaramente, aveva intuito subito chi potesse essere a ridere in quel modo, così sguaiatamente, senza alcun tipo di contegno; seguiva spesso quel paziente, e al momento era l'unico in struttura che non sembrava trarre beneficio dai farmaci. Era un ragazzo giovane e piuttosto educato, che passava la maggior parte del tempo a leggere e scrivere. A volte, le veniva quasi da pensare che fosse stato ricoverato per sbaglio, tanto le sembrava sano di mente rispetto agli altri occupanti della clinica. Eppure, altre volte, si abbandonava a quelli che sembravano essere dei banali episodi psicotici, ma che per qualche strano motivo le medicine non riuscivano a placare.

La giovane infermiera entrò tranquilla nella stanza di quel paziente, mentre egli, afflosciatosi su una sedia, le braccia e la testa abbandonate, stava quasi letteralmente soffocando dalle risate. Con un movimento ben preciso e senza la minima esitazione, la ragazza gli sostenne la testa, assicurandosi che il suo collo rimanesse ben dritto e le sue vie respiratorie libere, fino a che la risata nervosa non fosse passata.

«Oggi, quando sono entrata, non mi ha neanche salutato », disse l'infermiera al paziente, per cercare di distrarlo e agevolare la fine dell'episodio. Per qualche motivo si davano del lei, anche se i due avevano circa la sua stessa età.

Quello, fra una risata e l'altra, riprese parzialmente il controllo delle braccia ed indicò qualcosa sulla sua scrivania.

Lei prese, sempre sorreggendogli la testa

Il Corriere della Sera, prima pagina, taglio medio:
TORNA RAGAZZA SCOMPARSA A BARCELLONA
I GENITORI LA RINNEGANO: NON È LEI

Per quanto la storia potesse essere buffa, non la fece ridere neanche un po'. Lei non rideva mai lei, quando era al lavoro. La sua aura di apatia in qualche modo parve contagiare il suo paziente, che riprese gradualmente il controllo di sé.

«Come va col suo lavoro, Fontanelli? » lo distrasse la ragazza, mentre gli sentiva il polso.

«Credo\ldots credo di aver finito », ansimò quello.

«Butterà via tutto e ricomincerà da capo? »

«Credo proprio di sì. »

L'infermiera soppresse il sorriso che le si voleva dipingere sul volto. Non doveva sorridere, a lavoro. «Guardi che non deve comportarsi da pazzo solo perché la teniamo qui »

Lui cercò il suo sguardo.

«Ah, no? » le disse, ostentando un disarmante candore.

La ragazza si lasciò sfuggire uno sbuffo pericolosamente simile ad una risata.

«Mi scusi per stamani », proseguì il ragazzo. «Ero\ldots di pessimo umore. »

«Si figuri », fece lei automaticamente. «Adesso, invece? Si sente di buon umore? »

Lui non rispose subito. La guardò per qualche istante dritta negli occhi, uno strano scintillio in quello sguardo. Quando finalmente parlò, le disse: «Diciamo\ldots che ho appena visto un'opportunità di fare qualcosa di divertente. »

Lei lo studiò un istante. Era la prima volta che lo vedeva così animato; forse avrebbero dovuto rincarare la sua dose di medicine. «E cosa sarebbe? Vuole raccontarmelo? »

Il ragazzo esibì un sorriso inquietante: gli angoli delle sua labbra si erano arricciati in un modo che era semplicemente malvagio, perverso.

«No, per ora credo di no », dichiarò con l'aria di nascondere chissà cosa. «Credo che sarebbe assai più appropriato se glielo dicessi domani mattina.»

***

L'infermiera non si stupì più di tanto quando, l'indomani, nella stanza assegnata a Fabio Fontanelli non trovò nessuno. Poche parole scritte di fretta su di un fogliaccio annunciavano il suo addio.

Cara inf. S. F.,

capita la battuta? Spero di no, perché mi accorgo ora che potrebbe rimanerci male.

Ho una cosetta simpatica da fare, perciò starò via per un po'. Non si preoccupi per me, ora che non sono più lì da voi non mi sento più moralmente obbligato ad avere qualche psicosi una volta ogni tanto.

Spero di non averla messa in un bel casino. Non so chi è responsabile per me in quella metaforica-ma-neanche-del-tutto gabbia di matti: se è lei, sappia che mi dispiace. Casomai questo la portasse a dichiarare che la sua vita è ormai rovinata, e che non le rimane altro da fare che ricominciare da capo cambiando nome e connotati, mi sento di consigliarle caldamente di non farlo - se non altro, perché quella è la mia follia e, se deve impazzire, che se ne trovi una tutta sua.

Non mi dimenticherò di lei, anzi: le prometto che, appena mi annoierò di quello che sto andando a fare, verrò a trovarla.

Suo,
F.F.

Per la prima volta da quando lavorava lì, quella ragazza si concesse un vero, sincero sorriso.

P.S. Io non avrò più il tempo di comportarmi da pazzo, è vero, ma qualcuno dovrà pur farlo: le lascio in eredità un piccolo assaggio del mio tormento: l'ultima stesura delle mie regole - con cui ormai dovrebbe avere una certa familiarità. Le legga, se le va; poi le butti via e le riscriva da capo, cambiando poco o niente. Ripeta questa procedura fino a due volte al giorno, preferibilmente dopo i pasti. Se possibile, le peggiori ad ogni iterazione, e si lamenti costantemente di quanto era migliore la stesura originale. Confido che lo farà.

Il sorriso dell'infermiera si allargò.

Stare al Mondo

Parte prima: Criptici Avvertimenti

1 - Non perdere mai di vista l’orizzonte, ma soprattutto guarda dove metti i piedi.

1 bis - Non fissare il sole troppo a lungo, o non ti accorgerai che ci sono anche altre stelle.

2 - Esitare è facile, ma raramente è utile. Pensare è costoso, ma spesso è saggio.

3 - Gli impulsi vanno presi in considerazione; non semplicemente seguiti, né soppressi.

Parte seconda: Relazionarsi con La Realtà

4 - Le sue interpretazioni possono essere tante, ma la realtà è una sola.

5 - Il mondo dentro la tua testa ed il mondo reale sono ben distinti. Dedica ogni sforzo possibile per mantenerli - se non coincidenti - almeno simili.

6 - Il modello binario del bene e del male è pura fantasia, ma comunque molto efficace. Se lo usi, scegli a priori le definizioni di bene e male con estrema attenzione.

Parte terza: Relazionarsi con Te Stesso

7 - Tempo ed energia sono tutto ciò che hai: usale con parsimonia.

8 - Sii severo nell’individuare i tuoi errori, ma indulgente nel perdonarteli.

9 - Se sei forte, il tuo interesse è apparire innocuo; se sei debole, il tuo interesse è diventare forte.

10 - Scegli bene le battaglie che vuoi combattere. Se riesci, è meglio scegliere quelle che puoi vincere.

Parte quarta: Relazionarsi con Gli Altri

11 - Non valutare una persona solo in base ai suoi interessi o alle sue preferenze, ma non ignorarne le implicazioni.

12 - Quando sei frainteso, anche se qualcuno è effettivamente troppo stupido per capire, tu comunque non sei stato in grado di spiegare.

12 bis - Discorsi semplici per menti semplici; serba la complessità per chi è in grado di capirla.

13 - Quando sei tentato di disperarti per la stupidità di qualcuno, pensa a quelle volte in cui tu sei stato ancora più stupido.

14 - Essere gentili è faticoso, ma spesso vale lo sforzo: la gentilezza è gratuita, farsi perdonare no.

15 - Menti solo quando non puoi assolutamente evitarlo. La verità è la tua bandiera, la menzogna il tuo pugnale.

La ragazza piegò con cura quelle carte e le nascose sotto il camice. Qualcosa in lei era cambiato.

\chapter{Titoli di Coda, parte 2}

Casa Circondariale di Prato "La Dogaia", un altro giorno a caso.

Un uomo barbuto, alto ma un po' torto, rimirava pensieroso il prodotto di quello che, dall'inizio della prigionia, era stato il suo unico passatempo.

La Repubblica, prima pagina, taglio medio:
TORNA DOPO MESI E FA ARRESTARE UN MORTO
LA FIDANZATA: "CHE CAZZO SUCCEDE?"

La Nazione, prima pagina, titolo principale:
RAGAZZO SCOMPARSO DENUNCIA IMPRENDITORE MORTO: 
CARABINIERI ARRESTANO TUTTI E DUE

Il Tirreno, prima pagina, articolo di fondo:
IMPRENDITORE CREDUTO MORTO È VIVO: ARRESTATO
SOSPETTATA PISTA MAFIOSA NELLA VENDITA DEL SUO LANIFICIO

Notizie di Prato, articolo online:
BAGONGHI È VIVO, LEKA E' MORTO E FONTANELLI E' TORNATO
SCONCERTANTE LA VERITA' DIETRO LA SURREALE VICENDA

La svolta nella vicenda che ha coinvolto l'ormai tristemente famoso Lanificio Bagonghi, a dir poco, rasenta il ridicolo: Fabio Fontanelli, la cui scomparsa era stata denunciata ormai da mesi, si è presentato ieri pomeriggio alla stazione dei carabinieri di Poggio a Caiano in compagnia di Bruno Bagonghi, il noto rampollo della famiglia Bagonghi che aveva ceduto lo scorso mese il suo storico, omonimo lanificio ad una holding cinese, appena qualche giorno prima di darsi per morto con tanto di finto funerale. Fontanelli ha poi consegnato una pistola, dei documenti falsi ed una certa quantità di sostanze stupefacenti, ed ha inoltre confessato una lunga serie di crimini. I due sono stati immediatamente arrestati, e [...]

Con molta calma, l'uomo staccò gli occhi quei frammenti di articoli, concedendo uno sguardo alla copia - intera - de La Repubblica che aveva appena ricevuto: il titolo dell'articolo di fondo era per lui di estremo interesse.

SCOMPARE, RITORNA CON UN MORTO E SCOMPARE DI NUOVO
LA (EX?) FIDANZATA: "ORA MI SONO ROTTA IL CAZZO"

L'uomo barbuto lasciò perdere per un attimo la sua rassegna stampa, e rivolse un'occhiata assente al poco che vedeva del mondo esterno. Si lasciò tangere il volto dalla timida luce del sole che filtrava attraverso le sbarre, quasi come se temesse di passare per quella minuscola finestra. 

Sospirò.

"Fabio... questa te la faccio ricacare", pensò per l'ennesima volta, lasciandosi corrompere dal rancore. "Non so come e non so quando, ma giuro che questa te la faccio proprio ricacare".

Lo sguardo gli corse nuovamente al giornale. Sorrise: la partita non era ancora persa, era solo cambiato il suo avversario.


\end{document}
