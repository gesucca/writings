\chapter{Il Remo}

\begin{chapquote}{Author's name, \textit{Source of this quote}}
``This is a quote and I don't know who said this.''
\end{chapquote}

% pulled

Una nave da crociera solcava pigramente il Mar Mediterraneo. Era una bellissima giornata, fredda ma soleggiata. Tutti i passeggeri erano intenti a godersi il viaggio, osservando la costa, il mare o dilettandosi in qualche attività sul largo ponte scoperto; tutti, tranne uno. Fabio fissava pensieroso un minuscolo taccuino. Il vento spiegazzava incurante quelle piccole pagine, gli schizzi delle onde ogni tanto le bagnavano. Lui ignorava tutto, immerso nelle parole confuse che aveva scritto.

"Regola 27: le buone maniere sono gratis, abusane più che puoi."

Qualcosa non gli tornava. Pensò ancora qualche istante, poi tracciò altri segni confusi.

"Regola 28: i tuoi interessi invece sono cari, molto cari; non sacrificarli in nome della cortesia."

Le cose che non gli tornavano continuarono a non tornargli. L'istinto gli diceva di fondere quelle due regole in una sola, ma le varie combinazioni che gli venivano in mente somigliavano troppo a una citazione che aveva letto da qualche parte. Avrebbe voluto riportarla per intero, però non ricordava la formulazione esatta e tanto meno sapeva a chi attribuirla. Forse sarebbe stato meglio lasciare le due regole divise, rendendo la ventotto un comma della ventisette...

Ponderò relativamente a lungo la faccenda, poi decise di lasciar perdere. Non c'era bisogno di impazzire per questioni del genere, non quando stava già impazzendo senza fare ulteriori sforzi. Bloccarsi a rimuginare sui dettagli sarebbe stato utile se, ad esempio, quella fosse stata la sua unica occasione di stilare la costituzione mondiale; visto che non la era, avrebbe dovuto tirar via a buttare giù una bozza completa invece che esitare su ogni singolo punto, o sarebbe andato poco lontano. Improvvisamente arrivò un'intuizione: prese a scribacchiare.

"Regola 29: esitare è facile, ma raramente è utile."

Quella era decisamente una citazione di un qualche libro, ma non poteva costringersi a ricordare dove la aveva letta. Obbedendo alla regola, non esitò e passò oltre. Gli venne subito un'altra idea:

"Regola 30: finché qualcuno non reclama una proprietà intellettuale, è tua."

« Che scrivi adesso? », flautò una voce da un altro mondo.

La bolla di follia in cui Fabio si era isolato scoppiò, rivelandogli i dintorni. Comparvero una ragazza magra e bionda, il ponte di una nave e il mare.

« Le stesse cose di stamani » sospirò Fabio, sconsolato.

Strappò l'ultima pagina del suo taccuino e se la cacciò in bocca.

« Ancora le tue regole? Non stai facendo molti progressi, vero? », tubò Vittoria.

Fabio non si disturbò a rispondere: era troppo intento a masticare i suoi scritti.

« Io non me ne intendo », mise le mani avanti lei, « ma sai che forse faresti meglio a lasciar perdere? Va bene pensare, ma figa a una certa anche basta. »

Fabio interruppe la masticazione e la fissò per qualche istante.

« No, non scriverlo, ti prego! », si affrettò ad aggiungere lei. « Dai, molla quel quadernino. È da stamattina che sei in sbatti, oggi non ti ho visto sorridere nemmeno una volta. »

La replica arrivò fin troppo in fretta:

« Vonvollio vovviee - puh! Non voglio sorridere! »

« E invece sorriderai, perché io ti voglio vedere contento. Dai, facciamo qualcosa insieme! Guarda che bel sole, il mare è bello, non fa troppo freddo... facciamoci mezza goccia in due! Ci passa prima ancora di arrivare. »

Fabio rifletté per un istante. Senza pensarci, strappò un'altra pagina di scritti e fece per sbranarla, ma Vittoria lo bloccò repentinamente. Lui non protestò; mise semplicemente il taccuino in tasca, estrasse un pacchetto di sigarette e se ne accese una. Fece un tiro profondo e, finalmente, rispose:

« La droga, eh? Potrebbe essere la risposta, sì, ma non adesso. Tu fai pure, io credo proprio di non volermi buttare in mare dalla disperazione. Ricordo benissimo quanto ho sofferto l'ultima volta. »

Vittoria si incupì.

« Ti ricordi che sei stato male, però hai dimenticato quanto siamo stati bene insieme. »

Fabio sospirò. Prese un'altra boccata di fumo, poi gettò la sigaretta in mare nonostante avesse fatto soltanto due tiri. Distrattamente, ne estrasse un'altra dal pacchetto e la accese. Esitò per dei lunghi istanti prima di rispondere:

« Ti sbagli, non l'ho affatto dimenticato. »

Fece una breve pausa, poi aggiunse:

« Su una cosa però hai ragione, devo proprio prendermi una pausa dai miei pensieri. Andiamo a prendere una cosa da bere e godiamoci il viaggio. »

Così fecero. Il bar era ben fornito e la crociera in effetti avrebbe potuto rivelarsi molto piacevole, ma Fabio non riusciva proprio a stare sereno. C'era qualcosa che tornava poco nella loro situazione; anche se non riusciva a formulare concretamente il suo dubbio, avvertiva distintamente la sensazione di stare commettendo un grave errore.

Stranamente, sembrava andare tutto per il meglio. All'imbarco nessuno aveva chiesto documenti o si era interessato al contenuto dei loro bagagli; avevano solo preso il biglietto - caro asserpentato, non aveva mancato di commentare Fabio - ed erano subito saliti a bordo senza tante storie. Riguardo al loro intento di usare la crociera per emigrare da clandestini, il piano era così semplice da non poter andare storto: quando la nave avrebbe fatto porto per i rifornimenti, loro sarebbero scesi di soppiatto; se qualcuno li avesse beccati, sarebbe stato messo in condizione di non parlare. Non era certo quello che lo preoccupava; aveva alle spalle esperienze così intense da rendere lo sfuggire alla sorveglianza di una nave una mera bazzecola.

La faccenda di Leghorn, però, a Fabio proprio non tornava. Avrebbe giurato di non essersi mai imbattuto nel nome di quella città in tutta la sua vita, ma un'immagine continuava a comparire nella sua testa: il ricordo di un cartello con scritto proprio quel termine. Che fosse dannato se fosse riuscito a ricordarsi dove diavolo lo aveva visto. E poi, Leghorn: gamba-corno? Che sarebbe dovuto significare? Inizialmente si era limitato a etichettare la città come un qualsiasi porto sulle coste inglesi, ma nel corso del viaggio la sua convinzione si era progressivamente sgretolata, al punto di rendergli quasi impossibile ignorare il fatto che in realtà non sapeva dove diavolo si trovasse la sua destinazione.

Il comandante della nave aveva parlato varie volte dagli altoparlanti, probabilmente illustrando le caratteristiche delle località che stavano costeggiando, ma Fabio non aveva capito molto. Forse per l'inglese strascicato o per il rimbombo metallico che accompagnava ogni comunicazione, fatto sta che in media era riuscito ad intendere una parola su dieci. E quelle poche che era riuscito ad udire correttamente non gli tornavano per niente.

« Brr - Bzz - Mar -sei - krrrrr »

« hhhhgggg - Naiss - ssssssrrrrr »

Marsiglia? Nizza? Che diavolo ci facevano sulle coste della Francia? Si sarebbe aspettato che gli venisse indicato Valencia, Gibilterra e poi qualche posto in Portogallo, prima che la nave facesse finalmente porto a Leghorn. Avrebbe pensato di aver sentito male - in effetti, da quell'impianto venivano fuori tanti di quei ronzii da diventar scemi ad ascoltarli - però, stando sulla prua della piccola nave, la costa rimaneva a sinistra. Se due indizi facevano una prova, stavano proprio costeggiando la Francia, diretti verso...

Nonostante Fabio cercasse di ignorare tutto questo con tutte le sue forze, l'illuminazione si innescò quando la voce metallica - tra una robotica gracchiata e l'altra - annunciò abbastanza chiaramente che stavano costeggiando il Principato di Monaco. Dovette accettarlo: non stavano andando in Inghilterra. Ma allora, in nome del cielo, dove erano diretti? Che c'era dopo la Francia? Cosa avrebbe potuto chiamarsi Leghorn?

Era appoggiato sul candeliere di prua a sorseggiare un ottimo Gin Tonic quando finalmente lo capì, e per poco la bevuta non gli cadde in mare. Quel cartello nella sua testa... Era un ricordo vecchio di quasi vent'anni. In automobile, al porto di Olbia, lui giocava con il Game Boy, non riusciva a sconfiggere la settima palestra, suo padre che bestemmiava in cerca dell'imbarco corretto...

Non stavano affatto andando in Inghilterra. Quanto era stato stupido, era ovvio, ovvio! Non era proprio possibile andare in Inghilterra da Barcellona in così poco tempo! Come aveva fatto a convincersi del contrario, come era stato possibile ingannarsi così?

« Uuuh, guardaaaa! Io lì ci sono stata! », civettò Vittoria, anche lei abbandonata sul candeliere.

« Tra poco saremo in posti dove sono stato anche io », mormorò Fabio con voce funerea.

Lei lo ignorò, completamente presa dal paesaggio.

« Lì c'era l'albergo dove eravamo! Abbiamo visto il Gran Premo dal terrazzo di camera! »

Fabio bestemmiò. L'unica volta che aveva assistito dal vivo a un gran premio di Formula 1 aveva speso un patrimonio per stare in un prato fangoso, facendo a cazzotti per intravedere mezza curva da una rete.

« Dai, non fare così » gli disse la ragazza, appoggiandosi a lui. « Lo stiamo facendo, stiamo andando in un posto nuovo a rifarci una vita! Sei sempre triste? »

Fabio scrutò l'orizzonte. La sua espressione era così dura che sembrava scolpita nella roccia. Dopo una lunga pausa meditabonda, rispose:

« Mi dispiace fare il bastian contrario, ma ogni cosa che hai detto è sbagliata. Non stiamo andando in un posto nuovo. Non ci rifaremo una vita e no, non sono triste. »

Lei lo abbracciò. Lui non ricambiò l'abbraccio, ma neanche lo respinse.

« A me invece non dispiace, e ti dico che non mi importa niente di quello che hai detto. Tranne che non sei più triste. »

Fabio ignorò il moto di insofferenza che lo investì. Replicò secco:

« Non sono neanche felice, se ti può interessare. »

Lei lo strinse a sé ancora di più.

« Vedrai che fra poco starai meglio. Saremo in una città nuova, nessuno ci darà fastidio. Solo io e te. Mi prenderò cura di te così bene ti farò passare la voglia di essere cattivo. »

Fabio provò l'impulso di fingere un conato di vomito, ma lo soppresse. Lo aveva scritto qualche ora prima, in uno degli innumerevoli foglietti finiti in mare o nel suo stomaco: non c'era alcun vantaggio nel ferire gratuitamente i sentimenti della ragazza. Riluttante, ricambiò l'abbraccio e tacque.

***

Le ore trascorsero; il sole sprofondò nel mare, lasciando spazio ad un magnifico cielo stellato. L'aria era pungente: sul ponte ormai non c'era più nessuno, tutti i passeggeri si erano rifugiati sottocoperta. La voce metallica del comandante gracchiò:

« kkkaaa- Leee - ghorn - drrrggoff de ship for any rrreezzzonzzz - brrr »

Enio Filippi dormiva sodo negli alloggi riservati all'equipaggio. Un collega lo svegliò senza tante cerimonie: toccava a lui fare rifornimento. Enio maledisse il clero, il cielo e varie divinità, ma tutto sommato si alzò di buon umore: era nella sua Livorno, e fare porto a casa dava sempre una bella sensazione.

Caricare i rifornimenti era sempre un letale miscuglio di fatica e noia: lo sforzo fisico impediva di fare quattro chiacchiere coi colleghi, la noia costringeva la mente a vagare in cerca di qualcosa con cui intrattenerla. Quella sera poi, con l'equipaggio ridotto al minimo, le sue braccia erano le uniche su cui poteva contare per tirar dentro le casse abbandonate sulla banchina dal fornitore. Ma Enio era un uomo di sostanza, non si sarebbe certo tirato indietro di fronte a un compito ingrato, ma tutto sommato banale. Così, cuffie nelle orecchie e maniche arricciate nonostante il freddo, il suo corpo si muoveva automaticamente, come un meccanismo ben oliato che scorreva nelle guide scavate da una ferrea memoria muscolare. Immerso in quel trance lavorativo, per Enio sarebbe stato molto facile perdersi qualche dettaglio di ciò che stava succedendo in quella stiva buia.

Tuttavia, una bella biondina che sbirciava oltre una pila di bancali non poteva certo passare inosservata. Trattenendo a stento un sorriso, mollò sul posto la cassa di bibite che aveva fra le braccia e le andò incontro.

« Bimba, 'sa ci fai vi? Devi tornà di sopra coll'altri! » le disse, tiratosi via un auricolare.

Lei lo guardò con un'espressione indecifrabile.

« Sei sporco » gli disse sognante. « Hai un odore strano e parli in modo buffo. Sei interessante, però ora... foeura di ball! »

Enio rimase completamente senza parole.

Mentre pensava a come avrebbe potuto replicare a delle affermazioni del genere, percepì un movimento dietro di sé. Si voltò di scatto e vide una figura maschile, seminascosta dall'ombra.

Si impaurì istantaneamente e sbottò:

« Dé, ma se' scemo? Mi fai veni' un coccolone, te e quer tegame di tu ma'! »

Il figuro rise; una risata moderatamente acuta, un po' gracchiante: forse era solo un ragazzo.

« Ti ho spaventato? Le mie scuse » fece, beffardo. «Vittoria, bel tentativo, ma devi parlare nella sua lingua per farti capire. Ora, vediamo... il mio livornese è un po' arrugginito, ma il pratese andrà comunque bene: nini, sorti da i' cazzo. Che ha' 'nteso? »

« Te capì? », fece eco la bionda.

Enio si rilassò. Passato lo spavento, la situazione non sembrava pericolosa. Un ragazzo e una bella figliola che cercano privacy in una stiva buia, beh, era stato giovane anche lui e sapeva fare due più due. Mise una mano sulle spalle di entrambi i ragazzi e disse:

« Bimbi, io c'ho da fare. Se vi riesce di tenevvi addosso le mutande per dieci minuti, io quando ho fatto mi le'o di 'culo e vi ci fate 'osa vi pare. Va bene? »

Il giovanotto si tolse di dosso la mano di Enio con deliberata precisione. Non sorrideva più.

« Ascoltami, marinaio » disse a voce bassa. « Ascoltami bene, e fai del tuo meglio per capire. Noi scendiamo. Tu ci lasci passare, riprendi il tuo lavoro e ti dimentichi di tutto questo. Non ho molta voglia di assecondarti e la tua vita per me non vale nulla: mi obbedirai, o sarà peggio per te. Detto in altre parole, e queste sono sicuro che le intenderai: 'canzati, o piglio 'l remo. »

Enio non rispose. Chi diavolo si credeva di essere quel giovanotto? Fece scrocchiare le dita delle mani e gli lanciò uno sguardo ammonitore. Lo avrebbe riportato di peso sul ponte se non si fosse dato una regolata.

« Molto bene » proseguì il ragazzo. « Io ti avevo avvertito. Ora piglio 'l remo. »

Enio non si sarebbe mai immaginato quello che seguì.

Il prossimo, pirandelliano ma breve capitolo sarà pubblicato il 31 maggio 2018.

-Simone



