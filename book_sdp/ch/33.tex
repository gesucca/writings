\chapter{Titoli di Coda, parte 2}

Casa Circondariale di Prato "La Dogaia", un altro giorno a caso.

Un uomo barbuto, alto ma un po' torto, rimirava pensieroso il prodotto di quello che, dall'inizio della prigionia, era stato il suo unico passatempo.

La Repubblica, prima pagina, taglio medio:
TORNA DOPO MESI E FA ARRESTARE UN MORTO
LA FIDANZATA: "CHE CAZZO SUCCEDE?"

La Nazione, prima pagina, titolo principale:
RAGAZZO SCOMPARSO DENUNCIA IMPRENDITORE MORTO: 
CARABINIERI ARRESTANO TUTTI E DUE

Il Tirreno, prima pagina, articolo di fondo:
IMPRENDITORE CREDUTO MORTO È VIVO: ARRESTATO
SOSPETTATA PISTA MAFIOSA NELLA VENDITA DEL SUO LANIFICIO

Notizie di Prato, articolo online:
BAGONGHI È VIVO, LEKA E' MORTO E FONTANELLI E' TORNATO
SCONCERTANTE LA VERITA' DIETRO LA SURREALE VICENDA

La svolta nella vicenda che ha coinvolto l'ormai tristemente famoso Lanificio Bagonghi, a dir poco, rasenta il ridicolo: Fabio Fontanelli, la cui scomparsa era stata denunciata ormai da mesi, si è presentato ieri pomeriggio alla stazione dei carabinieri di Poggio a Caiano in compagnia di Bruno Bagonghi, il noto rampollo della famiglia Bagonghi che aveva ceduto lo scorso mese il suo storico, omonimo lanificio ad una holding cinese, appena qualche giorno prima di darsi per morto con tanto di finto funerale. Fontanelli ha poi consegnato una pistola, dei documenti falsi ed una certa quantità di sostanze stupefacenti, ed ha inoltre confessato una lunga serie di crimini. I due sono stati immediatamente arrestati, e [...]

Con molta calma, l'uomo staccò gli occhi quei frammenti di articoli, concedendo uno sguardo alla copia - intera - de La Repubblica che aveva appena ricevuto: il titolo dell'articolo di fondo era per lui di estremo interesse.

SCOMPARE, RITORNA CON UN MORTO E SCOMPARE DI NUOVO
LA (EX?) FIDANZATA: "ORA MI SONO ROTTA IL CAZZO"

L'uomo barbuto lasciò perdere per un attimo la sua rassegna stampa, e rivolse un'occhiata assente al poco che vedeva del mondo esterno. Si lasciò tangere il volto dalla timida luce del sole che filtrava attraverso le sbarre, quasi come se temesse di passare per quella minuscola finestra. 

Sospirò.

"Fabio... questa te la faccio ricacare", pensò per l'ennesima volta, lasciandosi corrompere dal rancore. "Non so come e non so quando, ma giuro che questa te la faccio proprio ricacare".

Lo sguardo gli corse nuovamente al giornale. Sorrise: la partita non era ancora persa, era solo cambiato il suo avversario.
