\chapter{L'Ingegnere}

% pulled

Vittoria malediceva la pioggia e il suo compagno Jorge mentre, incappucciata, si faceva largo in una piccola foresta di ombrelli che sciamavano verso il parcheggio del piccolo cimitero di Chiesanuova.

« Signore! Signoreee! », chiamò a gran voce, cercando di sovrastare il rumore della pioggia. « Quell'uomo con l'ombrello grigio chiaro! Dico a lei, signore! SIGN - figa, però, eh! Mica posso urlare! »

Se solo ne fosse stata sicura, Vittoria avrebbe urlato il suo nome. Govi? Le pareva Govi, ma... No, non se lo ricordava proprio. In ogni caso, ne aveva avuto abbastanza di strillare: si fece largo fra la gente con assai poca grazia e, quasi saltando, raggiunse e agguantò per una spalla il tizio per cui si era sgolata.

« Scusi, eh! », esordì aggressiva, con un tono di voce che poco si addiceva a delle scuse. « Sono due ore che la chiamo! »

Quello non si scompose affatto. Era un uomo snello e molto alto, dall'età indecifrabile. Vittoria si sentì immediatamente un po' a disagio al suo cospetto.

« Le chiedo scusa », disse lui a voce alta, ma tranquilla. « Fra le persone e la pioggia, non l'ho proprio sentita. Ah, ma non ha l'ombrello, è completamente bagnata! Venga sotto il mio, ci stiamo in due, la accompagno alla sua automobile. Ci conosciamo? »

« Ehm... diciamo che la conosco di parola », rispose Vittoria, togliendosi il cappuccio e incamminandosi a fianco di quella persona. Un grande imbarazzo stava cominciando a farsi largo in lei; maledì mentalmente Jorge, e si rimproverò per non aver insistito sul restare insieme.

« Mi conosce di parola... », le fece eco quel signore, squadrandola con evidente sospetto.

Passò qualche istante di imbarazzante silenzio, durante il quale Vittoria si accorse che forse approcciare in quel modo uno sconosciuto non era stata poi una grande idea. Cercò di darsi coraggio e decise che, se fosse sopravvissuta alla situazione, avrebbe preso Jorge e gli avrebbe ficcato l'ombrello di quel tizio dritto nel...

« No, mi arrendo », sospirò finalmente il figuro, sfoggiando un sorriso incoraggiante. « Così non riesco proprio a indovinare. Mi dia qualche indizio, la prego. »

« Facciamo così », tentò lei, un po' rincuorata dall'atteggiamento del suo interlocutore. « Io le dico tutto quello che so, ma lei non mi uccide. »

L'uomo rise in modo assai strano; si poteva azzardare che fosse una risata sincera, ma la compostezza con cui rideva era così impeccabile da far pensare che in realtà facesse finta.

« Oh, cielo », flautò non appena riprese fiato. « Mi scusi, ma la sua battuta mi ha fatto proprio divertire. »

Vittoria rispose istintivamente:

« Mi scusi lei, ma visto che è un amico di Jorge... insomma, lo sa, uccidere è un po' il vostro trend! »

L'uomo smise di camminare di colpo e Vittoria, che se ne accorse un instante dopo, si ritrovò sotto l'acqua a capelli scoperti. Con un solo passo egli le si avvicinò rapidamente, coprendola di nuovo con l'ombrello.

« Mi dica subito chi diamine è lei », le disse con voce tranquilla, torreggiandole davanti e guardandola dall'alto in basso.

Il cuore di Vittoria le saltò in gola.

« Io... Ecco, cioè, lei non mi conosce, che... che le devo dire? Mi manda Jorge, dice che è un suo amico, vuole sapere delle cose, ma... ma io che ne so! »

L'uomo alto la guardò intensamente per dei lunghi momenti. Sembrava che stesse soppesando una decisione.

« Mi descriva il suo mandante, per favore », chiese dopo un po'.

« E'... è alto, non come lei ma comunque alto, e... »

« E' Fontanelli », fece con decisione a sé stesso, interrompendo bruscamente Vittoria e lasciandola interdetta.

« Lei non ha un mezzo, vero? », proseguì egli, cordiale. « Venga con me, così può raccontarmi tutto. Non la ucciderò, stia tranquilla. Quando avremo finito di informarci a vicenda, la lascerò andare viva e vegeta. »

Questo ultimo commento fu accompagnato da una risata sommessa. Vittoria sorrise a sua volta, più a disagio che mai in quella situazione. Acconsentì; non che avesse avuto altra scelta.

Cercò comunque di non perdersi di spirito.

« Può prestarmi il suo ombrello, dopo? », civettò, mentre quel signore le apriva la portiera di una elegante auto tedesca. « Avrei da farci una certa cosa... »

***

In una polverosa e muffita stanza di un capannone, un uomo tanto snello quanto distinto stava finendo di spiegare gli esiti di una certa situazione a una giovane ragazza bionda.

« ...così il signor Bagonghi ha offerto al Fontanelli la possibilità di ottenere una nuova identità. Ora le è più chiaro il perché del comportamento del suo amico, signorina Meis? »

Vittoria annuì, sconvolta dal sentirsi confermare per filo e per segno tutte le cose che aveva sempre sospettato sul suo Jorge. Quel Govidi - di, Govi-di, non Govi! - le aveva rivelato tutto, schiettamente, senza allusioni o vaghe attenuazioni per addolcire una pillola che, come si aspettava, si stava rivelando amarissima da mandare giù.

Fabio si chiamava, altro che Jorge! Fabio Fontanelli, fidanzato e convivente da un'eternità con una sua coetanea, Lavinia Gori. Una brutta situazione lavorativa, dei dubbi sulla fedeltà della sua compagna di vita e l'abuso di alcool e droghe leggere lo avevano portato dalla depressione fino alla follia, facendolo scappare da tutto quanto, in cerca di chissà che cosa. E in effetti qualcosa, nel suo folle viaggio lontano da sé stesso, Fabio Fontanelli lo aveva trovato: lei, Vittoria. Ma, evidentemente, lei non era quello di cui lui aveva bisogno. Nei lunghi mesi che avevano passato insieme, il cammino di Jorge - di Fabio! - verso l'abisso non si era arrestato. Di tutto quello che aveva appena capito sul suo amato, per qualche ragione, quello era sicuramente l'aspetto che la destabilizzava di più.

« La vedo delusa », proseguì il signor Govidi. « Spero di non averla ferita troppo raccontandole queste vicende in modo così diretto, ma il tempo purtroppo è tiranno, non potevo fare altrimenti. »

« Non - non si preoccupi, si figuri » fece Vittoria con un filo di voce, a un passo dal pianto.

« Per quello che può valere, sono molto dispiaciuto per lei. Sembra chiaro che prova qualcosa per il signor Fontanelli, ma - se vuole il mio consiglio - lo lasci perdere. Detto in confidenza, sono rimasto vagamente sorpreso nell'apprendere sia ancora vivo. »

Vittoria non rispose: era troppo occupata a trattenersi dal gettarsi a terra dalla disperazione.

Lo sguardo di Govidi vagò per qualche istante lontano da Vittoria.

« Maledetto bastardo », sussurrò fra sé, « ...eppure gli hanno fatto il culo più che a una puttana. »

Vittoria si riscosse improvvisamente dal suo trance pensieroso.

« Cosa? », fece sorpresa, « Ma dice a me? »

Govidi esibì di nuovo la sua risata intensa, ma composta.

« No, no, si figuri se mi riferivo a lei! », fece divertito. « Lasci perdere, stavo solo... Ah, ma che mi importa? Questo lo posso raccontare: stavo ripensando che ero così sicuro della sorte che avrebbe atteso il Fontanelli, che l'ho dato per suicida entro un mese in una scommessa col signor Bagonghi. Una cena contro un caffé, capisce quanto lo davo probabile? Ero proprio certo di averci indovinato, già pregustavo il caffé che avrei vinto, invece... »

« Invece ha perso, ma ha comunque risparmiato la cena, no? », civettò Vittoria, senza riuscire a trattenersi.

Govidi le scoccò uno sguardo penetrante per un lungo istante.

« Scusi... », fece lei, la voce piccola piccola.

« Non serve nessuna scusa, la battuta poteva essere divertente », replicò Govidi, che non sembrava affatto divertito. La stava guardando in un modo molto strano.

« Anzi » fece, scomparendo in un cassetto della sua scrivania, « a proposito di questo... »

Un piccolo aggeggio venne lanciato bruscamente verso Vittoria, che lo prese al volo. Sembrava un telefono cellulare molto vecchio, con uno schermo piccolissimo.

« Ci siamo fatti domande e dati risposte » disse Govidi, riemergendo dal caos del suo cassetto. « Ora però devo farlo una richiesta. »

Vittoria si limitò a guardare il suo interlocutore, in attesa. Notò che la luce calda della vecchia lampada a incandescenza si rifletteva in modo non omogeneo sulla faccia di Govidi; le sembrava come se una certa zona del suo volto fosse più opaca delle altre. Che si fosse truccato? Scacciò quel pensiero, quasi divertita, e abbassò lo sguardo. Si sentiva di nuovo un po' a disagio.

Dopo un lunghissimo istante, Govidi parlò:

« Mi dica: Fontanelli è in sé? »

Vittoria sgranò gli occhi. Non sapeva proprio che cosa rispondere a una domanda come quella.

« Credo... », tentò, « credo di sì, cioè, è stato male, poi fa cose che... oh, fa niente, lasci perdere, non glielo so proprio dire! »

Govidi insisté: « Si sforzi, la prego. Devo saperlo. »

« Il punto è che... cioè, a questo punto, non so proprio cosa pensare di lui! »

Vittoria cominciò a singhiozzare. Govidi non la incalzò, ma stava chiaramente aspettando che lei dicesse qualcos'altro.

« Cre-credevo che fosse un ragazzo... forte! un... un figo! Ora scopro che... che è una merda, ecco! Non è pazzo, è... è proprio una merda, non so come altro dirlo! »

Govidi continuò a scrutarla imperturbabile. Quando fu palese che Vittoria non avrebbe saputo dirgli altro, parlò con voce calma:

« Non si preoccupi, va bene così. Conosco il Fontanelli da tanto quanto il signor Bagonghi: se lei mi dice che è una merda, significa che è totalmente in sé. »

Govidi chiuse gli occhi e si massaggiò per un attimo le tempie.

« Ecco la mia richiesta », annunciò, gli occhi ancora chiusi. « Ora le darò delle istruzioni precise, che lei seguirà alla lettera. Mi è sembrata una ragazza sveglia: confido che lei sia in grado di ricordare esattamente quanto le sto per dire. »

Vittoria annuì, ma Govidi non aveva ancora riaperto gli occhi per vederla. Proseguì comunque, senza aspettare risposta:

« Appena lo incontrerà di nuovo, lei ricorderà a Fabio Fontanelli il favore che ha promesso al signor Bagonghi. Se non dovesse rammentarlo, o non volesse più farlo, lei convincerà, persuaderà od obbligherà con qualsiasi mezzo il Fontanelli a visitare la tomba del signor Bagonghi. »

Vittoria si sforzò per tenere a mente quanto le era stato detto, ma le venne subito un dubbio.

« Mi scusi », fece a Govidi, « io ci posso provare, ma lei lo sa quanto è testardo Jorge - sì, insomma, Fabio! ».

« Lei lo convincerà, persuaderà od obbligherà con qualsiasi mezzo », ripeté lui. « Non c'è altra via: gli eventi dovranno seguire esattamente il percorso che le ho descritto. »

Vittoria si esibì in una perfetta espressione di totale impotenza. Govidi si bloccò per qualche secondo, poi sospirò profondamente e si stropicciò gli occhi con le mani.

« Senta », fece stancamente, « sarò un po' meno sottile di quanto è mio solito essere, e farò persino appello al suo lato umano. Mi creda: lei non ha idea di quanto dovrò rompermi i coglioni se non ci fosse verso di portare il Fontanelli in quel cazzo di cimitero. Lei è in gamba, perciò faccia pure ricorso a tutta la sua creatività. Lo seduca, lo inganni, lo accoppi e ce lo trascini, faccia un po' come le pare. La prego: lo porti lì il prima possibile. »

« Farò del mio meglio », pigolò Vittoria, per niente convinta di farcela. Preoccupata, aggiunse: « Ma non so mica se ci riuscirò, eh! »

« Il suo meglio basterà, ne sono certo », sentenziò Govidi. « Appena sarete nel cimitero, lei userà il cercapersone che le ho dato per avvertirmi, e io manderò qualcuno ad aiutarvi. Pensa di poterlo fare? »

Vittoria annuì con vigore. Sentirsi dire che poteva essere aiutata le aveva infuso improvvisamente un po' di speranza.

« Molto bene », fece lui in tono definitivo. « Come le ho già detto, il tempo è tiranno. Mi perdoni la sgarbatezza, ma devo congedarla. Non posso proprio accompagnarla in auto, dovrà cavarsela da sola. Prenda questi, chiami un taxi - anzi, fa niente, glielo chiamo io fra un istante, ma adesso la prego di lasciarmi solo. »

Senza dire altro, Vittoria uscì dall'ufficio di quello strano tipo. Era quasi a metà del corridoio quando la voce di Govidi la raggiunse:

« Signorina Meis! Prenda il mio ombrello! Ormai avrà smesso di piovere, ma mi aveva detto che le sarebbe servito! »