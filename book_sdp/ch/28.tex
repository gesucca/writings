\chapter{cap 28}

Il sole finalmente sorse, e la luce dell'alba inondò senza pietà il triste cimitero di Chiesanuova. Fabio strinse gli occhi e maledisse il cielo, rimuginando sulla natura della sua avversione per il sole
mattutino.

Un irritante stridolio proveniente dalla cappella lo riscosse dai suoi vaneggiamenti: «\emph{Eeeeeeeeek} --- »

Un sordo \emph{thud} lo interruppe bruscamente.

« \ldots{}cazzo te strilli, CHE CAZZO TE STRILLI?! » tuonò un'altra voce impastata, più aggressiva ma assai meno irritante dell'altra.

«I miei capelli sono un disastro\ldots{}» piagnucolò quella che sicuramente era Vittoria.

Fabio rise più forte di quanto sarebbe stato opportuno fare in un cimitero, e fu subito investito da una valanga di striduli insulti.

Ci fu un altro colpo secco e lo starnazzio cessò. Dalla cappella emersero Vittoria e Denise, avvinghiate in quella che sembrava essere una mezza presa di wrestling.

«Ma povera, lasciala subito!» scimmiottò Fabio.

Denise sbadigliò copiosamente prima di rispondere. «Sta' zitto» biascicò, la voce distorta dallo sbadiglio. «Non sono una persona mattiniera. E comunque, buongiorno.»

Vittoria prese a divincolarsi e ricominciò a stridere: «\emph{Lasciamiiiii} --- »

In un attimo, Denise cambiò la presa e soffocò il fastidioso lamento con un braccio. «\emph{Buongiorno}, ho detto», scandì minacciosa, come se volesse scolpire quel concetto direttamente sul cranio di Vittoria.

Fabio non aveva idea di come reagire a tutto quello. «Voi due siete qualcosa di male», disse sconsolato. «Temo di aver commesso un grave errore a farvi conoscere.»

Le due ragazze si esibirono all'unisono in una folle risata isterica.

«Allora», fece Denise, non appena le due si furono calmate. Lanciò a Fabio un cauto sguardo indagatore, e proseguì: «Sù, non lasciarci sulle spine: come ti senti?»

Fabio trasse un profondo, stanco sospiro. «Questa, vecchie mie, è proprio una bella domanda.»

«Lo so», lo incalzò Denise. «Mi piacciono le belle domande. La sai la risposta?»

Fabio sospirò ancora. Esitò un attimo, poi rispose quasi in un sussurro: «Mi sento strano.»

«Si capisce», rispose subito Denise. «Questo acido è una roba incredibile, non avevo mai provato niente del genere. Ho visto cose --- ho \emph{fatto} cose che\ldots{} beh, non avrei mai creduto possibili.»

Vittoria le lanciò una fugace occhiata e arrossì violentemente. Fabio proseguì, apparentemente non interessato alla faccenda: «Anche io ho visto cose impossibili; è una droga allucinogena, in fondo. Ma non è quello che mi fa sentire strano. Piuttosto, è quello di cui mi sono accorto che mi turba. Ho avuto varie\ldots{} intuizioni, chiamiamole così, vari assaggi di sentimenti di cui non mi ero ancora accorto. Però, se dovessi dirti che ho capito tutto, ti direi sicuramente una cazzata. »

Sul volto di Denise fu chiaramente visibile il dispiacere. «Non è servito proprio a niente?», chiese con la voce piccola piccola. 

«Oltre a prendere tutti e tre una bellissima polmonite?» sbottò Fabio, un po' più tagliente di quanto non avrebbe voluto essere.

Denise abbassò lo sguardo. «Le droghe psichedeliche favoriscono l'introspezione», sussurrò, una tristezza assoluta intrisa nelle sue parole. «Pensavo che potesse aiutarti. Davvero. A me ha aiutato, credo. A te non è servito, si capisce; niente di quello che faccio serve mai a nessuno.»

Se non avesse conosciuto bene la su amica, Fabio avrebbe potuto giurare di aver sentito un accenno di singhiozzo. «Denise\ldots{}» disse piano, tendendo una mano incerta verso la sua amica.

«Non importa», fece lei, riscuotendosi. Guardò Fabio dritto negli occhi. «Quello che ho fatto l'ho fatto perché lo ritenevo giusto, e se non è servito a niente, o se addirittura ti ha fatto male\ldots{} ti chiedo scusa e ne trarrò le conseguenze. Ma se tornassi indietro lo rifarei, perché l'ho fatto perché mi sembrava giusto.»

Fabio non sapeva cosa dirle; evidentemente, l'esperienza mistica non aveva scosso soltanto lui.

«Hai fatto la cosa giusta», disse improvvisamente Vittoria, la voce bassa ma sicura.

Denise sbatté più volte le palpebre. «Grazie», le disse incerta. 

«Anche lui pensa che tu abbia fatto bene», proseguì. Un'impercettibile, strano sorriso incurvò lievemente le sua labbra. «E anche io, o non sarei stata d'accordo a farlo.»

«La pivella ha ragione», concordò Fabio, cogliendo l'espressione di Vittoria e provando a sdrammatizzare.

// una nuvola deve coprire il sole quando si intristiscono

Quando nessuna reazione arrivò, tornò serio e proseguì. «Non ce l'ho con voi. Sono grande e vaccinato, e ho acconsentito a farlo. Non sono certo un tipo da soccombere alla peer pressure, se non mi fosse andato bene semplicemente non ve lo avrei fatto fare. E comunque, Denise, non è stato inutile: non ci ho capito molto in quel turbinio di pensieri che mi ha investito, ma credo\ldots{}» il suo sguardo si spostò su Vittoria, «\ldots{} credo proprio di aver capito l'unica cosa importante, la chiave per arrivare a tutte le altre.»

Senza preavviso, Vittoria si avvicinò a Fabio e lo abbracciò.

«E' da tanto che -» disse, e la sua voce fu rotta da un violento singhiozzo. Si strinse a Fabio ancora più forte. «E' da quando sei tornato dal porto che - che lo so, e che cerco di aiutarti, ma - non sapevo che fare - poi »

Il sole riapparve improvvisamente.

Fabio la strinse a sé con decisione, cercando di tranquillizzare il suo pianto. Scoccò un'occhiata a Denise, che lo guardava triste. «Ebbene sì, care mie», disse sorridendo. «Sono proprio partito di cervello. Pazzo al cento per cento, e non un pazzo tranqui -»

Il mondo si congelò all'istante.
