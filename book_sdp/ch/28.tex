\chapter{Titoli di Coda}


Casa Circondariale di Prato "La Dogaia", un giorno a caso di un mese a caso. Un uomo barbuto, alto ma un po' torto, rimirava pensieroso il prodotto di quello che, dall'inizio della prigionia, era stato il suo unico passatempo.

La Repubblica, prima pagina, taglio medio:
TORNA DOPO MESI E FA ARRESTARE UN MORTO
LA FIDANZATA: "CHE CAZZO SUCCEDE?"

La Nazione, prima pagina, titolo principale
RAGAZZO SCOMPARSO DENUNCIA IMPRENDITORE MORTO: CARABINIERI ARRESTANO TUTTI

Il Tirreno, ...
IMPRENDITORE CREDUTO MORTO È VIVO: ARRESTATO

Notizie di Prato
BAGONGHI È VIVO, LEKA E' MORTO E FONTANELLI E' TORNATO
SCONCERTANTE LA VERITA' DIETRO LA SURREALE VICENDA

Una storia che ha dell'incredibile: Fabio Fontanelli, la cui scomparsa era stata denunciata ormai da mesi, si è presentato ieri pomeriggio alla stazione dei carabinieri di Poggio a Caiano in compagnia di Bruno Bagonghi, il noto rampollo della famiglia Bagonghi che aveva ceduto lo scorso mese il suo storico omonimo lanificio ad una holding cinese, appena qualche giorno prima di darsi per morto, con tanto di finto funerale. Fontanelli ha poi consegnato una pistola, dei documenti falsi ed una certa quantità di sostanze stupefacenti, ed ha inoltre confessato una lunga serie di crimini.

Raggiunta con difficoltà, l'ex fidanzata di Fontanelli ha rilasciato poche, confuse dichiarazioni.

L'uomo barbuto lasciò perdere per un attimo la sua rassegna stampa, rivolgendo uno sguardo assente al poco che vedeva del mondo esterno. "Fabio... questa te la faccio ricacare", pensò per l'ennesima volta. "Non so come, non so quando, ma giuro che questa te la faccio proprio ricacare".


un manicomio chissà dove

Una ragazza riccia e una sua collega più anziana discutevano svogliatamente del più e del meno, sorseggiando di tanto in tanto un pessimo caffé. Il fumo della sigaretta di una di loro non riusciva a uscire del tutto dal minuscolo abbino che, aperto, costituiva l'unica fonte di luce naturale di cui godeva quel corridoio. L'aria ingrigita di quell'ambiente era perfettamente in linea con il loro umore: dopo anni di servizio in quel manicomio, una profonda apatia era l'unica difesa possibile contro il peso schiacciante di una sanità mentale che lì, era più unica che rara.

Una risata sguaiata eruppe in quel corridoio, sovrastando il sommesso chiacchiericcio delle due infermiere. Senza dare il minimo segno di sorpresa, la più anziana resse la sigaretta in bocca e preparò la mano chiusa a pugno. La più giovane la imitò, e le due cominciarono il loro consuero gioco di morra cinese, che aveva in palio il resto della pausa.

La giovane perse, e si avviò verso la fonte di quel rumore, che nel frattempo non aveva dato alcun cenno di cessare. Ovviamente, sapeva già chi era a ridere in quel modo; quel paziente non dava quasi mai problemi, ricordò senza particolare emozione, era sempre educato e la maggior parte del tempo faceva quasi pensare che fosse in quella struttura per sbaglio, tanto sembrava sano di mente. Eppure, a volte, si abbandonava a quelli che sembravano essere dei banali episodi psicotici, ma che per qualche strano motivo i farmaci non riuscivano a placare.

La giovane infermiera entrò tranquilla nella stanza di quel paziente, mentre egli, abbandonato su una sedia, le braccia e la testa abbandonate, stava quasi letteralmente soffocando dalle risate. Con un movimento ben preciso e senza la minima esitazione, l'infermiera gli sostenne la testa, assicurandosi che il suo collo rimanesse ben dritto e le sue vie respiratorie libere fino a che la risata nervosa non fosse passata.

< Oggi, quando sono entrata, non mi ha neanche salutato >, disse l'infermiera al paziente, per cercare di distrarlo e agevolare la fine dell'episodio. Gli dava del lei, anche se probabilmente aveva la sua stessa età.

Quello, fra una risata e l'altra, riprese parzialmente il controllo delle braccia ed indicò qualcosa sulla sua scrivania.

Lei lo prese, sempre sorreggendogli la testa

Il Corriere della Sera, prima pagina, taglio medio:
TORNA RAGAZZA SCOMPARSA A BARCELLONA
I GENITORI: NON È LEI

Per quanto la storia poteva essere buffa, a lei non face ridere neanche un po'. non rideva mai lei, quando era al lavoro.

<>

< Mi scusi per stamani. Ero... di pessimo umore >

< Si figuri. Adesso invece? si sente di buon umore? >

Lui la guardò dritta negli occhi, uno strano scintillio in quello sguardo.

< Diciamo... che ho appena visto un'opportunità di fare qualcosa di divertente. >

< E cosa sarebbe? Vuole raccontarmelo? >

< No, per ora credo di no. Credo che sarebbe assai più appropriato se glielo dicessi domani mattina.>

L'infermiera non si stupì più di tanto quando, l'indomani, nel letto assegnato a Fabio Fontanelli non trovò nessuno.
