\chapter{Titoli di Coda}

Casa Circondariale di Prato "La Dogaia", un altro giorno a caso.

Un uomo barbuto, alto ma un po' torto, rimirava pensieroso il prodotto di quello che, dall'inizio della prigionia, era stato il suo unico passatempo.

La Repubblica, prima pagina, taglio medio:
TORNA DOPO MESI E FA ARRESTARE UN MORTO
LA FIDANZATA: "CHE CAZZO SUCCEDE?"

La Nazione, prima pagina, titolo principale
RAGAZZO SCOMPARSO DENUNCIA IMPRENDITORE MORTO: CARABINIERI ARRESTANO TUTTI

Il Tirreno, ...
IMPRENDITORE CREDUTO MORTO È VIVO: ARRESTATO
SOSPETTATA PISTA MAFIOSA NELLA VENDITA DEL SUO LANIFICIO

Notizie di Prato
BAGONGHI È VIVO, LEKA E' MORTO E FONTANELLI E' TORNATO
SCONCERTANTE LA VERITA' DIETRO LA SURREALE VICENDA

Una storia che ha dell'incredibile: Fabio Fontanelli, la cui scomparsa era stata denunciata ormai da mesi, si è presentato ieri pomeriggio alla stazione dei carabinieri di Poggio a Caiano in compagnia di Bruno Bagonghi, il noto rampollo della famiglia Bagonghi che aveva ceduto lo scorso mese il suo storico omonimo lanificio ad una holding cinese, appena qualche giorno prima di darsi per morto, con tanto di finto funerale. Fontanelli ha poi consegnato una pistola, dei documenti falsi ed una certa quantità di sostanze stupefacenti, ed ha inoltre confessato una lunga serie di crimini.

Raggiunta con difficoltà, l'ex fidanzata di Fontanelli ha rilasciato poche, confuse [...]

Con molta calma, l'uomo staccò gli occhi quei frammenti di articoli, concedendo uno sguardo al giornale intero che aveva appena ricevuto.

IL TIRRENO
FONTANELLI SCOMPARSO

L'uomo barbuto lasciò perdere per un attimo la sua rassegna stampa, rivolgendo un'occhiata assente al poco che vedeva del mondo esterno. Si lasciò tangere il volto dalla timida luce del sole che filtrava attraverso le sbarre della minuscola finestra della sua cella, e sospirò. "Fabio... questa te la faccio ricacare", pensò per l'ennesima volta. "Non so come, non so quando, ma giuro che questa te la faccio proprio ricacare".

Lo sguardo gli corse nuovamente al giornale e sorrise. La partita non era finita: era solo cambiato il suo avversario.
