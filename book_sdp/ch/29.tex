\chapter{cap 29}

// separare fabio e lavinia punti di vista

Il cuore di Fabio si era fermato. Sulla soglia del grosso, rugginoso
cancello, inondata dallo scintillante sole mattutino, si stagliava
un'inconfondibile figura. Occhi gonfi e sguardo basso, passo lento e
rassegnato, un crisantemo fresco in mano: Lavinia Gori si trascinava sul
vialetto del cimitero con l'aria di chi ha subito qualche colpo di
troppo dalla vita.

Né Denise né Vittoria osarono muoversi o fiatare. Ne avevano parlato la
sera precedente, dopo essere scampate al finimondo per un pelo, quando
per puro miracolo Lavinia e Fabio non si erano notati. Le due ragazze
erano d'accordo: sarebbe potuto andare disastrosamente male, ma se Fabio
voleva avere anche solo una minima speranza di tornare in sé, doveva
affrontare apertamente quello spettro del passato.

Senza parlare, senza respirare o anche solo sbattere le palpebre, Fabio
lentamente si mosse; la testa alta, il passo incredulo ma fremente,
prese il vialetto e andò incontro al proprio destino.

(\textless{}Scommettiamo che la uccide?\textgreater{}, bisbigliò
Vittoria alla sua compagna, la quale la zittì con una violenta gomitata
nelle costole.)

Non appena i due furono a opchi passi di distanza, Lavinia alzò
distrattamente lo sguardo da terra e si fermò.
\textless{}Sì?\textgreater{} chiese distrattamnete al tizio che le si
era parato davanti.

Fabio non rispose subito. Vedere la sua ex compagna di vita in quello
stato gli causava una strana sensazione. Portava il vestito buono, il
più elegante che aveva; tuttavia era spiegazzato, non propriamente
sporco ma comunque si vedeva che non era fresco di armadio. Lo avevano
comprato insieme, in un caldissimo sabato pomeriggio di luglio, in una
delle rarissime volte in cui Fabio aveva acconsentito a fare un uscita
mirata per fare shopping a buon mercato. Un'increspatura quasi apparve
sulle sue labbra al ricordo: Fabio odiava quelle spedizioni, ma quasi
sempre tornava a casa con un bottino migliore della sua compagna.
L'increspatura scomparve subito non appena realizzò l'ovvia
implicazione: si era messa quel vestito perché era l'unico vestito nero
che possedeva. Non amava vestire di nero, Lavinia Gori; non si vestiva
di nero quasi mai, né nelle serate metal al Cipher, né al concerto
stesso degli Iron Maiden; nemmeno alle serate eleganti per le quali
aveva appositamente comprato quel capo d'abbigliamento. Solo la morte di
un caro amico la aveva fatta vestire di nero.

Fabio si sentiva molto strano. Molti ricordi di vita quotidina con
Lavinia gli stavano balenando in mente, scuotendo intensamente le sue
fondamenta; era come se la memoria gli stesse tornando dopo tanto tempo,
ma Fabio non credevadi aver mai dimenticato niente. Aveva evitato quei
ricordi perché gli facevano male, ed anche adesso che gli si
presentavano all'attenzione contro la sua volontà non era sicuro di che
cosa provava.

La voce di Lavinia lo riscosse. , fece, la voce piccola e distratta,
\textless{}avrei da pass - \textgreater{}

I due sguardi si incontrarono, e il mondo intero rabbrividì di nuovo.

Fabio si sentì morire dentro nel rivedere quegli occhi. Luminosi come il
sole, chiari come il cielo. Gonfi, stanchi, specchio di un'anima ferita
e disperata.

, fece Fabio con un filo di voce.

Lavinia era come pietrificata.

Fabio proseguì: \textless{}Dimmi, Lavinia, se una luce ne oscura altre è
sempre luce?\textgreater{}

?? Una nuvola passeggera coprì coreograficamente il sole.

Lavinia non parve far caso a quelle parole. La sua attenzione era fissa
su Fabio, il suo sgurado lo esplorava avidamente, come se volesse capire
se era davvero lui. \textless{}Come stai?\textgreater{} gli chiese con
le lacrime che stavano cominciando a sgorgare dai suoi occhi.

\textless{}Se un bene fa del male, è davvero bene?\textgreater{} domandò
di nuovo Fabio, gli occhi fissi sulla sua ex compagna di vita.
\textless{}Se per amore menti, o rubi, o uccidi, sei comunque una
persona buona?\textgreater{}

Un barlume rosso attraversò brevemente gli occhi di Fabio. Sul suo volto
si dipinse un'espressione dura, risoluta, e i suoi occhi divennero duri,
insostenibili, come se stesse cercando di incendiare la sua vecchia
compagna di vita con il solo sguardo.

(\textless{}Dai, venti euro su lei morta, niente morti si
patta?\textgreater{}, bisbigliò Vittoria alla sua compagna, la quale
rispose nuovamente con un altro violento colpo di gomito.)

Fabio tirò dritto col suo discorso. \textless{}Se ogni tanto sbirci
nell'abisso solo per vedere l'effetto che fa, e ogni tanto ti concedi
qualche piccola deviazione dal sentiero giusto, per prenderti quella
piccola soddisfazione che ti fa andare avanti, sei sempre
buono?\textgreater{}

La sua voce si indurì ancora di più. Una terribile furia si era come
impadronita di ui.

\textless{}Se invece a un certo non ne puoi semplicemente più e dici
basta, e cominci a cercare di vivere peggio che puoi, tuffandoti in un
abisso oscuro senza la minima esitazione ogni volta che ne hai
l'occasione con la folle, disperata speranza di perderti, di non essere
più lo stesso quando torni su. Ma sono solo momenti, e quando passano ti
rendi conto che in realtà sei sempre il solito coglione di sempre,
soffocato dallo stesso dolore, condannato a patire per sempre senza
avere altro che insignificanti vendette contro il mondo ogni volta in
cui fai del male a qualcuno! Dimmi, Lavinia - se questo è il tuo
destino, sei una persona buona?\textgreater{}

Lavinia parve aver ascoltato attentamente di ciò che Fabio aveva detto.
I loro sguardi si incrociarono di nuovo, e nel suo cominciarono a
comparire delle lacrime. \textless{}Questo --- questo che c'entra? E'
così che stai?\textgreater{} chiese, la voce piccola e già rotta dal
pianto.

La faccia di Fabio si raggrinzì su sé stessa, come se si fosse
improvvisamente rotta. Tutto il suo dolore parve mostrarsi su di essa. ,
disse in un patitico squitìo.

Durò solo un'istante.

L'ardore ritornò potente nelle sue parole quando proseguì. Lo sguardo
che Fabio le restituì avrebbe potuto ucciderla all'istante.
\textless{}Scusami, non è importante. Sto esattamente come l'ultima
volta che mi hai visto. Non ti interessava allora, non vedo perché
dovrebbe interessarti adesso.\textgreater{}

Qualcosa dentro Lavinia parve rompersi; quasi inciampò sui suoi piedi
pur essendo ferma, ed emise un penoso lamento interrogativo.

Fabio non le dette tempo di aggiungere altro. \textless{}Lasciamo
perdere, parliamo un momento di te. Se una ragazza non ha né arte né
parte in tutto il mondo, e vuole solo raccogliere dalla vita il meglio
che trova, magari può trovarsi un coglione qualunque che la desidera per
viverci insieme e dividersi le fatiche della vita. Se poi ne trova
addirittura un altro, e se in fondo gli sta più che bene perché alla
fine le attenzioni gli piacciono, può succedere che alla fine lei si
inganni per addolcire la realtà, per mascherarsi il fatto che in realtà
è una stronza opportunista che mantiene una relazione solo per il comodo
di dividere gli oneri della vita, magari in attesa del momento opportuno
per fare il ribaltone e andare con l'altro tizio. E se questa ragazza
non si accorge - o addirittura non gli importa dell'effetto che il suo
essere troia fino al midollo fa alle persone che gli stanno intorno e
che magari le vogliono bene, questa ragazza è o non è una
puttana?\textgreater{}

Fabio aveva le fauci asciutte. Ribolliva di rabbia, il suo discorso
aveva seguito uncrescendo di itnensità tale dall'aver ringhiato le
ultime parole.

Lavinia aveva ascoltato rapita ogni cosa che aveva detto Fabio. Era
pallida, sudata, addolorata dal sentir dire quelle parole dalla persona
che amava; eppur enon disse ancora niente. Le sue mani stringevano così
forte il crisantemo che aveva preso per il Bagonghi da avern quasi
stritolato il gambo.

\textless{}Comunque, non ha più importanza\textgreater{}, concluse
Fabio, ora calmo, una nota di rassegnazione nella voce. \textless{}Sono
solo cose che mi sono accorto che ti avrei voluto dire, ma non abbiamo
più avuto occasioni di parlare con franchezza da tanto, tanto
tempo.\textgreater{}

Fabio si voltò con una certa, bizzarra solennità. \textless{}Offri pure
quel fiore al povero Bruno, dagli l'ultimo saluto come meglio credi, poi
vattene.\textgreater{}

Singhiozzò.

\textless{}Continua ad essere felice, te che riesci a farlo. Non hai
bisogno di Bruno\ldots{} e nemmeno di me.\textgreater{}

Se ne andò con passo lento verso le due ragazze, che Lavinia aveva
totalmene ignorato fino a quel momento.

Una sola parola, piatta, incredula, uscì dalla bocca della ragazza
mentre guardava la figura del suo compagno di vita allontanarsi da lei:
.

Incrociò lo sguardo di Denise, la quale si limitò a guardarla e ad
alzare leggermente le mani in segno di rassegnazione.

, ripeté con più vigore.

Il suo ragazzo era scomparso. Un loro amico aveva scritto a tutti di
averlo visto, poi era scomparso anche lui. Un altro loro amico era stato
ammazzato dall'ennesimo amico, che era a sua volta sparito.

, ripeté con ancora più vigore.

Ora, il suo ragazzo si era presentato dopo \emph{mesi}, concio come le
bestie, per dire cose a caso e prenderla a parole.

\textless{}Cosa \emph{cazzo} - \textgreater{}

Lavinia esplose.

Buttò via il crisantemo ormai stritolato, si tolse una scarpa e la
lanciò dietro a Fabio con tutta la forza che aveva.

 urlò, completamente fuori di sé. \textless{}FERMO --- FERMO, \emph{DIO
B ---}\textgreater{} lanciò l'altra scarpa.

Fabio si fermò, ma non si voltò.

\textless{}Ora te\ldots{} sì, *ora te `ttu vieni qui\emph{\textgreater{}
ringhiò, la voce vibrante per la rabbia. \textless{}}Ora 'ttu pigli, tu
'tti metti fermino\emph{, sì, e }'ttu* mi spieghi un po' CHE CAZZO STA
SUCCEDENDO!\textgreater{}

L'ultimo grido riecheggiò sinistramente in tutto il cimitero.

(Poco lontano, Vittoria bisbigliò: \textless{}Ultima offerta, dieci euro
morto lui!\textgreater{} Denise non si disturbò a rimproverarla)

Lavinia continuò, schiumante di ira: \textless{}*E' da quande t'ha'
preso e ttu se' spari'o che gl'è tutto un succede' di roba che nemmeno a
i' telegiornale! Ho chiamato e' harabinieri, la polizia, la guardia
nazionale, i' porco d'Idd --- nulla! Poi gl'è spari'o anche i' Brogelli!
Ora i' Bagonghi s'è fatto ammazzare da i'Leka, e tu'sentissi icché mi
vien'a 'ddire quell'attro rintrona'o d'i Gazzi, tu gli
stiaccerest'iccapo!\textgreater{}

Fabio si voltò di scatto a guardare la sua vecchia compagna di vita, il
volto solcato dal dolore e da un'improvvisa rabbia. , sbottò a denti
stretti, \textless{}il tuo amichetto Giacomo farà presto la fine che
---\textgreater{}

\textless{}CHETO!\textgreater{}, ruggì ferocemente Lavinia,
interrompendolo e quasi spettinandolo. La ragazza fece un aggressivo
passo in avanti; Fabio non indietreggiò, ma non poté fare a meno di
sussultare di fronte a quella furia.

Lavinia sputò le parole come fossero veleno. \textless{}Amichetto?
Quella serpe!? *Se `ttu sentissi la metà d'icché m'ha racconta'o* --- te
un tu'nnha' un'idea! Ogni volta che ne succedeva una, lui l'era sempre a
dire male di quello e di quell'altro, se uno mi diceva coppe lui mi
veniva a dire che era picche! Io sta'o `n pena, e lui gl'era lì a dì
male di te, di me e di tutti quelli che m'hanno ma' detto una parola di
conforto!\textgreater{}

Il mondo di Fabio si inceppò. I colori, ancora fin troppo vividi a causa
della droga, si spensero di colpo, ed il suo occhio sinistro si
contrasse di scatto per alcune volte.

, disse piano, senza emozione.

\textless{}Cosa?\textgreater{}, gli fece eco Lavinia.

Fabio prese un bel respiro. , scandì cautamente \textless{}che
\emph{sai} che il Gazzi è una serpe. Lo \emph{sai} che per lui mettere
zizzania è come respirare, e lo \emph{sai} che ti ha sempre raccontato
un sacco di cazzate. Stai --- stai dicendo questo?\textgreater{}

\textless{}Eh!\textgreater{} fece Lavinia, insofferente. \textless{}Che
mi se' diventa'o grullo? Tu lo sa'`nche te come fa! Gl'ha scritto su i'
gruppo che t'ha' fatto degl'affari che neanche la mi nonna quand'e
guarda Quarto Grado! Prima t'ave'i la russa, poi la cinese, poi e'
debiti di gioco* - \textgreater{} La voce della ragazza fu rotta da un
singhiozzo. \textless{}\emph{Te un tu'nn'ha' un'idea\ldots{}} Io\ldots{}
Mesi e mesi a sentirmi dire queste cose su di te\ldots{} Senza potergli
dire di chiudere quella fogna e di andare a nascondersi, senza potergli
andare n'i'vviso a digli --- lascia fare, guarda!\textgreater{}

Il cerchio si chiuse, e Fabio andò finalmente a sbattere alla velocità
della luce contro il granitico muro della realtà.

Il suo più intimo terrore era lì, di fronte a lui. Non aveva via
d'uscita: lo disse ad alta voce. \textless{}Pensavo che tu non mi
volessi più. Che non vedessi l'ora di stare con lui.\textgreater{}

Lavinia spalancò la bocca per lo stupore e si bloccò per alcuni secondi.
\textless{}EH?\textgreater{} urlò finalmente, non appena riuscì a dar
voce alla sua incredulità.

(\textless{}Ecco, ora si rimettono insieme\ldots{}\textgreater{} si
lagnò Vittoria, prima di essere violentemente zittita per la terza
volta.)

Lavinia sembrava non trovare le parole adatte per esprimere quello che
provava. \textless{}Fabio\ldots{}\textgreater{} tentò, ma il suo
tentativo si estinse subito.

Fabio non disse niente; non aveva proprio altro da dire.

Il silenzio si prolungò e i due se ne stettero lì, immobili, per diversi
istanti, entrambi traditi dalla loro eloquenza.

Una voce rassicurante ruppe lo stallo. \textless{}Siamo tutti un po'
troppo agitati\textgreater{}, disse piano ma con decisione Denise, che
nel frattempo si era avvicinata a Fabio e Lavinia. \textless{}Agitati
tutti quanti, noi tre addirittura scompigliati e infreddoliti. Che ne
dite di fare un bel respiro, sospendere per un attimo tutto quello che
stava succedendo e andare a raccattare i nostri cocci? Il Bagonghi tanto
non scappa, torneremo a trovarlo quando ci saremo presi cura di
noi.\textgreater{}

Il buonsenso di quelle parole permeò subito tutti quanti.

 fece Fabio. \textless{}Io ho freddo. Cazzo se ho freddo -- non mi ero
accorto fino a ora di avere freddo.\textgreater{}

 fece Lavinia.

 pigolò Vittoria. \textless{}Devo farmi la doccia, e poi la piastra ---
avete una piastra e una spazzola da prestarmi? Ah, e anche una
doccia.\textgreater{}

\textless{}Te chi sei?\textgreater{} chiese debolmente Lavinia, ma
Denise la bloccò subito: \textless{}Prima le docce e le piastre, poi le
presentazioni. Abbiamo tante cose da raccontarcdi l'un l'altro, si
capisce.\textgreater{}

Senza dire altro, la strana comitiva che si era appena formata si
apprestò a uscire dal cimitero, con la speranza di lasciarsi alle spalle
non soltanto quel luogo maledetto, ma anche tutta la confusione, i dubbi
e, soprattutto, il freddo che ognuno di loro aveva patito.
