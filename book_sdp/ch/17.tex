\chapter{Uno e Nessuno}

\begin{chapquote}{Author's name, \textit{Source of this quote}}
``This is a quote and I don't know who said this.''
\end{chapquote}

% pulled

Una piccola nave era ormeggiata al porto di Livorno. La sua stiva era fredda, umida e buia. Quel poco che si vedeva era illuminato solo dalla fioca luce che penetrava dalla banchina. C'era rumore e odore di salmastro, ma i sensi più acuti avrebbero potuto individuare dei singhiozzi e un caratteristico sentore metallico. Erano in tre in quella stiva: un elfo, un nano... ah, no, quella è un'altra storia. [1] In questa c'erano un ragazzo alto, una bella biondina e un uomo adulto che piangeva come un bambino, rannicchiato a terra nel suo sangue; se quello fosse stato l'inizio di una barzelletta, nessuno avrebbe potuto dire di averla già sentita.

Il ragazzo alto si avvicinò all'uomo, esaminando divertito la ferita da arma da fuoco che gli aveva causato.

« Dai, dai... », commentò con fare canzonatorio, « non fare così, ti ho appena regalato qualche giorno di malattia, no? »

Il giovane ridacchiò, divertito dalla sua stessa battuta.

« Farò il signore », proseguì, « ed eviterò di dirti che ti avevo avvertito. Se non mi sbaglio, ora dovresti essere disposto a prendermi sul serio. Possiamo dire quindi che siamo una squadra? Che lavoriamo tutti e tre per raggiungere il mio obiettivo? »

L'uomo non rispose; gemeva e piangeva, in preda ad un dolore terribile. Il ragazzo si rivolse allora alla sua compagna:

« Diglielo tu, Vittoria: siamo una squadra? »

Lei non sembrava a proprio agio in quella situazione.

« Sì », tentò, « ma non sarebbe meglio andarcene, tipo, non so... adesso? »

Il ragazzo rise. Una risata fredda, precisa, senza la minima parvenza di allegria. Sul suo volto si dipinse un sorriso malvagio.

« Ma come? Vuoi andare via proprio adesso? La festa è appena cominciata! »

Pestò con forza la gamba ferita dell'uomo a terra, che cacciò uno straziante urlo di dolore. Il sorriso sul suo volto vacillò appena.

« Ssshhh! » fece beffardo. « Piano, piano. Non vogliamo che ci senta qualcuno, vero? Non vogliamo altra compagnia, giusto? »

L'uomo forse intravide una speranza in quella provocazione. Prese ad urlare con tutto il fiato che aveva, dimenandosi per cercare di liberare la gamba che veniva torturata. Fu tutto inutile: in uno scatto fulmineo il ragazzo gli fu addosso, la pistola puntata in mezzo agli occhi, e lui si arrese.

« No, no! », gli fece animoso. « Questo non è carino. Credevo che noi fossimo una squadra, che non ci servisse l'aiuto di nessuno! »

Nessuno osò ribattere niente. Sulla stiva scese un silenzio totale, perturbato solo dal respiro irregolare dell'uomo a terra.

Il ragazzo alto si alzò, la sua arma sempre fissa sul bersaglio.

« Sono un tipo orgoglioso, sai? », disse a voce molto bassa. « Orgoglioso e possessivo. Immaginami come una fidanzata appiccicosa. Come reagirei se ti sentissi chiamare a gran voce una tua amica? Oh, reagirei male, molto male. Mi metterei ad urlare, ti prenderei a schiaffi. Sarei così fuori di me che potrei fare qualche pazzia. Ma non succederà, perché tu non vuoi che succeda. Noi non siamo due fidanzatini, siamo adulti. Siamo più furbi, vero? Noi lavoriamo insieme, non ci facciamo questi dispetti. »

Il silenzio tornò per qualche istante, poi la voce della ragazza bionda lo ruppe. Parlò fermamente, ma non riuscì ad occultare del tutto il profondo disagio che provava.

« Jorge, te lo devo proprio dire. C'è davvero bisogno di... ehm... perdere tempo così? Lo so che non lo faresti se non ci fosse un motivo, ma cioè, non è meglio semplicemente andarcene via? »

Il ragazzo le rivolse uno sguardo intenso, pensando a chissà cosa.

« Hai ragione » le disse finalmente. « Non posso controbattere, hai perfettamente ragione. Però è un vero peccato, non credi? Si stava creando un bel rapporto fra me e il nostro marinaio. »

Il suo sguardo si perse in lontananza, nella direzione generica dell'uomo a terra. Un qualcosa di lui sembrò improvvisamente cambiare.

« Lo sai che non posso lasciarti vivere, vero? » disse a voce estremamente bassa.

Non arrivò nessuna risposta, ma il giovane non sembrava aspettarsene una.

« Vittoria, lo sai dove siamo? » chiese alla ragazza, lo sguardo sempre fisso sul vuoto.

« In un posto dove non dovremmo essere. » fece pronta lei. « Per favore, dobbiamo andare via! »

« Sempre concreta » replicò il ragazzo, per niente turbato dal senso di urgenza della sua compagna. « Sempre presente, sempre coerente. Sei capace di astrarti, di vedere oltre, anche troppo oltre a volte, ma rimani sempre ancorata nel tuo centro. Come fai, Vittoria? È davvero la droga che ti aiuta a rimanere te stessa, a non perderti dietro le maschere che indossi? »

« Non... non lo so », rispose lei.

« Non lo so neanche io. »

Per dei lunghi istanti nessuno fiatò. Poi, all'improvviso, il ragazzo si riscosse:

« Basta, leviamoci di torno. Marinaio, tu vuoi vivere o morire? »

L'uomo a terra cercò di mettersi seduto, ma il meglio che riuscì a fare fu sorreggere il busto con le braccia.

« Bah, lascia perdere, non importa » fece sbrigativo il ragazzo alto. « È chiaro che vuoi vivere, a quanto pare tutti vogliono vivere. Beh, sei fortunato, oggi hai vinto il resto della tua vita. »

L'uomo si lasciò cadere pesantemente a terra. Prese a singhiozzare e pronunciò delle parole di ringraziamento.

« Sì, certo, prego! » sbottò il ragazzo. « Quello che mi dice grazie dopo aver preso una pallottola mi mancava. Giuro, un giorno o l'altro capirò che diavolo avete in testa voialtri! Ma ormai non mi stupite più, e forse in realtà nemmeno me ne frega un cazzo. Forza Vittoria, su! O non avevi tanta fretta? »

Senza tante altre cerimonie, i due giovani scomparvero a passo svelto nella notte.

***

Qualche ora più tardi, un arrabbiato barista in cerca di caffè scese nella stiva, trovandovi Enio Filippi incosciente e ferito. Il marinaio fu soccorso ed ebbe fortunatamente salva la vita, pur avendo perso molto sangue. Non seppe dire niente riguardo al suo aggressore; raccontò solo di aver avuto strane allucinazioni, forse causate dal dolore.

Da quella notte in poi, Enio prese a provare una strana diffidenza nei confronti dei ragazzi alti e delle giovani, attraenti ragazze bionde. Sua moglie ne fu più che felice.

NOTA DELL'AUTORE
[1] : scusatemi, non ho potuto resistere!

Il prossimo, mostruoso capitolo sarà pubblicato il 30 giugno 2018.

- Simone



