\chapter{Paranoia}

Ormai si era fatta notte, forse il momento meno adatto per gironzolare nelle profondità de La Barçeloneta.

Cioè, dipende: se sei un tizio losco e pure armato, come Fabio, puoi andare un po' dove ti pare. É proprio per via di gente come lui che certi posti sono più inquietanti di altri; come al solito, il posto in sé non fa niente, è la gente che lo popola il problema.

Quello era un quartiere come un altro, e non faceva eccezione.

Tutto taceva più o meno tranquillo. In lontananza si udivano gli schiamazzi della movida, urla di turisti ubriachi intenti a perdere la loro dignità; qualcuno alloggiava proprio lì, in quei palazzi adornati di asciugamani stesi alle finestre, e non si era neanche disturbato ad uscire per fare festa.

Niente di strano, sospetto o pericoloso; ma per qualche motivo, quella zona male illuminata emanava una sensazione di degrado e di malcelata criminalità.

Non sentendosi al sicuro, Fabio mise una mano sulla pistola e affrettò il passo.

Si sentiva costantemente osservato, controllato dal quartiere stesso. "Non è paranoia se vogliono davvero prenderti", pensò, ben conscio che in realtà nessuno in tutta Barcellona ce l'aveva specificatamente con lui --- per lo meno, non ancora.

Camminò a lungo, accompagnato dalla sua inquietudine.

Ogni palazzo gli sembrava losco quanto l'altro, ogni vicolo tanto deserto quanto il precedente; persino i pochi locali aperti gli facevano uggia, invece di aiutarlo ad orientarsi in quel dedalo di strade

Si perse varie volte, poi, finalmente, dopo un lungo peregrinare, trovò l'edificio che stava cercando.

A un primo sguardo poteva apparire uguale agli altri, ma sul muro aveva quella sorta di segno distintivo di cui il suo amico Bruno gli aveva parlato: una svastica con gli uncini al contrario, tracciata a bomboletta sopra un manifesto di propaganda per l'indipendenza catalana.

Stava per succedere.

Fabio fece un bel respiro, cercando di chiamare a raccolta tutta la sua determinazione.

Si sentiva un po' strano: forse l'esperienza del pomeriggio aveva lasciato il segno, o forse era soltanto una reazione a quello che stava per accadere. Era come se fosse distaccato, scollegato dal suo vero sé; come se stesse pilotando un robot con le sue sembianze.

La paura che provava era un'informazione, non un'emozione.

Non c'era più tempo per farsi venire dei dubbi. Non si trattava più di giocare al delinquente, la cosa era tremendamente più importante: si trattava di \emph{modificarsi}, nel senso più fisico del termine. La faccenda era più seria che mai. Fabio si era isolato, protetto, riparato da qualunque cosa potesse fargli cambiare idea. Non era mai stato sicuro di niente, Fabio. L'incertezza lo aveva sempre attanagliato, strozzato e limitato, ma in compenso, se mai il suo carattere avesse avuto un pregio che fosse uno, era proprio quello di sapersi costringere ad andare sempre fino in fondo alla strada intrapresa, nel bene o nel male.

Era il momento di andare fino in fondo.

Traboccante di ansia, Fabio suonò al citofono e disse la frase che doveva dire, pregando varie divinità in cui non credeva che il Bagonghi non lo avesse preso in giro.

La porta si aprì, e salì al piccolo appartamento.

Successe tutto molto velocemente.

Il dottor Gambino, che sembrava un medico tanto quanto Fabio sembrava una teiera, lo ricevette senza alcun convenevole. Prese i soldi faticosamente rubati, li contò e mostrò una specie di menù con i vari trattamenti. Fabio ci pensò un attimo, cercando di far entrare più modifiche possibili nel suo budget.

Alla fine, decise.

«Il mento più forte, ma non tocchi la barba, se può. Le orecchie, un po' meno...sì, esatto. E la gobba del naso, la tolga, già che c'è.»

«Mento a culo, \emph{'}recchie da Legolas e naso senza gobba?» riassunse il sedicente chirurgo plastico, la voce bassa e impastata. «Molto bene. Stenditi che ti drogo.»

Fabio esitò un istante.

«Secondo lei, dottore» chiese cautamente, «questo è...abbastanza, diciamo così, per rendermi irriconoscibile? Legalmente parlando, intendo.»

Quell'uomo ridacchiò. «Dottore?» grugnì divertito. «Io sono perito elettronico!»

Fabio rimase tradito dalla sua eloquenza.

Il ceffo eruppe in una beffarda risata, ma si ricompose subito. Mise una mano sulla spalla del suo paziente --- o, per meglio dire, della sua vittima --- e parlò.

«Stammi a sentire: c'hai pochi soldi e lo capisco, ma non sei messo bene. Certo, puoi metterlo nel culo a qualche guardia che ti cerca con la foto. A quelli, già solo con i capelli li fai fessi. Però, se si mettono a indagare...cerca di non metterti in guai troppo grossi, ok?»

Alzò le mani come per dire \emph{io ti ho avvertito, e ho fatto anche troppo}.

«E soprattutto», proseguì, la voce un po' indurita, il dialetto marcato, «ora non prendertela, ma io te lo devo dire, lo sai, è la procedura. Insomma, tu non mi hai mai visto. Tu qui non ci sei mai stato, è molto semplice. Sembri un tipo sveglio, e non importa che ti dica che come teniamo per le palle il tuo caro Bagonghi, teniamo per le palle anche te. Non importa che ti dica quanti dei tuoi amici e parenti possono ritrovarsi una bomba nel culo se parli di me, se scrivi di me, se accenni a me o se anche solo pensi a me mentre qualcuno ti guarda negli occhi.»

Fabio non disse niente, ma quell'uomo non sembrava aspettarsi risposta.

«Ora stenditi e dammi il braccio, ti faccio un buchetto.»

Fabio era più inquietato che mai. Non tanto per la minaccia, anzi, forse quella era la parte che lo impensieriva di meno; sapeva perfettamente qual era il tipo di gente con cui aveva a che fare. Era proprio la natura di quello che stava per attenderlo a stritolargli l'anima.

Aveva già deciso, ed agonizzare ancora non gli avrebbe certo fatto riconsiderare la sua scelta. Obbedì e permise al carnefice di fare il suo lavoro, ma non poté fare a meno di cominciare a pensare.

Una plastica totale del viso sarebbe stata molto meglio, ma non se la poteva proprio permettere; né ora, né fra un mese, nemmeno rapinando altre dieci coppie di ricconi. Con un po' di fantasia, comunque, riusciva a farsi piacere gli interventi che aveva scelto: il mento forte, prominente, con carattere; le orecchie affusolate, aerodinamiche; un naso dritto, armonioso ma integerrimo. Con ancora più fantasia, invece, si poteva pensare che quei piccoli ritocchi fossero addirittura ben studiati: nessuno che volesse nascondere la sua identità alla società si sarebbe cambiato i connotati con caratteristiche così appariscenti. A nessun poliziotto sarebbe venuto in mente di indagare più approfonditamente sulla sua identità, a meno che non si fosse fatto coinvolgere in un grosso pasticcio.

Fabio era già in un grosso pasticcio. Era morto, e assomigliare a un morto proprio non gli avrebbe giovato.

Il dubbio tornò, attanagliando la sua coscienza come una pressa idraulica: era proprio sicuro di voler intraprendere questa strada?

Pensò a quello che stava lasciando indietro. Erano cose brutte, molto brutte...

Non poteva sopportarle.

Si soffermò un po' più del solito su queste ultime riflessioni, sebbene non fossero poi così complesse per i suoi standard. Cercò di chiedersi il perché, ma comporre pensieri era diventato molto più difficile del solito: come pianeti in formazione da una nebulosa, l'astratto diffuso nella sua testa aveva bisogno di molto tempo per combinarsi e solidificarsi in concetti concreti.

Si sentiva confuso, obnubilato, distaccato da se stesso.

Il braccio gli pizzicava forte: tentò di guardarlo, ma l'impresa di volgere la testa si dimostrò ben oltre le sue momentanee capacità. Si chiese blandamente che cosa gli stesse succedendo.

«Stai buono, che \emph{la keta} sta facendo effetto», disse un ceffo alla sua destra.

\emph{"La keta?"}, pensò. La domanda sembrò rimbombare nella sua testa, rimbalzando sulle pareti del cranio. "Ah, \emph{la keta...}", concluse.

Non si ricordava più il motivo esatto per il quale era lì. In fondo era un posto piuttosto bruttino per stendersi, perché non era a casa sua? Perché? Non solo non riusciva a rispondersi, ma non ricordava nemmeno la domanda.

La comprensione di quel che stava succedendo era ormai un ricordo: si arrese.

La sua mente si stava avviluppando su sé stessa, lasciandolo passivo in balia dei suoi sensi. Il braccio gli bruciava da morire, si sentiva invadere da uno strano fluido ardente che, inesorabilmente, sopraffaceva ogni parte del corpo che riusciva a raggiungere. Intanto, il soffitto si faceva sempre più grande ogni volta che prestava attenzione a ciò che i suoi occhi mostravano, come se la stanza si espandesse con il ridursi della sua capacità di discernimento.

Gradualmente, il mondo si spense: rimase solo con sé stesso, assistendo impotente agli artefatti sensoriali della sua mente.