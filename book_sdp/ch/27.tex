\chapter{Titoli di Coda, parte 1}

Presidio ospedaliero di Prato "Misericordia e Dolce", un giorno a caso di un mese a caso.

Una ragazza riccia e una sua collega più anziana discutevano svogliatamente del più e del meno, sorseggiando di tanto in tanto un pessimo caffé. Il fumo delle loro sigarette non riusciva a uscire del tutto dal minuscolo abbaino che, aperto, costituiva l'unica fonte di luce naturale di cui godeva quel corridoio. L'aria ingrigita di quell'ambiente era perfettamente in linea con il loro umore: dopo anni di servizio in quel manicomio, una profonda apatia era l'unica difesa possibile contro il peso schiacciante della loro sanità mentale, che lì era più unica che rara.

Una risata sguaiata eruppe in quel corridoio, sovrastando il sommesso chiacchiericcio delle due infermiere. Senza dare il minimo segno di sorpresa, la più anziana resse la sigaretta in bocca e preparò la mano chiusa a pugno. La più giovane la imitò, e le due cominciarono il loro consuero gioco di morra cinese che aveva in palio il resto della pausa.

La giovane perse, ma provò a rilanciare.

< Un cicchino che è il Fontanelli >, fece spassionata, mentre spengeva rassegnata quella che era la penultima sigaretta che aveva.

L'altra infermiera sorrise appena, scuotendo il capo.

La ragazza più giovane si avviò allora verso la fonte di quel rumore, che nel frattempo non aveva dato alcun cenno di cessare. 

Chiaramente, aveva intuito subito chi potesse essere a ridere in quel modo, così sguaiatamente, senza alcun tipo di contegno; seguiva spesso quel paziente, e al momento era l'unico in struttura che non sembrava trarre beneficio dai farmaci. Era un ragazzo giovane e piuttosto educato, che passava la maggior parte del tempo a leggere e scrivere. A volte, le veniva quasi da pensare che fosse stato ricoverato per sbaglio, tanto le sembrava sano di mente rispetto agli altri occupanti della clinica. Eppure, altre volte, si abbandonava a quelli che sembravano essere dei banali episodi psicotici, ma che per qualche strano motivo le medicine non riuscivano a placare.

La giovane infermiera entrò tranquilla nella stanza di quel paziente, mentre egli, afflosciatosi su una sedia, le braccia e la testa abbandonate, stava quasi letteralmente soffocando dalle risate. Con un movimento ben preciso e senza la minima esitazione, la ragazza gli sostenne la testa, assicurandosi che il suo collo rimanesse ben dritto e le sue vie respiratorie libere, fino a che la risata nervosa non fosse passata.

< Oggi, quando sono entrata, non mi ha neanche salutato >, disse l'infermiera al paziente, per cercare di distrarlo e agevolare la fine dell'episodio. Per qualche motivo si davano del lei, anche se i due avevano circa la sua stessa età.

Quello, fra una risata e l'altra, riprese parzialmente il controllo delle braccia ed indicò qualcosa sulla sua scrivania.

Lei prese, sempre sorreggendogli la testa

Il Corriere della Sera, prima pagina, taglio medio:
TORNA RAGAZZA SCOMPARSA A BARCELLONA
I GENITORI LA RINNEGANO: NON È LEI

Per quanto la storia potesse essere buffa, non la fece ridere neanche un po'. Lei non rideva mai lei, quando era al lavoro. La sua aura di apatia in qualche modo parve contagiare il suo paziente, che riprese gradualmente il controllo di sé.

< Come va col suo lavoro, Fontanelli? > lo distrasse la ragazza, mentre gli sentiva il polso.

< Credo... credo di aver finito >, ansimò quello.

< Butterà via tutto e ricomincerà da capo? >

< Credo proprio di sì. >

L'infermiera soppresse il sorriso che le si voleva dipingere sul volto. Non doveva sorridere, a lavoro. < Guardi che non deve comportarsi da pazzo solo perché la teniamo qui >

Lui cercò il suo sguardo.

< Ah, no? > le disse, ostentando un disarmante candore.

La ragazza si lasciò sfuggire uno sbuffo pericolosamente simile ad una risata.

< Mi scusi per stamani >, proseguì il ragazzo. < Ero... di pessimo umore. >

< Si figuri >, fece lei automaticamente. < Adesso, invece? Si sente di buon umore? >

Lui non rispose subito. La guardò per qualche istante dritta negli occhi, uno strano scintillio in quello sguardo. Quando finalmente parlò, le disse: < Diciamo... che ho appena visto un'opportunità di fare qualcosa di divertente. >

Lei lo studiò un istante. Era la prima volta che lo vedeva così animato. < E cosa sarebbe? Vuole raccontarmelo? >

Il ragazzo esibì un sorriso inquitante: gli angoli delle sua labbra si erano arricciati in un modo che era semplicemente malvagio, perverso.

< No, per ora credo di no >, dichiarò con l'aria di nascondere chissà cosa. < Credo che sarebbe assai più appropriato se glielo dicessi domani mattina.>

***

L'infermiera non si stupì più di tanto quando, l'indomani, nella stanza assegnata a Fabio Fontanelli non trovò nessuno. Poche parole scritte di fretta su di un fogliaccio anunciavano il suo addio.

Cara inf. S. F.,

capita la battuta? Spero di no, perché mi accorgo ora che potrebbe rimanerci male.

Ho una cosetta simpatica da fare, perciò starò via per un po'. Non si preoccupi per me, ora che non sono più lì da voi non sono più moralmente obbligato ad avere qualche psicosi una volta ogni tanto.

Spero di non averla messa in un bel casino. Non so chi è responsabile per me in quella metaforica-ma-neanche-del-tutto gabbia di matti: se è lei, sappia che mi dispiace. Casomai questo la portasse a dichiarare che la sua vita è ormai rovinata, e che non le rimane altro da fare che ricominciare da capo cambiando nome e connotati, mi sento di consigliarle caldamente di non farlo - se non altro, perché quella è la mia follia e, se deve impazzire, che se ne trovi una tutta sua.

Non mi dimenticherò di lei, anzi: le prometto che, appena mi anoierò di quello che sto andando a fare, verrò a trovarla.

Suo,
F.F.

Per la prima volta da quando lavorava lì, quella ragazza si concesse un vero, sincero sorriso.

P.S. A ripensarci, io non avrò più il tempo di comportarmi da pazzo, ma qualcuno deve pur farlo, le lascio in eredità un piccolo assaggio del mio tormento: l'ultima stesura delle mie regole - con cui ormai dovrebbe avere una certa familiarità. Le legga, se le va; poi le butti via e le riscriva da capo, cambiando poco o niente. Se possibile, le peggiori ad ogni iterazione, e si lamenti costantemente di quanto era migliore la stesura originale. Ripeta questa proceura fino a due volte al giorno, preferibilmente dopo i pasti. Confido che lo farà.

Il sorriso dell'infermiera si allargò.

Stare al Mondo

Parte prima: Criptici Avvertimenti

1 - Non perdere mai di vista l’orizzonte, ma soprattutto guarda dove metti i piedi.

2 - Esitare è facile, ma raramente è utile. Pensare è costoso, ma spesso è saggio.

3 - Gli impulsi vanno presi in considerazione; non semplicemente seguiti, né soppressi.

Parte seconda:

tempo ed energia sono tutto ciò che hai: usale con parsimonia.

Le sue interpretazioni possono essere tante, ma la realtà è una sola.

Va bene fare i coglioni, ma fino a un certo punto

Menti solo quando non puoi assolutamente evitarlo.

Quando sei frainteso, anche se qualcuno è effettivamente troppo stupido per capire, tu comunque non sei stato in grado di spiegare.

Le opinioni vanno bene, ma i fatti vanno meglio.

Quando parli male di qualcuno o qualcosa, limitati ad esporre esattamente i fatti che hanno generato la tua opinione.

Se non ti senti molto loquace, dillo.

Non valutare una persona in base ai suoi interessi o alle sue preferenze.

Sii severo nell’individuare i tuoi errori, ma indulgente nel perdonarteli.

Quando sei tentato di disperarti per la stupidità di qualcuno, pensa a quelle volte in cui tu sei stato ancora più stupido.

Se sei forte, il tuo interesse è apparire innocuo; se sei debole, il tuo interesse è diventare forte.

Finché qualcuno non reclama una proprietà intellettuale, usala pure a tuo piacimento - senza però attribuirtene esplicitamente la paternità.

Discorsi semplici per menti semplici; serba la complessità per chi è in grado di capirla.

Nessuno può pretendere il massimo da te se non da il massimo per te, o viceversa.
Essere gentili è faticoso, ma spesso vale lo sforzo: la gentilezza è gratuita, farsi perdonare no.

Il modello binario del bene e del male è fin troppo riduttivo, ma comunque molto efficace. Se lo usi, scegli a priori le definizioni di bene e male con estrema attenzione.

Il mondo dentro la tua testa ed il mondo reale sono ben distinti. Dedica ogni sforzo possibile per mantenerli se non coincidenti, almeno simili.
Scegli bene le battaglie che vuoi combattere. Se riesci, è meglio scegliere quelle che puoi vincere.
​
La ragazza piegò con cura quelle carte, e le nascose sotto il camice. Qualcosa in lei era cambiato.
