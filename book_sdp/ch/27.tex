Presidio ospedaliero di Prato "Misericordia e Dolce", un giorno a caso di un mese a caso.

Una ragazza riccia e una sua collega più anziana discutevano svogliatamente del più e del meno, sorseggiando di tanto in tanto un pessimo caffé. Il fumo delle loro sigarette non riusciva a uscire del tutto dal minuscolo abbino che, aperto, costituiva l'unica fonte di luce naturale di cui godeva quel corridoio. L'aria ingrigita di quell'ambiente era perfettamente in linea con il loro umore: dopo anni di servizio in quel manicomio, una profonda apatia era l'unica difesa possibile contro il peso schiacciante della loro sanità mentale, che lì era più unica che rara.

Una risata sguaiata eruppe in quel corridoio, sovrastando il sommesso chiacchiericcio delle due infermiere. Senza dare il minimo segno di sorpresa, la più anziana resse la sigaretta in bocca e preparò la mano chiusa a pugno. La più giovane la imitò, e le due cominciarono il loro consuero gioco di morra cinese, che aveva in palio il resto della pausa.

La giovane perse, ma provò a rilanciare.

< Un cicchino che è il Fontanelli >, fece spassionata, mentre spengeva rassegnata quella che era la penultima sigaretta che aveva.

L'altra infermiera sorrise appena, scuotendo il capo.

La ragazza si avviò allora verso la fonte di quel rumore, che nel frattempo non aveva dato alcun cenno di cessare. Chiaramente, aveva intuito subito chi potesse essere a ridere in quel modo, così sguaiatamente, senza alcun tipo di contegno; seguiva spesso quel paziente, e al momento era l'unico in struttura che non sembrava trarre beneficio dai farmaci antipsicotici. Era un ragazzo giovane e piuttosto educato, che passava la maggior parte del tempo a leggere e scrivere. A volta le veniva quasi da pensare che fosse stato ricoverato per sbaglio, tanto le sembrava sano di mente rispetto agli altri occupanti della clinica. Eppure, a volte, si abbandonava a quelli che sembravano essere dei banali episodi psicotici, ma che per qualche strano motivo le medicine non riuscivano a placare.

La giovane infermiera entrò tranquilla nella stanza di quel paziente, mentre egli, afflosciatosi su una sedia, le braccia e la testa abbandonate, stava quasi letteralmente soffocando dalle risate. Con un movimento ben preciso e senza la minima esitazione, la ragazza gli sostenne la testa, assicurandosi che il suo collo rimanesse ben dritto e le sue vie respiratorie libere fino a che la risata nervosa non fosse passata.

< Oggi, quando sono entrata, non mi ha neanche salutato >, disse l'infermiera al paziente, per cercare di distrarlo e agevolare la fine dell'episodio. Per qualche motivo si davano del lei, anche se probabilmente i due avevano la sua stessa età.

Quello, fra una risata e l'altra, riprese parzialmente il controllo delle braccia ed indicò qualcosa sulla sua scrivania.

Lei prese, sempre sorreggendogli la testa

Il Corriere della Sera, prima pagina, taglio medio:
TORNA RAGAZZA SCOMPARSA A BARCELLONA
I GENITORI: NON È LEI

Per quanto la storia poteva essere buffa, a non la fece ridere neanche un po'. Lei non rideva mai lei, quando era al lavoro.

<>

< Mi scusi per stamani. Ero... di pessimo umore. >

< Si figuri. Adesso invece? si sente di buon umore? >

Lui la guardò dritta negli occhi, uno strano scintillio in quello sguardo.

< Diciamo... che ho appena visto un'opportunità di fare qualcosa di divertente. >

< E cosa sarebbe? Vuole raccontarmelo? >

< No, per ora credo di no. Credo che sarebbe assai più appropriato se glielo dicessi domani mattina.>

L'infermiera non si stupì più di tanto quando, l'indomani, nel letto assegnato a Fabio Fontanelli non trovò nessuno.









Stare al Mondo

Parte prima: Criptici Avvertimenti

1 - Non perdere mai di vista l’orizzonte, ma soprattutto guarda dove metti i piedi.

2 - Esitare è facile, ma raramente è utile. Pensare è costoso, ma spesso è saggio.

3 - Gli impulsi vanno presi in considerazione; non semplicemente seguiti, né soppressi.

Parte seconda: 

tempo ed energia sono tutto ciò chehai: usale con parsimonia.

Le sue interpretazioni possono essere tante, ma la realtà è una sola.

Va bene fare i coglioni, ma fino a un certo punto

Menti solo quando non puoi assolutamente evitarlo.

Quando sei frainteso, anche se qualcuno è effettivamente troppo stupido per capire, tu comunque non sei stato in grado di spiegare.

Le opinioni vanno bene, ma i fatti vanno meglio.



Quando parli male di qualcuno o qualcosa, limitati ad esporre esattamente i fatti che hanno generato la tua opinione.

Se non ti senti molto loquace, dillo.

Non valutare una persona in base ai suoi interessi o alle sue preferenze.

Sii severo nell’individuare i tuoi errori, ma indulgente nel perdonarteli.

Quando sei tentato di disperarti per la stupidità di qualcuno, pensa a quelle volte in cui tu sei stato ancora più stupido.

Se sei forte, il tuo interesse è apparire innocuo; se sei debole, il tuo interesse è diventare forte.



Finché qualcuno non reclama una proprietà intellettuale, usala pure a tuo piacimento - senza però attribuirtene esplicitamente la paternità.

Discorsi semplici per menti semplici; serba la complessità per chi è in grado di capirla.

Nessuno può pretendere il massimo da te se non da il massimo per te, o viceversa.
Essere gentili è faticoso, ma spesso vale lo sforzo: la gentilezza è gratuita, farsi perdonare no.


Il modello binario del bene e del male è fin troppo riduttivo, ma comunque molto efficace. Se lo usi, scegli a priori le definizioni di bene e male con estrema attenzione.

Il mondo dentro la tua testa ed il mondo reale sono ben distinti. Dedica ogni sforzo possibile per mantenerli se non coincidenti, almeno simili.
Scegli bene le battaglie che vuoi combattere. Se riesci, è meglio scegliere quelle che puoi vincere.
​