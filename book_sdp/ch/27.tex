\chapter{cap 27} 

(offuscamento?)

«Allora?» incalzò la figura. Era un uomo alto, dalla folta barba scura, parte del volto nascosta dal cappuccio di un enorme impermeabile.

«Boh, penso di sì», fece Fabio spassionatamente, parlando con fatica; la droga e la nottata all'umido avevano reso la sua voce era roca ed impastata. «Anche se, a essere sincero, non ne sono più così sicuro quanto prima. Vuoi essere annoiato con qualche sega mentale sul concetto stesso di realtà? Là c'è una ragazza che ti può aiutare.»

La figura rise. «Credici o no, ho già avuto notizia delle tue compagnie. Sembra tu sia beato tra le donne, almeno stasera.»

Fabio sbuffò sarcastico. «La beatitudine non è esattamente il mio stato d'animo.» Fissò per qualche momento il cielo terso, poi scoccò uno sguardo penetrante all'uomo incapucciato. «Non mi hai ancora risposto. Sei reale?»

«Certo che lo sono», rispose egli.

«Reale come una sequenza di correnti nel mio cervello, o reale come una persona?»

«Tu cosa sospetti che sia?»

Fabio sospirò. «Sospetto che tu ti stia divertendo a prendermi per il culo e credimi, questo gioco mi sta stancando.»

Il cappuccio di quel tizio lasciò intravedere un sorriso. «Non hai paura di me?»

Stavolta fu Fabio a ridere. «E perché mai dovrei averne? Se sei un'allucinazione, mettiti pure là con le altre. Vuoi farmi credere di essere un fantasma, o uno spirito? Sono pazzo, mica scemo. E se sei davvero una persona\ldots{} beh, se tu avessi voluto farmi del male l'avresti fatto e basta, non mi avresti offerto del fuoco --- che, tra l'altro, sto ancora aspetttando.»

Senza dire niente, l'uomo incappucciato gli lanciò un'accendino.

«Nero come l'anima», fece distrattamente Fabio, mentre si accendeva il sigaro. «Sai, c'era una certa persona che comprava sempre e solo accendini neri. Non aveva un vero motivo, non gli importava davvero del colore; però, quando qualcuno glielo chiedeva, lui diceva sempre così. Di che altro colore avrebbe dovuto prenderli?»

L'uomo misterioso ridacchiò. Un po' goffamente, si mise a sedere per terra accanto a Fabio. «Sembra un tipo niente male, questa persona».

«Già, niente male davvero», sussurrò Fabio, assente. «Povero diavolo\ldots{}»

«Che fine ha fatto?» domandò l'altro, accendendo il sigaro a sua volta.

«Siamo seduti sulla sua tomba», fece Fabio, amaramente. «Fa un po' cliché, vero?»

L'uomo non rispose, e i due stettero in silenzio per un po'. 

Fabio prese una copiosa boccata di fumo e provò a sbuffarlo in dei cerchi concentrici, fallendo miseramente. Non era mai riuscito a farlo; né lui né Bruno si erano mai presi la briga di imparare quel genere di cose. Amava le cose utili, il buon Bruno Bagonghi.

«Sai», disse Fabio dopo un po', «sto fumando questo sigaro sulla tomba del mio amico perché lui mi ha chiesto di farlo. Riesci a crederci?»

«Sì», rispose semplicemente l'altro. «So molto di te, caro il mio Fabio Fontanelli.»

Fabio ridacchiò. «Sai solo leggere i giornali e fare due più due, caro il mio custode del cimitero.»

«Anche tu sembri saper fare due più due. Forza, spiega.»

«Sono un tizio strano che di notte se ne sta seduto sulla tomba dell'imprenditore morto l'altro giorno. Chi potrei mai essere, se non il suo amico scomparso da diversi mesi?»

L'uomo fece un tiro profondo, nascondendosi in una nuvola di fumo profumato. «No, quello è ovvio», fece sbrigativo. «Spiega come mai pensi che io sia il custode del cimitero.»

Fabio rimase un po' interdetto. «Non sei un darkettone e tra un po' è mattina, quindi devi aprire questo posto. Non è ovvio anche questo?»

Non ci fu risposta, ed il silenzio tornò fra i due per dei lunghi istanti.

«Dimmi, Fabio», parlò il presunto custode, «perdona la mia curiosità: perché sei scomparso? Era intuibile che non ti fosse successo niente, che te ne fossi andato di tua volontà --- allora dimmi, perché hai voluto andartene?»

Fabio fu secco: «Non mi va di parlarne.»

«Capisco», fece l'uomo con l'aria di chi la sa lunga. «Allora forse ti va di parlare di questo: perché sei tornato?»

Fabio produsse un sospiro pesante. «Te l'ho detto, perché dovevo fumare questo sigaro con il mio amico a qualunque costo.»

Il custode lo incalzò: «Una lealtà ammirevole la tua, ma mi cheido se in realtà ci sia un altro motivo. Capirei se in realtà tu fossi tornato a vendicarlo --- o addirittura a vendicare te stesso.»

Fabio si voltò a guardare l'uomo dritto nell'oscurità del suo cappuccio. «Sai dove posso trovare il Leka?»

«Oh, lui? Sì\ldots{}» fece quello, la voce carica di un'emozione indecifrabile. «Sì, so proprio dove trovarlo --- ah, ed è più vicino a noi di quanto tu possa pensare!»

I muscoli di Fabio si irrigidirono di scatto, ma durò solo un momento; con lo sgaurdo perso nel vuoto, si abbandonò di nuovo contro la lapide. 

«Non mi interessa», disse stancamente. «A che servirebbe? Non mi ridarebbe il Bagonghi, né metterebbe a posto la mia vita. Non servirebbe a niente se non a farmi diventare sempre più scemo, sempre più diviso, sempre meno\ldots{} beh, sempre meno me. Non voglio più uccidere.»

«Ti manca la fermezza per farlo?» chiese l'uomo, la voce tagliente e severa. «Hai paura di non esserne in grado, nemmeno con qualcuno che merita tutta la tua ira?»

«Oh, ci riuscirei, eccome. Anzi, credo che uccidere sia una delle cose meno peggio che ho fatto  ultimamente.»

La voce dell'uomo si ammorbidì. «Non puoi dirmi una cosa del genere e sperare che me la beva così. Ora voglio sapere che diavolo hai combinato di peggio.»

«Oh, un sacco di cose. Me ne sono accorto solo stanotte, sai? Ho ammazzato, questo non lo nego. Questa notte ho rivissuto per tante volte i miei omicidi, lì ho\ldots{} diciamo che li ho sognati --- »

L'uomo rise forte. «Stai confessando di essere un assassino, ma ti premuri di nascondere che tu e le tue amichette vi siete drogati stanotte?»

Anche Fabio rise. «Sì, scusa, forse è l'abitudine --- oh, ma lascia perdere. Sai qual'è la cosa interessante di tutto questo? Che ogni volta che ho ucciso, l'ho fatto per proteggere qualcuno. Cristo santo, stasera ogni cosa che dico è un cliché, che diavolo mi succede? Insomma, dicevo\ldots{} quando invece avrei potuto uccidere ma non l'ho fatto\ldots{} quando ho assaporato quel potere, quando ho goduto nello sbandierare la mia possibilità di concedere la vita invece della morte\ldots{} è stato allora che ho toccato il mio fondo.»

L'uomo stette in silenzio per un po', come se stesse pensando velocemente a qualcosa. Quando finalmente parlò, lo fece sussurrando: «Tu mi parli del potere di uccidere? Ingenuo. Il vero potere è quando abbiamo ogni giustificazione per uccidere, ma non lo facciamo».

«Sì, l'ho visto anch'io quel film», rispose Fabio. «Anche se credo che la morale lì fosse un pelo diversa dalla mia.»

Ci fu un altro lungo silenzio fra i due.

«Mi hai detto», disse lentamente l'uomo, «che uccidere Leka non ti ridarebbe la tua vecchia vita. Hai ragione. Ma quando parlavo di vendicarti, in realtà non mi riferivo a lui.»

«Vai avanti.»

L'uomo esitò un attimo, poi riprese: «Sai, non ho molto da fare, mi piacciono i pettegolezzi sui fatti di cronaca. Ho seguito con interesse il tuo caso, sulla stampa e parlando con qualcuno che conosce una
certa\ldots{} com'è che si chiama? Lavinia Gori, mi sembra.» 

Il respiro di Fabio si interruppe.

«Ah, vedo che ricordi ancora il nome della tua fidanzata», proseguì l'uomo. «Insomma, mi sono stati raccontati vari retroscena che non sono stati scritti sui giornali. Pare che lei se la facesse con un certo Giacomo Gazzi, il quale ha messo in moto diversi eventi per impossessarsi di varie cose che non gli appartenevano.»

Fabio continuò ad ascoltare, pietrificato.

L'uomo fece una pausa, prendendo solennemente una boccata di fumo. «Si dice che dietro la morte del tuo amico imprenditore ci sia proprio questo Gazzi. Oh, non mi guardare così, in fondo ha senso: lavorava per Bagonghi, magari ha intravisto un modo per fare carriera liberandosi di lui; e l'esecutore materiale era un suo collega, è probabile che fossero d'accordo, o che Gazzi avesse esercitato una qualche leva su di lui. Ho letto molto su Anton Leka, molto più di quanto valesse la pena di scrivere riguardo a uno come lui; pare che avesse mille motivi per odiare il suo datore di lavoro, e non puoi che concordare che fosse una persona facilmente manipolabile da qualcuno che avesse l'interesse a sfruttare la sua rabbia.»

Il tempo si era come fermato.

Finalmente, tutto aveva un senso. Come poteva non esserci il Gazzi dietro tutto quello che era successo al povero Bruno? Gazzi gli aveva portato via il suo amico. Gazzi gli avva portato via la donna. E se due più due faceva quattro, anche Daniele era morto per colpa di Gazzi, in qualche strano e distorto modo. Aveva perfettamente, pericolosamente senso.

Una strana visione riaffiorò nella mente di Fabio, un alto e pallido Gazzi,bizzarramente incarnato in un simulacro dell'arcinemesi della saga di Harry Potter, il terribile Lord Voldemort; il male puro e semplice, senza sfumature, senza ambiguità. Il significato di quell'immagine era fin troppo chiaro: Giacomo Gazzi era l'opposto di Fabio Fontanelli, il nemico giurato con cui non ci sarebbe mai potuta essere pace.

\emph{Nessuno dei due può vivere se l'altro sopravvive\ldots{}}

La voce dell'uomo nel cimitero penetrò nei pensieri di Fabio, rimbombante come quella della profezia che aveva ricordato. «Quindi», annunciò con tono definitivo, «se mai decidessi che in fondo non c'è niente di male ad avere la tua meritata vendetta\ldots{} beh, non sono affari miei, ma credo che dovresti dirigerla verso di questo Gazzi.»

Fabio annuì assente, perso nei meandri della sua mente. Per dei lunghi istanti fissò il nulla davanti a sé, riflettendo sulla metafora che le visioni lisergiche gli avevano donato, mentre un Gazzi-Voldemort danzava balordo nella sua immaginazione.

«Tutto bene?» disse l'uomo dopo un po'.

«No», piagnucolò Fabio, tornando alla realtà. «Perché penso queste cose? Non ho mai neanche letto Harry Potter!»

Il tizio tossì la sua boccata di fumo, come se fosse sbiottito. «Prego?»

«Niene, lascia perdere. Credevo avessi capito che sono matto da legare.»

L'uomo rise. «Beh, chi non lo è?»

Fra i due calò nuovamente il silenzio, e nessuno lo ruppe per un po'.

Dopo qualche istante, o forse qualche ora, lo strano uomo sbuffò un ultima nuvola di fumo profumato e posò pesantemente una mano sulla spalla di Fabio, spingendosi in piedi.

«Caro mio, è stato un piacere fumare con te. I toscanelli al caffé sono i miei sigari preferiti», disse. «Purtroppo, il tempo è tiranno e sono sicuro che abbiamo entrambi del lavoro da fare.»

Fabio lo seguì con lo sgaurdo mentre si allontanava a passo lento.

«Non piangere troppo il tuo amico Bagonghi», aggiunse ad alta voce, senza voltarsi. «Lui ha solo scelto di seguire il suo destino. E se anche te seguirai il tuo\ldots{} beh, se lo facessi, lui sarebbe fiero di te, ne sono certo.»

Fabio non disse niente. Non aveva proprio niente da dire.
