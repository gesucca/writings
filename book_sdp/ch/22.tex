\chapter{Qualcosa}

Non molto lontano da lì, in un palazzo della stessa zona industriale, c'erano un ragazzo e una ragazza. I due vecchi amici si erano raccontati quello che sapevano riguardo alle morti di due uomini a loro cari. La ragazza, che non si era data pena di truccarsi dopo aver pianto molto, guidava una una vecchia auto un po' ammaccata, per accompagnare l'altro in un certo luogo. Non parlavano: si erano già detti tutto.

« Io... eh, io non lo so se ho ancora qualcosa per cui vivere », aveva confessato Fabio, seduto e chino su sé stesso.

Lei gli sveva scoccato il più ardente dei suoi sguardi, e non contenta gli aveva mollato anche un sonoro schiaffo.

« Sei patetico », aveva sputato fuori con disgusto. « Patetico come lui, vile anche di più e stupido peggio che mai! Pensi che io abbia qualcosa per cui vivere? Credi che la mia esistenza voglia dire qualcosa per qualcuno? Non certo per il mio ragazzo, che a quanto pare è andato a farsi ammazzare a Barcellona senza dirmi un cazzo di niente! Neanche per il mio migliore amico, che ha preso e si è levato dai coglioni, andando per i cazzi suoi al minimo problema. Mi hai mai sentito lamentarmi? No, e sai perché? Perche tanto non fregherebbe a nessuno! »

Lui aveva incassato il colpo, senza darsi pena né di reagire, né di smentirla. Lei si era subito pentita di averlo trattato così.

« Scusa, non ti volevo aggredire. Ma non ritiro niente di quello che ho detto. Ti stai comportando come Daniele, e lui era uomo un quarto di quanto lo sei te. Sei scappato da dei problemi che, scusami, ma dai, non sono poi 'sti grandi problemi, eh! Reagisci, cazzo! Rimboccati le maniche e trovati un altro lavoro, riparti da zero e progetta come rilanciarti! Non posso credete che uno come te si sia dato per finito. Gazzi ti ruba la donna? Vai a prenderlo a schiaffi! Fatti valere! Lavinia fa la troia con Gazzi? Affrontala! Chiamala puttana e lasciala! Hai paura di restare solo? Non ci credo neanche morta - e comunque io ci sarò sempre! Pensavo che lo sapessi, che ti importasse! »

Per un attimo le era parso che neanche queste parole lo avessero toccato. Ma poi, sorprendentemente, il suo vecchio amico si era animato all'improvviso.

« Sei stata illuminante », le aveva detto, alzandosi di scatto. « Mi importa, eccome. Mi importa di tante cose. Ho sbagliato, ho sbagliato proprio tutto. »

L'improvviso ardore di Fabio la aveva spaventata.

« Cosa fai? », gli aveva chiesto allarmata.

« Questa storia deve finire. Io sono scappato, hai ragione te. Ma in realtà, quello da cui volevo fuggire non era Lavinia, o il Gazzi, o il lavoro. Non te lo nascondo, ho disprezzato Daniele tante volte proprio per questo, proprio mentre stavo commettendo lo stesso sbaglio, finendoci dentro cento volte piùdi lui. Ho fatto del male, ho dato il mio peggio a chicchessia, tutto per cosa? Non sono pazzo, sono solo stupido! Ed è mille volte peggio: ho fatto tutto quello che ho fatto... perché? Per non affrontare la mia umanità, accettarla e mostrare un briciolo di palle a chi invece forse lo meritava? Sono un cretino. »

A quel punto Fabio si era alzato, e si era rimesso la pistola nei pantaloni.

« E ora dove vai? » aveva squittito Denise, spaventata.

« A chiudere delle questioni. »

« No! Non ti lascio andare in giro con una pistola dopo un discorso del genere. Non mi importa un cazzo di chi puoi ammazare, ma se perdo anche te chi mi rimane? Sono sola, cazzo, sola come una bestia! Non ho più niente per cui vivere - a parte te! Ti prego, non... non farlo, qualunque cosa tu voglia fare. Ho paura che ti succeda qualcosa. Se proprio non vuoi farlo per te stesso, ti prego, fallo per me! »

L'ovvio teatrino che le era toccato mettere in piedi aveva fatto rinsavire Fabio, facendogli ritrovare la lucidità.

« Credi veramente che ci cashi? », aveva sibilato, trattenendo a stento un sogghigno.

« Certo che no », aveva replicato immediatamente lei, « però che cazzo, seriamente. Non te ne puoi partire per la tangente e piantarmi qui, dopo quello che è successo! Sei proprio una merda. »

Fabio stava sorridendo.

« Hai ragione », le aveva detto, « ma io devo comunque fare delle cose. Vieni con me - anzi, accompagnami che sono a piedi. Intanto ti faccio conoscere una persona, poi penseremo a cosa fare. »

Ed era proprio incontro a quella persona che i due amici stavano andando. Denise scollò per gli occhi dalla strada per un attimo, scoccando uno sguardo al suo silenzioso compagno di viaggio: sembrava immerso in qualche piacevole pensiero.

In effetti, lo era. Spostarsi per Prato con quella specifica automobile, che tante volte lo aveva riportato a casa mentre era in preda a tossici mix di alcool e droghe leggere, gli dava una magnifica sensazione di solennità. Ne era passato di tempo dai bagordi orgiastci adolescenziali; tante cose erano successe e molte altre erano semplicemente cambiate, ma dopo tutto quanto lui si trovava ancora lì, a farsi scarrozzare da Denise perché lui - per un motivo o per un altro - non poteva guidare. Lo raggiunse una fitta dritta al cuore: Daniele in quell'auto non ci sarebbe più salito. E nemmeno Bruno, che nonostante facesse tutto il sostenuto col Mercedes, non sdegnava mai lo scroccare un passaggio alla sua amica per poter bere come un dannato senza rischiare la patente.

Fra un pensiero e l'altro, il duo arrivò nei pressi del cimitero di Chiesanuova, dove Fabio e Vittoria avrebbero dovuto ritrovarsi e aspettarsi, una volta completata la loro missione. Non c'era ancora nessuno. Parcheggiarono e, sempre restando in religioso silenzio, si misero in attesa davanti al cancello.

Fabio osservò per qualche momento Denise: sembrava totalmente serena.

« Tutto a posto? », le chiese, un po' titubante. « Sei a disagio? »

« Sono in pace », rispose semplicemente lei.

Fabio non aggiunse altro. Aveva capito perfettamente cosa intendeva.

« Secondo te ho fatto bene a mandarla da sola? », fece Fabio, dopo una decina di minuti, mentre un taxi si fermava poco lontano da dove erano.

« E io che ne so? », replicò disinteressata Denise. « Non so nemmeno perché siamo qui, a dirtela tutta. »

« Non hai chiesto, e io non ho detto. Comunque, non importa: credo che sia là. »

Una ragazza bionda si stava dirigendo a grandi passi verso di loro. Si fermò ad appena un passo di distanza da Fabio, fulminandolo con uno sguardo estremamente minaccioso.

« Cara mia », fece Fabio, rivolto a Denise, « lei è Vittoria, la mia compagna di disavventure. Come ti accorgerai presto, è antipatica, viziata e ormai bruciata dalla droga, ma ha anche dei difetti. »

L'espressione di Vittoria rimase dura come l'acciaio temprato.

« STRONZO! » esordì con veemenza, avventandoglisi addosso e mollandogli una sonora sberla.

Fabio si trattenne vistosamente dal reagire, e si limitò a mugolare un lamento:

« Perché oggi mi schiaffeggiano tutti? »

« Dieci volte stronzo! », proseguì Vittoria, sottolineando ogni accusa con un pugno o con una spinta. « Falso, bugiardo, schifoso puttaniere! Sei fidanzato! Convivi da anni! TI CHIAMI FABIO, IO ADORO IL NOME FABIO! »

Denise non riuscì a trattenersi: scoppiò scompostamente a ridere.

« Confermo tutto », fece divertita alla furente ragazzina. « Gli voglio bene, ma è proprio pessimo! »

« E ora tu chi cazzo sei, eh? » la aggredì Vittoria.

« Sono Denise, una sua amica - con benefici, ovvio », rispose lei, ammiccante.

« Dai, non ti ci mettere anche te! », sbottò Fabio verso la sua amica-con-benefici. Lei gli mostrò un certo dito, beffarda.

« Quindi questa è un'altra puttana che ti scopi? », proseguì Vittoria, sempre più furiosa.

L'espressione sul volto di Denise cambiò rapidamente.

« Oh, puttana a chi? », disse animosa a Vittoria. « Non so dove 'sto disgraziato ti abbia raccattato, ma a casa mia si porta rispetto per la gente più grande! Fabio, ma sei diventato un prete? Guarda che questa è una bambina! »

Vittoria aveva appena preso fiato per urlare qualche rispostaccia, quando Fabio le mise una mano su una spalla. Lei fece per scansarla con violenza, ma fu intrappolata con una leva articolare sul braccio.

« Ora parlo un attimo io, e te stai zitta », le ordinò Fabio. « Mi dai retta? Ti posso lascaire? »

Lei si limitò a mugolare, cercando inutilmente di liberarsi.

Denise lo guardò disgustato.

« Ma povera, lasciala subito! », intimò al suo amico. « Sei grezzissimo, sei sempre il solito! »

Fabio obbedì, ma rispose: « Se magari non ti fossi messa a provocarla, non ce ne sarebbe stato bisogno. »

Terminato il parapiglia, fra i tre calò un breve, imbarazzante silenzio. Fu Denise a romperlo:

« Vittoria, giusto? Come aveva detto di chiamarsi, l'infame? »

« Jorge », sibilò lei.

Denise scoppiò nuovamente a ridere.

Fabio sospirò profondamente. Era chiaro che le due ragazze, alla fine, sarebbero sicuramente andate molto d'accordo, e lui avrebbe dovuto sopportarle. Guardò Denise e le sorrise.

« Mentre voi mi fate nero », le disse, nascondendo un improvviso fremito di emozione, « io vado a vedere dove hanno messo il povero Bruno. »

Denise non rispose, ma gli fece l'occhiolino. Non appena Fabio varcò la soglia del cancello del cimitero, le due ragazze gli parlarono all'unisono, entrambe in preda a un certo moto d'urgenza:

« Fermo! Quella là non sarà mica - »

« Aspetta! Quell'uomo mi ha detto - »
