\chapter{Un cavaliere fra tre dame}

Non molto lontano da lì, in un palazzo della stessa zona industriale, un ragazzo e una ragazza, due vecchi amici si erano raccontati quello che sapevano riguardo alle morti di due uomini a loro cari. La ragazza, che non si era data pena di truccarsi dopo aver pianto molto, guidava una una vecchia auto un po' ammaccata, per accompagnare l'altro in un certo luogo. 

Non parlavano: si erano già detti tutto.

« Io... eh, io non lo so se ho ancora qualcosa per cui vivere », aveva confessato Fabio, seduto e chino su sé stesso.

Lei gli sveva scoccato il più ardente dei suoi sguardi, e non contenta gli aveva mollato anche un sonoro schiaffo.

« Sei patetico », aveva sputato fuori con disgusto. « Patetico come lui, vile anche di più e stupido, peggio che mai! Pensi che io abbia qualcosa per cui vivere? Credi che la mia esistenza voglia dire qualcosa per qualcuno? Non certo per il mio ragazzo, che a quanto pare è andato a farsi ammazzare a Barcellona senza dirmi un cazzo di niente! Neanche per il mio migliore amico, che ha preso e si è levato dai coglioni, andando per i cazzi suoi al minimo problema. Mi hai mai sentito lamentarmi? No, e sai perché? Perche tanto non fregherebbe a nessuno! »

Lui aveva incassato il colpo, senza darsi pena né di reagire, né di smentirla. Lei si era subito pentita di averlo trattato così. 

« Scusa, non ti volevo aggredire. Ma non ritiro niente di quello che ho detto. Ti stai comportando come Daniele, e lui era uomo un quarto di quanto lo sei te. Sei scappato da dei problemi che, scusami, ma dai, non sono poi 'sti grandi problemi, eh! Reagisci, cazzo! Rimboccati le maniche e trovati un altro lavoro, riparti da zero e progetta come rilanciarti! Non posso credete che uno come te si sia dato per finito. Gazzi ti ruba la donna? Vai a prenderlo a schiaffi! Fatti valere! Lavinia fa la troia con Gazzi? Affrontala! Chiamala puttana e lasciala! Hai paura di restare solo? Non ci credo neanche morta - e comunque io ci sarò sempre! Pensavo che lo sapessi, che ti importasse! »

Per un attimo le era parso che neanche queste parole lo avessero toccato. Ma poi, sorprendentemente, il suo vecchio amico si era animato all'improvviso.

« Sei stata illuminante », le aveva detto, alzandosi di scatto. « Mi importa, eccome. Mi importa di tante cose. Ho sbagliato, ho sbagliato proprio tutto. »

L'improvviso ardore di Fabio la aveva spaventata.

« Cosa fai? », gli aveva chiesto allarmata.

« Questa storia deve finire. Io sono scappato, hai ragione te. Ma in realtà, quello da cui voglio fuggire non era Lavinia, o il Gazzi, o il lavoro. Non te lo nascondo, ho disprezzato Daniele tante volte proprio per questo, proprio mentre stavo commettendo lo stesso sbaglio. E ho fatto de male, ho dato il mio peggio a chicchessia, tutto per non affrontare la mia umanità, accettarla e mostrare un briciolo di palle a chi invece forse lo meritava. »

« E ora dove vai? »

« A chiudere delle questioni. » aveva chiosato lui, spaventandola ancora di più.

« No! Non ti lascio andare in giro con una pistola dopo un discorso del genere. Non mi importa un cazzo di chi puoi ammazare, ma se perdo anche te chi mi rimane? Sono sola, cazzo, sola come uma bestia! »

L’ovvio teatrino che le era toccato mettere in piedi aveva fatto rinsavire Fabio, facendogli ritrovare la lucidità.

« Credi veramente che ci cashi? », aveva sibilato, trattenendo a stento un sogghigno.

« Certo che no », aveva replicato immediatamente lei, « però che cazzo, seriamente. Nom te ne puoi partire per la tangente e piantarmi qui, dopo quello che è successo! Sei proprio una merda. »

Fabio stava sorridendo. 

« Hai ragione », le aveva detto, « ma io devo comunque fare delle cose. Vieni con me - anzi, accompagnami che lo sai, sono a piedi. Intanto ti faccio conoscere una persona, poi penseremo a cosa fare. »
​
Ed era proprio incontro a quella persona che i due amici stavano andando. Denise scollò per un attimo lo sguardo dalla strada, rivolgendo un veloce sguardo a Fabio

Dalla prospettiva di denise
Fabio rivede la sua Prato. si sente rinascere, è contento. Anche lei è serena, ora che sapeva che il suo daniele era in pace.

< Cara mia >, fece Fabio, rivolto a Denise, < lei è Vittoria, la mia compagna di disavventure. Come ti accorgerai presto, è antipatica, viziata e ormai bruciata dalla droga. Purtroppo ha anche dei difetti. >

L'espressione di Vittoria rimase dura come l'acciaio temprato.

< STRONZO! > esordì con veemenza, mollandogli un sonoro schiaffo.

Fabio non reagì. Forse sapeva di meritarselo, pensò Denise.

< DIECI VOLTE STRONZO! >, proseguì, sottolineando ogni accusa con un pugno o con una spinta. < FALSO, BUGIARDO, SCHIFOSO PUTTANIERE! SEI FIDANZATO! CONVIVI DA ANNI! TI CHIAMI FABIO, IO ADORO IL NOME FABIO! SEI UNO STRONZO, TI ODIO!>

Denise non riuscì a trattenersi: scoppiò scompostamente a ridere.

< Confermo tutto > disse alla furente ragazzina. < Gli voglio bene, ma è proprio pessimo! >

