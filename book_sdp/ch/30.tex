\chapter{cap 30}

Il destino, si sa, a volte è simpatico; il più delle volte, però, può
risultare dispettoso o inopportuno, se non addirittura una\ldots{}

\textless{}Merda!\textgreater{}, fece Denise, non appen si accorse di
chi stava percorrendo in tutta fretta quel dannato vialetto di quel
dannato cimitero, trottando proprio verso di loro.

\textless{}Lav!\textgreater{} latrò quell'uomo, la voce
insopportabilmente untuosa. \textless{}Cucciola, ti ho cercata
dappertutto!\textgreater{}

Denise scattò in avanti verso di lui, decisa a fermare o per lo meno
rallentare l'ennesima catastrofe che si stava per abbattere sul gruppo
di persone sfortunate.

, cominciò aggressiva, \textless{}guarda, non è proprio aria. Abbiamo
già abbastanza cose per il capo, e chi mancheresti soltanto te, capisci?
Dagli pace un secondino alla Lavinia, sù!\textgreater{}

Il tono di Giacomo Gazzi cambiò repentinamente; era come se le sue
parole contenessero un ringhio di sottofondo, presagio di malcelata
ostilità. \textless{}Ero solo preoccupato, ok? Ora ci penso io a
lei.\textgreater{}

\textless{}no, guarda, non hai proprio capito\textgreater{}, gli tenne
testa Denise. \textless{}Ora ci lasci un attimo in pace a tutti quanti,
poi fai pure quello che ti pare. Te lo ridico: ora ci mancavi soltanto
te.\textgreater{}

\textless{}Senti, ma che cazzo vuoi, eh?\textgreater{} sbottò il Gazzi,
sfoderando la sua aggressività senza filtri. \textless{}Lav sa badare a
sé stessa! Non ha bisogno di te! Se non mi vuole vedere sarà lei a
dirmelo!\textgreater{}

\textless{}Gazzi ma che cazzo stai dicendo? Certo che lei sa
-\textgreater{}

\textless{}E poi te, chi cazzo ti credidi essere? Non te ne è mai
importato niente di lei, con che coraggio ora ti metti - \textgreater{}

\textless{}Ma non è questo il punto, ti ho solo detto di lasciarci un
attimo in pace -\textgreater{}

\textless{}Lei ha bisogno di me! Ferisci lei se cerchidi tenermi lontano
da lei!\textgreater{}

Denise considerò seriamente per un attimo di portarlo via di peso prima
che gli raggiungessero il loro parapiglia sul vialetto.

Si rassegnò a compiere un ultimo diplomatico tentativo:
\textless{}Ultimo avvertimento Gazzi: vai via più veloce della luce. Ti
chiamo io dopo pranzo, o dopo cena, o quando sarà il caso, promesso.
\textgreater{}

\textless{}Non esiste. Devo vederla e parlarle,
\emph{ora}.\textgreater{}

Denise si arrese. Pronunciò la sua risposta coems e fosse una condanna a
morte: \textless{}Fai come ti pare. Per me erano successe abbastanza
disgrazie, ma se proprio ci tieni a farne succedere altre, fai
pure.\textgreater{}

Il Gazzi grugnì soddisfatto. \textless{}Fatti da parte. non sei mai
stata una protagonista, mettiti al tuo posto. \textgreater{}

Denise sospirò, mentre il gazzi si avviava inesorabilemnte verso quello
che sarebbe stato l'ennesimo disastro di quella settimana.

// dove sta scritto che debba tramare una sola persona per volta?

con gazzi:

\textless{}Per me!?\textgreater{} fece Lavinia, minacciosa.
\textless{}Io non la voglio di certo la tua carità! Ci penso da sola a
vendicarmi --- e tra l'altro non mi voglio vendicare, e di certo non di
lui!\textgreater{}

Si rivolse al Gazzi con uno sguardo intriso del più puro disprezzo.
\textless{}Lo so che persona sei, stronzo, viscido e marcio fino al
budello! Ti piace, ti ecciti proprio a ficcare il dito nelle piaghe
degli altri, a cercare di smerdare chiunque sia in difficoltà per
sembrare più bello, più grande, più importante! Mi fai schifo, e per
ogni cattiveria che hai sputato su di me e chi mi sta caro te ne farò
ingoiare tre appena ne avrò l'occasione, puoi contarci. Ma
\emph{te}\textgreater{}

Fabio fece un patetico salto all'indietro, come a volersi scansare
dall'incandescente dito accusatorio che Lavinia gli aveva puntato
contro.

\textless{}Con \emph{te} farò proprio i conti,\textgreater{}
