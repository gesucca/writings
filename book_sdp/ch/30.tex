\chapter{cap 30}

Il destino, si sa, a volte è simpatico. Altre volte invece è dispettoso, oppure addirittura inopportuno. Certe volte, però, può essere semplicemente una\ldots{}

«Merda!», esplose Denise, non appena si accorse di chi stava trottando in tutta fretta su quel dannato vialetto di quel doppiamente dannato cimitero, dritto e rapido verso di loro come un traballante treno merci stracolmo di guai.

«Lav!» latrò quell'uomo, la voce insopportabilmente untuosa. «Cucciola, ti ho cercata dappertutto!»

Denise scattò in avanti verso di lui, decisa a fermare o per lo meno a rallentare questa ennesima catastrofe. «Gazzi, \emph{sta' bòno}», cominciò aggressiva, «guarda, non è proprio aria. Abbiamo già abbastanza cose per il capo, ci mancheresti soltanto te. Capisci? Dagli pace un secondino alla Lavinia, sù!»

Il tono di Giacomo Gazzi cambiò repentinamente; era come se le sue parole contenessero un ringhio di sottofondo, un chiarissimo presagio di malcelata ostilità. «Ero solo preoccupato, ok? Levati di torno, ora ci penso io a lei.»

«No, guarda, non hai proprio capito», gli tenne testa Denise. «Ora ci lasci un attimo in pace, tutti quanti, poi fai quello che ti pare. Te lo ridico: ora ci mancavi soltanto te.»

«Senti, ma che cazzo vuoi, eh!?» sbottò il Gazzi, sfoderando la sua aggressività senza filtri. «Lav sa badare a sé stessa! Non ha bisogno di te! Se non mi vuole vedere sarà lei a dirmelo!»

«Gazzi ma che stai dicendo? Certo che sa ---»

«E poi te, chi cazzo ti credi di essere? Non te ne è mai importato niente di lei, con che coraggio ora ti metti ---»

«Ma che c'entra! Non è questo il punto! Ti ho solo detto di lasciarci un attimo in pace ---»

«Lei ha bisogno di me! Non ferisci me, ferisci lei se cerchi di tenermela lontana!»

«Ma chi ti vuole ferire!? Ma chi ti caga!? Io non --- senti, basta, lasciamo perdere.»

Denise  per un attimo considerò seriamente di portarlo via di peso prima che gli altri raggiungessero il loro parapiglia sul vialetto. 

Si rassegnò a compiere un ultimo, diplomatico tentativo: «Io te lo dico, Gazzi: qui siamo in un bel casino. Se non ti vuoi fare ammazzare, vai via più veloce della luce. Poi ti spiego, ti chiamo io; dopo pranzo, o dopo cena, o quando sarà il caso, promesso. Ora però per piacere, davvero, \emph{vai via}.»

«Non esiste. Devo vederla e parlarle, \emph{ora}.»

Denise si arrese. «Fai come ti pare», gli disse con la voce intrisa di tutta la sua tristezza. «Per me erano successe abbastanza disgrazie, ma se proprio ci tieni a farne succedere altre, fai pure.»

Il Gazzi grugnì soddisfatto. «Fatti da parte. Non sei mai stata una protagonista, rimettiti al tuo posto.»

Denise gli rivolse un ultimo sguardo interrogativo, ma poi scosse la testa e si fece da parte. Il Gazzi si avviò tronfio e inesorabile verso quello che sarebbe stato l'ennesimo disastro di quella settimana.

\begin{center}
***
\end{center}

Fortunatamente, Fabio pareva non essersi ancora accorto dell'intrusione del suo acerrimo nemico.

Non aveva occhi che per Lavinia: se la guardava e riguardava come se non l'avesse mai vista, come se ogni increspatura ed ogni ombra che il dolore aveva causato su quel bel volto liscio e pulito potesse raccontargli un'incredibile storia in cui lui, ovviamente, era l'assoluto antagonista. 

C'era rimorso nei suoi occhi, c'erano sensi di colpa e tanto, tanto dispiacere al pensiero di che cosa era riuscito a perdersi. Eppure, nonostante tutta questa negatività, nonostante tutto quello che aveva fatto succedere a causa della sua stupidità, Fabio non si sentiva triste: il poter guardare Lavinia, il solo fatto che lei sopportasse ancora la sua presenza, che lui potesse posarle lo sguardo addosso, tutto questo poteva significare soltanto che il mondo non si era ancora guastato del tutto. Per quanto male poteva averle fatto, lei era comunque lì, esisteva, viveva, era rimasta sé stessa anche dopo la tempesta di pura e semplice merda che Fabio le aveva stupidamente fatto pivoere addosso.

Niente sarebbe stato più come prima con lei. Questo, Fabio lo sapeva.

Non avrebbe mai osato sperare in un perdono delle sue malefatte, ed in cuor suo neanche lo voleva. Gli bastava che lei fosse lì, che non rifiutasse il confronto con lui e che fosse pronta a giudicarlo per le sue colpe, se mai lui avesse deciso di confessarle.

Perso in questi pensieri, Fabio vedeva e ascoltava un solo aspetto dell'ambiente a lui circostante: Lavinia.

Quando il suo ritrovato angelo parlò, la voce melliflua ma dritta, penetrante, il suo vernacolo bello come la più aulica delle lingue, Fabio non capì del tutto cosa stesse dicendo, e soprattutto a chi.

«\emph{Oh Gia'omino, ma che mi de'i venì dietro dietro 'om un cane!}»

«Mi hai fatto preoccupare!» qualcuno le rispose con una familiare voce unta e lamentosa.

«Te l'avevo scritto ieri sera che stamani avevo da fare! Che passavo dal cimitero e poi andavo a fa' la spesa, e che semmai si sentiva dopo desinare! Preoccupare di che?»

«Cucciolina, ma non te ne devi andare in giro da sola, lo sai che su di me ci puoi contare per ---»

Il tizio dalla voce unta e lamentosa non voleva che la sua Cucciolina andasse in giro da sola. Fabio emerse bruscamente dal suo mondo interiore, ed un brutale fuoco lo avvampò non appena cominciò a rendersi conto di chi si era manifestato al suo cospetto.

«Intanto non sono da sola, poi --- lascia fare. Senti, ma te non c'hai proprio null'altro da fare? Dai, fa' per bene, è una giornatina niente male anche questa, lasciami un po' respirare, ci si risente dopo desinare.»

Gazzi fece per replicare, ma un sommesso sghignazzio gli gelò il sangue prima che potesse aprire bocca.

Fu Fabio a rompere il breve silenzio che seguì, la voce così bassa che sarebbe potuta passare tranquillamente per un borbottio del cielo, nuovamente, improvvisamente e piuttosto coreograficamente coperto; bassa, ma carica di una chiara, chiarissima sfumatura omicida, una sfumatura gelida e affilata come la lama di un coltellaccio appena uscito dall'arrotino.

«Giacomo Gazzi», disse. Solo questo.

Il Gazzi guardò stupidamente l'uomo che lo aveva nominato. «In persona» rispose, piuttosto a disagio. «Tu chi sei?»

Fabio riprese a sghignazzare. Anche l'osservatore meno attento si sarebbe potuto accorgere che qualcosa non andava in lui: la bocca gli tremava e si deformava come se stesse cercandod i trattenersi, il suo sguardo era marmoreo, fisso come quello di un predatore sull'uomo che odiava. Incapace di controllarsi, la sua risata crebbe fino a diventare un terribile ululato, un animalesco annuncio di cattive intenzioni.

Denies si stava letteralmente cacando sotto. Sapeva quello che stava per succedere, ma sapeva anche di essere totalmente impotente. La situazione sarebbe degenerata, Fabio avrebbe ucciso il Gazzi e tutto sarebbe andato ai maiali. Niente di quello che lei poteva fare avrebbe potuto cambiare il corso di questi eventi.

Ci provò comunque.

Si avvicinò a Fabio lentamente, gli prese una mano fra le sue e gli sussurrò all'orecchio «Fai per bene. non so che altro dirti, se non ti prego, ti scongiuro, fai per bene»

La folle risata di Fabio si trasformò in una ringhiata, folle risposta. «Oh sì, vecchia mia, farò proprio per bene»

Fu solo questione di un momento. Gazzi si ritrovò una pistola puntata dritta in mezzo agli occhi.

Vittoria urlò e LAvinia imprecò. Denise si sedette semplicemente per terra, sconfitta. 

«Giacomo Gazzi» ripeté fabio, col tono solenne di chi stava pronunciando una sentenza. «Il bagonghi è morto. il Brogelli è morto. tu sei ancora qui, e questo non va bene.» 

GAzzi realizzò finalmente chi aveva davanti.

«non è possibile», si lagnò pietosamente. «Sei, sei davvero te?»

Fabio esibì un sorriso così ampio che avrebbe mostrato tranquillamente una quarantina di denti, se li avesse avuti.

«In persona» ghignò.

Il Gazzi lo stava guardando come se fosse il Cristo in persona, sceso dal cielo e rivelatosi a lui con il solo scopo di ucciderlo brutalmente. « Fabio » mugolò balbettando, «io - io sono - d - dove - siamo amici noi, lo siamo!»

«Amicissimi» fece Vittoria, avvicinandosi con decisione a Fabio e appoggiando delicatamente una mano sul braccio con cui egli reggeva la pistola.

«Fabio la degnò di un solo, breve sguardo, prima di rimettersi a fissare il Gazzi col suo migliore sguardo inceneritore. «Non intrometterti» le sibilò. «Questo verme non avrà salva la vita»

«Salva la vita?» ripeté lei. «Cazzomene, ma chi lo conosce. Volevo dire, fai quello che ti pare, ma veloce, che qui c'è gente che ha freddo e che deve darsi una sistemata»

Un denso silenzio scese su tutto il gruppo, fino a che Denise non scoppiò a ridere. «Scusate» fece non appena si fu ricomposta. «questa ragazza mi spezza, ha queste uscite che... boh. Comunque ha ragione, eh. Fabio, sbrigati con questa vendetta che una bella doccia calda non mi farebbe proprio scomodo»

Lavinia intervenne. «Vendetta? Che gli ha fatto il Gazzi?»

«Ti ha portata via da lui, si capisce. O comunque, questo è quello che pensa Fabio.» 

Lavinia aprì e chiuse la bocca varie votle, incapace di formulare una frase.

Fabio, senza accennare a voler abbassare la pistola, dichiarò al mondo intero: «Non temete, presto sarà tutto finito e ci faremo tutte le docce calde che vogliamo. Gazzi: sei responsabile - direttamente o indirettamente, non lo so, non mi interessa - della morte del Bagonghi, del Brogelli, del Leka, di aver tramato alel mie spalle per portarmi via l'affetto più importante della mia vita e di aver reso quest'ultima un inferno tramite l'influenza della tua mera esistenza.» Tirò indietro il cane della sua 9mm silenziata, sebbene fosse un modello semiautomatico e non ce ne fosse alcun bisogno. «Neghi di non essere colpevole di tutto questo?»

«No!» strillò il Gazzi. «No, no! Non mi ammazzare!»

«Perfetto» sentenziò Fabio. «Se tu avessi contato le negazioni nella mia domanda, forse non avresti confessato. Non che questo di avrebbe risparmiato, parliamoci chiaro. Ebbene, Giacomo Gazzi, se avessi imparato il passo di Ezechiele che dice il nero di Pulp Fiction, adesso te lo reciterei. Sappi comunque che questa è la mia vendetta, e che mi macchierò del tuo assassinio per vendicare i torti subiti da me, dalla Denise e dalla Lavinia!»

Un ordine venuto dal cielo fermò la sommaria amministrazione di giustizia. «Abbozzala subito»

Era stata Lavinia a parlare. Con un rapido guizzo si mise fra il Gazzi e Fabio, e con u nmovimento repentino si impossessò della pistola di quest'ultimo.

«AscoltaZ» disse, fumante di rabbia come non lo era mai stata prima di allora. «ascoltami molto, molto bene, Fabio Fontanelli. Fai pure quello che vuoi della tua vita, diventa pure la canaglia che aspiri a diventare, ma non provare neanche per un momento a tirarmi in ballo in tutto questo. Per troppo tempo ho pensato che fosse colpa mia se sei impazzito, che fosse colpa mia se non ho fatto miracoli per impedire che avvenisse. Sbagliavo, e non sbaglier; pi#. Ammazza pure questo scemo, fai veramente che cazzo ti pare. Ma non azzardarti a darmi anche un puttanesimo di responsabilit' per quello che sei diventato!»

Si fece da parte e gli porse la pistola. «Prego, continua pure per la tua strada, per tutto il bene che può farti»

Fabio la prese, esterrefatto.

Fabio aveva ascoltato tutto qusto a bocca aperta. «Io non ti sto dando proprio niente», le disse debolmente. «Voglio vendicarmi, e visto che ci sono volevo farlo anche per conto tuo»

// dove sta scritto che debba tramare una sola persona per volta?

con gazzi:

«Per conto mio!?» fece Lavinia, minacciosa. «Io non la voglio di certo la tua carità! Ci penso da sola a vendicarmi --- e tra l'altro non mi voglio vendicare, e di certo non di lui!»

Si rivolse al Gazzi con uno sguardo intriso del più puro disprezzo. «Che cazzo c'è da vendicarsi di quest'ometto! Lo so che persona sei, stronzo, viscido e marcio fino al budello! Ti piace, ti ecciti proprio a ficcare il dito nelle piaghe degli altri, a cercare di smerdare chiunque sia in difficoltà per sembrare più bello, più grande, più importante! Mi fai schifo, e per ogni cattiveria che hai sputato su di me e chi mi sta caro te ne farò ingoiare tre appena ne avrò l'occasione! §Io son bona e cara, ma quande m incazzo vu lo sapete, le cose le dico male: vai in culo, Giacomino, te e quella zoccola di to ma!»

Fece un bel respirone come per calmarsi, ma in realtà stava soltanto riprendendo fiato.

«Lui l'è sistemato» fece con l'aria di chi ha appena spuntato dalla lista delle cose da fare un lavoro faticoso e pesante. «Ma \emph{te}...»

Fabio fece un patetico salto all'indietro, come a volersi scansare dall'incandescente dito accusatorio che Lavinia gli aveva puntato contro.

«Con \emph{te} non ho ancora finito. Certo, nini, s'ha da fare i conti, \emph{io e te}» ringhiò al suo probabilmente ex compagno di vita.

Dopoq quest'ultiam dichiarazione, sul gruppo calò un pesante silenzio.

Lavinia sbuffò, questa volta cercando di calmarsi davvero. <Due volte> dissse, la voce sottile per il poco fiato rimastole. <Fabietto, in neanche un'ora mi hai fatto incazzare due volte. Non ti strozzo solo perché sei alto e al collo ci arrivo male.>
// un po' brutto e repentino questo passaggio, smussare e legare


LA chiosa della ragazza non aiutò nessuno a rompere lo spesso muro di imbarazzo. Solo Denise, dopo dei lunghissimi inensi momenti, ebbe il coraggio di esprimere ad alta voce la domanda che tutti si stavano facendo
<E ora? Che si fa?>

<Che si fa?> le fece eco FAbio con sarcasmo. <Che vuoi che si faccia? Ho appena avuto l'ennesima prova che io sono del tutto scemo. L'hai sentita? A lei il Gazzi gli fa schifo, come a tutti insomma, ma lasciamo perdere -

Abbassò lo sguardo e si mise a fissare meditabondo la sua pistola.io ho perso la testa, ragazzi., cosa vuoi che mi metta a fare?Non posso ammazzare nessuno che sennò lei si incazza, e meno male, perché tanto lo so che alla fine non risolverei proprio niente. 

Si rimise solennemente la pistola nei pantaloni. e dedicò a tutti i presenti un lungo, penetrante sguardo.

<Ragazzacci> dichiarò, <io mi arrendo. Sono malato, penso tante cose ma le penso tutte male. Ho sbagliato tutto da sei mesi a questa parte, ma proprio tutto. Ho provato a farcela da solo, ma... beh, lo avete visto tutti, no?> la sua voce cominciò a strozzarsi
<aiutatemi, per favore. Anche te, Gazzi. Non ce la faccio a decidere se è davvero colpa tua e quindi ti devo odiare, o se hai solo fatto il tuo, e la colpa è mia che ci sono cascato. Non so più che cosa è bene e che cosa è male. Non capisco più niente, ragazzi.>

Cominciò a piangere copiosamente, senza il minimo freno. 

Nessuno dei quattro presenti si mosse per andare a consolarlo; quello che avevano visto lli aveva impietriti più che mai.

// vittoria chiama la polizia col telefono di Fabiomi hai rubato il telefono
e tu mi hai rubato l'iphone, credi forse che l'abbia dimenticato?
