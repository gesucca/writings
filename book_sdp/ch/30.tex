\chapter{cap 30}

Il destino, si sa, a volte è simpatico; il più delle volte, però, può
risultare dispettoso o inopportuno, se non addirittura una\ldots{}

<Merda!>, fece Denise, non appen si accorse di
chi stava percorrendo in tutta fretta quel dannato vialetto di quel
dannato cimitero, trottando proprio verso di loro.

<Lav!> latrò quell'uomo, la voce
insopportabilmente untuosa. <Cucciola, ti ho cercata
dappertutto!>

Denise scattò in avanti verso di lui, decisa a fermare o per lo meno
rallentare l'ennesima catastrofe che si stava per abbattere sul gruppo
di persone sfortunate.

, cominciò aggressiva, <guarda, non è proprio aria. Abbiamo
già abbastanza cose per il capo, e chi mancheresti soltanto te, capisci?
Dagli pace un secondino alla Lavinia, sù!>

Il tono di Giacomo Gazzi cambiò repentinamente; era come se le sue
parole contenessero un ringhio di sottofondo, presagio di malcelata
ostilità. <Ero solo preoccupato, ok? Ora ci penso io a
lei.>

<no, guarda, non hai proprio capito>, gli tenne
testa Denise. <Ora ci lasci un attimo in pace a tutti quanti,
poi fai pure quello che ti pare. Te lo ridico: ora ci mancavi soltanto
te.>

<Senti, ma che cazzo vuoi, eh?> sbottò il Gazzi,
sfoderando la sua aggressività senza filtri. <Lav sa badare a
sé stessa! Non ha bisogno di te! Se non mi vuole vedere sarà lei a
dirmelo!>

<Gazzi ma che cazzo stai dicendo? Certo che lei sa
->

<E poi te, chi cazzo ti credidi essere? Non te ne è mai
importato niente di lei, con che coraggio ora ti metti - >

<Ma non è questo il punto, ti ho solo detto di lasciarci un
attimo in pace ->

<Lei ha bisogno di me! Ferisci lei se cerchidi tenermi lontano
da lei!>

Denise considerò seriamente per un attimo di portarlo via di peso prima
che gli raggiungessero il loro parapiglia sul vialetto.

Si rassegnò a compiere un ultimo diplomatico tentativo:
<Ultimo avvertimento Gazzi: vai via più veloce della luce. Ti
chiamo io dopo pranzo, o dopo cena, o quando sarà il caso, promesso.
>

<Non esiste. Devo vederla e parlarle,
\emph{ora}.>

Denise si arrese. Pronunciò la sua risposta coems e fosse una condanna a
morte: <Fai come ti pare. Per me erano successe abbastanza
disgrazie, ma se proprio ci tieni a farne succedere altre, fai
pure.>

Il Gazzi grugnì soddisfatto. <Fatti da parte. non sei mai
stata una protagonista, mettiti al tuo posto. >

Denise sospirò, mentre il gazzi si avviava inesorabilemnte verso quello
che sarebbe stato l'ennesimo disastro di quella settimana.

Fortunatamente, Fabio pareva non essersi proprio accorto dell'intrusione del suo acerrimo nemico. Non aveva occhi che per Lavinia: se la guardava e riguardava come se non l'avesse mai vista, come se ogni increspatura e ombra che il dolore aveva causato su quel bel volto un tempo liscio e pulito potesse raccontargli un'incredibile storia in cui lui ovviamente era stato la causa di tutto. C'era rimorso nello sguardo di Fabio, c'erano sensi di colpa e tanto, tanto dispiacere al pensiero di che cosa era riuscito a perdersi. Eppure, nonostante tutta questa negatività, nonostante tutto quello che aveva fatto succedere a causa della sua stupidità, Fabio non si sentiva triste. Guardare LAvinia, il solo fatto che lei si lasciassea ncora guardare da lui, che lui potesse posare lo sguardo su di lei, poteva significare soltanto che nonostante tutto quello che aveva fatto, il mondo non si era ancora guastato del tutto. Per quanto male poteva averle fatto, lei era comunque lì, esisteva, viveva, aveva resistitio alla tempesta di pura e semplice *merda* che Fabio le aveva fatto pivoere addosso.

Niente sarebbe stato più come prima con lei, questo Fabio lo sapeva. Non avrebbe mai osato sperare in un perdono completo delle sue malefatte, ma in cuor suo neanche lo voleva. Gli bastava che lei fosse lì, e che lo ascoltasse se lui avesse deciso di parlarle, di confessarle tutte le sue vicissitudini e aspettare l'ovvio gidizio che lei gli avrebbe dato.

Quando il suo ritrovato angelo parlò, Fabio non capì del tutto cosa stesse dicendo, e soprattutto a chi.

<Oh Giacomino, ma che mi de'i venì dietro dietro 'om un cane!>

<Mi avevi fatto preoccupare!>

<T'ave'o scritto ieri sera che stamani c'aveo da fare, e anda'o ai cimitero e poi a fa la spesa, e che semmai ci si sentiva dopo desinare! Preoccupare d'i' che?>

<Cucciolina, ma non te ne devi andare in giro da sola, lo sai che ci puoi contare su di me per --- >

<Intanto un sono da sola, poi - lascia fare. Senti ma te un c'hai proprio null'altro da fare? Gnamo fa' per bene, è una giornatina niente male anche questa, lasciami un po' respirare e ci si risente dopo desinare!

Gazzi fece per replicare, ma un sommesso sghignazzio lo interruppe prima he potesse aprire bocca.

La voce di Fabio era così bassa che sarebbe potuta passare tranquillamente per un borbottio del cielo, nuovamente, improvvisamente e piuttosto coreograficamente coperto, se non avessea vuto una chiara, chiarissima sfumatura omicida, una nota gelida come e affilata come la lama di un coltellaccio appena uscito dall'arrotino.

<Giacomo Gazzi>, disse. Solo questo.

Il Gazzi guardò stupidamente l'uomo che lo aveva nominato. <In persona> rispose piuttosto a disagio. <Tu chi sei?>

Fabio riprese a sghignazzare. Anche l'osservatore meno attento si sarebbe potuto accorgere che qualcosa niona dava in lui: la usa bocca tremava e si deformava come se stesse cercandod i trattenersi, il suo sguardo era marmoreo, fisso come quello di un predatore sull'uomo che odiava. Incapace di controllarsi, la sua risata crebbe fino a diventare un terribile ululato, un animalesco annuncio di cattive intenzioni.

Denies si stava letteralmente cacando sotto. Sapeva quello che stava per succedere, ma sapeva anche di essere totalmente impotente. La situazione sarebbe degenerata, Fabio avrebbe ucciso il Gazzi e tutto sarebbe andato ai maiali. Niente di quello che lei poteva fare avrebbe potuto cambiare il corso di questi eventi.

Ci provò comunque.

Si avvicinò a Fabio lentamente, gli prese una mano fra le sue e gli sussurrò all'orecchio <Fai per bene. non so che altro dirti, se non ti prego, ti scongiuro, fai per bene>

La folle risata di Fabio si trasformò in una ringhiata, folle risposta. <Oh sì, vecchia mia, farò proprio per bene>

Fu solo questione di un momento. Gazzi si ritrovò una pistola puntata dritta in mezzo agli occhi.

Vittoria urlò e LAvinia imprecò, Denise si sedette semplicemente per terra, sconfitta. 

<Giacomo Gazzi> ripeté fabio, col tono solenne di chi stava pronunciando una sentenza. <Il bagonghi è morto. il Brogelli è morto. tu sei ancora qui, e questo non va bene.> 

GAzzi realizzò finalmente chi aveva davanti.

<Fabio>






// dove sta scritto che debba tramare una sola persona per volta?

con gazzi:

<Per me!?> fece Lavinia, minacciosa.
<Io non la voglio di certo la tua carità! Ci penso da sola a
vendicarmi --- e tra l'altro non mi voglio vendicare, e di certo non di
lui!>

Si rivolse al Gazzi con uno sguardo intriso del più puro disprezzo.
<Lo so che persona sei, stronzo, viscido e marcio fino al
budello! Ti piace, ti ecciti proprio a ficcare il dito nelle piaghe
degli altri, a cercare di smerdare chiunque sia in difficoltà per
sembrare più bello, più grande, più importante! Mi fai schifo, e per
ogni cattiveria che hai sputato su di me e chi mi sta caro te ne farò
ingoiare tre appena ne avrò l'occasione, puoi contarci. Ma
\emph{te}>

Fabio fece un patetico salto all'indietro, come a volersi scansare
dall'incandescente dito accusatorio che Lavinia gli aveva puntato
contro.

<Con \emph{te} farò proprio i conti,>


// vittoria chiama la polizia col telefono di Fabiomi hai rubato il telefono
e tu mi hai rubato l'iphone, credi forse che l'abbia dimenticato?