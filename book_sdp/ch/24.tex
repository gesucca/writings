\chapter{Qualcosa}

Lo spazio era bizzarro, un cosmico pulviscolo oscuro dentro il quale Fabio fluttuava senza alcun controllo sui suoi movimenti. Sentiva l'odore di quel nulla, così terribile e al tempo stesso così seducente; ne avvertiva la presenza al tatto, come se quei colori freddi e sinistri lo potessero toccare. Fabio lo sapeva di aver perso il cervello, ma la conoscenza non aveva potere quel luogo misterioso: lui era quello che provava, e provava ciò che era diventato.

Senza preavviso, priva di qualunque motivo per essersi manifestata, una voce lo raggiunse: era tagliente, acuta e gelida, ma aveva un timbro estremamente familiare.

...lo designerà come suo eguale, ma egli avrà un potere a lui sconosciuto... // a me sta storia continua a non tornarmi
...l'uno dovrà morire per mano dell'altro, perché nessuno dei due può vivere se l'altro sopravvive...

La voce di Lavinia tacque, ma gli echi di quelle rimbombanti frasi rimbalzarono a lungo fra la gabbia di follia in cui Fabio era rinchiuso. Era chiaro, fin troppo chiaro a chi si riferisse quella profezia. Un senso di determinazione lo pervase, un vivo incendio alimentava la sua ira; totalmente in balia di quell'ardente sensazione, si sentì prendere letteralmente fuoco, fino a diventare tutt'uno con le fiamme.

Appena il fuoco svanì, Fabio lo vide. Più bianco di un teschio, con grandi, lividi, occhi rossi, il naso piatto come quello di un serpente, due fessure per narici: privato dalle sue maschere, quello era il vero aspetto della sua nemesi.

« Sei stato uno sciocco a venire qui stanotte, Giacomo », gli disse, ostentando una tranquillità che non aveva.

Egli parlò, la voce acuta, distorta, ma inconfondibile:

« Per allora me ne sarò già andato, e tu sarai morto. »

Successe qualcosa di terribilmente strano, eppure Fabio non ne fu stupito: in qualche senso, se lo aspettava. Lo scheletrico Giacomo Gazzi pronunciò delle parole arcane, e una forza oscura esplose verso di lui. Fabio schivò con sdegnosa eleganza il pericoloso raggio verde e poi, estratto repentinamente un lungo e nodoso bastoncino dalla tasca, rispose al fuoco nemico con un ardente lampo di colore rosso acceso.

« Non vuoi uccidermi, Fontanelli? » gli gridò Gazzi, gli occhi scarlatti socchiusi. « Sei superiore a tanta brutalità, vero? »

« Sappiamo entrambi che ci sono altri modi per distruggere un uomo, Giacomo » replicò Fabio con amarezza. La sua maschera di perfetta tranquillità fu incrinata appena da una violenta ondata di rancore. « Ammetto che non mi darebbe soddisfazione toglierti soltanto la vita... »

// spezzare di più il cambio di scena
All'improvviso, un raggio color cremisi balenò nell'aria, e il suo nemico cadde pesantemente a terra, colpito. Questo, Fabio proprio non se lo era aspettato. Si voltò di scatto terrorizzato e, sopra una roccia che prima non c'era, vide qualcosa di completamente folle: un uomo basso, totalmente nudo e senza sesso, la pelle perlacea e liscia, con la testa e le spalle di uno sgradevole colore. Bruno Bagonghi non aveva mai avuto quell'aspetto, ma quell'orribile figura era inconfondibilmente una sua incarnazione.

Qualcuno accanto a lui parlò:

« Voglio combattere! Devo vendicare il mio amico Giacomo! »

Era Daniele. Aveva un vestito strano, era più basso e aveva un ridicolo raglio di capelli, ma non c'erano dubbi che fosse lui. Stava guardando Fabio con aria sconvolta, ma terribilmente determinata.

Fabio provò una travolgente sensazione di rabbia. Rivolse lo sguardo verso quella disgustosa cosa che era Bruno, e con tutta la grinta che riuscì a trovare, ordinò al suo amico:

« Vattene subito! È me che vuole, non te! Se resti qui, farai la sua stessa fine! »

La pallida imitazione di Bruno Bagonghi parlò con la voce melliflua, leggera, ma intrisa di malvagità:

« Sei un illuso! Che cosa ti fa pensare che lo lascerò andare via? »

Un gelido sorriso, una risata e un semplice gesto con un dito: Daniele fu sollevato da terra, totalmente in balia della volontà del suo aggressore.

« Sei così ridicolo, moscerino! »

Fabio non poté fare niente: il suo amico urlò e si contorse, poi esplose.

Fabio rimase immobile, totalmente privo della forza necessaria per comprendere quello che stava accadendo. Perché proprio Daniele? Era indifeso e debole, non era una minaccia da eliminare! Dopo tutto il male che il pallido Bagonghi aveva fatto, dopo tutte le cose meschine e crudeli di cui si era reso responsabile... Una terribile rabbia si impadronì di Fabio, una furia devastante dalla sua schiena fluiva in ogni anfratto del suo corpo; un'aura dorata gli comparve attorno, ed urlò al nemico tutta la sua collera. Bagonghi non la avrebbe passata liscia.

Fabio estrasse nuovamente il suo bastoncino e si scagliò verso il Bagonghi, che aveva abbandonato il suo pallore in favore di una maschera nera. In pochi istanti, Fabio si trovò a brandire una verde spada luminosa contro quello che era sempre stato il suo perfetto opposto, il suo acerrimo nemico, tutto quello che non sarebbe mai diventato fintanto che si fosse voluto bene. Egli rispose all'attacco con un'arma simile alla sua, ma di colore rosso; le due spade lucenti collisero in una tempesta di violente scintille, e i due si trovarono a combattere un duello senza esclusione di colpi su una stretta passerella sospesa sul vuoto.

Per quanto Fabio si potesse impegnare, non c'era niente da fare: il suo nemico era più forte. Bastò un attimo di incertezza, uno spiraglio aperto nella sua guardia, e un fendente gli tagliò di netto una mano. Un dolore terribile lo straziò: cadde a terra urlando, aspettando la sua fine.

« Sei battuto. È inutile resistere », sentenziò la nera figura, la voce offuscata dalla maschera che indossava. « Non hai scampo. Non lasciarti distruggere come ha fatto Daniele. Fabio, tu non ti rendi ancora conto della tua importanza! Hai solo cominciato a scoprire il tuo potere. Vieni con me e io completerò il tuo addestramento. Unendo le nostre forze possiamo mettere fine a questo conflitto distruttivo e riportare l'ordine a Prato! »

Fabio fu colpito da quelle parole, ma non esitò un istante a strillare la giusta risposta:

« Non verrò mai con te! »

La figura oscura non si scompose.

« Se tu solo conoscessi il potere del lato oscuro... » disse, lasciando che una certa dolcezza filtrasse attraverso la sua maschera. « Giacomo non ti ha mai detto cosa accadde a suo padre? »

Fabio rimase un attimo interdetto. Qualcosa era andato in frantumi.

« A... suo padre? » chiese confuso. « A mio padre, semmai! »

Fabio non capiva. Per qualche ragione, credeva di sapere già dove doveva andare a parare quella conversazione - ma quell'ultima battuta era sbagliata. Eppure sembrava esserci tutto: la passerella sul vuoto, le spade laser, il tizio mascherato di nero...

« Scusa - scusa una cosa... » tentò, guardingo. « Non è che mi puoi ripetere l'ultima cosa che hai detto? »

Dopo un lungo istante di silenzio, dalla maschera uscirono delle parole:

« Non lasciare che ti distrugga. »

Fabio si sentiva sempre più confuso, Si grattò la testa con la mano destra, ma poi si ricordò che la aveva persa pochi momenti prima. Qualcosa decisamente non tornava.

« Non lasciare che ti distrugga come ha fatto Giacomo », continuò imperterrita la figura ammantata.

« No, non quella cosa. Ascolta, hai detto qualcosa sul padre di Giacomo, ma... »

« Non lasciarti distruggere come ha fatto Anton. »

« No, no! » fece Fabio, frustrato. « Non questa, quella sul padr - aspetta, hai detto Anton? Hai distrutto Anton? Che significa? »

« Giacomo non ti ha mai detto cosa accadde a suo padre? »

« Ooh! » esclamò trionfante Fabio. « Quello volevo risentire! Ascolta, mi sa che hai confuso un po' il copione, dovresti alludere al fatto che tu sei mio padre, no? »

Con snervante lentezza, il figuro oscuro si tolse la sua maschera; il volto di Bruno Bagonghi era invecchiato e impallidito, ma perfettamente riconoscibile.

Esibì uno strano sorriso condiscendente, e parlò:

« Tuo padre? Guarda che abbiamo più o meno la stessa età, coglione! »

In quel preciso istante, Fabio seppe di essere pazzo. La sua sospensione di incredulità perse finalmente ogni appiglio sul suo essere, e lui capì di non capire proprio più niente. Scoppiò a ridere, senza freni, senza inibizioni, senza il minimo ritegno.

Delle parole arrivarono dal Bagonghi:

« Cerca dentro di te; tu sai che è vero. »

Fabio rise ancora più forte. Si accasciò a terra, impotente, e si sentì improvvisamente cadere. Qualcosa di terribile gli stava accadendo. Serrò istintivamente gli occhi, ma era impossibile proteggersi dalle forme dai bizzarri colori che gli danzavano intorno. Ebbe la fortuna di riuscire a utilizzare con successo il briciolo di senno che gli era rimasto: capì che cosa gli stava succedendo, e si arrese.

Impotente e volutamente passivo, si abbandonò al vorticante del maelstrom di follia in cui si trovava. Passarono quelli che a Fabio sembrarono giorni: successero ancora cose e comparvero altre persone ma, alla fine, quella bolla di finzione solipsistica si dissolse, e la realtà coninciò lentamente a riacquistare la sua usuale solidità.