Lo spazio era bizzarro, un cosmico pulviscolo oscuro dentro il quale Fabio fluttuava senza alcun controllo sui suoi movimenti. Sentiva l'odore di quel nulla, così terribile a al tempo stesso così seducente; ne avvertiva la presenza al tatto, come se quei colori freddi e sinistri lo potessero toccare. Fabio lo sapeva di aver perso il cervello, ma la conoscenza non aveva potere quel luogo misterioso: lui era quello che provava, e provava ciò che era diventato.

Senza preavviso, priva di qualunque motivo per essersi manifestata, una voce lo raggiunse: era tagliente, acuta e gelida, ma aveva un timbro estremamente familiare.

...lo designerà come suo eguale, ma egli avrà un potere a lui sconosciuto...
e l'uno dovrà morire per mano dell'altro, perché nessuno dei due può vivere se l'altro sopravvive...

La voce di Lavinia tacque, ma gli echi di quelle rimbombanti frasi rimbalzarono a lungo fra la gabbia di follia in cui Fabio era rinchiuso. Era chiaro, fin troppo chiaro a chi si riferisse quella profezia. Un senso di determinazione lo pervase, un vivo incendio alimentava la sua ira; totalmente in balia di quell'ardente sensazione, si sentì prendere letteralmente fuoco, fino a diventare tutt'uno con le fiamme.

Appena il fuoco svanì, Fabio lo vide. Più bianco di un teschio, con grandi, lividi, occhi rossi, il naso piatto come quello di un serpente, due fessure per narici: privato dalle sue maschere, quello era il vero aspetto della sua nemesi.

« Sei stato uno sciocco a venire qui stanotte, Giacomo », gli dsse, ostentando una tranquillità che non aveva. « Gli Auror stanno per arrivare. »

Egli parlò, la voce acuta, distorta, ma inconfondibile:

« Per allora me ne sarò già andato, e tu sarai morto. »

Successe qualcosa di terribilmente strano, eppure Fabio non ne fu stupito: in qualche senso, se lo aspettava. Lo scheletrico Giacomo Gazzi prununciò delle parole arcane, e una forza oscura esplose verso di lui. Fabio schivò con sdegnosa eleganza il pericoloso raggio verde, poi, estratto repentinamente un lungo e nodoso bastoncino dalla tasca, rispose al fuoco nemico con un ardente lampo di colore rosso acceso.

« Non vuoi uccidermi, Fontanelli? » gli gridò Gazzi, gli occhi scarlatti socchiusi. « Sei superiore a tanta brutalità, vero? »

« Sappiamo entrambi che ci sono altri modi per distruggere un uomo, Giacomo » replicò Fabio con amarezza. La sua maschera di perfetta tranquillità fu incrinata appena da una violenta ondata di rancore. « Ammetto che non mi darebbe soddisfazione toglierti soltanto la vita… »

All'improvviso, un lampo color cremisi balenò nell'aria, e la sua nemesi cadde pesantemente a terra, colpita. Questo, Fabio proprio non se lo era aspettato. Si voltò di scatto terrorizzato e, sopra una roccia che prima non c'era, vide qualcosa di completamente folle: un uomo basso, totalmente nudo, la pelle perlacea e liscia, con la testa e le spalle di uno sgradevole colore. Bruno Bagonghi non aveva mai avuto quell'aspetto, ma quella orribile figura era inconfondibilmente una sua incarnazione.

Qualcuno accanto a lui parlò:

« Voglio combattere! Devo vendicare il mio amico Giacomo! »

Era Daniele. Aveva un vestito strano, era più basso e con un ridicolo raglio di capelli, ma non c'erano dubbi che fosse lui. Lo stava guardando con aria sconvolta, ma determinata. 

Fabio provò una travolgente sensazione di rabbia. Rivolse lo sguardo verso quella disgustosa cosa che era Bruno, e ordinò con tutta la grinta che riuscì a trovare:

« Vattene subito! È me che vuole, non te! Se resti qui, farai la sua stessa fine! »

La pallida imitazione di Bruno Bagonghi parlò con la voce melliflua, leggera, ma intrisa di malvagità:

« Sei un illuso! Che cosa ti fa pensare che lo lascerò andare via? »

Un gelido sorriso, una risata e un semplice gesto con un dito: Daniele fu sollevato da terra, totalmente in balia della volontà del suo aggressore.

« Sei così ridicolo, moscerino! »

Fabio non poté fare niente: il suo amico urlò, si contorse e semplicemente esplose. 

Dart Fener: Sei battuto. È inutile resistere. Non lasciarti distruggere come fece Obi-Wan. Non hai scampo. Non lasciare che ti distrugga. Luke, tu non ti rendi ancora conto della tua importanza. Hai solo cominciato a scoprire il tuo potere. Vieni con me e io completerò il tuo addestramento. Unendo le nostre forze possiamo mettere fine a questo conflitto distruttivo e riportare l'ordine nella galassia.
Luke Skywalker: Non verrò mai con te!
Dart Fener: Se tu solo conoscessi il potere del lato oscuro. Obi-Wan non ti ha mai detto cosa accadde a tuo padre!
Luke: Mi ha detto abbastanza: che sei stato tu ad ucciderlo!
Dart Fener: No, io sono tuo padre![1]
Luke: No! Non è vero! Non è possibile!!
Dart Fener: Cerca dentro di te! Tu sai che è vero!
Luke: Nooo!! Noo!!
Dart Fener: Luke, tu puoi distruggere l'Imperatore. Lui lo ha previsto. Questo è il tuo destino. Unisciti a me e insieme potremo governare la galassia, come padre e figlio. Vieni con me. È l'unica strada.