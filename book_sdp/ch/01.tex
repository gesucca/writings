\chapter{Cuore Non Duole}

\begin{chapquote}{Pop X, \textit{La Prima Rondine Venne Ier Sera}}
	Figli di puttana senza padri e senza mamma\newline
	Siamo battiti animali, siamo porci senza le ali
\end{chapquote}

% pulled
% tex syntax
% rough editing

Fabio si tormentava la barba. L'aria salmastra di Barcellona la rendeva crespa, stopposa, appiccicosa. Non riusciva a sopportarla. ``Che stress'' pensò, ricacciando nel groviglio rossiccio un ricciolo particolarmente ribelle. Non ci poteva fare niente: non riusciva a togliersi le mani da barba e capelli. Erano lunghi e folti, e prudevano da impazzire. Nella sua vecchia routine quotidiana, il rasoio era stato il suo migliore amico. Ma la pacchia adesso era finita, le circostanze gli imponevano un aspetto radicalmente diverso dal passato; sempre nervoso e privo di pazienza, Fabio si trovava in seria difficoltà.

Senza perdere d'occhio la coppia che stava seguendo, cercò di distrarsi dal suo tormento con una sigaretta. Non era affatto un fumatore incallito, anzi, di solito non fumava quasi mai. Tuttavia, per la gioia dei suoi amici, si era sempre concesso il pacchetto in tasca. Nemmeno la solitudine lo aveva convinto a privarsi del suo accessorio e, visto che da qualche tempo nessuno gli scroccava più le sigarette, era finito a fumarsele tutte da solo. Non era abituato ad un tale consumo di tabacco: il saporaccio in bocca, l'arsura costante, la tosse, la sensazione di essere sporco, contaminato\ldots\ Erano disagi sopportabili, confrontati con il piacere di succhiare pulviscolo rovente da un bastoncino incendiato. E poi Fabio era morto, almeno per quanto ne poteva sapere il resto del mondo, ed ai morti fumare non fa male.

I suoi due bersagli finalmente si mossero. Con snervante lentezza, si alzarono dalla panchina che aveva ospitato le loro natiche da almeno un'ora.

«\textit{Oh, e l'era l'ora, serpe d'Id\ldots}» imprecò Fabio, incapace di trattenersi.

Odiava aspettare le persone, anche quando doveva derubarle. Trattenendo a stento l'impazienza, si mise a seguire la coppia con goffa discrezione. Mentre camminava, si tastò la cintura e sospirò. L'arma era ben nascosta alla vista, ma il metallo a contatto con la nuda pelle ricordava costantemente a Fabio di esserne in possesso; ce l'aveva già da qualche giorno, ma si sentiva ancora insicuro nel portarla in giro, soprattutto sapendo di doverla usare: non aveva mai infranto la legge prima di allora, eccetto stupidaggini come il codice della strada ed il divieto di possedere droghe leggere. Non c'era più tempo per vuoti timori o scrupoli morali: era scomparso o addirittura morto, cosa avrebbe dovuto temere? Sospirò di nuovo. ``Non ho nulla da perdere'' si disse, sapendo di mentire. Sapeva anche che la sua coscienza non si sarebbe zittita a suon di menzogne: doveva semplicemente pensare di meno ed agire di più.

Affrettò il passo, avvicinandosi ai due tizi che si era prefissato di rapinare. Li aveva seguiti per tutto il pomeriggio, senza perderli di vista un momento: erano certamente dei turisti, entrambi sui quarant'anni, forse sposati e di certo molto facoltosi. Americani? Inglesi? Qualunque fosse la loro provenienza, il loro abbigliamento, tanto appariscente quanto ridicolo, non lasciava molto all'immaginazione: era la classica coppia di ricconi in vacanza, annoiata e già pentita di aver abbandonato il limbo dorato da cui proveniva. In altre parole, rappresentavano in pieno lo stereotipo di persone che Fabio odiava di più in assoluto. Come primo bersaglio erano perfetti, bastava solo continuare l'inseguimento ad oltranza, fino a che i riccastri non si fossero messi in condizione di essere predati facilmente. All'inizio del pedinamento era parsa proprio questione di momenti, la donna si comportava proprio come una stupida oca e l'uomo non aveva l'aria di saper badare a se stesso. Avrebbe dovuto essere facile, ma il sole era già tramontato e Fabio ormai stava per perdere la pazienza.

Poco prima di crollare di nervi e fare una strage, ecco che li vide entrare in una piccola pineta sul lungomare, vicino a La Barçeloneta. Il crepuscolo ormai stava cedendo il passo alla notte ed il posto era praticamente deserto. I due turisti si accasciarono su una panchina, evidentemente esausti per aver trasportato le loro membra per duecento metri. Fabio fremette: era quella la sua occasione!

Prima di agire, si prese un attimo di tregua. Fu fatale: come se non stesse aspettando altro, il cuore gli saltò in gola. Inspirò profondamente, si morse il labbro inferiore ed assaporò il salmastro che la brezza ci aveva depositato. Ora che il sole era calato, il vento marino non era più fresco sollievo, ma pungente fastidio. In qualche strano modo, quei ventosi spilli gelidi lo eccitavano: era uno strano freddo, un freddo inconfondibilmente estivo che lo stuzzicava nel profondo. Guardò attraverso i rami dei pini, rivolto verso un cielo che sovrastava fin troppe persone\ldots

Fino a che punto la sua coscienza gli avrebbe impedito di fare quello che la sua volontà gli imponeva? Aveva veramente intenzione di uccidere? In fondo, quello era il suo intento originale. Non era tanto per provare la pistola --- quello lo avrebbe fatto comunque, in un modo o nell'altro --- ma per testare se stesso. L'avventura che intendeva intraprendere imponeva la capacità di uccidere a sangue freddo: ne sarebbe stato in grado? Pensò che assassinare due innocenti non era proprio il massimo per iniziare; avrebbe preferito giustiziare per esempio un pluriomicida, o uno stupratore, o un politico. Ma, tutto sommato, quelle amebe non erano sicuramente innocenti e rappresentavano ciò che odiava. Sarebbe stato il battesimo del fuoco, la prova perfetta per temprarlo definitivamente: uccidere ciò che lui aveva deciso essere indegno di vivere. Sapeva benissimo, nel profondo, che l'unica ragione del suo odio per la gente del genere era l'invidia: loro erano ricchi e senza preoccupazioni mentre lui no. Sarebbe bastata quella pseudo colpa a giustificare a se stesso la loro esecuzione a sangue freddo? Trasse un altro respiro profondo, assaporò la tensione\ldots\ Sapeva come comportarsi, ed era consapevole delle conseguenze: decise che era tempo di agire e non di pensare.

Estratta l'arma e tolta la sicura, girò attorno alla panchina e si avvicinò alla coppia più silenziosamente che poté. La sua cautela fu inutile, dato che lo starnazzare della sua vittima femmina avrebbe impedito a chiunque di notarlo anche se si fosse messo a cantare.

«\textit{I ain't gonna eat in that shack!}» berciava la snob. Gli ammennicoli appesi alle sue braccia tintinnavano furiosamente mentre gesticolava: avrebbero sovrastato perfino il chiasso di un carro armato.

«\textit{Dear, it's a de\ldots}» provò a rispondere l'uomo, ma altri starnazzi lo chetarono.

Fabio provò un moto di compassione per quell'uomo. ``Se la faccio secca, ci sta che mi ringrazi\ldots'' pensò, in un feroce slancio di cinismo.

«\textit{I ain't gonna eat there! Go find a real place!}» proseguì lei.

Fabio era già stufo della sua voce. Si caricò di determinazione, fece un bel respiro e si manifestò con l'arma in mano.

«\textit{You ain't gonna eat any fuckin' where, if you don't shut the fuck up!}» eruppe, sfoggiando un perfetto inglese da film d'azione.

La donna strillò alla vista della pistola. L'uomo invece sgranò gli occhi e guardò con apprensione il suo aggressore, senza emettere alcun suono.

«\textit{Give me money, quick. No time to waste.}» disse Fabio, aggressivo.

Calò un silenzio assoluto, quasi imbarazzato, rotto soltanto dai mugolii spaventati della riccona avvinghiata stretta al braccio del suo cavaliere, il quale sembrava sul punto di piangere.

«\textit{Don't be a motherfuckin' cunt}» gli disse Fabio, impaziente. «\textit{You are disgustingly rich, I'm not. I've got a weapon, you don't. Can you see how things combine perfectly? Come on! Your money for my clemence.}»

Per l'effetto che ebbero le sue parole, avrebbe potuto parlare alla panchina. L'immobilismo dei due stava per fargli perdere il controllo. \textit{``Mostro di\ldots! Che ti mòvi?''} imprecò nella sua testa. Puntò la pistola dritta in mezzo agli occhi dell'uomo.

«\textit{I said: your money\ldots}» ripeté, mimando i soldi con la mano libera, «\textit{\ldots for my clemence. Don't you think it's a good deal?}»

Nessuno dei due emise alcun suono, e l'atmosfera si fece così densa che Fabio si stupì di riuscire a respirare. La collera lo stava divorando dall'interno. Perché il tizio non faceva niente? Lo stava minacciando per avere i suoi soldi, non sembrava affatto una situazione ambigua! Non riusciva proprio a capire perché le sue vittime si fossero pietrificate, invece di sbrigarsi a consegnare i soldi per aver salva la vita.

Finalmente, con tremenda lentezza, l'uomo annuì. Fabio sorrise, sollevato. Abbassò l'arma e tese pronto la mano, stampandosi in volto un falsissimo sorriso da venditore porta a porta. Successe qualcosa di incredibile: la donna trasse un enorme respiro, scoccò all'uomo uno sguardo di puro disprezzo e gli tirò uno schiaffo, cominciando a strillare furiosa.

«\textit{Are ya givin' up?! This' the way you are protectin' me?! Fight like a man, protect me!}»

Fabio ci rimase di stucco: questo proprio non se l'era aspettato. Scoppiò a ridere, completamente spiazzato dall'assurdità della scena, ma puntò con fermezza la pistola verso la signora imbizzarrita.

«\textit{I want nothing from you, you maggot!}» le disse, divertito.

Lei si chetò all'istante.

«\textit{Shut up and let your man give me some money}», proseguì. «\textit{Be happy for I'm not raping you\ldots}»

La donna si impietrì di nuovo, terrorizzata, e la situazione parve risolversi: l'uomo estrasse dei soldi dal suo costoso portafogli e li depose a terra. Fabio intimò alla coppia di stare indietro e li raccolse, soffermandosi a contarli.

Non soddisfatto, si rivolse all'uomo:

«\textit{I saw what you did, you still got some. Would you mind give me another green one? Come on, I bet you earn one of this in a day.}»

Lui obbedì, ma non proferì parola. La povera stolta, innescata dalla vista di Fabio che contava i soldi che avrebbe voluto spendere lei, ricominciò a starnazzare. Con un teatrale, grazioso movimento, percosse con la borsa il suo compagno in un improvviso slancio di follia isterica. L'uomo non reagì, anzi sospirò, probabilmente sollevato per aver avuto salva la vita. Fabio rise di nuovo. Si sentiva molto più leggero ora che avvertiva la presenza dei soldi nella tasca. Il suo proposito omicida gli sembrava un bizzarro ricordo. Perché avrebbe dovuto ucciderli? Aveva già quello che voleva.

Prima che realizzasse quanto fosse stupido intrattenersi ancora con le sue vittime, parlò:

«\textit{Leave him alone! He was just robbed. Can you please shut the fuck up?}»

L'aria si congelò di nuovo, anche se Fabio ormai aveva messo via la pistola. Cosa diavolo stava per succedere? L'uomo lo guardò eloquentemente, spaventato come mai era stato fino ad allora. La donna emise un orrendo sibilo e, senza alcun preavviso, si scagliò su Fabio, sputandogli addosso e cominciando ad insultarlo con i suoi starnazzi. Lui fece un balzo indietro, confuso dall'improvvisa aggressione, ma ben pronto a reagire.

«Come osi, stupida!», ringhiò, e le assestò un perfetto yoko-geri che la spinse indietro di parecchi metri.

Il karateka che era in lui non si sentì affatto a disagio nel colpire una donna: un'aggressione chiama sempre una difesa ed un contrattacco, indipendentemente dalle circostanze. Ma l'azione improvvisa, come un fruscio nel sottobosco per un cacciatore, aveva evocato un Fabio ben più pericoloso del karateka: il Fabio predatore.

Anche se il colpo aveva atterrato l'isterica, non la aveva certo zittita. ``Non ha proprio istinto di sopravvivenza\ldots'' pensò Fabio, mentre estraeva nuovamente la pistola.

All'improvviso, per la prima volta, l'uomo fece uso della parola: «\textit{Cecilia, quiet!}»

La sua voce, inaspettatamente ferma e dal timbro profondo, risuonò per il parco in una sorta di inquietante eco. La folle si zittì immediatamente, ma ormai era inutile: Fabio moriva dalla voglia di farlo. In uno di quei rari momenti in cui l'azione bypassa completamente l'intelletto, puntò la sua nove millimetri silenziata sulla donna, sorrise all'uomo e sparò. Non fu rumoroso: il colpo produsse solo un suono sordo, simile ad uno starnuto. Era l'arma perfetta per i suoi scopi, pensò Fabio prima di riconnettersi alla realtà: nessuno avrebbe riconosciuto quel rumore per quello che era.

La snella figura si contorceva a terra: era stata colpita sul fianco. Resosi conto di ciò che aveva fatto, Fabio fu pervaso da un panico molto strano, soprattutto inatteso. Guardò il corpo spasimare dal dolore e macchiarsi di sangue: la donna non sembrava ferita mortalmente. Non urlava, ma piangeva. Cercando di calmarsi, si leccò le labbra, sempre appiccicose di salmastro, e gli parve di assaporare le lacrime della sua vittima. Il panico si trasformò immediatamente in eccitazione, una selvaggia e crudele eccitazione. Non aveva mai provato una sensazione del genere, si sentiva come ubriaco, ma con il controllo della situazione. Non abbassò l'arma, ma si avvicinò alla donna ferita. Lei cercò di trattenere il pianto e chiuse gli occhi, tremante.

«\textit{I can't stand those like you}», le disse piano, scandendo le parole per assicurarsi che lei lo capisse. «\textit{You deserve this pain.}»

L'uomo nel frattempo si stava esibendo nella sua migliore performance di immobilità: era totalmente interdetto, sopraffatto dagli eventi.

Fabio si rivolse a lui, il volto deformato dalla malvagità, l'arma puntata verso la sua vittima:

«\textit{Should i do it?}»

Si aspettava che negasse, che lo supplicasse di lasciare in vita la sua amata. Era già pronto a ridere di lui. Invece l'uomo, celato dietro un'espressione grave, sembrava soppesare la possibilità di rispondere in disaccordo con il copione dell'ovvio. I secondi passavano, l'adrenalina stava svanendo, l'eccitazione stava lasciando il posto ad un più consono disagio. E se gli avesse chiesto di farlo? Lo avrebbe fatto davvero? Anche a sangue freddo? Lentamente, tremando, l'uomo scosse il capo.

Fabio abbassò l'arma.

«\textit{Thank you}», gracchiò egli, trattenendo a stento le lacrime.

Fabio sgranò gli occhi allibito, e per poco non gli cadde la pistola. Mai avrebbe scommesso, neanche un fagiolo contro un lingotto d'oro, che una sua vittima lo ringraziasse. Di cosa, poi?

«\textit{She's a\ldots but\ldots please, let her live\ldots}»

Una folata di vento un po' più veloce delle altre alzò un lieve turbinio di aghi di pino. La situazione aveva raggiunto un livello di assurdità ormai non più tollerabile. Fabio provò il bizzarro impulso di aiutare la donna e di scusarsi con l'uomo, ma aveva ancora abbastanza autocontrollo da soffocarlo. Ripose la pistola e, senza dire altro, andò via a passo svelto. Era estremamente turbato: il suo velo di risolutezza stava per cadere, lasciandolo faccia a faccia con la realtà di ciò che era successo.

Raggiunto un punto del litorale abbastanza lontano dal luogo del delitto, fece dei respiri profondi e si calmò. La passeggiata si stava riempiendo di gente pronta per la movida notturna, ma in spiaggia c'erano ancora alcuni temerari. Fabio li guardò, perso nei suoi pensieri. Si mise una sigaretta in bocca, ma non la accese. Prese a tormentarsi la barba\ldots

Non era andata come aveva programmato, non aveva ucciso nessuno, almeno materialmente. Però, nel momento di premere il grilletto, era diventato comunque un assassino: aveva deciso di uccidere, ed aveva posto in essere un'azione apposita per raggiungere quello scopo. Era moralmente un omicida tanto quanto lo sarebbe stato se il suo proiettile avesse colpito un punto vitale. Poteva giovarsi di questa sicurezza: era in grado di uccidere. Ma la cosa, invece di corroborarlo, lo turbava. Cosa diavolo era diventato?

E poi la reazione dell'uomo\ldots Alla fine si era lasciato andare, sicuramente disperato al pensiero di poter vedere la sua amata morire, ma per qualche istante aveva accettato l'avvenimento ed addirittura valutato la possibilità di farlo accadere. In quel momento, prima di rendersi conto dei torbidi lidi nei quali i suoi pensieri avevano vagato, il povero derubato era stato freddo, cinico, calcolatore: si era immaginato, in modo perfettamente razionale, come sarebbe potuta evolvere la sua vita senza quella palla al piede. Se fosse riuscito a resistere ai sentimenti, a restare razionale, sensato, solo per un altro minuto, avrebbe guadagnato la libertà. Ma non sarebbe potuto succedere\ldots Se l'uomo avesse posseduto quella forza, avrebbe saputo anche spezzare il suo legame con quella stupida oca senza aver bisogno di farla ammazzare. O addirittura, non avrebbe mai provato dei sentimenti per una persona così immeritevole\ldots

Ma che ne sapeva lui? Aveva troppi pregiudizi, troppa supponenza, ancora troppa empatia nei confronti dei suoi simili. La donnicciola magari non era come gli appariva, forse era lei dalla parte giusta della relazione; forse era il pover'uomo ad essere il suo peso. O forse doveva smetterla di pensare a chiunque tranne che a se stesso. Oppure, addirittura, come si trovava a dirsi fin troppe volte, doveva smettere di pensare e muoversi.
