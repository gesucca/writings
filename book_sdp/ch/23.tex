\chapter{Qualcosa}

Fu questione di un momento: le due anime, così vicine da essersi allontanate, si sfiorarono appena nell'attraversare il cancello del cimitero; Fabio per un verso, Lavinia per un altro. L'aria si era come congelata, terrorizzata da quello che sarebbe potuto accadere se quelle due persone si fossero riconosciute. Denise era muta, le mani sulla bocca che stringeva le labbra fra i denti; Vittoria era guardinga, anche lei improvvisamente cheta, conscia che due masse di materiale fissile avevano rischiato di collidere.

Invece, non successe niente: Fabio proseguì nel cimitero come se niente fosse, e Lavinia continuò la sua marcia verso la sua automobile, i capelli e i vestti fradci, gli occhi gonfi e lo sguardo a terra.

Denise e Vittoria si scambiarono un'intensa occhiata.

< Quella lì... >, disse Vittoria a voce molto bassa. Aveva capito.

Denise annuì, senza osare togliere le mani dalla bocca.

< Porca vacca >, fece Vittoria, scrollandosi di dosso i brividi che le erano venuti. < Però, porca vacca davvero >, aggiunse.

Denise sospirò a sua volta. < Meno male non l'ha vista >, sussurrò con voce tremante < cazzo, avrebbe fatto una strage! >

< No, dicevo porca vacca per l'ombrello di quel Govidi. L'ho lasciato nel taxi! >

< Ah. >

< E poi - cazzo! Quello mi ha dato un coso, lo devo pigiare o no? > 

< Eh? >

< Va beh, cazzi suoi, io lo pigio! >

Denise non sapeva che pesci prendere. < Fai come ti pare >, disse rassegnata alla ragazzina.

Fra le due calò il silenzio. Denise osservò per un attimo la sua nuova ammica: quella spontaneità, la capacità di provare emozioni intense e scrollarsele di dosso in un attimo, in qualche senso le ricordava un po' Lavinia. Un'amara nostalgia la pervase al pensiero che, un tempo ormai remoto, pensava che la nuova ragazza del suo migliore amico fosse una tipa tutto sommato niente male. Che Fabio rivedesse nella sua giovane conquista qualcosa che, in fondo, un po' gli mancava?

< Il nostro caro amico, prima, ha rammentato della droga >, fece vagamente Denise, colta da un'improvvisa ispirazione.

Vittoria non si scompose nemmeno. < Mi vuoi fare compagnia? >, le chiese spassionatamente.

Denise sorrise. 

< Fammi un po' vedere che roba hai... >

***

Fabio credeva di essere pronto. Si era immaginato molte volte quel momento, cercando di rafforzarsi il più possibile contro le emozioni che sarebbero sorte alla vista della tomba del suo migliore amico. Tuttavia, non appena delle impietose lettere incise nel marmo gli urlarono in faccia la verità, non poté farci niente: crollò.

Bruno Bagonghi
16 febbraio 1990
1 dicembre 2018

Si il n'était pas mort, il serait encore en vie.

Si lasciò cadere sulle ginocchia, arrendendosi di fronte alla morte di una delle persone a lui più care. Appoggiò la fronte sul gelido marmo della tomba, e pianse per quello che parve un secolo.

Qualcuno lo toccò gentilmente, ma lui non si mosse. Sentì Vittoria singhiozzare alle sue spalle. Non avrebbe dovuto permettere che lei vedesse quel lato di sé, ma niente di quello che gli era accaduto aveva senso in confronto a quelle incisioni.

Come se gli avesse letto nel pensiero, Denise parlò: < se vuoi ti lasciamo solo, ma non ti devi vergognare.>

< Lo - lo ha fatto... > tentò Fabio, la voce rotta dal pianto.

< Si... si capisce >, proseguì Denise, abbracciando Fabio da dietro. < Se non ti scalfisce neanche una cosa così... cosa - cosa altro dovrebbe? >

Vittoria tirò su con il naso, e si avvicinò.

< Maledetto coglione >, sibilò Fabio, senza fiato. < Lo ha fatto - davvero... >

...