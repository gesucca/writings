Lo spazio era bizzarro, un cosmico pulviscolo oscuro dentro il quale Fabio fluttuava senza alcun controllo sui suoi movimenti. Ne sentiva l'odore, così terribile a al tempo stesso così seducente, ne avvertiva la presenza al tatto, come se quei colori freddi e sinistri lo potessero toccare. Fabio lo sapeva di aver perso il cervello, ma la conoscenza non aveva potere quel luogo misterioso: lui era quello che provava, e provava ciò che era diventato.

Senza preavviso, priva di qualunque motivo per essersi manifestata, una voce lo raggiunse: era tagliente, acuta e gelida, ma aveva un timbro estremamente familiare.

...lo designerà come suo eguale, ma egli avrà un potere a lui sconosciuto...
e l'uno dovrà morire per mano dell'altro, perché nessuno dei due può vivere se l'altro sopravvive...

La voce di Lavinia tacque, ma gli echi di quelle rimbombanti frasi rimbalzarono a lungo fra la gabbia di follia in cui Fabio era rinchiuso. Era chiaro, fin troppo chiaro a chi si riferisse quella profezia. Un senso di determinazione lo pervase, un vivo incendio che alimentava la sua ira; totalmente in balia di quell'ardente sensazione, si sentì prendere letteralmente fuoco, fino a diventare tutt'uno con le fiamme.

Appena il fuoco svanì, Fabio lo vide. Più bianco di un teschio, con grandi, lividi, occhi rossi, il naso piatto come quello di un serpente, due fessure per narici: privato dalle sue maschere, quello era il vero aspetto della sua nemesi. 

« Sei stato uno sciocco a venire qui stanotte, Giacomo », gli dsse, ostentando una tranquillità che non aveva. « Gli Auror stanno per arrivare. »

Egli parlò, la voce acuta, distorta, ma inconfondibile:

« Per allora me ne sarò già andato, e tu sarai morto »