\chapter{Qualcosa}

Fu questione di un momento: le due anime, così vicine da essersi allontanate, si sfiorarono appena nell'attraversare il cancello del cimitero; Fabio per un verso, Lavinia per un altro. L'aria si era come congelata, terrorizzata da quello che sarebbe potuto accadere se quelle due persone si fossero riconosciute. Denise era muta, le mani sulla bocca che stringeva le labbra fra i denti; Vittoria era guardinga, anche lei improvvisamente cheta, conscia che due masse di materiale fissile avevano rischiato di collidere.

Invece, non successe niente: Fabio proseguì nel cimitero come se niente fosse, e Lavinia continuò la marcia verso la sua automobile, i capelli e i vestti fradci, gli occhi gonfi e lo sguardo a terra.

Denise e Vittoria si scambiarono un'intensa occhiata.

< Quella lì... >, disse Vittoria a voce molto bassa. Aveva capito.

Denise annuì, senza osare togliere le mani dalla bocca.

< Porca vacca >, fece Vittoria, scrollandosi di dosso i brividi che le erano venuti. < Però, porca vacca davvero >, aggiunse.

Denise sospirò a sua volta. < Meno male non l'ha vista >, sussurrò con voce tremante < cazzo, avrebbe fatto una strage! >

< No, dicevo porca vacca per l'ombrello di quel Govidi. L'ho lasciato nel taxi! >

< Ah. >

< E poi - cazzo! Quello mi ha dato un coso, lo devo pigiare o no? > 

< Eh? >

< Va beh, cazzi suoi, io lo pigio! > (??? se lo pigia, arriva bagonghi mentre sono drogati)

Denise non sapeva che pesci prendere. < Fai come ti pare >, disse rassegnata alla ragazzina.

Fra le due calò il silenzio. Denise osservò per un attimo la sua nuova amica: quella spontaneità, la capacità di provare emozioni intense e scrollarsele di dosso in un attimo... In qualche senso, le ricordava un po' Lavinia. Un'amara nostalgia la pervase; un tempo ormai remoto, prima che le avversità della vita corrompessero la genuinità dei rapporti, prima ancora che l'esperienza le impartisse le lezioni necessarie a renderla adulta, aveva pensato che la nuova ragazza del suo più vecchio amico fosse una tipa niente male. Che Fabio rivedesse nella sua giovane conquista qualcosa che, in fondo, un po' gli mancava?

< Il nostro caro amico, prima, ha rammentato della droga >, fece vagamente Denise, colta da un'improvvisa ispirazione.

Vittoria non si scompose nemmeno. < Mi vuoi fare compagnia? >, le chiese spassionatamente.

Denise le sorrise. < Fammi un po' vedere che roba hai... >

***

Fabio credeva di essere pronto. Si era immaginato molte volte quel momento, cercando di rafforzarsi il più possibile contro le emozioni che sarebbero sorte alla vista della tomba del suo migliore amico. Tuttavia, non appena delle impietose lettere incise nel marmo gli urlarono in faccia la verità, non poté farci niente: crollò.

Bruno Bagonghi
16 febbraio 1990
1 dicembre 2018

*Si il n'était pas mort, il serait encore en vie.*

Si lasciò cadere sulle ginocchia, arrendendosi di fronte alla morte di una delle persone a lui più care. Appoggiò la fronte sul gelido marmo della tomba, e pianse per quello che parve un secolo.

Qualcuno lo toccò gentilmente, ma lui non si mosse. Sentì Vittoria singhiozzare alle sue spalle. Non avrebbe dovuto permettere che lei vedesse quel lato di sé, ma nessun contegno aveva senso in confronto a quelle incisioni.

Come se gli avesse letto nel pensiero, Denise gli parlò. < Se vuoi ti lasciamo solo, ma non ti devi vergognare. >

< Lo - lo ha fatto... > tentò Fabio, la voce rotta dal pianto.

< E' stata una giornatina niente male per tutti e due >, gli fece Denise, chinandosi per abbraciarlo da dietro. < Si capsice che sei sconvolto. Se non ti scalfisce neanche una cosa così... che altro dovrebbe? >

Vittoria tirò su con il naso, e si avvicinò.

< Maledetto coglione >, sibilò Fabio, senza fiato. < Lo - lo ha fatto scrivere - davvero... >

Con gentilezza, le due ragazze lo aiutarono a rialzarsi. Fabio tenne lo sguardo basso, in un patetico tentativo di mantenere nascosto il volto segnato dal pianto. Gli ci volle un po', ma dopo qualche istante di lotta interna riuscì a ricomporsi, regolarizzando il respiro.

Le due nuove amiche si scambiarono un breve cenno di intesa. Fu Denise a parlare:

< Il mio Daniele era una persona peggiore di te. >

La voce le tremava, ma lo aveva detto con tutta la calma del mondo, quasi avesse appena ribadito un'ovvietà così banale da non meritare particolare enfasi. Fabio si costrinse a guardarla negli occhi, ma non riuscà a dire niente.

< Ma sai, su varie cose ti stava un bel pezzo avanti >, proseguì Denise con una strana animosità. < Appena un paio d'ore fa mi hai raccontto di averlo visto morire. Sei tutt'ora convinto che sia stata colpa tua, e scusami se prima non mi sono data pena di contraddirti, ma ero tipo disperata - e te giustamente mi hai dato quel poco di conforto che potevi darmi, nessuno dice nulla su quello.  >

Denise si fermò per riprendere fiato, ma Fabio ancora non replicò. Quando lei riprese, nella sua voce era comparsa una generosa dose di risentimento.

< Hai versato almeno una lacrimuccia per lui? Ti sei sentito anche solo un po' triste per quello che era successo al tuo amico - non azzarti a dire di sì, perché ormai ho capito cosa fai! Ti ho sempre pensato come uno di quelli che non si fa mai vedere quando piange, che mantiene un contegno a tutti i costi per non contagiare gli altri col suo dolore - ma no, sono stata stupida, sono *sempre* stata stupida! Stupida e *cieca*, con Daniele come con te! >

Fabio ancora non reagì; sembrava scolpito nella pietra.

< Non ti è mai - *mai* - importato un cazzo di proteggere gli altri dal tuo dolore: sei *te* quello che si vuole nascondere, quello che ha paura di essere debole, fragile e - e basta, cazzo! >

Finalmente, Fabio si riscosse dalla sua immobilità: sospirò.

< Ah, ma ormai ti ho capito! > proseguì Denise, imperterrita. < Prima ti ho detto delle cose, ho cercato di riscuoterti, ti darti coraggio - ma te *non vuoi* il coraggio! Sai, fai sempre il duro, il sacrastico, il superiore, una alla fine può anche finire per crederci, eh? Vederti spezzato mi ha ferita - e te *non vuoi* vedermi ferita. La tua nuova amica mi ha detto tutto, sai? Si capisce che sei sensibile, l'ho sempre saputo, ma arrivare a rinnegare tutto quello di buono che c'è nel mondo pur di non patire scusami, ma è come buttarsi in un fosso quando piove per non bagnarsi! E che cazzo! Sei proprio un cretino. C'è voluta questa ragazzina per farmi vedere quello che avevo davanti agli occhi, ma ora, finalmente, *ho visto*. E ti assicuro che adesso mi torna tutto! Ora *lo so* perché te ne sei andato, perché non hai affrontato la Lavinia o il Gazzi, perch - >

< Hai finito? >

La voce di Fabio era stata così tagliente che Vittoria aveva sussultato.

< No >, fece Denise risoluta. < Devo anche dirti che forse stiamo per fare la cosa sbagliata, ma sappi che lo facciamo con le migliori intenzioni. >

Fabio sospirò di nuovo. < Fate pure. Tanto, peggio di così... >

Vittoria gli si avvicinò lentamente, ma Fabio la bloccò.

< Sputalo e dammene uno nuovo. Ormai l'hai assorbito tutto te, è mezz'ora che ce l'hai in bocca >, le disse rassegnato.

Vittoria obbedì, lasciando cadere dalla bocca un piccolo foglietto di carta.

< Volevo... >

Per la terza volta, Fabio sospirò.

< Fate un po' che cazzo vi pare >

---
o bagonghi redivivo arriva quando sono tutti e tre belliin drogati
ah, aspetta
Uhm
perché no
Direi la seconda
dai dai, ho già in mente anche la scena
il focus sul guardiano del cimitero, che spiega la presenta di loro tre oltre l'orario di chiusura
Lol
e che viene corrotto pure da bagonghi per farlo entrare
Mi piace l'idea
e questo lega anche con le visioni di fabio da drogato
e poi bagonghi gli lascia un sigaro
così che possa mantenere la promessa
sì, dai, posso farlo funzionare
