\chapter{cap 26}

Il cielo era inaspettatamente terso. Il firmamento si esibiva in quello
che sarebbe stato l'ultimo atto di quella notte: presto il sole sarebbe
sorto, sopraffacendo il flebile bagliore delle stelle con la sua luce.

Dolcemente immerso in una piacevole, stordita incredulità, Fabio
osservava la volta celeste. Era incantato dal mistero della grande luce
che ne avrebbe oscurate altre; riusciva a percepire quell'ossimoro in
tutta la sua potenza, lasciandosi sedurre da una rima che solo lui
riusciva cogliere. Poteva esistere una luce così splendente da
sopraffare le altre? Era sempre luce, anche se ne annientava altre? Era
un pensiero stupido, ma Fabio intuì la presenza di una potente metafora
in quel concetto: poteva un bene essere così intenso da renderne altri
irrilevanti - annullare il bene, senza diventare male?

Assurdo. Il bene ed il male non esistevano, questo Fabio lo sapeva con
certezza. Eppure si sentiva certo anche di qualcos'altro\ldots{}

Niente vale il debole bagliore del cielo stellato a confronto con lo
splendere del sole, così come il sorriso di uno sconosciuto è nulla
rispetto al complimento di un amico; eppure, quello stesso amico può
contare poco o anche niente, se la tua vita è illuminata dall'amore di
una famiglia unita ed in salute. Poi, quando il sole inevitabilmente
tramonta, le stelle tornano visibili, pronte a rischiarare al loro
meglio il buio della notte. Sono lì, sono sempre state lì; nascoste,
offuscate, forse addirittura dimenticate, ma presenti.

Fabio cadde dalla lapide sulla quale si trovava. Non sapeva come ci era
finito, né in particolare perché ci si era seduto, ma non gli
interessava granché. Aveva capito il suo ennesimo errore, forse il più
grave di tutti --- e realizzando uno squisito contrappasso dantesco, la
sua eccessiva devozione verso la più intensa delle luci gli aveva
inflitto l'oscurità eterna.

Se stai una vita intera a fissare il sole, appena scende la notte è
ovvio che non ti accorgi delle stelle: sei diventato cieco.

Il dolore della botta iniziò a farsi sentire, e quella preziosa
rivelazione cominciò a dissolversi. Fabio provò un repentino moto
d'urgenza, perché sapeva di dover catturare quel pensiero a tutti i
costi. Si frugò confusamente fra le vesti e lesto estrasse il suo fidato
quadernino, triste souvenir di Barcellona orgogliosamente trafugato a El
Corte Inglès. Alla vista delle molte pagine strappate, il ricordo del
viaggio in nave tornò e la tristezza lo invase prepotente. Aveva buttato
via una buona metà di quei momenti ossessionandosi, imponendosi di
sancire nero su bianco dei paletti entro i quali avrebbe confinato la
sua esistenza, delle ferree regole da seguire alla lettera. Perché aveva
anche solo pensato una cosa del genere? Era già pazzo allora?

C'erano tante cose che Fabio non avrebbe dovuto pensare, nemmeno per
scherzo. Quella stessa notte, uno dopo l'altro, i suoi errori gli si
erano parati davanti in bizzarre visioni oniriche. Richiamare alla mente
quegli episodi gli era costato un profondo disagio, non tanto per il
dolore di rivivere uno strano remix di tutti i suoi sbagli, ma per la
consapevolezza di non aver imparato un bel niente da tutto quello.
Forse, semplicemente Fabio non era proprio in grado di imparare quel
tipo di cose; si era sempre sentito confuso riguardo ai suoi sentimenti,
ed era riuscito di riflesso a confondere anche le persone che gli
stavano intorno.

Quel treno di pensieri portava in una direzione terribile: se lui non
sapeva che cosa stava facendo, ed era riuscito a confondere Denise,
allora\ldots{} chissà chi o che cosa altro aveva confuso.

Non riuscì a fare il collegamento ovvio; si rifiutò.

Riportò il pensiero sulla sua amica Denise. Le cose che lei gli aveva
detto lo avevano scosso, ma in fondo non era niente che non sapesse già.
Fabio sapeva di avere una malsana paura del dolore, lo aveva sempre
saputo. Era sempre stato al corrente che quella paura aveva distorto la
sua ragione e gli aveva fatto prendere delle pessime decisioni --- ma
quella consapevolezza era stata peggio che inutile. L'impulso di fuga
era stato semplicemente irresistibile, un comportamento istintivo agente
ad un livello così profondo da non poter essere combattuto
coscientemente.

Sapere di commettere un errore mortale, ma non potersi impedire di
compierlo: che cos'era quella se non l'ennesima conferma che Fabio, in
fondo, era matto da legare?

Già, ormai era ovvio: Fabio era pazzo; ma non di quelli tranquilli,
sempre persi chissà dove e solo a volte un po' agitati, Fabio era
proprio un pazzo di quelli da legare. Chissà quando era cominciata la
sua discesa verso gli inferi della follia --- e, tanto per cominciare,
chissà se la sua realizzazione di essere pazzo non fosse essa stessa un
pensiero psicotico. Se è solo la follia che ti fa pensare di essere
impazzito, significa che in realtà sei sano?

Fabio fissò per dei lunghi istanti le pagine bianche sul suo quadernino,
urlando più silenziosamente che poteva.

Quando finalmente riuscì a liberare i suoi pensieri dal pericoloso
percorso circolare che essi avevano intrapreso, cercò di richiamare alla
mente la preziosa rivelazione di prima. Era stata qualcosa di profondo,
di incredibilmente intenso. Per quanto si sforzasse, non riusciva a
metterla in parole; eppure era sicuro che se fosse riuscito a scriverla,
quella pagina non la avrebbe strappata.

Dopo qualche altro momento di lisergica catatonia, Fabio realizzò che la
tomba sulla quale si era era casualmente inginocchiato era quella del
suo amico Bagonghi. Una fitta di angoscia si materializzò in lui
nell'osservare l'epitaffio elegantemente inciso nel marmo; sentì come
una vaga sensazione di confusione, come se si stesse perdendo qualcosa.
\emph{Se non fosse morto, sarebbe ancora in vita}, recitava la scritta.
Se uno l'avesse letta attentamente, si sarebbe accorto che
l'implicazione non era doppia: quella frase non diceva niente riguardo
al caso in cui Bruno fosse effettivamente morto.

Fabio sospirò, tracciando mentalmente un gigantesco ``SÌ'' nel documento
immaginario che gli chiedeva se avesse delle malattie mentali.

Meccanicamente, senza pensare, fece per accendersi una sigaretta.
L'accendino fallì; evidentemente, una nottata all'umido non aveva fatto
bene al fuoco tascabile. Sospirando, fissò assente il cadavere del suo
amato marchingegno. Qualcosa di già rotto si ruppe ancora dentro di lui;
non era mai stato affezionato alle cose --- né aveva mai avuto una
qualunque cura dei suoi averi, se è per questo --- ma per uno strano
scherzo del fato che ancora non aveva notato, quell'accendino nero era
sopravvissuto durante tutta la sua lunga avventura, finendo per morire
sulla tomba del suo precedente padrone.

L'amarezza per quella poetica coincidenza durò poco. Fabio notò qualcosa
ai piedi della lapide, un qualcosa che qualche ora prima sicuramente non
c'era; gettò via la sigaretta e afferrò di scatto il misterioso oggetto.

Era una scatola di Toscanelli al caffé. Sigillata e perfettamente
asciutta, nonostante la pioggia della sera e l'umidità della notte.

Le parole sgorgarono naturali dalla bocca di Fabio: «Ma che
cazzo\ldots{} ?»

Delle altre parole, pronunciate diversi mesi prima dal suo compianto
amico, sovvennero intense alla sua attenzione: ``Vienimi a trovare e
fuma un sigaro con me. Occhio, un toscano dei miei, non uno qualsiasi!''

Era ovvio, no?

Fabio doveva mantenere la sua promessa; come avrebbero potuto non
comparire dei sigari sulla tomba del suo amico? La drammaticità del
momento lo richiedeva, e a quanto pare quella era una ragione
sufficiente per innescare una materializzazione spontanea di tabacco
lavorato.

Fabio cominciò a sentirsi un po' a disagio per quanto la situazione
fosse sinistra, ma si tranquillizzò quasi subito, ricordandosi di essere
pazzo.

Con rassegnata lentezza, scartò la scatola e ne estrasse un sigaro. Non
apprezzava particolarmente i Toscanelli, ma una promessa era una
promessa: se lo mise in bocca ed attese fiducioso che il destino lo
accendesse con un fulmine a ciel sereno.

Non successe niente.

« Ho del fuoco, se ti serve », disse improvvisamente una voce profonda,
« ma non ti lascerò fumare da solo. »

Fabio non sobbalzò, né diede alcun segno di sorpresa; si aspettava un
fulmine, ma anche una persona con un accendino funzionante sarebbe
andata bene. Alzò stancamente lo sgaurdo verso l'imponente figura che si
stagliava vicino a lui, e non disse niente.

Quella voce rimbombante lo incalzò: « Allora? Me ne dai uno o no? »

Fabio lanciò pigramente la scatola verso la montagna incappucciata che
lo sovrastava. Un'accenno di confusione si fece largo in lui: essere
pazzo non poteva bastare per creare dal nulla una persona vera, e quella
sembrava fin troppo concreta per essere un'allucinazione.

« Sei reale? », gli cheise.

La figura rise. « Ti direi di sì anche se non lo fossi. Piuttosto, tu lo
sei? »

Fabio sospirò pesantemente. Sarebbe mai finita quella notte?
