\chapter{La Città dello Sconforto}

\begin{chapquote}{Author's name, \textit{Source of this quote}}
``This is a quote and I don't know who said this.''
\end{chapquote}

% pulled

Hai scagliato te stesso così in alto, – ma ogni pietra scagliata deve cadere! Condannato a te stesso, alla lapidazione di te stesso: o Zarathustra, è vero: tu scagliasti la pietra lontano, – ma essa ricadrà su di te!

Friedrich Nietzsche - Così parlò Zarathustra


L'aria era densa, permea d'acqua. Molte persone silenziose erano riunite attorno ad una bara chiusa. L'atmosfera rispecchiava il diffuso sentimento di tristezza: perfino il tempo piangeva il buon Bruno Bagonghi.

Una coppia di merli osservava discreta la cerimonia funebre, forse in attesa che il piccolo cimitero di Chiesanuova si svuotasse. Una ragazza li notò. La irritavano, aspettavano che tutto finisse per poter cacciare i lombrichi sulla terra smossa, non gli importava niente di Bruno. Fece come per scacciarli, ma poi si bloccò: a che sarebbe servito? Erano merli, anche loro avevano bisogno di mangiare. Riprese a fissare la bara, immersa nella tristezza più profonda di tutta la sua vita.

A pochi chilometri di distanza, due loschi figuri incappucciati fissavano un cartello che recitava 'PRATO'. Come la coppia di merli del cimitero, anche loro sembravano attendere qualcosa.

Cominciò finalmente a piovere. Radi, pesanti goccioloni presero a precipitare dal plumbeo cielo.

«Dobbiamo stare a fissare questo cartello ancora per molto?», si lagnò la figura più bassa.

«Eccoci, Vittoria», annunciò l'altro, ignorando totalmente i commenti della sua compagna. «Il cerchio si chiude\ldots»

«Jorge\ldots non mi prendere per matta, forse è la pioggia che mi fa ansia, ma\ldots avverto una sensazione strana», disse l'altra figura. «Sei sempre sicuro di volerci tornare?»

Non si udì risposta. Incurante dell'acqua che cadeva dal cielo, l'uomo che un tempo era Fabio Fontanelli si tolse il cappuccio. Sul suo volto era scolpita una maschera di dura determinazione.

«Bruno\ldots», sussurrò al vento.

Aveva fatto una promessa. Doveva mantenerla, a qualunque costo.

«Prato, città dello sconforto!», gridò al cartello.

Fece un passo e mise finalmente piede sul suolo pratese.

Senza alcun evidente nesso di causa ed effetto, quel passo innescò un'incredibile serie di eventi. L'intensità della pioggia aumentò vigorosamente: una fragorosa bufera prese a infrangersi su tutta la città. Nel piccolo cimitero dove si commemorava il Bagonghi, la folla si disperse. Anche la coppia di merli cercò riparo, rimandando il suo banchetto di lombrichi a dopo la pioggia. 

Solo una ragazza rimase impietrita dove era, come paralizzata dalla pressione dell'aria. Prima ancora che avvenisse, avvertì quello che stava per succedere.

Un enorme lampo squarciò il cielo plumbeo e si infranse sulla campana in vetta al campanile della cappella. Il suono che produsse fu spaventoso: pareva il ruggito della morte stessa. Un vento formidabile sferzò l'aria, mulinando l'acqua che cadeva martellante. La ragazza sorrise; un sorriso folle, disperato. La sensazione che la aveva paralizzata ora la possedeva: aprì le braccia e scoppiò a ridere. Forse fu il vento che fischiava, o forse lo scrosciare dell'acqua per terra, ma nella sua testa la sentì chiaramente, un timbro vocale che mai avrebbe dimenticato ma che non sentiva da tanto tempo: "Prato, città dello sconforto\ldots".

Corroborata da una strana sensazione, convinta che tutto quello che stava succedendo non fosse casuale, completò la formula che si stava realizzando tutta insieme, urlandola al cielo:

«O piove, o tira vento, o sòna a morto!»

E per diverse ore piovve e tirò vento, come per lavar via una folle speranza durata un istante.


Il prossimo capitolo ve lo beccherete\ldots eh, prima o poi. Non temete: non resterete a bocca asciutta per molto. Buone vacanze a chi le fa!

- Simone



