\chapter{Vendetta}

\begin{chapquote}{Author's name, \textit{Source of this quote}}
``This is a quote and I don't know who said this.''
\end{chapquote}

% pulled

«Ma chi ti si incula, Gazzi! Sei come lui, sei una merda!»

Anton Leka era furioso. Come osava il Gazzi presentarsi di fronte a lui, dopo quello che era successo? Credeva forse che gli avrebbe permesso di farsi beffe di lui, della sua condizione di vita?

«Anton, per favore, monta in macchina\ldots»

Giacomo Gazzi non appariva affatto a suo agio nell'affrontare il suo ex amico, e ne aveva ben donde.

«Per favore? PER FAVORE?!», lo aggredì verbalmente Anton. «Leccami le palle, PER FAVORE! Sei una merda, non mi devi parlare! Levati dai coglioni!»

«Calmati, Anton\ldots devi ascoltarmi.»

Leka gli avrebbe volentieri chiuso la testa nello sportello del suo SUV, ma si limitò a proseguire l'invettiva: 

«Ascoltarti? ASCOLTARTI? L'ultima volta che hai aperto bocca mi hai fatto licenziare! "Dai, un po' di coca, portami la coca", porcodd\ldots la coca! Io non spaccio, te l'avevo detto, ma te no, dovevi avere quella cazzo di droga da ricchi! L'unico rischio che mi sono preso in tutta la mia cazzo di vita\ldots Per te! Capito, infame schifoso? Per te, uomo di merda! SONO NELLA MERDA PER COLPA TUA!»

Gazzi esitò un istante, prima di azzardarsi nuovamente a dire qualcosa.

«Anton\ldots per favore\ldots» fece, con un filo di voce.

«Allora parla, forza!» lo aggredì di nuovo il Leka. «Sentiamo cosa ha da dire Giacomino, invece di andare a prendere il posto alla mensa!»

La voce di Giacomo riacquistò improvvisamente vigore.

«Alla\ldots oh, andiamo, ti porto a cena se è questo il problema!»

«Ti\ldots ti porto a - A CENA?»

Dopo tutto quello che era successo, sapendo ciò che stava passando, Giacomo Gazzi osava offrirgli una cena?

«Ti prego\ldots», lo implorò l'impomatato cicisbeo. «Bisogna che tu mi ascolti\ldots voglio aiutarti!»

Aiutarlo? Se Anton fosse stato un vulcano, avrebbe eruttato tutte viscere della Terra.

«TU - TU CHE?! Ma sei rincoglionito? HAI CAPITO CHE COSA MI HAI FATTO?!»

«Davvero, Anton, devi ascoltarmi\ldots» piagnucolò il Gazzi.

Uscì dall'auto e tentò di mettere una mano sulla spalla del Leka, ma questi la scansò brusco, senza dire una parola.

«Per favore!» proseguì deciso. «Ho scoperto delle cose! Cose importanti su quello che è successo!»

Anton gli scoccò il più intenso sguardo intriso di disgusto che riuscì. Si dovette sforzare per tradurre la sua furia in parole acide.

«Oh, il Gazzi ci ha pensato e si è inventato la storiella! E dimmi, è bella? Finisce bene? Perché non mi pare proprio che per me sia finita bene!»

***

Già, era stata proprio una bella storia. Ripensando alla fatica fatta per trattenersi dall'aggredire il suo interlocutore, Anton quasi non credeva che fosse stato possibile farsi convincere ad andare a cena con lui. Era sicuro che la fame avesse giocato un ruolo da protagonista nella faccenda, ma non riusciva comunque a perdonarsi di aver dato retta a quell'imbecille.

Era stato in imbarazzo per tutta la sera. Il Gazzi lo aveva portato in un noto ristorante nel centro storico, uno dei posti più costosi di tutta Prato; un posto che lui stesso conosceva molto bene. Prima di venire licenziato, frequentava spesso quel locale: ci portava le ragazze che sperava di rimorchiare, spendendo montagne di denaro in piatti e vini costosi per impressionarle. Aveva addirittura confidenza col proprietario! Presentarsi in quello stato, conciato come un barbone, senza l'ombra di un quattrino e totalmente dipendente dalla carità del suo detestato amico, era stata una grande umiliazione. Inoltre, la presenza stessa del Gazzi non lo aveva certo messo a suo agio: quel bastardo inamidato se ne stava lì, tronfio, a decantare vini e piatti, avvolto dal suo completo di alta sartoria, con la camicia cifrata e le scarpe di marca tanto lucide da potercisi specchiare. Il contrasto fra loro due non avrebbe potuto essere più netto. // decantare??? uhm occhiom poi ripeti vini e piatti

Ma aveva sopportato. Era da tanto che Anton non mangiava e beveva così bene. Anzi, ad essere onesti, era quasi un giorno che non mangiava affatto. Con il bere si era potuto arrangiare; per quel che ne sapeva lui, gli alcolici scadenti da due soldi erano altrettanto efficaci a stregare i sensi quanto quelli più pregiati. Ma investendo nell'oblio dell'alcool tutto il poco denaro che riusciva a raccattare, non poteva far altro che affidarsi alla mensa dei poveri per ottenere del cibo. Il senso di vergogna che provava era così insopportabile che spesso preferiva saltare un pasto piuttosto che abbassarsi ad elemosinare. Ogni volta che si metteva in fila, che chiedeva una scodella di zuppa e un cantuccio di pane, ogni singola volta che ringraziava per la carità che gli era stata fatta, la verità si ergeva impietosa davanti a lui: Anton Leka non era più nessuno, solo un patetico barbone.

Il Gazzi si era professato disposto ad aiutarlo. Aveva detto che gli avrebbe dato dei soldi, che gli avrebbe trovato un lavoro, che lo avrebbe strappato a tutti i costi dalla spirale di disperazione che lo avviluppava. Non poteva vederlo in quello stato, non avrebbe permesso che un suo amico si riducesse così. Giacomo Gazzi aveva detto tante cose quella sera. Di tutto quel blaterare, niente lo aveva toccato. Solo la pronuncia di nome era riuscita a catturare l'attenzione di Anton: Bruno Bagonghi.

E così erano finiti a parlare di lui, il suo acerrimo nemico, colui che lo aveva condannato all'umiliazione togliendogli il lavoro. Pareva che l'impresa del Bagonghi navigasse in cattive acque: quello non era affatto un segreto, eppure Gazzi si era comportato come se stesse confessando i più terribili misteri mentre raccontava arzigogolate storie di contese con la mafia cinese, attività illegali e sotterfugi degni del peggior telefilm giallo. In un'altra situazione, ad Anton questi pettegolezzi non sarebbero interessati granché. Avrebbe bollato tutto come le ennesime storielle di Giacomo Gazzi, pizzichi di realtà mischiate a fantasia di scarso interesse. Tuttavia, il tremendo risentimento che provava verso Bruno la pensava diversamente: era stato licenziato in tronco, "per giusta causa" diceva la lettera, senza alcuna possibilità di fiatare in sua difesa.

Fu quello l'amo che il Gazzi predispose, ed al quale lui abboccò. Anton Leka si abbandonò all'odio, senza preoccuparsi di filtrare quello che gli veniva detto. Credette tutto quanto: che Bruno avesse premeditato il suo licenziamento; che avesse complottato abilmente contro di lui, ricattando Giacomo affinché chiedesse della cocaina. Credette che avesse chiamato lui stesso i carabinieri per inscenare una perfetta giusta causa di licenziamento in tronco. Tornava tutto, nella mente stritolata dal rancore di Anton Leka. Il Bagonghi voleva licenziarlo, aveva sempre voluto farlo, ma era pieno di debiti e non poteva permettersi la sua liquidazione. 

Era ovvio che fosse andata in quel modo, che fosse o meno Giacomo Gazzi a dirlo.

Forte della sua nuova convinzione e corroborato dal vino, Anton Leka si recò nella notte alle porte del Lanificio Bagonghi e Gori, portando con sé un malcelato intento omicida. Sarebbe stato facile, aveva detto Gazzi, entrare nello stabile passando dal parcheggio, aprire la porta a vetro con la chiave nascosta nel sottovaso ed assassinare Bruno quella sera stessa\ldots ma erano solo parole di sfogo, non diceva sul serio, aveva addirittura supplicato Anton di non farsi venire strane idee.

Fu così che Leka arrivò indisturbato al cospetto della porta dell'ufficio del suo nemico, l'unica fra le tante di quell'orribile e polveroso corridoio ad avere una parvenza di eleganza.

Il cuore gli martellava prepotente contro le costole, segnalando il rischio; ignorarlo era difficile, ma doveva imporsi la calma ad ogni costo. Delicatamente, premette un orecchio contro il legno massiccio che lo separava dal Bagonghi. Sentì una voce: era lui! Almeno a proposito di questo, Gazzi aveva detto il vero. Ma stava davvero lavorando, alle undici di venerdì sera? Anton ascoltò: la voce di Bruno era l'unica a risuonare in quella stanza. Forse parlava al telefono?

«\ldotsavevamo già detto. No, niente neanche lui: solo tu e l'ingegnere. Non è che non mi fido, è che\ldots bravo, hai capito. Certo, infatti\ldots»

Sì, parlava proprio a telefono.

«\ldotsgià firmati. No, post-datati. Domani. Ti prego, segui il piano alla lettera. No, non è per quello. No, no, assolutamente no! Non gli permetterò di mettere le zampe su tutto quello che ho costruito! Sì, scusa. Giusto, ma ho già pensato a tutto. Esatto. Bravo: fatta la magia, il patrimonio della società sarà immune a qualsiasi pretesa. Tu sarai a posto, mia sorella pure e quel cane sarà servito. Che ti importa di me? Ti sei forse affezionato? Dai, stavo scherzando. No, non passerò da Gambino\ldots»

Di cosa diavolo stava blaterando il Bagonghi? Con chi ce l'aveva? Anton era così concentrato sull'ascolto che non si accorse affatto di un ceffo alto e snello che si era materializzato nel corridoio.

Fece un salto di quasi un metro quando sentì una mano gentile posarsi sulla sua spalla.

«Lei è Leka, giusto? Il signor Bagonghi è in ufficio, ma temo proprio non possa riceverla. Non può aspettare fino a lunedì?»

Anton rimase senza fiato dalla paura. Era stato l'ingegner Govidi a spaventarlo: un ragazzo tanto distinto quanto spilungone, tanto pacato quanto fastidioso. L'appellativo ingegnere, a quanto ne sapeva Anton, non aveva niente a che vedere con nessun titolo di studio, ma era solo un nomignolo simpatico per descrivere l'aura da grigio accademico che emanava la sua figura. Si occupava lui delle scartoffie del lanificio. Era stato lui a consegnargli la lettera di licenziamento senza scomporsi di una virgola, la sua maschera di cortese distacco intatta.

Un pugno rabbioso, dritto in faccia: l'ingegnere cadde a terra, e lì rimase.

Anton fiondò nuovamente l'orecchio sulla porta. Bagonghi aveva smesso di parlare; non c'era più alcun rumore in quella stanza, nessuno che Anton riuscisse a captare. Era probabile che il siparietto con Govidi fosse stato notato. Doveva agire, prima che fosse troppo tardi! Si frugò nelle mutande con mani tremanti e ne estrasse un fagotto di stoffa. Il coltello da carne che aveva trafugato dal ristorante stava per squarciare il maiale più grande della sua carriera. Fece un bel respiro, si armò dell'odio più nero che riuscì a trovare nel suo animo e spalancò la porta con un calcio.

Bruno Bagonghi non si scompose nemmeno di un millimetro: lo attendeva appoggiato alla scrivania, sigaro in bocca e fucile in mano.

«Che c'è?» chiese brusco. «Ti aspettavi che sobbalzassi? Che urlassi di paura? Povero te\ldots»

Fece un sospiro e si stampò in faccia un sorriso senza allegria. Il fucile non puntava verso l'intruso ma era rivolto verso l'alto, retto da una mano sola.

Leka rimase senza parole, interdetto dalla reazione della sua nemesi. Il suo sguardo rimbalzava intermittente fra il volto del Bagonghi e la sua arma da fuoco.

«Caro Anton» proseguì quello, la voce ora fastidiosamente affabile, «dovresti essere più silenzioso quando accoppi i miei dipendenti. L'ingegner Govidi, sebbene sia snello, ha fatto un bel tonfo cadendo a terra!»

L'odio eruppe di nuovo in Anton, potente più che mai. Non avrebbe sopportato le beffe del suo nemico, nemmeno se gli avesse puntato contro un pezzo d'artiglieria.

«Come - come osi - NON TI PROVARE A PRENDERMI PER IL CULO! Chiudi subito quella fogna, maiale ebreo!»

«Oh, fammi un favore», sbottò lui, «smettila con queste cazzate da razzista. Intanto, dare dell'ebreo a qualcuno è considerabile un'offesa solo in qualche barbaro gergo, usato da pochi ancor più barbari figuri, non certo da me. Poi, il fatto che la maggioranza degli ebrei sia stata storicamente più ricca del ceto sociale più incline a credere alle fake news del periodo --- e ti prego di notare la lunga perifrasi che mi tocca fare per non darti direttamente dello stupido --- beh, quello non basta certo per rendere vera l'implicazione contraria.»

Ci fu silenzio per qualche istante.

«Non ci hai capito molto, vero? Comprensibile. Lo dirò in parole povere, e se ancora non capirai saranno cazzi tuoi: non sono ebreo, sono solo ricco.»

La lama di Anton sferzò l'aria, carica di rabbia.

«RICCO UN CAZZO!», urlò. «Sei nella merda, pensi che non lo sappia? Non ci hai pensato nemmeno un secondo prima di buttarmelo in culo per salvare te stesso!»

Bagonghi si accigliò.

«Penso che tu sappia solo ciò che ti hanno detto», disse senza emozione. «Il che non è molto di concreto, in realtà. Ma d'altronde, da uno che crede all'autenticità dei Protocolli di Sion non mi aspettavo certo di meglio.»

«Per l'ultima volta - NON OSARE PRENDERMI PER IL CULO! Credi che sia proprio scemo, vero? Un povero imbecille, che se le beve tutte! Lo so benissimo che il Gazzi è un cazzaro patentato, se credi che dia retta a tutte le sue stronzate non mi conosci affatto!»

Bruno sorrise, ma il suo sguardo si indurì.

«Il Gazzi, eh? Ottimo. Potevo anche indovinarlo, ma tanto valeva controllare. E così, tra una cazzata e l'altra, ti sei fatto manipolare da Giacomino per venire fin qui a compiere la sua opera? Lasciatelo dire, sei veramente un coglione: aggraziato come un pachiderma, armato con uno stupido coltellino da tavola, che speravi di fare? Non ti smentisci mai, sei un disastro ambulante, ora come sempre da quando ti conosco. D'altro canto\ldots lui mi spiazza.»

*Clack-clack.* Il fucile a pompa si rivolse verso Anton.

«Perché il Gazzi ti ha mandato al macello in questo modo? Non poteva sperare veramente che bastassi tu per uccidermi. Che cosa intendeva ottenere, mandandoti qui a farmi perdere tempo? Sei un diversivo? Parla!»

Anton si senti rimpicciolire. Non poteva lasciargli dire quello che voleva, avrebbe dovuto infilzarlo, sgozzarlo, fargli più male possibile, anche a costo della vita. Eppure, un po' per la minaccia armata, un po' per uno strana sensazione che lo stava invadendo, rispose:

«Non so un cazzo di quello che voleva o non voleva fare il Gazzi. Sono qui per conto mio, per chiuderti per sempre quella fogna sputasentenze! Credi che quel cretino mi possa intortare come gli pare e piace?»

«Sì», rispose semplicemente il Bagonghi.

«BEH, TI SBAGLI! Te ne stai lì, dietro quel cazzo di cannone, credi di essere al sicuro, di tenermi sotto tiro! Tu non sai niente, NIENTE di me!»

Bagonghi sospirò. Si tolse il fucile di mano, appoggiandolo sulla scrivania.

«Molto bene», fece senza entusiasmo. «Non avevo voglia di interpretare questa parte, ma se le cattive maniere non funzionano\ldots»

Assunse un'espressione di sufficienza e declamò con tono monotono:

«Anton Leka, sei nato a Scutari il 30 gennaio del '91, abbandonato dopo il parto da una sconosciuta montenegrina. Hai vissuto in un orfanotrofio fino ai nove anni, quando sei stato adottato da una vecchia coppia di Durazzo. Sei scappato di casa a diciassette anni ed emigrato in Italia da solo; i tuoi genitori adottivi non si sono mai dati la pena di venirti a cercare. Hai conseguito un attestato di frequenza biennale alle scuole serali, e lavori per me da quando avevi diciannove anni. Non hai mai avuto una relazione stabile, sei un tabagista ed un alcolista. Non credo di sia altro da sapere su di te: non sei niente di più che un mucchietto di dati su una delle mie schede.»

Anton rimase completamente senza parole. Bruno Bagonghi proseguì:
// sta roba la dice anche govidi, renderlo voluto?
«Ascoltami. Sarò meno sottile di quanto non sia di solito: il Gazzi sta tramando contro di me, ed io di riflesso mi sto difendendo. Vedilo pure come un gioco fra noi due: tu, in questo gioco, sei soltanto una pedina. E non sono stato certo io il primo a muoverti.»

La strana sensazione aveva ormai avvolto completamente Anton. Si sentiva gabbato, truffato, ingannato. Ma non riusciva ancora razionalizzare quello sgradevole presagio. Dove stava andando a parare il Bagonghi, con tutti i suoi discorsi? Non era sicuro di volerlo sapere. Scosse forte la testa, raccolse tutto lo sdegno che riuscì a trovare in sé e disse:

«I tuoi giochi a me non interessano. Io voglio vendetta. Tu mi hai licenziato, hai rovinato la mia vita! Osi negarlo?»

Il suo nemico emise un sofferto sospiro.

«Le metafore non sono il tuo forte, eh?», si lamentò. «Va bene, Anton. Proverò a spiegarmi di nuovo, con un discorso meno allusivo. Mi segui? Bene. Giacomo-Gazzi-vuole-la-mia-azienda. Il suo intento è così chiaro che sono sinceramente sorpreso dal fatto che non lo avessi capito persino tu. Potrebbe impadronirsene in vari modi, uno dei quali è provocare la mia morte; non sto a spiegarti come o perché. Ti conosce, sei rancoroso e cedi facilmente alla rabbia: lui avrà sicuramente pensato che se tu ti credi rovinato per causa mia\ldots Andiamo, non guardarmi così! Te la sto mettendo più semplice che posso.»

Anton ringhiò. Sviare la colpa, si trattava solo di quello? Non se la sarebbe cavata così facilmente, neanche se avesse avuto ragione sul serio. Ululò la sua ira:

«Pensi che non lo sappia? Quando avrò finito con te, ammazzerò anche lui! Ammazzerò tutti! Finirò in galera, non mi importa niente, sono finito ormai! Sono un cazzo di barbone, non conto più niente per nessuno e nessuno conta più per me, non importa che cazzo mi possiate raccontare!»

Bagonghi ridacchiò. «Ti sbagli», fece tranquillo.

«Ah si? Mi dirai anche te che non mi lascerai a morire per strada, perché sono tuo amico eccetera? Ho già sentito questa storia, proprio stasera!»

Bagonghi continuò a ridacchiare. «No, no! Ti sbagli se credi di finire in galera. Sei una provvidenza per me, Leka: sto imbastendo delle cose per le quali mi potrebbe servire un uomo, e tu capiti proprio al momento giusto.»

Il mondo di Anton si fermò.

«Mi\ldots mi stai offrendo un lavoro?», chiese guardingo.

Bagonghi rise a lungo.

«No, non direi proprio», bofonchiò non appena riprese abbastanza fiato. «Non ho la più flebile intenzione di pagarti per i tuoi servigi, quindi non oserei mai chiamarlo lavoro\ldots»

Anton capì.

«Vuoi che uccida il Gazzi per te», disse senza emozione.

Sul viso del Bagonghi si dipinse un'espressione indecifrabile.
// pacing cazzo, non posso chiduere così una roba così' carica
Era passata da poco la mezzanotte quando, nella zona industriale di Prato, una vita si spense.


NOTA DELL'AUTORE

Il prossimo capitolo\ldots eh, ragazzi miei, ve lo metto il 23 di aprile.

Mi mancano due esami per laurearmi, in ufficio sono sommerso di lavoro ed ogni tanto vorrei anche vedere la luce del sole. Ho già scritto tutto il resto del libro: la sostanza c'è, ma la forma è impresentabile. Concedetemi un mese in più e vedrete che non resterete delusi.

In cambio della vostra pazienza, vi concedo un piccolo spoiler del prossimo capitolo: si intitolerà IL REMO.

Mentre aspettate fiduciosi\ldots potete sempre rileggere tutto e tentare di indovinare cosa succederà, no?

A presto,

Simone (o Gesucca, o Gegiùcca, chi lo sa\ldots)



