\chapter{Rendez-Vous}

\begin{chapquote}{Author's name, \textit{Source of this quote}}
``This is a quote and I don't know who said this.''
\end{chapquote}

% pulled

Plaça de Catalunya era piuttosto affollata. Vittoria fissava una vetrina di El Corte Inglès, osservando il riflesso del suo volto. La sua mente era altrove.

"Sei tua", le aveva detto. Vittoria sapeva badare a sé stessa, ma quella frase la aveva angosciata molto, ancor più di tutto il resto della situazione. Essere propri... voleva forse dire essere gli unici responsabili della propria sorte? "Se non vengo, sei tua". Quindi, fino ad allora era stata di Jorge? Lui si sentiva veramente responsabile per lei? Teneva a lei fino a quel punto?

Non lo sapeva. Jorge era un mistero per Vittoria; nei brevi mesi in cui erano stati insieme, non era riuscita a farsi un'idea della reale persona che aveva di fronte. C'era poco da fare: non lo capiva, non importava quanto si sforzasse. Le parole del ragazzo spesso contraddicevano le sue azioni, ed altrettanto spesso succedeva viceversa. Non contava quale stimolo potesse ricevere, niente di vero trapelava da lui.

Vittoria aveva meditato a lungo sul quel misterioso uomo; lo aveva messo al centro di molte sue estasi mistiche, usandolo come perno sul quale far ruotare tutta la sua mente durante quelle ore di intossicazione. Non era servito proprio a niente; a dire la verità, non si era nemmeno aspettata che funzionasse. Nessuna illuminazione le era giunta. Jorge le era rimasto impenetrabile, il suo misterioso fascino intatto. Era una maschera troppo efficace per essere frutto del caso: quel bel ragazzo si ostinava di proposito a nascondere il vero sé.

Che fosse segretamente innamorato di lei? Vittoria lo era di lui. Nemmeno tanto segretamente, a dire il vero.

Voltò le spalle alla vetrina e scrutò la piazza con malcelata disperazione. Non poteva finire tutto così, non era giusto. Lui doveva venire. Lei voleva che venisse. Quella mattina era stato tutto così intenso: non poteva non esserci un lieto fine. Costrinse la sua attenzione ad occuparsi dell'ambiente circostante, ma nella piazza non c'era niente di interessante, ed il centro commerciale non sembrava esercitare più alcuna attrattiva su di lei. Anche se, a ripensarci...

Un accenno di sorriso comparve sul suo volto. El Corte Inglés era stato uno dei principali motivi per cui aveva voluto farsi portare in vacanza a Barcellona. Beh, qualunque posto sarebbe stato migliore dei soliti paesini liguri che i suoi vecchi sembravano adorare. Rapallo, Camogli, le Cinque Terre: non c'era niente, assolutamente niente. Almeno Barcellona era una città, con dei locali da giovani ed una bella collezione di negozi interessanti! Al diavolo calamite, cartoline e varie reliquie di una vacanza all'insegna della noia più totale. Si sarebbe fatta comprare l'intero stand della Mac dai suoi genitori, ed al ritorno avrebbe fatto morire di invidia le sue amiche.

Ma ormai non gli importava più granché di tutto quello. Nella testa aveva soltanto lui, Jorge: la sua espressione tagliente, il suo cinismo, il suo coraggio...

Non poteva continuare a pensare al ragazzo, doveva distrarsi. Nessuno dei passanti sembrava far caso a lei: si frugò nel reggiseno e ne estrasse un foglietto. Lo osservò per un lungo istante, poi lo ripose. Nemmeno tutta la droga del mondo la avrebbe calmata in quella situazione. Anzi, probabilmente avrebbe rotto i freni coscienti che aveva posto alla sua angoscia. Si costrinse a restare presente, sforzandosi per non scoppiare a piangere dalla disperazione.

All'improvviso, un tocco lieve sulla spalla.

« Sono vivo. »

Il suo fragile autocontrollo crollò, ridotto in briciole da un'improvvisa valanga di Oh-Mio-Dio-Stai-Bene. Si gettò fra le braccia di Jorge e frignò tutte le lacrime che aveva.

« Cre-credevo... avevo paura... » balbettò dopo un po', tentando di controllarsi.

« Anche io avevo paura » mormorò lui, « ma ce l'ho fatta. »

Vittoria si asciugò gli occhi e lo guardò.

« S-stai... stai bene, v-vero? » gli chiese, intimorita.

Sul volto di Jorge era scolpita un'espressione così terribile da farla preoccupare. Sembrava che avesse passato cento anni di dolore.

« Sì » rispose semplicemente lui.

La ragazza lo osservò per un istante. Aveva vestiti diversi rispetto a quella stessa mattina, sembrava essersi appena fatto una doccia. Non aveva più con sé il fagotto con i suoi averi.

« Che... che è successo? S-sei ferito? », tentò.

Jorge non rispose. Si voltò e fece cenno a Vitoria di seguirla; i due cominciarono a camminare verso il centro della piazza. Il silenzio durò solo qualche passo.

« Oh, tutto bene? » insisté lei con vigore.

« Sì, va tutto bene. Perché sei nuda? Fa freddo. »

« E' una canottiera, non sono nuda! Ero tutta sporca di sangue, che dovevo fare, andarmene in giro in quel modo? »

Jorge si limitò a ridacchiare sottovoce. In qualche modo, quello fu troppo: la sospensione di incredulità creata dalla forte emozione andò in frantumi. Vittoria si fermò di colpo. Quello che stava succedendo non era genuino, doveva fare qualcosa.

« Ascolta, non puoi cavartela così! » disse bruscamente.

Prese Jorge per mano e lo strattonò verso la panchina più vicina, facendocelo sedere a forza. Lui non oppose resistenza.

« Lo vedo che sei sconvolto, sai? » proseguì. « Riesco a capire la situazione, sono una persona, mica una bestia! Ho quella cosa che si chiama empatia, posso sentire quando non stai bene! E se stai male tu, sto male anche io. Quindi ora mi parli dei tuoi problemi, io ti ascolto, tu stai meglio e io ti vedo felice. Te capì? »

Jorge sorrise un sorriso amaro, ma non aprì ancora bocca.

« Su! » esortò lei. « E' per quel ragazzo che è morto? E' perché hai visto morire tutta quella gente? E' perché il viaggio è saltato? Non fare il sostenuto, hai una faccia che è un pomodoro! Non devi essere di ghiaccio con me, non è che conti meno se mi fai vedere un po' di emotività, eh? »

La risposta fu glaciale:

« Sì, sì e no. Impara a non fare troppe domande in una frase sola. »

« Eh no figa, eh! Enne ci esse, proprio! Non ci siamo. E-mo-ti-vi-tà, ce la fai? Riproviamo. »

« Emotività » ripeté cupo Jorge, la voce quasi un sussurro. « Fa rima con stupidità. »

« Vedi che va meglio così? Parliamo. Ti sei convinto dei legami fra le parole? Perché l'altro ieri ti ho detto qualcosa del genere, ma non mi ricordo bene come è andata a finire. »

Jorge sospirò, facendo schiudere leggermente l'aria da funerale che lo avvolgeva. Parlò:

« E' andata a finire come tutte le altre sere: te che fissavi il soffitto, strafatta, ed io che mi chiedevo perché certi pensieri erano nella mia testa. »

Vittoria proseguì, sforzandosi per infondere tutto il calore che poteva nella sua voce:

« No perché, ora che non prendo la roba da qualche ora, non è che abbia più di tanto senso eh! Cioè, quasi tutti i nomi di cose astratte finiscono con la a accentata! Emotività, stupidità, bontà... ehm... calamità... »

« Acidità, aridità, virilità, validità, meschinità. Potrei dirtene altre venti senza mai prendere fiato. »

« Ah. Non me ne venivano in mente altri. Come sei intelligente... »

Jorge soffocò una risata beffarda, senza allegria. Si passò una mano sul viso, poi prese a tormentarsi il pizzetto. Parole sofferte gli uscirono di bocca:

« Intelligente? Ti ringrazio per il complimento, ma i fatti mostrano il contrario. Una persona intelligente avrebbe sputtanato un amico alla mafia e fatto scoppiare un conflitto armato? Lo avrebbe fatto per il solo motivo di aver seguito l'impulso della sorpresa senza pensare alle conseguenze? »

Vittoria gli scoccò uno sguardo intenso.

« Non significa che sei stupido » gli disse dolcemente. « Significa che sei umano. Gli umani sbagliano e... stanno male, poi. Va bene così, non siamo macchine. »

« Beh, allora non voglio più essere umano », proseguì amaramente Jorge. Una dolorosa scintilla di risolutezza apparve nei suoi occhi.

« Non fare così... » mormorò lei.

I toni si alzarono.

« Tu non sai... non puoi capire! Ahmed era umano. Mi ha aiutato, e guarda dove è finito! Per terra in una pozza di sangue! »

« Guarda che l'hai pagato, eh! Mica gliel'ha detto il dottore di aiutarti! »

« Daniele era un caro amico. » tirò dritto lui, ignorando l'interruzione. « Un amico vero e sincero. Pieno di difetti, tutto quello che vuoi, ma ci volevamo bene per davvero. E ora è li per terra pure lui! Ammazzato come una bestia, senza esitazione o ritegno. Lui era umano! A chi è importato della sua vita? A i'ccane didd.., ecco a chi! Sono bastate due parole sbagliate al momento sbagliato per distruggerlo! Questa è la verità, questa è la realtà! E allora, sai cosa? Se queste sono le cose che succedono agli umani, io mi chiamo fuori! Se il prezzo per essere umano è far succedere queste cose, io non voglio più esserlo! »

Un pizzico di peli di barba cadde a terra, strappato con violenza sul finire dell'ultima frase. Vittoria rimase senza parole, ma Jorge continuò come un fiume in piena: la sua voce non era più un lamento animoso e tremante, ma chiara e fredda. Una mano decisa ricompose il suo pizzo torturato e prese a lisciarlo delicatamente. Il suo sguardo era fisso su qualcosa che solo lui poteva vedere.

« Ho perso il conto di tutte le volte in cui ho indugiato in questi pensieri. Daniele era come me, in questo senso. Era messo peggio, ma condividevamo bene o male lo stesso abisso. Sai qual era la sua soluzione, quando era giù? Cercava di ammazzarsi, quel cretino. Non lo vedevi più in giro per qualche giorno, poi magicamente tornava, pieno di belle speranze pronte ad infrangersi alla prossima occasione. Capisci quanto era stupido? Non imparava mai.

« Ma io non sono come lui. Non mi è mai passato per la mente di uccidermi. Invece di fuggire dalla vita, sono scappato solo dalle situazioni che mi facevano stare male. Credevo di potercela fare... Confesso che ci speravo davvero. Ho mollato tutto per tentare questa via. Eppure sto sbagliando. Le cose sono due: o non è la strada giusta, o non l'ho percorsa tutta. Ormai è imboccata, non mi resta che stringere i denti ed arrivare in fondo. Se l'unico modo per sopravvivere è diventare un mostro, allora lo diventerò. »

E Vittoria finalmente capì. La luce del timido sole di novembre sembrò illuminare Jorge più di quanto non avrebbe fatto un riflettore. Vide per la prima volta l'uomo che aveva vicino, quell'uomo che non aveva ancora compreso: un'anima sola, fragile, che non accettava la propria debolezza.

« No, no! », negò con forza. « Non è così che funziona! Non puoi... Cioè, non puoi proprio dire... che non sopporti di stare male, allora... diventi cattivo e poi... e poi non provi più nulla! Non riesce... non funziona, non va bene! »

« Sei sicura? » sibilò lui, dopo un'intensa pausa.

Vittoria si stava agitando. Il suo Jorge non avrebbe dovuto fare così.

« Certo! » gli rispose, tentando di tenere la sua crescente disperazione lontana dalla voce. « È come - è una di quelle cose così ovvie che sono difficili da spiegare, ma - ma proprio perché sono ovvie! Non puoi diventare cattivo, o lo sei o non lo sei! E poi, anche se sei cattivo non diventi automaticamente uno psicopatico che non sente niente! »

Jorge mostrò un orribile ghigno senza allegria. Parlò duramente:

« Sai a cosa pensavo stamani, quando ti ho fatta scappare? A un bel niente. Ero pilotato totalmente dall'impulso di proteggerti. Avevo visto un conoscente e un amico morire nel giro di trenta secondi, ero schiavo dell'impeto di protezione. Sopraffatto dall'errore, cercavo disperatamente di limitare i danni. Mi sarei sacrificato pur di non vedere anche te lì per terra, in una pozza di sangue. Ma ad un certo punto... »

Il suo volto si deformò nell'espressione più terribile che Vittoria avesse mai visto sulla faccia di una persona: l'arricciatura degli angoli delle labbra era semplicemente malvagia, e gli occhi sembravano brillare di una luce perversa, immonda.

« ...sì, a un certo punto ho capito. Mentre danzavo da un nascondiglio all'altro come uno scemo, cercando più di non farmi sparare che di combattere - è stato allora che ho compreso che stavo sbagliando tutto. Cosa diavolo mi era saltato in testa? Non dovevo cercare di proteggerti, ma semplicemente smettere di provare quell'impulso alla protezione. Smettere di avere dei cari da proteggere. Rendermi indifferente al resto del mondo. È per quello che sono qui, a Barcellona, invece che a casa mia. Invece eccomi lì, a pregare di non morire per impedire che ti venisse fatto del male. Mi sono sentito così stupido - »

Vittoria non poteva sopportare di sentire quelle cose. Lo interruppe:

« Così sei scappato via... »

Parte del male dipinto su quel volto si dissipò: Jorge rise.

« Scappato via? E dove? La gente mi sparava addosso. No, ho fatto l'unica cosa furba di tutta la giornata: mi sono accasciato e ho fatto il morto. Col senno di poi, era la cosa ovvia da fare fin dall'inizio: ero già sporco di sangue e avevo una bella pozza su cui buttarmi. Sono stato un cadavere perfetto, una magnifica interpretazione. Per fortuna nessuno è venuto a controllare se respiravo, ma avevo pensato a qualche carta da giocare anche in quel caso. Tempo neanche dieci minuti la battaglia si è spostata un po' più in là: ho ripreso i miei soldi dal corpo di Ahmed, ho rubato qualcos'altro e me ne sono andato. Il resto lo puoi anche immaginare. »

« Allora » tentò timidamente la ragazza, « tutto è bene quel che finisce bene, no? »

Lo sguardo della follia tornò, gli occhi dei due si allacciarono in uno sguardo. La temperatura attorno a quella panchina sembrò precipitare.

« No » disse semplicemente Jorge. « Non è finita bene, ma ormai non mi importa più. Non hai capito, Vittoria? Io non baderò più a te, non avrei mai e poi mai dovuto cominciare a farlo. Devi andartene, se ci teni a te stessa: le cose d'ora in avanti saranno sempre più pericolose. »

La lacrime a lungo trattenute sgorgarono dagli occhi della ragazza:

« Io ti amo », singhiozzò lei.

Se quell'ammissione lo colpì, Jorge non lo diede a vedere.

« Questo, cara mia, è affar tuo », le disse deciso. « Io sono un assassino: ho ucciso e ucciderò ancora. Se verrai con me, non è escluso che ad un certo punto tu muoia per mano mia. »

« Non mi importa! Io voglio scappare con te... Rifarmi una vita insieme a te, come avevamo detto... »

« La mia vita temo che sarà piuttosto strana, d'ora in poi. La strada che ho scelto di percorrere non porta in nessun luogo bello. Sono cosciente di quello che faccio, e so bene che andrà a finire terribilmente male. Se resterai con me, o morirai o diventerai qualcosa che non saresti voluta diventare. »

« Non mi importa... » ripeté lei, avvicinandosi.

Lo baciò.

Lui non si scansò.

***

Un volantino, ignorato da tutti, svolazzava ormai da qualche giorno nella ventosa Barcellona di novembre.

"OFERTA ESPECIAL - VIARGE INAUGURAL DE NOU CREUER", diceva da un lato; "SPECIAL OFFER - FIRST JOURNEY OF THE NEW CRUISE SHIP", mostrava l'altro.

"Take a trip on the brand new GESUCCA cruise ship!" , proseguiva il foglio pubblicitario. "Don't let this SPECIAL OFFER slip away! With only € 299.90 you will coast Mediterranean Sea from Barcelona to Leghorn and be right back in just 2 days!"

Quel volantino, al termine del suo vagare, un bel giorno finì proprio in grembo ad un certo Fabio Fontanelli, mentre egli se ne stava seduto su una panchina a farsi i fatti suoi assieme ad una ragazza.

Un sorriso sincero si dipinse sul volto del giovane uomo quando i suoi occhi scorsero le microscopiche parole:

"Reservations open until 22/11, ship sails from Barcelona on 01/12 at 9:00. This is a sea-only cruise, no ID or passport is required to attend. Passengers are not allowed to get off the ship for any reasons during docking."

Forse qualcosa stava cominciando a girare per il verso giusto.


NOTA DELL'AUTORE:

Il prossimo, vendicativo capitolo verrà pubblicato il 2 marzo 2018, e tante grazie a me che ho trovato il tempo di sistemarlo. Avete rischiato grosso, io ve lo dico!

- Simone

EDIT: piccolo cambio di programma: ve lo metto domenica 4 marzo, ad urne chiuse, mentre guardo la maratona di Mentana.



