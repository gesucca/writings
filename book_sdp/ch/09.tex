\chapter{Compagnia}

\begin{chapquote}{Author's name, \textit{Source of this quote}}
``This is a quote and I don't know who said this.''
\end{chapquote}


% pulled

Fabio si precipitò verso la ragazza.

«Ti ha ferita?»

Lei non rispose. Tremava fortissimo ed era scossa da brevi singhiozzi; sanguinava da una gamba. Fabio cercò di rassicurarla come poteva.

«Su, su\ldots non è poi chissà cosa\ldots» le disse con un filo di voce.

Era ancora zuppo d'adrenalina, stava tremando anche lui. Istintivamente, cercò di tamponare il taglio sulla coscia della ragazza come meglio poteva, ma per poco non svenne. Detestava il sangue. Si costrinse a restare presente e lo esaminò: non era profondo, per fortuna, ma era molto esteso e si stava sicuramente infettando.

«Quello\ldots quello\ldots» rantolò la poveretta, indicando a caso nel buio.

«Stai tranquilla» le disse Fabio, cercando di suonare rassicurante. «Ora è tutto finito. La gamba non è grave, è tutto tranquillo.»

In realtà, Fabio non pensava proprio che potesse esistere una situazione meno tranquilla di quella. La ragazza era ferita e sotto shock, avevano accanto uno stupratore con un coltello nella schiena ed il cervello sull'asfalto e, per di più, Fabio stesso era un clandestino armato.

«Quello\ldots è morto\ldots?» mugolò lei, tremando violentemente.

«Ora è tutto finito» ripeté Fabio. Non sapeva proprio che altro dirle, temeva che la verità potesse sconvolgere quella povera creatura. Con tutta probabilità, quella sera aveva provato la paura più intensa di tutta la sua vita.

«Cazzo, quello è morto o no?» chiese lei con inaspettato vigore.

«Stai calma, sei ferita! Sì, quello è morto. Mi ha quasi staccato la testa a cazzotti, ma ora è morto.»

L'energia della ragazza fece riscuotere Fabio. Lo shock passò e si ritrovò di nuovo freddo e calcolatore. Il pericolo immediato era finito, doveva decidere in fretta che cosa fare. La cosa più sensata gli sembrò quella di levarsi di torno il prima possibile, chiamando un'ambulanza per soccorrere la poveretta, poi far sparire finalmente il quel maledetto smartphone gettandolo in mare. No, non poteva farlo: i soccorsi avrebbero ritrovato la ragazza insieme ad un cadavere con un buco in testa, lei avrebbe parlato, lo avrebbe descritto e lui sarebbe stato ricercato. Per un attimo fece l'orribile pensiero di sbarazzarsi di lei, ma ne fu così disgustato da spaventarsi. L'istinto gli aveva detto di salvarla e così aveva fatto, senza pensare alle conseguenze. Ormai quella strada era imboccata.

«Ce la fai a camminare?» le disse dolcemente, passandole un braccio dietro le spalle.

«Certo!» rispose Vittoria infastidita, allontanando il braccio di Fabio. Si rimise in piedi con sorprendente rapidità, ma gemette di dolore non appena tentò di muovere un passo.

«Sono caduta di culo,» piagnucolò con voce tremante, «mi ha dato proprio una bella spinta quella merda!»

«Fosse il male di quello, ti ha aperto una gamba!» sbottò Fabio. «Forza, aggrappati, non fare la stupida. So che è uno dei tuoi migliori talenti, ma per favore smetti di servirtene almeno per un po', ok? »

Con tre gambe, i due si incamminarono fuori dal vicolo.

«Ma dove vado così , sono mezza nuda\ldots» fece Vittoria dopo qualche metro.

«Non ti guardo né ti tocco, giuro. Forza, dobbiamo levarci di torno!»

«Sono mezza nuda!»

«Allora toh, prendi la mia maglia e legatela in vita! Così sono io mezzo nudo\ldots»
// qualche filler in questi dialoghi
«Grazie. Sei stato\ldots un eroe. Io credevo\ldots»

«Tutto quello che vuoi, ma adesso, nel nome del Cristo, ci vogliamo togliere di torno?»

La marcia riprese.

Riuscirono a trascinarsi fino alla fermata della metropolitana, lontano dalla scena del delitto. Ripresero fiato. Fabio si stirò la schiena ed imprecò: i colpi ricevuti lo avevano ammaccato non poco. Per fortuna la ragazza era leggera!

«Dove mi porti? Hai una casa? Non voglio tornare in albergo, sono quasi morta, i miei vecchi mi uccideranno appena vedranno come sono conciata\ldots» si lagnò Vittoria.

«Non ci penso nemmeno a portarti dai tuoi» disse seccamente Fabio. «Ora ti porto da me e ti sistemo la gamba, poi penserò a qualcosa.»

«Sei nei guai?»
// meno puntini
«Fai un po' te\ldots»

«Mi dispiace\ldots»

Le dispiaceva! Dopo quello che le era capitato, lei provava empatia per il responsabile. Fabio ne fu sbalordito. Da quando aveva intrapreso il suo viaggio verso la sua nuova esistenza, già due volte il comportamento delle persone lo aveva stupito. Perché la gente non capiva l'ovvio? Possibile che fosse lui a non vedere le situazioni nella loro interezza? Non poteva permettersi simili errori, ma non poteva nemmeno diventare un esperto di etologia umana da autodidatta in quell'istante. Si promise di tornare a quel pensiero in seguito, semmai gli fosse capitata l'occasione. Dopotutto, due casi isolati potevano tranquillamente essere una coincidenza.

Il silenzio regnò sull'improvvisata coppia, fino a che i due non arrivarono alla dimora di Fabio. Pareva passato un secolo da quando aveva rotto lo specchio a pugni, quella stessa mattina.

Provò un moto di vergogna: ora che erano in due, quel buco sembrava ancora più piccolo. Mai avrebbe pensato di avere ospiti! L'interruttore scattò e, dopo qualche secondo, la luce di una vecchia lampadina ad incandescenza scese tetra sulle orribili suppellettili dell'appartamento, esaltando al massimo il degrado di quel luogo.

Imbarazzato, Fabio fece stendere Vittoria sul letto e prese gli antibiotici che gli erano avanzati.

«Tieni, butta giù» le disse, porgendole una pasticca e dell'acqua. «Non è certo la roba più indicata per infezioni cutanee, ma è l'unica cosa che posso darti adesso. Domani cercherò di procurarmi un po' di acqua ossigenata, o roba così\ldots Scusami, ma non so proprio che altro fare, sono medico tanto quanto il tizio che mi ha fatto gli interventi\ldots»

«Grazie» rispose lei. Ingoiò la medicina e tacque per qualche istante. Fabio aprì la finestra, si accasciò su una sedia e si accese una sigaretta. Era esausto e dolorante. La ragazza invece era ancora vispa: sembrava avere una gran voglia di parlare.

«Per interventi intendi quelle cicatrici sulla faccia?» esordì.

«Accidenti a me! Fai finta che non abbia detto niente.»

«Pensavo avessi tipo la psoriasi o una malattia del genere. Ti sei fatto togliere dei porri, o roba simile? O hai fatto un lifting? No, non credo\ldots A proposito, quanti anni hai?»

// magari cerca di fare i cerchi? o similitudine di come fa il bagonghj?
Fabio si nascose in una nuvola di fumo, evitando di rispondere. Non aveva la forza di pensare a cosa inventarsi.

«Perché sei nei guai?» chiese ancora Vittoria.

«Per un botto di cose. Non riesci ad immaginartelo?» sbottò lui, soffocando uno sbadiglio.

«No. Dai, ti prego, dimmelo. Te l'ho detto che mi dispiace\ldots»

«Tanto per dire la più grave, io in pratica ti ho rapita. Ti staranno cercando per tutta Barcellona, ed era l'ultima cosa che volevo. Ma perché non mi sono fatto i cazzi miei, sul lungomare?»

«Cazzodici [1], è stata la scena più bella che abbia mai visto!» esclamò lei, illuminandosi di gioia. «Prima mia madre in para [2] e mio padre che sgrida il pakistano, poi sei arrivato tu e giuro, non sapevo più come fare, morivo dalle risa!»

Fabio sospirò e scosse la testa.

«Cara la tua risata\ldots Ora sono troppo stanco per preoccuparmene, ma la mia situazione - e, di riflesso, la tua - è gravissima. Se non mi viene in mente un modo per assicurarmi di essere estraneo alla tua scomparsa, credo proprio che non potrò lasciarti andare.»

«Lasciarmi andare? No! Ti prego, ti scongiuro, non voglio tornare dai miei genitori!»

// snellire
Fabio la guardò intensamente. In un certo senso, la capiva: a meno che sua madre non fosse già morta di crepacuore e a suo padre non fosse scoppiata una coronaria a forza di urlare per la disperazione, rientrare nei rigidi ranghi familiari non sarebbe stato indolore. Era comprensibile che la ragazza volesse rimandare il più possibile l'ora della cinghia. Certo, preferire la compagnia di uno sconosciuto armato a quella della propria famiglia la diceva lunga su quanto quella figliola avesse la testa sulle spalle. Non che il Fabio ragazzino fosse stato un perfetto esempio di responsabilità, ma sicuramente non era un viziatello ingenuo e tronfio. Lei invece, altro che! Una sicurezza ostentata con così tanta veemenza poteva voler dire insicurezza, oppure semplicemente un'errata percezione delle proprie abilità; comunque fosse, quella tipa non era assolutamente in grado di badare a sé stessa.

Fabio sorrise debolmente. Una vaga ombra di pensiero si affacciò sua alla mente\ldots Lei era molto bella e tutto sommato simpatica, anche se milanese, mentre lui era tanto solo. L'idea di avere un po' di compagnia nella sua crociata contro il nulla gli appariva irresistibile. Era una follia, certo che lo era! Ma anche intervenire in un litigio fra uno spacciatore e due estranei lo era stata. Anche rubare il telefono ad una giovane ragazza, e poi condurla in un vicolo cieco per terrorizzarla ed abusare di lei. Anche uccidere un aggressore che non stava più mostrando interesse per lui, per salvare quella che fino a pochi istanti prima era stata una sua vittima. // anche partire? riferimento ci starebbe

«Perché mi fissi?» chiese Vittoria.

«Nessun motivo in particolare» rispose Fabio, scuotendo rapidamente la testa, come per rimescolare i pensieri che vorticavano al suo interno.

«Sì, certo\ldots sei forse timido?»

«Ti sono sembrato timido, prima?»

«No, proprio no! Solo che mi fissi e non dici niente\ldots Pensavo che volessi prendere la tua ricompensa.»

Fabio sorrise. Ricompensa? Quella parola, per la prima volta da mesi, lo fece sentire in qualche modo apprezzato.

«Cosa ridi? Non vuoi prendere ciò che ti spetta?»

«Ti ho rapita, Vittoria. Ho una pistola, e anche se quel negro mi ha fatto venire due o tre emorragie interne, ho comunque più forza di te. Potrei avere da te quello che mi pare.»

«Come sei antipatico!» sbottò lei. «Non hai proprio sensibilità. Hai mai letto qualche libro, visto qualche film\ldots no, eh? La storia del guerriero che combatte, che uccide\ldots e poi si prende il suo trofeo\ldots»

Il sorriso di Fabio si allargò. Se la ragazza aveva così tanta voglia di avventura, tenerla con sé sarebbe stato fin troppo facile.

***

Un raggio di sole penetrò molesto nella stanza e Vittoria si svegliò. Che ore erano? Si alzò di scatto, ma la gamba ferita le diede una fitta; si ributtò sul misero letto sul quale aveva dormito.

Cosa diavolo ci faceva in un posto così brutto? Confusa e smarrita, cercò di ricostruire mentalmente la sera precedente. Ricordava tutto: il bel ragazzo divertente, lo smartphone, poi il vicolo ed infine l'aggressione. Istintivamente, sorrise. Era stata proprio una bella avventura, senza ombra di dubbio la più eccitante di tutta la sua vita. Certo, in dei momenti c'era mancato poco che se la facesse addosso\ldots Quando quella schifosa bestia le si era avventata addosso, aveva addirittura accettato il peggio\ldots Ma il ragazzo la aveva salvata! Proprio come in un film, dove lo stronzo di turno alla fine si converte, ed usa il suo potere per salvare la ragazza indifesa dal vero cattivo!

A proposito del ragazzo, Jorge, o qualunque fosse il suo vero nome\ldots dove era andato? La sua compagnia le piaceva molto: era sarcastico, intelligente e a modo suo premuroso. Se non altro, le aveva offerto un posto dove nascondersi dalla furia dei suoi genitori.

Una sgradevole sensazione la attraversò. Non voleva proprio pensare a quello che sarebbe successo quando, prima o poi, la sua fuga dalla normalità fosse giunta al termine. Non avrebbe mai voluto che succedesse: odiava la sua famiglia e l'oppressione bigotta che essa esercitava sulla sua vita. Lei era edonista e libertina, ed adorava il brivido del rischio. Sapeva come divertirsi, a differenza dei suoi vecchi. A che serviva essere responsabile, se ciò portava ad una vita noiosa ed insignificante? Piuttosto, avrebbe voluto vivere ogni giorno esperienze come quella della sera scorsa. Certo, magari stando un po' più attenta, ma non sacrificando tutta l'avventura nel nome della sicurezza! Sbuffò. Doveva proprio tornare alla sua vecchia vita?

Chissà per quanto tempo Jorge le avrebbe concesso di stare da lui\ldots


NOTE DELL'AUTORE:

[1]: (dialetto lombardo) Ma che dici!

[2]: (dialetto lombardo) confusa dalla paranoia.


Il prossimo, sorprendente capitolo verrà pubblicato il 30 settembre 2017.

- Simone



