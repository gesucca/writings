\chapter{Trama e Ordito}

% pulled

Forse il destino, forse solo un curioso guizzo del caso - o forse un programma ben preciso di un qualche narratore onniscente? - insomma, fatto sta che la trama di una brutta vicenda si stava finalmente per intrecciare con l'ordito della triste città di Prato.

Il cimitero di Chiesanuova, col favore del buio e dell'umidità, esibiva la sua solita atmosfera spettrale, tanto temuta dalle religiose frequentatrici abituali quanto adorata da un certo tipo di persone. Proprio per questo motivo i cancelli del cimitero, d'inverno, chiudevano alle diciotto in punto; ma i grossi lucchetti con cui venivano serrati servivano più a mantenere l'apparenza che a tenere fuori gli indesiderati visitatori notturni, dato che la porta si scardinava facilmente. 

Quella notte, però, nessun goth si affaccendava in dei tanto terrificanti quanto patetici rituali per evocare quello o quell'altro spirito.

Nascosti in una piccola cappella, coperti alla meno peggio come i barboni, c'erano un ragazzo e due ragazze. Avrebbero potuto sconfiggere il freddo invernale in modi interessanti, ma nessun bagordo orgiastico li animava: erano tutti e tre in balia degli effetti di una massiccia dose di acido lisergico.

A fissare quei tre c'era un'imponente figura, totalmente nascosta da uno scuro impermeabile; solo una folta barba bruna gli spuntava fuori dal cappuccio. Aveva riceuto il segnale che aspettava ed era accorso lì con la massima solerzia. Aveva dei piani da portare avanti, degli obiettivi da perseguire, delle cose da dire ad un suo vecchio amico - ma egli fissava il vuoto, completamente assorbito dagli effetti di una potentissima droga.

Il figuro barbuto soppresse faticosamente uno scortese gesto di stizza. Si professava paziente, e la questione era importante: poteva dedicarle il tempo che serviva.

***
