\chapter{Caronte}

\begin{chapquote}{Author's name, \textit{Source of this quote}}
``This is a quote and I don't know who said this.''
\end{chapquote}

% pulled

Il tempo volava per Fabio. Non sapeva con precisione da quanto si trovasse a Barcellona; aveva gettato il suo microscopico calendario secoli fa, ma poteva essere sicuro di essere lì ormai da qualche mese. Lo capiva dal clima, che non era più estivo: faceva quasi freddo, tanto che la brezza marina era diventata tagliente come un rasoio, e le foglie degli alberi inglobati dalla città erano già quasi tutte sui marciapiedi.

L'ultima volta che aveva messo gli occhi su un giornale, la data segnava il primo di settembre. Lo aveva comprato, quel giornale; era un corriere locale, di una decina di pagine scarse. Un euro e cinquanta per quei fogliaccio, che furto era stato! Ma aveva dovuto prenderlo, perché sulla prima pagina spiccava un titolo irresistibile. Faceva riferimento alla vicenda di una turista italiana scomparsa da qualche settimana, annunciando finalmente il ritrovamento del suo cadavere. Fabio non aveva la minima idea di chi potesse essere quella poveretta trovata morta, ma era certo che il Mossos non fosse stato a farsi troppe domande: una ragazza era scomparsa, un cadavere era comparso. Per quanto quella soluzione semplice fosse invitante, era sicuramente falsa. Un grossolano errore? Un cadavere provvidenziale, per farla finita con un caso scomodo? E chi poteva dirlo... L'unico fatto sicuro dell'intera vicenda era che, dal punto di vista di Fabio, la vera turista in questione era tutt'altro che dispersa.

Comunque, ne era passato di tempo da quella vicenda. Doveva essere circa ottobre, se non addirittura novembre. Non aveva importanza; quale che fosse il mese corrente, Fabio non aveva intenzione di passare l'inverno in Catalogna. Aveva lavorato duramente per preparare la sua migrazione, rastrellando Barcellona in lungo e in largo, raschiando fino all'ultimo centesimo che poteva da ogni attività. Si era dato da fare con spaccio, furti e rapine. Aveva persino fatto da guida turistica ad una coppia di giapponesi per qualche decina di euro. I suoi sforzi erano stati fruttuosi: aveva messo da parte diverse migliaia di sudati quattrini. Gli sarebbero serviti, ovunque fosse riuscito ad andare. Già, appariva buffo perfino a lui: aveva già organizzato quasi tutto, ma non era riuscito a decidere la meta.

Che poi, dicendo la verità, organizzare è un verbo completamente inadeguato a quella che era situazione; la destinazione ignota era solo la ciliegina su una torta fatta di incognite e di speranze. Il piano di Fabio per andarsene dalla Catalogna era fragile, incompleto e totalmente inadeguato per fronteggiare i vari intoppi che si sarebbero potuti presentare. D'altronde, quando la tua agenzia di viaggi in realtà è uno spacciatore libanese che conosce uno scaricatore del porto industriale, non è che puoi avere chissà quali pretese. Ma a Fabio non importava granché: si sarebbe accontentato di sbarcare almeno nel continente giusto, casomai ne avesse scelto uno.

A dire il vero gli importava, eccome! L'ansia di andarsene manovrava Fabio come un burattino, facendolo scalpitare dalla fretta di muoversi; sembrava non potesse restare a Barcellona neanche un'ora in più. Probabilmente, nel profondo della sua mente, giaceva pericolosa un'antica inquietudine: ancora una volta, si trovava a dover fuggire vigliaccamente dal poco che aveva costruito, preso dal bisogno di ricominciare tutto da capo. Stava forse diventando un vizio? In altri tempi si sarebbe psicoanalizzato, cercando di estrarre dalla sua mente il motivo del suo continuo desiderio di fuga, ma ormai non era più il caso. Doveva correre! Veloce a pensare, ad organizzare, ad agire! Una paura alimentava la sua follia: se si fosse fermato a riflettere, avrebbe sicuramente capito qualcosa di terribile.

Ma la frenesia da sola non era in grado di seppellire la verità. Per quanto si sforzasse, Fabio temeva che un giorno avrebbe finalmente preso coscienza di ciò che il suo subconscio aveva già realizzato da tempo. Mai e poi mai avrebbe ammesso apertamente a sé stesso che, in fondo, non sarebbe stato poi così male restarsene a Prato e tentare di rimettere insieme i cocci della sua vita, invece di buttare via tutto. Alla fine qualcuno che gli voleva bene lo aveva, anche se non tutto stava andando come aveva sempre sperato. E cosa aveva trovato invece, nella sua nuova avventura, a parte solitudine e guai? Fabio era terrorizzato da questa verità: aveva sacrificato tutto quanto, fuggendo, per trovare soltanto un'altra situazione da cui fuggire. E allora via, di corsa, a razionalizzare ciò che la sua inquietudine gli metteva in testa: si era già compromesso troppo, aveva fatto cose che non doveva fare ed incontrato persone che mai avrebbe dovuto incontrare. Era chiaro, ovvio che dovesse scappare da Barcellona. Per l'amor del cielo, aveva una ragazza rapita in casa sua! Cambiare emisfero era proprio il minimo che doveva fare prima di mettersi a costruire, finalmente, la sua tanto agognata nuova vita.

Anche Vittoria non vedeva l'ora di andarsene da Barcellona. Beh, in realtà lei non vedeva l'ora di fare qualunque cosa, se aveva con sé la sua boccetta di droga. Era diventata completamente un'altra persona rispetto alla ragazzina viziata e petulante che Fabio aveva derubato, terrorizzato e poi salvato qualche mese prima. Non erano solo i necessari cambiamenti estetici a renderla diversa - che diavolo voleva che fosse tagliarsi i capelli, era proprio il minimo, era ricercata! - la droga le aveva storpiato a fondo il carattere. Sotto l'effetto della sostanza appariva un po' stordita, ma era perfettamente in grado di fare le cose con il solito dinamismo; era sorprendente e Fabio per questo aspetto la invidiava molto, visto che a lui bastava un po' di erba per essere fuori gioco un paio d'ore. Da sobria, invece, si comportava da ragazza matura e faceva discorsi profondi, anche se conservava le insensate fobie tipiche delle donne - per l'amor del cielo, sono solo capelli!

Il tempo trascorreva tiranno, capelli o non capelli. Fu così che, una gelida mattina di quello che pareva essere dicembre, la strana coppia chiuse per l'ultima volta la porta dello squallido appartamento che era stato la loro tana per diversi mesi. Sarà stata l'aria fredda, sarà stato il cielo grigio, ma una strana malinconia avvolse Fabio mentre, imbacuccato, si incamminava verso il porto con la sua compagna al seguito. Non si era mai affiatato con Barcellona, ma la prospettiva di lasciare per sempre il suo continente natio lo intristiva. Alla fine quella città non era poi così' lontana da casa sua, anche se poteva considerarsi a distanza di sicurezza dagli spettri del suo passato. Cambiare continente però, era tutta un'altra storia. Scappare, sempre scappare... La fuga si era confermata una costante della sua vita. Prima dal suo paese natale, uno sperduto villaggio di campagna con sì e no cento abitanti, per raggiungere un paese più grande, dove i suoi genitori avevano il lavoro. Poi la città, Prato, per poter lavorare lui. Infine Barcellona, per fuggire dai suoi demoni. Ora neanche quella andava più bene... E se neanche il Brasile, o l'Argentina, o qualunque altro posto sulla Terra fosse andato bene? Dove si sarebbe rifugiato? Sulla luna?

Si sentì salire un groppo in gola. Cercò di non pensare al fatto che ogni marciapiede, ogni albero, ogni vicolo che attraversava lo avrebbe visto per l'ultima volta.

"Basta!", si impose.

Asciugò la lacrimuccia che gli scivolava sulla guancia e rivolse un sorriso a Vittoria. A quella ragazza non sarebbe importato un bel niente di quello che Fabio avrebbe deciso di fare. Forse, se proprio avesse voluto salvare qualcosa dell'esperienza barcellonese dalla definizione di fiasco totale, sarebbe stato proprio l'incontro con lei. Aveva trovato una persona a cui piaceva stare con lui, tanto gli bastava per darsi la carica. Ovunque fosse dovuto andare, lei lo avrebbe seguito. Per voglia di avventura, forse per affetto, non importava il motivo: non era da solo. Già per questo, si sentiva di dover volerle bene. Non c'erano sentimenti amorosi fra di loro, per lo meno non da parte di Fabio; si sentiva talmente saturo che forse non sarebbe stato mai più in grado di amare. Ma l'amore non gli interessava più, per lo meno consciamente. E neanche a lei sembrava interessare. A lui piaceva giocare con lei, e a lei piaceva stare al gioco. Se lo avessero voluto, sarebbero potuti essere compagni di giochi per il resto della vita. Sarebbe stato bene ad entrambi, Fabio ne era convinto.

Tra un pensiero e l'altro, arrivarono nei pressi del porto in un baleno. La bolla di malinconia che racchiudeva Fabio scoppiò, facendolo tornare nell'altrettanto tremendo mondo reale. Non sarebbe stato facile imbarcarsi senza farsi scoprire. Appoggiò su una panchina il fagotto che conteneva tutti i suoi averi e si fermò un attimo per riflettere.

« Cara mia, ora comincia il bello », disse a Vittoria. « Devo trovare Ahmed e sentire com'è la situazione. Non ti drogare nel frattempo, dobbiamo essere scaltri e svegli! »

« Tranquillo » disse Vittoria con fin troppa condiscendenza.

« Che te lo dico a fare, tanto sei già fatta... » sibilò Fabio, avendo cura di non farsi sentire.

Rivolse la sua attenzione all'ambiente circostante, cercando di capire se si trovava nel posto giusto. Non aveva mai avuto a che fare con un porto industriale prima di allora, non sapeva proprio da che parte rifarsi. Sembrava tutto squallido ed abbandonato; ogni cosa in quel luogo era di colore spento, opaco, sbiadito dalla continua esposizione al salmastro. Una piccola baracca a fianco di una sbarra mobile rappresentava l'ingresso in una bolgia di darsene e container. Fabio si incupì più che mai: la bruttezza di quel luogo lo aveva contagiato.

Stette per qualche istante ad osservare le persone che passavano di lì, poi sentì una mano sulla spalla e si voltò; il suo contatto Ahmed gli sorrideva. Era un giovane marocchino di bell'aspetto, addobbato di tutto punto con una tuta da lavoro.

« Tu ce l'ha, Jorge? » gli chiese.

Fabio fece un cenno con la testa verso Vittoria. Lei si frugò nelle mutande e tirò fuori un rotolo di banconote. Ahmed le prese e le annusò avido.

« Nel nome del Cristo, che schifo! » sbottò Fabio.

« Ma che ne sa... tu culattoni, io già detto! » gli rispose quello. Parlava un italiano passabile, ma aveva un buffo accento nordafricano.

« Tutto come concordato, allora? » domandò Fabio ansioso.

« No, casino. Oggi vero casino... Ma nave per São Paulo salpa oggi, se voi vuole lasciare Europa, non altra scelta. »

« Casino? Che casino? »

« Ieri carico, ma di notte nessuno. Carico è lì, su nave che tu vuoi andare... Qualcuno lo prende oggi, prima che salpa! »

« E a noi che ce ne importa, scusa? »

Il giovane scaricatore sbuffò ed alzò gli occhi al cielo.

« Bela ragazza, di qualcosa a tuo uomo! Lui stupido quanto te sei bella! »

Vittoria sorrise assente e rispose sognante:

« Grazie, sei gentile. Lui va in para, ma non è stupido. Il suo pensiero è... aggressivo. Non riflette mentre pensa, va dritto al punto e a volte... pensa troppo veloce, senza... pensare. »

Ahmed strabuzzò gli occhi. Era sicuramente un ragazzo sveglio, ma dava tutta l'impressione di trovarsi più a suo agio con i pensieri concreti; simili filosofie astratte non facevano certo per lui.

Infatti, chiese titubante:

« Cosa... cosa è quel che tu ha detto? »

Vittoria cercò di spiegare:

« Come quando parli senza pensare...ma un livello... sopra... capisci? »

Il bel marocchino guardò Fabio preoccupato.

« Lascia perdere, è la droga » disse lui, sbrigativo. La situazione lo stava facendo sentire un po' a disagio.

Un silenzio imbarazzante scese sul trio. Vittoria guardò Fabio intensamente per qualche istante, poi disse:

« Jorge ha capito, vero? Lui capisce quello che dico, perché è intelligente. Aggressivo, ma intelligente. »

Fabio sospirò. Non potevano perdere tempo a parlare de massimi sistemi, doveva tagliare corto.

« Sì, ho capito. » ammise. « Purtroppo per me, ho capito benissimo. Per favore, possiamo muoverci? »

Ahmed scosse la testa.

« Tu capisce cazzate di ragazza e non quello che io dico. C'è carico di droga su tua nave, mafia potente, prima che tolgono tu non prova a salire! Tu sei clandestino per guardie, ma se va su poi sei clandestino anche per mafia! »

« Questo si che è buffo» civettò Vittoria. « Somiglia a quella cosa che mi hai fatto leggere l'altro giorno sui computer! »

« Che?! » fecero in coro Fabio ed Ahmed.

« Sì... la clandestinità è... ricorsiva - era così la parola? - figa, non lo vedi? Non ci credo... Come quella sigla... Noi siamo clandestini in un contesto clandestino rispetto ad un altro contesto... »

Fabio era senza parole. Era un collegamento arguto, era vero, i livelli di clandestinità nei quali si stavano cacciando si potevano immaginare in quel modo, ma... cosa ci incastrava con il contesto? Da quando Vittoria era entrata in simbiosi con l'acido simili uscite erano all'ordine del giorno, eppure Fabio non si era ancora abituato alla strana sensazione che provava quando riusciva a cogliere il senso della riflessione senza avere idea del perché essa avesse un significato specifico.

La concretezza dello scaricatore nordafricano ruppe audacemente il silenzio che si era venuto a creare:

« Quando voi è Argentina, o Brasil, o dove va dopo che scende, voi va da psi... da pi... da dottore della testa! Voi pensa cose male! Voi non fa... »

Non seppero mai che cosa non avrebbero fatto. Ahmed si bloccò di colpo, lo sguardo fisso in un punto alle spalle di Fabio. Lui fece per voltarsi a guardare che cosa aveva pietrificato il suo contatto, ma quello gli strinse forte una spalla.

« Guai » disse Ahmed, preoccupato. « Non muove, non parla. Loro rivali di scarico, altra mafia, ma io conosce... »

Fabio annuì, ma non seppe resistere all'impulso di voltarsi; vide una dozzina di figure tutt'altro che amichevoli avvicinarsi verso di loro. Si sentì prendere la mano da Vittoria. Avrebbe preferito restare con le mani libere, ma non la scacciò: anche lui aveva paura.

« Amics? » urlò Ahmed, alzando le mani.

« Això depèn », rispose uno di loro, sguainando una pistola. La stretta di Vittoria sulla mano di Fabio si fece più forte.

« Buona... » le sussurrò lui.

« Avui dia és una porqueria, els altres han d'acabar » proseguì Ahmed.

Fabio non aveva idea di cosa il suo contatto stesse dicendo, ma non poté fare a meno di notare che in qualsiasi lingua parlasse, manteneva il suo buffo accento.

« Seva? » gracchiò uno dei figuri, indicando Fabio e Vittoria.

« Ambdós han d'emprendre. »

« Pagar! »

« No ha problema, que poden pagar. »

Fabio aveva intuito abbastanza del dialogo da capire che doveva metter mano al portafoglio. Sapeva che sarebbe potuto succedere, si era preparato; tuttavia non aveva proprio pensato all'evenienza di dover pagare ben due cricche mafiose per essere lasciato in pace. La sua agitazione galoppò al pensiero di vedersi estorcere tutta la sua piccola fortuna dalla malavita catalana.

Tuttavia, quello che sembrava essere un grosso problema diventò totalmente trascurabile nel giro di pochi istanti. Successero molte cose contemporaneamente: si sentì uno sparo, delle urla e, come evocate dal nulla, apparvero altre figure inquietanti intorno al gruppetto formato dai mafiosi, Ahmed, Fabio e Vittoria. C'erano più pistole in quei dieci metri quadrati che in un poligono di tiro.

Spaventato, Fabio alzò le mani in segno di resa, ma nessuno sembrava avere interesse per lui. Ci furono altre urla, poi un'accesa discussione in catalano scoppiò fra i membri delle diverse cricche mafiose. Urla, sputi e spintoni: se non fosse stato a tiro di arma da fuoco, Fabio avrebbe riso di quell'impietoso sfoggio di maleducazione. Si limitò ad osservare in silenzio quelle stupide scimmie che si esprimevano come potevano, cercando di non far notare il suo divertimento. Rivolse uno sguardo a Vittoria e strizzò un occhio; lei ricambiò, ma la sua espressione era tutt'altro che divertita. Evidentemente, l'enorme pericolo che stavano correndo le impediva di notare quanto quella scena in realtà fosse buffa. Fabio continuò ad osservare il parapiglia, che nel frattempo si era fatto sempre più violento. Le due fazioni mafiose non sembravano proprio trovare pace. Un elemento in particolare era decisamente arrembante: si era aggrappato al collo di un altro e gli urlava nell'orecchio: "Ha d'escoltar! Ha d'escoltar! Dividir els diner!".

Dopo qualche istante che lo osservava divertito, Fabio trasalì. Come aveva fatto a non riconoscerlo subito?! Sciupato, scheletrico e pelato, il partecipante più animoso della rissa era proprio Daniele Brogelli. Le parole di stupore gli uscirono di bocca senza che potesse farci niente:

« Daniele! Cristo santo, ma che ci fai con la mafia! »

Fabio avrebbe decisamente dovuto tenere il becco chiuso. La rissa si fermò di schianto e l'aria si congelò; tutti gli sguardi erano puntati su di lui. Il silenzio assoluto che scese sulla scena non presagiva niente di buono.

Daniele era stupefatto: sembrava avesse visto una strega.

« No... non può essere... Fabio... » ansimò disperato.

Il mondo esplose. Fabio non capì molto di ciò che accadde: era totalmente in balia degli eventi, intorpidito da una trance di incredulità.

« Espia! »

« Espia de merda! »

« Traïdor! »

Alle urla seguirono spinte, sputi e pugni. Qualcuno sparò un colpo. L'escalation della rissa verso la guerriglia fu brutalmente veloce; in pochi istanti il gruppo si disperse ed i vari guerriglieri occuparono ripari strategici, esibendosi in uno scontro a fuoco in piena regola.

Fabio non era certamente in grado di fronteggiare un evento del genere. Lo sbalordimento per aver perso il controllo della situazione lo aveva paralizzato. Per fortuna ci fu Ahmed, che spinse prontamente sia lui che Vittoria dietro un container, momentaneamente fuori dal pericolo.

Lei strillò. Ahmed le si era accasciato addosso e non accennava a togliersi. Vittoria non era abbastanza forte per reggere il peso del marocchino: lo lasciò cadere. Una scia rossa si dipinse sul vestito della ragazza.

« È s - è sangue! » balbettò lei in preda al panico.

Fabio non si stava ancora rendendo contro di cosa stesse succedendo. Tremava forte, gli fischiavano le orecchie ed elaborava a scatti le immagini che vedeva; era sopraffatto da ciò che si stava consumando intorno a lui. Cercò stupidamente di aiutare Ahmed ad alzarsi, ma lui gli sputò del sangue addosso, rantolò qualcosa e poi fece silenzio.

Provò la sensazione peggiore del mondo quando capi che in quel corpo non si celava più alcuna vita. Finalmente si rese conto di quello che stava capitando: delle persone si stavano ammazzando per colpa della sua stupidità. Fu sconvolgente, come svegliarsi da un gradevole torpore a colpi di bombarda. Il suo cervello cambiò marcia, passando dal sonno ad un pericoloso overdrive. Due soli pensieri frullavano nella sua testa: Ahmed e Daniele.

L'urlo di rifiuto sgorgò violento:

« No! NO! »

Non poteva essere, non voleva crederci: strinse forte un braccio a Vittoria, come per aggrapparsi alla vita. Lei lo guardò attonita e fece come per dire qualcosa, ma era senza parole. Fabio doveva sapere: lasciò la presa e si sporse dal suo riparo. La terra sotto i suoi piedi sembrò scomparire quando vide il corpo di Daniele Brogelli disteso a faccia in giù. Neanche si accorse che qualcuno aveva preso a sparare verso di lui.

Si lascio cadere sulle ginocchia, incapace di sostenere la responsabilità di ciò che era appena successo. Si sentì trascinare indietro, al sicuro; Vittoria lo aveva tolto di peso dal pericolo. Fabio la guardò: il suo volto era una maschera di paura. Provo a dirle qualcosa, ma dalla sua bocca non uscì alcun suono.

« Jorge... » tento di dire lei con un filo di voce. Il fragore dei colpi di pistola coprì le sue parole; la battaglia stava ancora infuriando.

Fabio non rispose. Era catatonico, gli occhi sbarrati dall'orrore, incapaci di nasconderlo dalla visione del corpo Daniele, ormai impressa nella sua mente.

« Jorge, che facciamo? Che facciamo? » chiese Vittoria con più vigore.

Fabio continuò a tacere, la bocca sigillata dall'angoscia. In che guaio la aveva cacciata? Era colpa sua, solo e soltanto colpa sua. Qualsiasi cosa sarebbe successo a quella povera figliola, lui ne era responsabile. Cercò di dirle qualcosa per rassicurarla, ma il fiato di Fabio era un grande assente in tutta quella faccenda. La situazione lo stava riempiendo di emozioni troppo intense per lui. Rivolse un'occhiata riluttante al corpo di Ahmed. Non era un suo amico, lo conosceva a malapena, eppure il suo ultimo gesto era stato quello di spingere lui e Vittoria lontano dal pericolo. Anche Ahmed era morto a causa sua.

La misura di Fabio era colma: traboccò. Un folle impeto lo pervase. Non sarebbe rimasto lì, nascosto, in attesa che tutto finisse. Che diavolo, era tutto sbagliato! La sua fuga dall'Europa non doveva andare così, nessuno avrebbe dovuto farsi male, né estranei né amici. Ormai era successo l'irrecuperabile, ma non poteva essere una scusa per evitare di agire. Cacciò un urlo per caricarsi ed estrasse la pistola. Non avrebbe permesso che accadesse qualcosa a Vittoria, la cui unica colpa era stata quella di fidarsi di lui! No, non se ne parlava: se doveva morire, lo avrebbe fatto cercando di difendere almeno lei.

« Jorge... JORGE, FERMO, CHE FAI?! » gridò lei, vedendo il suo compagno animarsi all'improvviso.

« Scappa! » le intimò Fabio. « Io cerco di fare pulito, te corri più lontano che puoi! »

Lei improvvisamente scoppiò a piangere.

« No... no! » singhiozzò.

« Fallo, dammi retta! È tutta colpa mia, te non c'entri niente in questo inferno! Vai, scappa! »

« No, non ti lascio! NON TI LASCIO! »

Fabio le tirò uno schiaffo. Si sentì morire dentro, ma non sapeva come altro farsi ascoltare.

« Devi darmi retta! FAI. COME. TI. DICO. Non voglio che ti succeda niente, hai capito? Scappa, vai verso l'entrata di questo cazzo di porto! Poi vai in un posto affollato, almeno sarai al sicuro... »

Una raffica di colpi si infranse sul container che li celava, facendo un rumore metallico assordante.

« Presto! MUOVITI! » abbaiò Fabio, allarmato.

« Ma muoviti dove?! Dove vado?! La gente spara! »

Altri proiettili colpirono la loro barriera.

« La gente sparerà a me! Vai, cazzo! VAI! »

« E l'America? Il viaggio, la nave! Voglio partire con te! »

« Questo è l'ultimo dei nostri problemi, adesso! Facciamo così: ci vediamo a mezzogiorno davanti al Corte Inglés. D'accordo? »

Lei non sembrava affatto rassicurata. Aveva capito quello che voleva fare Fabio: ingaggiare quei tizi per darle il tempo di fuggire, a costo di rimetterci la vita. Lo abbracciò fulmineamente, si alzò in punta dei piedi e lo baciò.

« Ti prego, non rendere tutto più difficile! » disse Fabio, svincolandosi dalla presa che lo avviluppava.

« E se non vieni? E se... »

« Se non vengo, sei tua. Sei forte, ce la farai. Ora vai! VAI! »

Vittoria scoppiò in lacrime e, finalmente, corse via. Fabio la guardò per un ultimo istante, prima di scattare allo scoperto. Sapeva quello che doveva fare: rapido, prima che potesse tradire la paura, cercò dei bersagli con lo sguardo e sparò.

Il suo unico pensiero, mentre affrontava quell'inferno, era Vittoria.

Il prossimo capitolo sarà pubblicato il 31 gennaio 2018. Buon anno, bifolchi in affanno!

-Simone



