\chapter{Sorpresa}

\begin{chapquote}{Author's name, \textit{Source of this quote}}
``This is a quote and I don't know who said this.''
\end{chapquote}


% pulled

 Fabio era quasi senza fiato.

« Hombre... hombre! [1] » chiamava, mentre inseguiva un catalano dall'aria arcigna. 

Da quando l'uomo si era accorto che Fabio lo stava seguendo, i due avevano camminato a passo svelto per almeno un chilometro, mantenendo più o meno la stessa distanza.

« Hombre, andiamo... Una pregunta [2], una soltanto! » ansimò. 

Ricevette solo fugaci occhiate spaventate, senza essere degnato di risposta. Erano così ovvie le sue reali intenzioni?

Lo strano inseguimento proseguì fino alla fine del lungomare, poi l'inseguito prese le scale che portavano alla zona del Porto Olimpico. Fabio soffocò una bestemmia: era un luogo troppo affollato per poter tentare una rapina. scocciato e sfinito, rinunciò al suo obiettivo e si abbandonò alla ringhiera della passeggiata. Si accese una sigaretta. Scrutò il mare, cercando nell'orizzonte risposte ai suoi dilemmi.

I quaranta euro che aveva in tasca, rubati ad un turista qualche ora prima, non sarebbero bastati a comprare quello di cui aveva bisogno. Che sudata aveva fatto per farseli consegnare! Non ne poteva più di inseguire la gente, sperando ogni volta di trovare un'occasione. Gli serviva un colpo grosso, da diverse migliaia di euro; così sarebbe stato a posto per un po'. Ma chi poteva mai portare simili somme in tasca? Non aveva certo la tempra per mettersi ad assaltare banche o negozi...

La risposta gli apparì davanti agli occhi, sotto forma di spacciatore.

« Hello man, wanna party tonight? [3] » gli fece quello, con il finto entusiasmo tipico del mestiere.

Fabio prese la palla al balzo:

« No, thanks man. Do you sell any drug? [4] »

Quello si smarrì un attimo; evidentemente non si aspettava una richiesta così diretta. Si guardò intorno, fece dei gesti ad un ceffo poco distante e si rivolse di nuovo a Fabio:

« Deutsche? Française? [5] Italiano? »

« Italiano. »

« Capito, amico. That guy can help you. [6] Prego, aspetta... he'll be on you in a moment [7]. Ciao. »

Gli ingranaggi del cervello di Fabio presero a girare velocemente. Ora che aveva trovato lo spacciatore, doveva trovare il modo di attirarlo in un posto isolato, così da poterlo rapinare senza problemi. Non fece in tempo a pensare niente che l'uomo della droga arrivò.

« Zio, su cosa andiamo? Oggi il bottino è grosso, ho della roba che non si vedeva da... »

La mandibola di Fabio quasi cadde dallo stupore.

« Daniele...!?» esclamò.

« No! Zio, che coincidenza! Ma che ci fai a Barcellona, amico? »

Fabio lo abbracciò con calore, ma il suo sangue era gelato. L'ultima cosa di cui aveva bisogno era imbattersi in qualcuno che conosceva.

''Merda...'', pensò.

« Sono... in vacanza. » rispose, sperando con tutte le sue energie che il Brogelli non volesse indagare oltre.

« Sì... anche io, zio! Cielo, ma che hai fatto al viso? »

Già, che aveva fatto? ''Merda! Merda, merda, merda!'', si disperò mentalmente. ''Che cazzo gli dico? Sù, cervello, inventa... Sono ferito... no, ho delle cicatrici! Sì, cicatrici. Perché... mi sono operato...oh, merda... eh...''

« ...dei porri! Ma me li hanno tolti, non era niente di grave, davvero, non ci pensare! »

''Bravo Fabio!'', si disse. ''Ora contrattacca!''

« Ma dimmi di te, non mi aspettavo di trovarti a Barcellona! Da quanto sei qui? Quando riparti? Eh? »

Daniele esitò; non era mai stato bravo a mentire al volo. Sembrava proprio che Fabio non fosse il solo ad avere qualcosa da nascondere.

« Vai tranquillo » gli disse, prima che l'altro riuscisse ad inventare qualcosa. « Facciamo così: non facciamoci domande e non raccontiamoci cazzate. Ok? »

« Meno male l'hai detto, zio! Stavo per entrare in crisi, paranoia pura, cosa potevo dirti? »

''Avresti potuto dire la verità, per quanto me ne frega'', pensò Fabio con stizza, ma in realtà era curioso. Perché il Brogelli si trovava lì? A meno che non avesse vinto alla lotteria, non poteva permettersi di far vacanze. Forse era scappato dall'Italia per sfuggire a certi amici, quelli ai quali si chiedono i soldi quando si è al verde. Era già successo almeno altre due o tre volte, Fabio ne era sicuro. Chissà da quanto tempo era via da Prato, e quando mai sarebbe tornato.

Fra tutte le persone che conosceva, Daniele era sicuramente il meno pericoloso per la segretezza della sua posizione; Bagonghi a parte, ovviamente. La sua passione per le droghe pesanti era ben nota a chi lo conosceva, pochi avrebbero creduto che avesse trovato Fabio a Barcellona per puro caso. No, averlo incontrato non era motivo di preoccuparsi.

« Dai zio, sono contento di averti rivisto. » disse lui, costringendo il cervello di Fabio a riconnettersi al resto della situazione. « Quanto stai? Uno di questi giorni ci becchiamo per un aperitivo, ti va? »

Ci mancava solo di andare a bere insieme! ''Niente panico, fai il vago'', si disse.

« Dovrei ripartire presto, tipo... fra qualche giorno, direi. Un momento per una bevuta lo troviamo, stai tranquillo. »

« Certo, amico! Però, lo sai, prima il dovere... Fumo, coca, emmedì [8], cosa ti serve? »

« Cosa mi serve? Eh, sapessi... mi serve... »

Non gli serviva proprio niente. Se avesse avuto un po' più soldi in tasca avrebbe preso volentieri dell'erba, anche solo per fargli un piacere e toglierselo di torno il più alla svelta possibile. Ma aveva solo quaranta euro, non poteva proprio farlo; quei soldi gli servivano per cose un pelo più importanti, tipo per comprare del cibo e un disinfettante per la gamba di Vittoria. Niente droga, non se ne parlava.

Certo, la sua idea era quella di rapinare uno spacciatore e, Brogelli o no, ne aveva uno proprio di fronte. Avrebbe potuto proseguire con il suo piano ed attirare Daniele in un luogo isolato, ma con che faccia lo avrebbe avrebbe poi minacciato per prendergli dei soldi? Con la stessa con cui gli aveva chiesto consiglio e consolazione? O con quella con la quale aveva condiviso momenti spensierati, malgrado l'oscurità che avvolgeva entrambi? Niente da fare: il Brogelli era un amico. Non era stato certamente l'amico su cui puoi sempre contare nel momento del bisogno, anzi, era stato più che altro l'amico di cui un po' di vergogni, ma tutto sommato gli voleva bene. Non poteva fargli del male. Doveva levarselo dalle scatole al più presto e trovare un'altra vittima. Come poteva fare?

Improvvisamente, gli venne un'ispirazione.

« Senti... sto cercando roba pesa. Ho un amico che è in una brutta situazione, gli sto facendo un favore. Sì, insomma, lo sto nascondendo perché non può proprio farsi vedere in giro. Non posso dirti di più, cerca di capire... »

« Tranquillissimo, zio. Ti puoi fidare. »

Fabio sospirò e si guardò attorno, simulando disagio. Storse la bocca e, sfoggiando una delle sue migliori performances recitative, fece con voce amara:

« Beh, arrivo al sodo... ha bisogno di eroina. »

« Cielo... »

« È un periodo intenso per lui, si sta nascondendo da certe persone... La dipendenza che ha non è fortissima, ma non riesce proprio a farne a meno. »

« Cielo, cielo... aiutalo Fabio, ti prego. Non lasciare che la roba lo divori, chiunque egli sia. Nessuno merita quel destino. »

Fabio trattenne un sorriso; aveva fatto centro. Rincarò la dose:

« Faccio quello che posso. La dipendenza non è troppo avanzata, ma ora non può proprio cominciare il percorso giusto per smettere. Te l'ho detto, si sta nascondendo. »

« Capisco... Senti, io non ce l'ho. Non posso proprio tenerla, non voglio neanche più vederla. Però so chi la tiene. »

« Puoi mandarmi dall'uomo giusto? »

« Sì, posso farlo. Segnati questo numero... »

Fabio estrasse lo smartphone di Vittoria. Un lampo lo colse: si era completamente dimenticato di disfarsene! Sperando con tutto il cuore di non togliere per errore la modalità offline, lo usò per segnarsi il numero di telefono e tutte le informazioni necessarie per comunicare con questo spacciatore di eroina.

La mente di Fabio lavorava frenetica. Ovviamente, avrebbe dovuto organizzarsi: non poteva certo azzardarsi a scrivere o telefonare con il numero di Vittoria; doveva procurarsi un altro apparecchio, con un altro numero, in modo da poter contattare chi gli pareva. Fatto questo, se fosse riuscito a fissare un appuntamento con lo spacciatore, avrebbe potuto finalmente rapinarlo. O magari conoscerlo, entrare nel giro e spacciare a sua volta. Ma spacciare era un lavoro vero e proprio, con dei capi e degli obiettivi, e Fabio non voleva più niente del genere nella sua vita. No, non c'era proprio questione sul da farsi: avrebbe contattato lo spacciatore, lo avrebbe incontrato e derubato della sua eroina, di tutti i contanti e degli oggetti di valore che portava con sé. Non aveva proprio bisogno di entrare in un business quasi onesto come lo spaccio di droga.

« Zio, io ti ho fatto un favore » disse il Brogelli.

Fabio riemerse dai suoi pensieri e lo guardò interrogativo.

« Avrei bisogno che me ne facessi uno anche te. »

« Va bene, dimmi pure. »

« Comprami qualcosa. Se torno a mani vuote anche oggi il señor [9] mi ammazza. »

"Eccoci, e ora che si fa?", piagnucolò Fabio nei suoi pensieri. "Come faccio a dirgli di no?''

« Guarda, lo sai che ti aiuterei volentieri, ma... ho speso tutto in questa vacanza, mi sono rimasti solo quaranta euro! »

« Dai, perfetto! Ho della roba zio, della roba... »

« Davvero Dani, mi servono quei quaranta euro, non posso spenderli in droga! »

« Zio, puoi permetterti un iPhone, non venirmi a fare storie per due spiccioli! »

« A dire il vero questo telefono non... niente, non importa. Davvero, ho solo quei contanti e niente altro. Non ho carte o altre amenità del genere con me. »

« Zio, ti scongiuro... Puoi sempre rivenderla se proprio ti vuoi fare l'ultimo drink, o scambiarla per altra roba, ma ti assicuro che dell'acido in boccette non l'avevi mai visto prima! »

« Dell'acido... in boccette? Acido lisergico? LSD [10]? » chiese Fabio, sinceramente interessato. "Non comprare niente!" gli intimò il suo cervello, ma la risoluzione di non spendere cominciava già a vacillare.

« Davvero! È da intenditori, non lo trovi in discoteca o nei parcheggi. Sapessi che fatica ho fatto per averne qualche dose... Non è che lo puoi prendere così, occhio! » disse Daniele, ficcandogli in mano un oggetto che assomigliava ad un flacone di smalto.

« C'è il contagocce, vedi... non ora, mettilo in tasca! » proseguì. « Una goccia, o due se proprio te la senti. Lo puoi mettere nell'acqua, oppure alla vecchia maniera sul cartoncino. Sono tanti, tanti trip! »

« Dani... una boccetta così saranno cento dosi su cartone. »

« Stai scherzando? Zio, sono cinquanta millilitri al venti per cento! Volendo puoi farci migliaia di dosi! »

Sul volto di Fabio era dipinto un bizzarro miscuglio di desiderio e tristezza. Non poteva, non doveva, non voleva comprare quella droga; eppure la desiderava.

« Capisco, beh, in effetti è proprio della bella roba », dichiarò incerto. « Dani, è fantastico, davvero, ma... il problema è che non posso permettermi di comprarlo. Ho solo quaranta euro. »

***

Nonostante la fermezza di Fabio nel non voler spendere i suoi soldi, qualche minuto dopo si ritrovò a passeggiare sul lungomare, senza un euro e con un quintale di droga in tasca.

Cosa diavolo ne avrebbe fatto? Di usarla su di sé non se ne parlava; non aveva mai provato allucinogeni così potenti e non poteva permettersi di rischiare. Nella sua mente c'erano troppe cose che stavano molto bene nel profondo degli abissi in cui le aveva sepolte. Certo che rinunciare ad un'occasione così ghiotta... Che aveva da perdere? La paura di un bad trip non era essa stessa un bad trip? Se avesse affrontato l'esperienza col terrore di star male, sarebbe stato male di certo. Ma anche se fosse riuscito ad affrontarla senza paura, cosa gli garantiva che la sua mente non si sarebbe soffocata fra le sue stesse spire? Non poteva controllare l'effetto della sostanza in alcun modo. Nella sua vecchia vita non avrebbe mai osato un tale azzardo, sicuro, ma adesso...

Maledì il cielo. Avrebbe fatto molto meglio a non comprarla, almeno non ci sarebbero stati dubbi sul da farsi. Nondimeno, ormai l'aveva in tasca e non poteva tornare indietro. Sbuffò. Decise che ci avrebbe pensato in seguito. La priorità in quel momento era raccattare qualche spicciolo per fare la spesa... di nuovo. Maledetto il Brogelli e la sua persuasività!


NOTE DELL'AUTORE:

[1] : (spagnolo) uomo.

[2] : (spagnolo) domanda.

[3] : (inglese) Ehi ragazzo, vuoi divertirti stanotte?

[4] : (inglese) no, grazie. Vendi qualche droga?

[5] : Tedesco? Francese?

[6] :(inglese) quel tizio può aiutarti.

[7] :(inglese) sarà da te fra un momento.

[8] : (gergo underground) hashish, cocaina, ecstasy. 

[9] : (spagnolo) signore, inteso come grossista dello spaccio di droga.

[10]: LSD, potente sostanza stupefacente dalle proprietà psichedeliche ed allucinogene, molto in voga negli anni '60.


Il prossimo, nostalgico capitolo verrà pubblicato il 31 ottobre 2017.

-Simone



