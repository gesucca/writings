\chapter{L'estasi}

\begin{chapquote}{Author's name, \textit{Source of this quote}}
``This is a quote and I don't know who said this.''
\end{chapquote}

% pulled

Vittoria si passò una mano sul volto. Sudava dal caldo e si annoiava a morte.

«Facciamo qualcosa?» cantilenò, rivolgendosi al ragazzo che stava trafficando intorno al tavolo.

«Eh\ldots cosa?» rispose distrattamente Fabio.

«Non lo so, qualsiasi cosa!»

«Ah\ldots eh, sì, aspetta che\ldots Oh Cristo di un Dio, stai fermo, fermo! Così\ldots»

«Dai, figa che palle! Cioè, veramente, che palle!»

«Ho quasi finito\ldots Solo un altro\ldots Se 'un volasse via d'ogni bene avre' bell'e finito, Madonna\ldots Chiudi un secondo la finestra, per favore, prima che rompa il tavolo a testate! Che stress\ldots»

«Ma fa caldo!» piagnucolò lei.

Fabio la ignorò: era totalmente assorbito dalla sua occupazione.

«E io mi annoio!» aggiunse lei al piagnisteo. «Voglio andare fuori! A cercare un po' di vita, un po' di avventura! Sto già bene, la gamba non mi fa più male, vedi?»

Scattò in piedi e si picchettò il taglio con la mano. Fu difficile mascherare le sforfie di dolore con un sorriso, ma credette di riuscirci.

«Non puoi andartene» le ricordò Fabio, senza staccare gli occhi dal suo lavoro. «Visto che sei in piedi, mi scorri giù la pagina? Ho le mani impegnate.»

«Ma io voglio uscire! Voglio vedere il mondo, non stare chiusa in gabbia! Voglio fare esperienze, sentirmi viva! Guarda che sennò me ne torno a casa eh! Almeno lì c'è la wi-fi\ldots e il mio iPhone non me lo tocca nessuno!»

«Il tuo iPhone? Tsk! Ormai ce l'ha il tizio del seminterrato in Poblenou.»

Ci volle qualche istante prima che Vittoria registrasse quell'informazione.

«Cosa\ldots che cosa!?» fece incredula.

«Già. In cambio ho avuto cento euro, questa cinesata qui e una scheda sim con cinquanta euro di credito. Mi aiuti, per favore?»

La ragazza non si mosse.

«Ma\ldots le mie robe! Le foto, le chat!»

«Tutto pulito. Un mago questo tipo, davvero. Non ho capito di preciso come ha fatto, mi ha detto che c'era della roba da sbloccare, ma ha resettato tutto. Tolta la sim, formattato e rimesso il sistema operativo. Del tutto indistinguibile da uno nuovo di scatola.»

Un grave silenzio riempì la stanza; più denso del salmastro, più soffocante dell'afa. Era insopportabile. La voce di Vittoria lo ruppe:

«Ah, sì? Va bene. No, ma va bene!» urlò. «Rovina pure tutta la mia vita digitale, cancella pure i miei ricordi! A chi importa!? Mi dici che vuoi la mia compagnia, e poi mi lasci qui, prigioniera! Mi dici che mi farai fare esperienze, che mi farai vedere il mondo, e poi cancelli i miei ricordi! Ma tanto a chi importa? A chi!? Non a te, certo che no! Tu mi hai rapita e con me ci fai quello che ti pare!»

Le si ruppe la voce. Fabio si voltò e la guardò: aveva gli occhi umidi e arrossati, ma il volto era contorto in un'espressione agguerrita.

«Ne avevamo già parlato» le disse freddamente, costringendosi a non perdere la pazienza.

Vittoria abbassò lo sguardo, forse pentita di aver urlato. Tirò su col naso e rispose:

«Sì.»

«Lo sapevi che lo avrei fatto.»

«Sì.»

«Mi dispiace per le foto e per il resto. Non c'era proprio altro modo. Sai la mia situazione, vero?»

«Sì»

«Dammi retta e vedrai più mondo di quello che riuscirai a fotografare.»

«Va bene.»

«Ora, di grazia, mi scorri questa pagina in giù? Non voglio farlo di nuovo col naso.»

Vittoria non rispose, ma obbedì. La guida per confezionare dosi di LSD che Fabio stava seguendo sullo smartphone mostrava ora il suo ultimo passaggio. Lui lo lesse e disse:

«Ah, beh, grazie\ldots»

«Hai finito?» gli chiese Vittoria.

«Sì, in pratica sì» rispose lui. «Metto a posto ogni cosa e poi si può andare a spacciare.»

«Con la droga in lungomare, andiamo a spacciare» canticchiò lei, molto più allegra.

«Pluralis majestatis.» la gelò immediatamente Fabio. «Ho detto si può, ma intendevo posso.»

L'umore della ragazza precipitò nuovamente.

«Dai! Cioè, ma che due coglioni! Davvero! Sto morendo, morendo dalla noia!»

Fabio sospirò. Tacque per un momento, pensieroso. Si annoiava, eh? La avrebbe sistemata lui. // troppi spostamenti di focus, occhio

«Oh!» lo incalzò lei. «Hai perso la lingua?»

Lui ancora non rispose. La guardò intensamente; lei sostenne il suo sguardo e sorrise.

«Vuoi fare\ldots roba?», tentò.

Fabio non rispose: era completamente assorbito dal suo flusso di pensieri. Sembrava che stesse ponderando una decisione difficile. Il silenzio regnò per diversi istanti, poi finalmente Fabio parlò.

«Quanta voglia di avventura hai?», chiese a Vittoria.

«Tutta quella che vuoi! MI. STO. ANNOIANDO. Te capì?»

«Non prenderla alla leggera», la ammonì Fabio. «E' da quando ce l'ho che ci penso. Da solo non mi azzarderei mai. In due, invece\ldots si possono fare\ldots cose.»

«La fai tanto lunga, neanche fosse la prima volta» disse lei, un po' delusa. «Abbiamo già fatto\ldots cose. Ma così, senza atmosfera, non ha senso! Io voglio avventura! Emozioni! Come quando ti ho incontrato! Come quando mi hai salvata!»

Fabio continuò a studiarla per qualche istante, poi sorrise.

«Sia. Meno parlare, più fare!» esclamò eccitato.

Riempì l'unico bicchiere a sua disposizione con una piccola porzione dell'acido diluito che aveva usato per confezionare le dosi da spacciare. Ne bevve metà, poi lo porse a Vittoria esclamando:

«Te la do io l'atmosfera, te la do io l'emozione! Giuro che non sarà noioso. Butta giù. E speriamo bene!»

***
// troppo peso, snellire
Vittoria fissava fuori dalla finestra, lasciando che le prime luci dell'alba le riempissero la vista. Si sentiva come se fosse appena nata: niente parole, niente pensieri, solo primitive emozioni. Il percepire le manifestazioni del mondo era fonte inesauribile di meraviglia. Si nutriva di quella sensazione come un assetato beve dell'acqua. Contemplava la luce, sorgente di bene, di vita; riusciva a sentire il rumore che faceva propagandosi, trasmettendo energia positiva a tutto ciò che toccava; riusciva a sentirne l'odore, una fragranza appagante come quella del pane appena sfornato. Non era mai stata così bene, così connessa con il mondo che la circondava.

Un pensiero la investì, potente, sbaragliante, come un lampo. Quel ragazzo\ldots lui non sentiva quello che sentiva lei. Non godeva degli stimoli che il mondo gentilmente concedeva loro, non percepiva l'immensa grazia che la luce e la brezza rappresentavano. O, almeno, non come riusciva a farlo lei. Durante l'amplesso lo aveva inglobato, lo aveva sommerso con la sua ricettività: la loro estasi era stata unica, condivisa. Ma dopo si era chiuso, intrappolato nei meandri della sua mente. Niente di quello che lei aveva provato a fare era riuscito a liberarlo, solo la fine dell'effetto della sostanza lo aveva fatto.

Cosa significava questo? Perché a lei quella droga spalancava le porte della percezione, mentre a Jorge rinchiudeva l'animo nelle segrete della sua psiche? Forse, lei era\ldots destinata a godere del mondo? Forse il mondo stesso la aveva creata perché lei si beasse a pieno di ogni sua caratteristica. Chiuse gli occhi e fece un respiro profondo. Sì, era senza dubbio così: perfino l'aria nei suoi polmoni glielo diceva. Lei era del mondo, ed il mondo era il suo harem. Di fronte a questa verità, ogni altra questione terrena perdeva senso.

***

Fabio non aveva mai provato niente del genere. Un'abbondante decina d'anni di esperienza su sesso e sostanze stupefacenti era andata in fumo in una giornata scarsa. Lui era rimasto lui, eppure era anche la ragazza. Era dentro di lei, ed era lei. Era tutto intero, tutto insieme, ma era anche ogni parte, ogni nervo, ogni anfratto dei loro corpi. Indescrivibile, incredibile, impossibile. Mai nella vita aveva sentito certe sensazioni, forse lontanamente paragonabili a quelle che si provano a fare sesso dopo aver fumato marijuana, ma dieci, cento, mille volte più potenti, più coinvolgenti.

Ma tanto erano stati lo stupore ed il godere, che tanto era ferma la sua sicurezza di non voler mai più ripetere un'esperienza del genere. Dopo il sesso, dopo quell'epifania estatica, era arrivato impietoso l'abisso. Tutti i suoi errori, tutti gli spettri di un passato che non voleva starsene al suo posto, tutti i ricordi di tutte le brutte sensazioni che aveva mai provato in vita sua, si erano riversati su di lui senza il minimo riguardo per la sua povera testa. Aveva temuto di impazzire, aveva sperato di morire\ldots e poi, quasi all'improvviso, era finito tutto. Nessun sollievo, solo terrore che ricominciasse.

Era stato insostenibile. Non avrebbe più preso quell'acido, non importa quanta goduria sarebbe stato in grado di trarne. Non avrebbe rischiato per nessun motivo al mondo di avere di nuovo un bad trip, c'erano troppe cose nella sua mente che dovevano restare al loro posto. Era stato bello, magnifico, ma anche tremendo. Per lui era abbastanza: aveva chiuso.

La ragazza invece aveva scoperto la manna. Per lei era stata estasi pura, senza nessuna ripercussione negativa. Appena una settimana dopo quella prima esperienza, le dosi confezionate erano ancora quasi tutte da spacciare, ma la bottiglia con la droga diluita era vuota. Vittoria la aveva consumata tutta da sola, assumendo la soluzione ad intervalli sempre più ravvicinati. Si beava di quel liquido come fosse l'unica sua ragione di essere, poi cercava sesso, o musica; a volte sedeva semplicemente vicino all'unica finestra del misero appartamento dove era confinata ed osservava fuori. Qualunque cosa stesse succedendo nella sua testa in quei momenti, sembrava che avesse trovato il suo nirvana.

Fabio l'aveva lasciata fare, se non altro per tenerla occupata durante le sue assenze. Girovagava spesso per Barcellona alla disperata ricerca di fondi per partire. Ormai lo spaccio non era più un tabù, ma la sua principale fonte di reddito; l'eroina che era riuscito a rubare allo spacciatore indicatogli da Daniele gli era fruttata un bel po' di soldi, ed era riuscito persino a vendere qualcuno dei cartoncini bagnati di acido a degli occasionali neo-hippie in Barceloneta, così si era deciso a chiedere a qualche collega dei contatti per avere dell'erba. Ora ne aveva a volontà e nonostante un tempo fosse la sua droga preferita, ne fumava pochissima, la vendeva quasi tutta. Con quel ritmo, entro poche settimane poteva metter su una piccola fortuna.

Ma non anelava al lusso, la vita semplice gli andava più che bene; mica era il Gazzi! Il vero scopo di accumulare soldi era quello di avere un tesoretto per scappare, per cominciare un'altra volta da zero. Si era già compromesso troppo per poter rimanere a Barcellona: aveva spacciato, rapinato, rapito e ucciso; ogni persona che incontrava, ogni parola che diceva, ogni istante che finiva nel campo visivo di un agente delle forze dell'ordine, ognuna di queste cose rischiava di essere quella che lo avrebbe messo in grossi guai. Aveva addirittura incontrato Daniele! No, non c'era proprio verso: ogni giorno che passava, aumentava la probabilità di essere raggiunto dal suo passato o, peggio, di lasciarci la libertà o l'osso del collo.

E poi c'era Vittoria. Lei a Barcellona non poteva proprio restare, né lo voleva. Non c'erano dubbi: sarebbe andata con Fabio; gli avrebbe proprio fatto comodo un po' di compagnia là dove doveva andare.

Il prossimo, fatale capitolo verrà pubblicato il 1 gennaio 2018. Godetevi l'ultimo mese dell'anno, bischeri!

- Simone



