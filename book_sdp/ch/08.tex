\chapter{La Preda}

\begin{chapquote}{Author's name, \textit{Source of this quote}}
``This is a quote and I don't know who said this.''
\end{chapquote}

% pulled

Scusate il ritardo, ma ogni tanto vado in vacanza pure io. Questo é peso, godetevelo.

- Simone

«Dammi lo smartphone!»

«No.»

Mai in vita sua Fabio aveva faticato così tanto per trattenere i suoi istinti violenti. Quella ragazza lo irritava nel profondo.

«Figa, dammi il mio smartphone!» ripeté lei, per la sedicesima volta.

«Senti me» le disse Fabio mentre accelerava il passo, nel disperato tentativo di seminare la scocciatrice, «perché non te ne torni dai tuoi? Non vorrai farli preoccupare, vero?»

«Cazzoméne [1], sono dei vecchi rompiballe! Sai che robe se gli dico che mi hai rubato l'iPhone! Dammelo, dammelo subito!»

Fabio sogghignò e disse, beffardo:

«Sei fin troppo disinibita per essere una bambina, con buona pace di tua madre. Quanti anni hai?»

«Diciannove» rispose lei, senza dar cenno di aver colto il doppio senso.

«Porco didd\ldots! E t'hai vent'anni! Bimba una sega! [2]»

«Eh? Voglio il mio smartphone! Cancellati il tuo video di merda e dammelo!»

«Non ci casco, come minimo hai iCloud [3] automatico. Di dove sei?» le chiese automaticamente, senza un reale interesse per la risposta.

«Milano. Figa, non si sénte?» disse, unendo le mani ed agitandole leggermente.

«Non mi intendo di dialetti sabaudi» ansimò Fabio. A furia di andare così svelto aveva quasi il fiatone. "Maledette sigarette" pensò, mentre si sforzava di tenere il passo spedito che aveva preso.

Per qualche metro regnò il silenzio: la ragazza aveva smesso di parlare. Fabio la osservò con la coda dell'occhio: guardava in giro, ostentando una moderata curiosità. Forse si stava chiedendo se sarebbe riuscita ad orientarsi da sola in Barcellona, una volta recuperata la refurtiva. Pur sprezzante, Fabio non poté fare a meno di essere colpito dall'intraprendenza della sua inseguitrice; era testarda e piena di energia, non aveva nessuna speranza di liberarsene camminando.

Avrebbe potuto renderle lo smartphone in cambio di soldi, che tra l'altro gli avrebbero proprio fatto comodo, ma come fare per avere la certezza di non apparire sul web in un video virale? Con tutti i servizi di archiviazione online automatica che esistevano, cancellare un file da un dispositivo non bastava più. Se solo avesse avuto due minuti di pace, sarebbe sicuramente riuscito a togliere ogni traccia della pericolosa ripresa. Ma aveva bisogno di stare fermo, se non voleva prendere in faccia qualche lampione. Doveva liberarsi della ragazza, o comunque riuscire a farla stare zitta e ferma. Senza neanche formulare il pensiero, capì che cosa doveva essere fatto e cominciò ad agire di conseguenza.

«Sono stato una volta sola a Milano» disse, prima che il disco rotto ripartisse, «all'esposizione internazionale del motociclo. Non mi è piaciuto più di tanto, l'ambiente è troppo\ldots artificiale, non so se mi spiego.»

«Ma quale Milano, le fiere le fanno a Rho!» rispose lei, punta nel vivo. «Io sono di Milanomilano! I miei vecchi hanno il loft in via Montenapoleone, eh! [4] Non ci andiamo quasi mai, è sempre affittato\ldots ma ce l'abbiamo!»

«Figa, una milanese d.o.c.! Allora senti una roba, mollami!» gli fece il verso Fabio, mentre prendeva una decisione cruciale. Risoluto, svoltò a destra, verso un posto che si ricordava di aver visto da qualche parte in Bogatell [5].

Continuò a parlare, sperando che la scocciatrice non facesse caso a dove la stava portando.
// questo dialogo non mi ha mai convinto, fa troppo amatoriale
«Senti, palle, s'ha a fare uno scambio culturale? C'ho una fame che pàian sei, te tu mi fai un quintale di riso a i'salto e io ti ridò i'ttelefono. Mi c'andrebbe un monte d'altra roba, ma dé, tusse' milanese, più che riso e cotoletta un si può pretendere! [6]»

«Il riso al salto? Non se lo incula più nessuno, ormai è roba da Cracco [7]» rispose lei, divertita.

«E allora dillo che un'se'bona a una sega! Ni' culo a' sandali di C\ldots [8]»

«Ascolta, fiorentino! Parla italiano, che non capisco cosa dici.»

«Fiorentino? No davvero. Ci sarò andato tre volte a Firenze.»

«Cazzodici [9], adesso cosa, la spari che sei di Napoli? Il dialetto si sente!»

«Sapete una sega, voi milanesi, del dialetto toscano! Il fiorentino è diverso dal pratese, che è diverso dal pistoiese, che è diverso dal senese e da quello del val d'Arno! Sulla costa poi, lasciamo fare!»

«Si, va beh, ciao. Il tuo che roba è?»

«Mmh, vediamo\ldots Un po' Firenze, sì, un minimo Livorno e parecchio, parecchio Prato.»

«A me sembra comunque f\ldots !!!»

Fabio si fermò improvvisamente, facendo inciampare la ragazza su di lui. Dove diavolo erano finiti? Voleva distrarre la ragazza con le chiacchiere, ma alla fine si era distratto pure lui.

Prese a lisciarsi il pizzo, pensando e guardandosi intorno alla ricerca di un punto di riferimento. Avrebbe dovuto esserci un vicolo cieco alla loro destra, invece c'era un minimarket. Maledisse mentalmente la Catalogna e la stirpe di ogni ingegnere civile che aveva lavorato a quelle strade, ma era colpa sua: aveva smarrito del tutto l'orientamento.

«Ti sei perso?» lo canzonò la ragazza.

«Sì» rispose lui, nervoso ma sincero. «Dove cazzo è la fermata della metro?»

«Laggiù, c'è la emme rossa. Mi restituisci lo smartphone, per favore?»

«No. Quindi se la fermata è lì\ldots e se la guardo le torri mi rimangono alle spalle\ldots io devo andare\ldots di là!»

«Vai dove ti pare,» lo schernì la ragazza, «ma io non ti mollo!»

«Brava, seguimi» rispose lui, ancora intento a pensare alla strada da prendere.

Aveva proprio difficoltà ad orientarsi fra i vari luoghi di Barcellona; qualcosa di quella città gli rimaneva ostile, anche se non sapeva dire esattamente che cosa.

Tremendamente inopportuno, il pensiero corse con nostalgia alla sua Prato. Gli mancava proprio quella vecchia, sudicia città. Ogni aspetto, ogni contraddizione, tutte le cose che aveva detestato adesso gli comparivano in mente come piacevoli ricordi. Quasi si commosse nel ripercorrere col pensiero le stradine del centro storico; lui poteva essere lontano quanto voleva, ma quei luoghi avrebbero continuato a significare casa per lui. E non importa chi tu sia, non importa quanto lungo sia stato il viaggio, prima o poi tutti vogliono tornare a casa. Sì, sarebbe tornato a Prato, chissà quando e chissà come, ma ci sarebbe tornato. Se non altro, per il favore che doveva al Bagonghi.

«A che pensi?», lo interruppe una voce femminile.

Fabio sospirò ed emerse dai suoi pensieri, costringendosi a tornare nel mondo reale. La nostalgia sparì, lasciando spazio ad un fremito di eccitazione. Erano vicini alla meta; lì avrebbe potuto fare alla sua inseguitrice tutto quello che gli pareva. Sperando che continuasse ad andargli dietro senza rendersi conto del pericolo che stava correndo, le rispose:

«Pensavo alla mia città. E a Barcellona. Ti piace Barcellona?»

«Mah\ldots sinceramente, ho visto di meglio» sentenziò la ragazza con fare snob.

Fu solo per via di un colpo di tosse che Fabio evitò di farle il verso.

«Sei in vacanza anche tu?» proseguì lei.

«Diciamo - cough, che cazzo! - diciamo che sono\ldots in trasferta. Questa città sembra un po' Viareggio, solo più moderna e sei volte più grande. Sei d'accordo?»

«Viareggio è quel posto vicino al Forte [10]? Non ci sono mai stata. Come ti chiami?»

"Tsk! Come ti chiami? Sul serio?" pensò Fabio, divertito. Ci mancava solo che le dicesse come si chiamava!

«Sì, è quello. Non ti perdi niente.» le rispose, evasivo.

«Oh! Come ti chiami ho detto!»

A pensarci bene, lui un nome ce lo aveva, per brutto che fosse. Perché non cominciare ad usarlo?

«Eh\ldots Jorge» gracchiò, per nulla convinto di quello che stava dicendo.

«Bum! Dai come ti chiami?»

«Jorge!» rispose Fabio, teatralmente offeso.

«Uno di Firenze non si chiama Jorge!»

«Io invece sì. Ho\ldots mio padre che\ldots è fan di Jorge Lorenzo [11]. Che ti importa? Non puoi prendere per buono che mi chiami Jorge?»

«I nomi sono importanti!» tubò lei con veemenza.

«Non per me» tagliò corto Fabio.

«Nel tuo nome c'è scritto chi sei! Io sono Vittoria, e vincerò tutte le sfide che affronterò nella vita.»

Fabio non riuscì a trattenersi. «Questa cosa è stupida», sillabò velenoso. «Il nome è solo una parola che la gente dice per chiamarti.»

«Assolutamente no! Se non mi chiamassi Vittoria, non sarei la persona che sono!»

«Quante cazzate\ldots Se ti chiamassi Patata, saresti la stessa ragazzina petulante che sei adesso, non un tubero. Per quanto mi riguarda, ci potremmo chiamare tutti con dei numeri, o dei codici, o anche non chiamarci affatto.»

«Che persona triste che sei. Oh, ma guarda dove vai, di qui non si passa!»

Erano arrivati. Il cuore di Fabio prese a battere forte. Lì poteva abusare di lei in qualunque modo gli venisse in mente. Non voleva andare troppo oltre, gli sarebbe bastato terrorizzarla per farla stare zitta e ferma per un po', mentre lui faceva sparire ogni traccia del video, oppure farlo fare a lei sotto minaccia.

Fingendo di essersi perso di nuovo, si aggirò distrattamente per il vicolo, mettendo la sprovveduta fra sé e il muro. Era in trappola: nessuno avrebbe potuto vederli o sentirli, a parte forse i topi nei cassonetti. Si parò davanti a lei e, per la prima volta, la osservò con attenzione. Era veramente una bella ragazza: aveva i lineamenti morbidi, ed era ben proporzionata. Fabio si sentì fremere, scosso da istinti che venivano dal basso. Assaporò il momento, lasciando che l'istinto trasportasse la sua immaginazione.

Fu lei a rompere il silenzio:

«Facciam quelli che vanno [12]? Questi cosi fanno caldo!» si lamentò, indicando le unità esterne dei climatizzatori degli uffici.

Ma era tardi, gli uffici erano chiusi e l'aria condizionata spenta. La ragazza avvertiva il pericolo, voleva andarsene da lì. Il suo volto angelico e disinvolto cominciava a tradire un po' di nervosismo. Forse, finalmente, si sentiva a disagio in quel vicolo buio con uno sconosciuto. Forse aveva addirittura paura.

«Vittoria, hai detto? Sì, era Vittoria\ldots» le disse Fabio a voce bassa.

«Jorge, hai detto? No, ma chi ci crede\ldots» gli fece il verso lei.

Fabio ridacchiò. Si mise le mani in tasca e parlò al suolo:

«Il tuo autocontrollo è incredibile, Vittoria. Ad averlo io, mi farebbe proprio comodo.»

Alzò lo sguardo. La ragazza si era irrigidita: non appariva tranquilla.

«Che vuoi dire? Ce ne andiamo, per favore?» disse a voce alta, ma incerta.

«Ce ne andiamo?» la schernì lui, avvicinandosi. «Sei te che mi hai voluto seguire. Quello che accadrà sarà soltanto colpa tua.»

Finalmente, la ragazza gettò la maschera. Indietreggiò verso la parete del vicolo, e cominciò a parlare molto velocemente:

«Sai cosa? Puoi tenertelo l'iPhone, non lo voglio più! Contento? Ora vado, salutami Firenze, ciao! Fammi passare! Non\ldots»

Fabio le mise una mano sulla spalla e lei si ammutolì di colpo. Sentire quella pelle liscia sotto le dita lo scosse. Assaporò un attimo quella sensazione, poi si costrinse a controllare i suoi istinti; doveva assicurarsi la distruzione del video, il resto non era importante.

«Ascoltami bene» le disse, parlando sinceramente. «Questa situazione è molto pericolosa. Sia per te che per me. Non vorrei proprio ritrovarmi a fare cose che non mi conviene fare. Capisci, vero?»

«Per favore, voglio andarmene\ldots» mugolò lei, ad un passo dal pianto.

Fabio provò un moto di insofferenza verso di lei. Si era cacciata da sola in quel pasticcio, cosa le sarebbe costato - a parte uno smartphone costoso - lasciarlo andare via in santa pace?

«Ubbidiscimi» rispose lui. «e non ti farò niente.»

Lei annuì, tremante.

«Tieni il tuo telefono» le disse, porgendoglielo.

Appena ci mise le mani sopra, si bloccò un attimo, come se avesse avuto l'idea di scappare all'improvviso.

Fabio fece scivolare la mano che aveva sulla sua spalla su fino al suo collo.

«Non ci siamo capiti» le disse amareggiato.

«Non ho fatto niente, sono qui ferma!» squittì lei, riportando con cautela la mano di Fabio sulla spalla.

Il panico della ragazza lo faceva sentire strano. In fondo non voleva farle del male, ma adorava terrorizzarla, soprattutto ora che poteva farlo semplicemente essendo sincero. Inebriato da quelle sensazioni, Fabio non diede importanza ai segnali di movimento che avrebbe potuto percepire.

«Voglio che tu capisca in che mani sei» le sussurrò, la voce intrisa di malvagità.

«Che vuoi dire?»

«Voglio dire che metterò tutte le carte sul tavolo. Non ha senso bluffare, ho io la mano più forte e tu non puoi tirarti indietro. Quando la vedrai, accetterai di aver perso. E sarà tutto più facile!»

«Non capisco\ldots Ti prego, smettila!»

«Le metafore non sono il tuo forte, eh? Va bene, proviamo così: tu, da brava persona scaltra, stai pensando a come uscire da questa situazione con il minimo danno. Ma se non conosci le mie vere intenzioni - o i miei limiti, per fregarmi! - beh, allora rischi di trascurare elementi utili per valutare la situazione e, alla fine, uscirne peggio di come non ne saresti uscita dandomi retta. Hai capito che cosa intendo?»

Lei appariva sconvolta. Una lacrima le scese da un occhio, e le labbra cominciarono a tremarle.

«Credo di sì» disse con voce malferma «ma perché mi dici questo? Tieniti il telefono e lasciami andare, per favore!»

«Non credo che tu sia in una posizione tale da poter avanzare richieste, sai? Ora io ti mostrerò una cosa. Ti renderai finalmente conto che seguirmi è stata una pessima idea. Consideralo\ldots un favore personale che ti faccio.»

Senza scollare lo sguardo dal suo volto, Fabio si sganciò la cintura. Lo sguardo di lei guizzò, intriso di terrore. Appena si accorse della fibbia ciondolante, tirò sorprendentemente un sospiro di sollievo.

Inspirò forte col naso e mugolò, visibilmente sollevata:
// sta roba della pistola va pensata un po' meglio, dove diavolo la tiene?
«Oddio, pensavo\ldots Non lo so, un coltello, una pistola\ldots Bastava dirlo\ldots non è un problema così grande\ldots se poi mi lasci\ldots ti posso fare\ldots»

Fabio sghignazzò.

«In effetti, sto proprio tirando fuori una pistola» le disse, puntandole l'arma addosso. Lei si terrorizzò all'istante e indietreggiò improvvisamente, andando a sbattere la schiena contro il muro del vicolo.

«No» gemette disperata. «Non mi sparare!»

«Vedi, quest'arma è proprio uno dei motivi di cui ti parlavo prima» sibilò Fabio, traboccante di malvagità. «Ce ne sono ovviamente altri, ma non credo siano rilevanti in confronto a questo, no? Ora, da brava, cancella il video.»

Lei, con mani tremanti, armeggiò per qualche istante sul suo cellulare. Fabio assaporò il silenzio, godendo della situazione di potere che aveva su di lei. Una volta finito di giocare con la tipa, avrebbe potuto lasciarla in pace e tornare alla sua vita. Forse avrebbe comprato un po' di erba da qualche spacciatore, giusto per fare qualcosa prima di addormentarsi. Ma l'istinto maschile non lo aveva affatto abbandonato. Doveva evitare in tutti i modi di prendersi altri rischi, e una violenza sessuale si poteva considerare come estremamente rischiosa. Ma come diavolo avrebbe fatto a non stuprarla? Se mi lasci\ldots ti posso fare\ldots Lei si era addirittura offerta di sollazzarlo per aver salva la vita. Davanti a delle pulsioni così profonde, rimanere razionali era molto difficile. I balbettii disperati di Vittoria lo costrinsero a tornare alla realtà.

«Non\ldots n-non si può\ldots è pending\ldots non ho il 3G\ldots s-siamo all'estero\ldots»

// questa cosa non credo funzioni
«Fammi indovinare: il telefono sta cercando di mandare il file ad iCloud nonostante tu non abbia internet, e non ti ci fa fare niente nel mentre?» tentò di tradurre Fabio.

«Sì! Esatto! Proprio quello! Non è colpa mia! Non posso fare niente! Lo.. lo cancello in albergo! Lì ho la connessione wi-fi! Giuro!»

// se poi glielo rendo, c'è qualcosa che non torna qui
Fabio sbuffò. «Lascia stare. Dallo a me, e che si fotta la Apple e le sue diavolerie. Vedrai che dal fondo del mare non caricherà proprio nulla su iCloud!»

Si rimise in tasca lo smartphone della ragazza. Era il momento, se voleva davvero abusare di lei. Il volto della povera vittima era deformato dall'insieme di emozioni che aveva provato. Le fece una carezza al volto, senza pensare. Quel gesto fece accadere qualcosa nella testa di Fabio: con suo immenso sollievo, la pietà prevalse sulla fame.

«Visto? Fatto! Sei libera!» le disse, mettendo via la pistola.

«Dai, non fare quella faccia» continuò, tentando di consolarla. «So che questi cosi costano uno stipendio. Ne avevo uno anche io prima di\ldots beh, prima. Se mi dici dove posso trovarti, nei prossimi giorni ti riporto i soldi del suo valore.»

«No, non è per quello\ldots i miei sono vecchi bacucchi pallosi, ma hanno tantissimi soldi! È solo che\ldots che..»

Scoppiò improvvisamente a piangere.
// troppo brusca sta transazione, rallentare
«Mi hai fatto veramente tanta paura! Credevo di morire!»

«Stupida ragazzina\ldots» mormorò Fabio. «Vedila così: ti servirà da lezione. Cosa cazzo ti salta in mente di seguire un tipo come me, tutta sola? Ti è andata bene, sei viva e non ti ho nemmeno stuprato, ma parliamoci chiaro, non ne trovi a dozzine di malviventi che ti fanno questi garbi. Sei giovane, indifesa, ingenua e terribilmente attraente!»

Lei sorrise debolmente, prendendo evidentemente quello di Fabio come un complimento. Lui non ci fece caso e riprese:

«Ma se fossi stato una carogna che se ne frega, e che ammazza senza pensarci due volte? È stata una mossa stupida, totalmente stupida. Così tanto da farmi quasi cambiare idea. Sono sempre in tempo per abusare di te, sai?»

«Te l'ho già detto,» biascicò Vittoria tra i singhiozzi, «quello è il minore dei problemi! Sono maggiorenne e vaccinata, posso farti quello che vuoi! Voglio dire\ldots tu non sei mica\ldots UN NERO COL COLTELLO!»

«Ah, siamo razzisti, eh?» la canzonò Fabio, ma Vittoria lo ignorò, sopraffatta da una nuova paura.

Stava indicando qualcosa alle spalle di Fabio. Lui si girò e quasi perse i capelli dallo spavento.

Un individuo li stava fissando a pochi metri di distanza, e sembrava davvero armato di coltello. Da dove diavolo era spuntato? Neanche il tempo di rendersi conto della situazione che l'uomo si avvicinò a passo svelto e si scagliò su Fabio. Il suo cuore fece un doppio salto mortale. Totalmente impreparato ad un attacco, non era in grado di reggere il confronto fisico con l'aggressore.

// è il climax della situazione, occhi al pacing!
Schivò a pelo un affondo di lama e mandò a segno un perfetto hiza geri [13] all'addome, ma nel giro di un istante subì una serie di colpi pesanti come cannonate al fianco e alla testa, e si ritrovò a terra rantolante.

Confuso e senza fiato, sentì la ragazza gridare ed il tintinnare della lama che cadeva a terra. Si rialzò in piedi, appoggiandosi al muro, cercando di non vomitare. Quella montagna di violenza era addosso alla ragazza, che si dimenava con tutte le sue forze. Fabio non pensò a nient'altro che a lei. Estrasse in tutta fretta la pistola e gridò:

«Qui! Sono qui, figlio di troia!»

Non ottenne la sua attenzione. Vittoria ormai aveva smesso di gridare, completamente immobilizzata. Non poteva sparare, rischiava di colpire anche lei!

Raccogliendo tutto il fiato ed il coraggio che poteva, in uno slancio prese il coltello dell'uomo da terra e glielo piantò nella schiena.

«Toh! Senti nella groppa, sai! [14]» gridò Fabio, in preda all'adrenalina.

Quello urlò, un ululato simile ad una bestia, un verso animalesco di sofferenza. Si alzò in fretta dalla povera ragazza, ma ricadde subito sulle ginocchia, tentando freneticamente di togliersi la lama dalla schiena. Senza la minima esitazione, Fabio approfittò del momento e prese la mira.

Una scintilla, un rumore sordo e quello cadde, immobile.

NOTE DELL'AUTORE

[1] : (gergo giovanile lombardo) Che mi importa?

[2] : (dialetto toscano) Accidenti! Hai vent'anni! Non sei più una bambina!

[3] : servizio di archiviazione online offerto da Apple.

[4] : importante strada di Milano, piena di negozi, appartamenti dai prezzi assurdi ed attività commerciali di ogni genere.

[5] : zona di Barcellona vicina ad una delle spiagge principali della città.

[6] : (dialetto toscano) Ascolta, facciamo uno scambio culturale? Ho tanta fame, se mi prepari il riso al salto io ti restituisco il telefono. Mi piacerebbe mangiare altre cose, ma sei milanese e non posso pretendere altro che riso e cotoletta!

[7] : Carlo Cracco, famoso chef e personaggio televisivo.

[8] : (dialetto toscano) Allora dillo che non sei buona a nulla! Maledizione!

[9] : (gergo giovanile lombardo) Ma che dici, non ti credo.

[10]: Forte dei Marmi, località turistica dell'alta costa toscana.

[11]: Jorge Lorenzo, famoso pilota di moto originario di Maiorca; ha vinto cinque titoli iridati nel Campionato Mondiale di Velocità.

[12]: (gergo lombardo) Andiamo via?

[13]: (giapponese) colpo inferto con il ginocchio.

[14]: (dialetto toscano: livornese) Senti che male ti faccio alla schiena!

Il prossimo, meno cruento capitolo sarà pubblicato il 31 agosto 2017, sempre che non me ne scordi anche questa volta.

- Simone



