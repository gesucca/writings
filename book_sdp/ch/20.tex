\chapter{Oltre la Maschera}

% pulled

Una ragazza si allontanava dal cimitero a grandi passi, maledicendo gli dei e la pioggia. Non c'erano dubbi, Denise avrebbe proprio preferito starsene a casa. D'altronde, del povero Bagonghi in realtà a lei importava il giusto e quanto a lui, beh, tanto era morto, che lei ci fosse stata o meno non si sarebbe certo lamentato.

Si rimproverò mentalmente. Lei non era una persona religiosa - aveva la certezza che nessuno la avrebbe mai giudicata per i suoi pensieri, fintanto che fossero rimasti nella sua testa - ma fare un discorso del genere nei riguardi di un suo amico la faceva sentire comunque un po' a disagio. Nonostante questo, non poteva farci niente: provava una discreta quantità di ingiustificata insofferenza verso il Bagonghi, anche se quel povero cristo non le aveva fatto niente di male. Era qualche settimana che si sentiva apatica verso il mondo, e non si era stupita quando non aveva provato granché nell'apprendere la notizia della morte di un suo amico.

Aveva sperato che il funerale le avrebbe fatto rendere conto della situazione: disprezzava la religione e si definiva atea, ma capiva e temeva l'aspetto mistico della ritualità, cristiana o occulta che fosse; conosceva la presa che semplici gesti e parole potevano avere sull'anima, sapeva cogliere il valore di quello che agli occhi di un cinico poteva sembrare solo un ridicolo teatrino. Nonostante credesse questo, Denise era rimasta completamente indifferente tutto il rituale, finendo per prendersi una messa per niente: le parole dette nell'omelia erano state solo ciance di un prete, dell'offerta di fiori sul feretro aveva visto soltanto della gente che poggiava cose costose su una scatola di legno, ed alla sepoltura nemmeno si era data pena di assistere. La potenza del rito funebre, semplicemente, non era bastata a perforare la sua ferrea barriera di apprensione.

In realtà, non si aspettava davvero che uno stupido funerale le restituisse la capacità di accorgersi del mondo. L'unico, vero motivo che la aveva spinta a presentarsi lì era stato, ovviamente, il desiderio di ritrovare Daniele. Ormai era passato un bel pezzo dall'ultima volta che lo aveva visto, ma non era preoccupata. Non era raro per lei vederlo assentarsi dalla sua vita addirittura per dei mesi. In fondo, la loro relazione si reggeva sull'assoluta libertà; chi era lei per impedirgli di andare dove più gli aggradava? Tuttavia, ci avrebbe tenuto molto ad incontrarlo perché, semplicemente, le mancava più di quanto non le mancasse chiunque altro. Fabio era scomparso e Lavinia si era isolata, Daniele si era reso irrintracciabile, il Cipher aveva chiuso ed il vecchio gruppo di metallari si era disperso, ora pareva che il Leka avesse addirittura ammazzato il Bagonghi... nel giro di tre mesi, aveva visto dissolversi gran parte di quella che era la sua vita; era rimasto solo il Gazzi, che rompeva ancora le scatole come se nulla fosse successo. Di tutto questo, forse le importava mezzo accidente di Fabio e l'altro mezzo sicuramente di Daniele.

Così era finita lì, a cercare il suo ragazzo a un funerale. Lui aveva un rapporto del tutto personale con il culto dei morti, era probabile che si sarebbe presentato a dare l'ultimo saluto ad un amico. Eppure non c'era.

Immersa in questi pensieri e frastornata dal rumore della pioggia che batteva incessante sul suo ombrello, Denise non si accorse che due individui la stavano seguendo. Improvvisamente, uno di loro le si parò davanti con un balzo. Era alto, snello e totalmente avvolto in un impermeabile scuro. Il suo volto era nascosto dal cappuccio, ma aveva una mano in bella mostra ed esibiva una pistola.

Subito lo spavento la avvolse:

« Non ho niente, ti giuro! Non ho niente! », esclamò agitando il braccio, come per evidenziare la mancanza di una borsa con degli averi.

Quello non si scompose.

« Dei tuoi giuramenti non ho mai saputo che farne. Voglio risposte », disse con una voce maschile piuttosto familiare.

Denise rimase completamente spiazzata.

« Non so nulla! », fece istintivamente.

« Già », concordò la figura alta. « Non hai mai saputo nulla. Eppure ascolti, eppure a volte parli. »

Senza aspettare replica, l'aggressore si girò verso l'altro ceffo incappucciato.

« Credo che Govidi sia quello con l'ombrello argento. Vai », disse seccamente.

Rimasero in due. Denise non ci stava capendo niente. Quella non sembrava proprio una rapina, e poi... aveva detto Govidi?

« Chi sei? », tentò.

Lui rise per un lungo momento.

« Non hai bisogno di chiedermelo », rispose finalmente. « Puoi arrivarci da sola. Non sei stupida, Denise. »

Lei capì subito. Quel ceffo conosceva il Govidi, sapeva il suo nome ed era troppo alto per essere Daniele.

« Fabio... » mormorò lei. Il suono di quel nome si perse nel fracasso della pioggia.

Una viva speranza si accese prepotente nell'animo di Denise.

« Fabio! » esclamò con vigore.

Dal figuro incappucciato non arrivò nessuna risposta. Con deliberata lentezza, quello le si avvicinò e le puntò la pistola dritta in mezzo agli occhi. Lei rimase immobile come paralizzata, ma non indietreggiò di un millimetro. Non si era mai tirata indietro nelle situazioni intense, e se quello era davvero il Fontanelli, qualunque fosse il gioco a cui stava giocando, non le avrebbe fatto niente; ma soprattutto, forse le avrebbe saputo dire che fine aveva fatto il suo Daniele.

« Sei tornato per Bruno », gli disse.

« Preferisco fare io le domande », ribatté lui.

« Comprensibile. Ma la mia non era una domanda. »

Non arrivò replica, ma solo qualche sommessa risata.

« Se vuoi l'ultima parola », lo incalzò lei, « te la servo: dove hai preso questo affare? Me lo levi dal viso, per favore? »

Ci fu un istante di pericoloso silenzio. Fu il misterioso individuo a romperlo:

« Sei sempre stata così, Denise. Sottile, arguta quanto basta, ma soprattutto dura e testarda. Non hai paura che ti uccida? »

Lei lanciò il suo sguardo più intenso dentro a quell'oscuro cappuccio.

« Non mi fai paura », gli disse duramente. « Puoi dire, pensare o fare quello che ti pare - e che mi venga un colpo se so che cazzo stai facendo - ma se sei il Fontanelli che conosco non mi farai niente di male. »

Un rabbioso sibilo filtrò dal cappuccio:

« Fabio Fontanelli è morto! »

La replica di Denise fu tagliente:

« Eppure è proprio qui, davanti a me. »

Con la stessa deliberata lentezza con cui il figuro incappucciato le aveva puntato contro la pistola, Denise abbassò la mano armata del suo aggressore e gli tolse il cappuccio.

Sorrise nel guardare chi aveva di fronte. Certo un po' diverso, forse sciupato, con quelle cicatrici sul naso e sulle orecchie; più oscuro, in un certo senso. In ogni caso, non c'erano dubbi che dietro quello sguardo glaciale ci fosse lo stesso ragazzo che conosceva fin dai tempi della scuola, nella stessa versione angosciata che aveva visto una certa sera di diverso tempo fa.

I due vecchi amici si guardarono per qualche istante in silenzio, forse pensando  a quanto le loro anime fossero simili, ma opposte, speculari.

Per Denise, Fabio era sempre stato una sorta di riflesso che negli anni le aveva mostrato che cosa sarebbe potuta diventare se avesse avuto più talento, ma anche quanto in basso avrebbe potuto scendere se avesse costantemente seguito l'impulso ad essere sé stessa. Lui le piaceva, ovviamente, ma l'idea che loro due potessero formare una coppia romantica non l'aveva mai sfiorata. Denise era una persona sì ingombrante, ma con dei confini; il Fontanelli invece non aveva proprio limiti. Lui stesso, in una qualche nottata di stupidaggini, aveva entusiasmato tutti annunciando di non essere solido, ma gassoso: ed aveva senso, perché Fabio sentiva la pressione dei suoi impulsi e, per abbassarla, si espandeva fin quanto poteva, invadendo ogni centimetro di spazio metaforico che riusciva a raggiungere. Le era chiaro come il sole che loro due avrebbero lottato in eterno per la supremazia, così come le era cristallino che lei avrebbe sempre perso. E siccome non avrebbe mai accettato di fare da comprimaria nelle dinamiche di coppia, era ovvio che la cosa non avrebbe potuto semplicemente funzionare. Comunque, non aveva mai soppresso il desiderio irrazionale di avere quel ragazzo vicino il più possibile, per bearsi dell'aura di puro potere che emanava il suo semplice esistere.

Se c'era una cosa che Denise non aveva mai capito, era chi o cosa fosse riuscito a rinchiudere il suo amico nella gabbia di depressione in cui lo aveva visto prigioniero prima che partisse, e nella quale probabilmente si trovava ancora. Proprio lei, che era stata vicina al suo Daniele nei momenti peggiori e ormai chiamava per nome i vari mostri che potevano piegare l'anima di una personcina semplice come Daniele, non riusciva a pensare a niente che avrebbe potuto anche solo scalfire Fabio Fontanelli. Per tutto il bene che potesse volere a Lavinia - poco, in realtà - non poteva proprio esistere che fosse stata lei la causa del sua profonda tristezza di Fabio. Proprio lei, una ragazzetta fra le più semplici nel mondo, non poteva aver pesato così tanto nel bilancio del'umore di Fabio Fontanelli. Proprio lui, che con un distratto colpo d'arguzia avrebbe potuto spazzar via ogni pensiero negativo, che con un distratto balzo di ironia avrebbe potuto scavalcare tutte le situazioni avverse. Denise ne era sicura: non poteva essere stato solo un amore andato male a sconfiggerlo.

Se quelli erano i pensieri che affollavano la mente di Denise, Fabio invece non pensava proprio niente. Abbandonandosi al suo istinto, una verità gli uscì di bocca con tutta la sua spaventosa gravità:

« Daniele è morto. »

Uno scenografico fulmine squarciò il cielo, come a fare da cornice alla rivelazione. Se non ci fosse stato il fragore del tuono, si sarebbe distintamente udito il rumore di una persona che andava in frantumi.

« È stata colpa mia », confessò Fabio.

Denise non parve reagire. Rimase immobile, il volto contratto in un'espressione indecifrabile. Fabio non parve attendere risposta:

« Mi rendo conto che non è un buon momento, ma devo farti delle domande. E mi dovrai rispondere. »

« Sei qui per Bruno, si capisce », squittì lei con un filo di voce.

Fabio annuì. Senza il minimo pensiero, si avvicinò a Denise e la abbracciò.

« Andiamo da qualche parte, magari all'asciutto », le sussurrò, mentre lei versava un fiume di lacrime sulla sua spalla. « Abbiamo tante cose da dirci. »
