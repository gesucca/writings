\chapter{Occhio Non Vede}

Il sole fece capolino da dietro la coltre di nubi, inondando tutta Prato di un piacevole sole invernale. Le tre ragazze e i due ragazzi, stanchi morti ed emotivamente sconvolti, si squadravano a vicenda, maledicendo il destino che li aveva riuniti tutti quanti in quel cimitero.

<E ora che si fa?> chiese finalmente Vittoria.

<Che si fa...> ripeté Denise, meditabonda e imbarazzata. <Se Fabio... no, senza se: Fabio ha ragione. Non ci si può mettere a giustiziare un figlio di troia in un cimitero.>

<Anche se sono pazzo, comincio ad avere ragione con un'allarmante frequenza.> intervenne Fabio allegro. <Siete sicuri di stare bene, voialtri?>

Nessuno gli rispose.

Ad un certo punto, Vittoria si animò. <Io - com'è che dici te? Senza saper leggere o scrivere - >

<Per non saper né leggere né scrivere> la corresse al volo Fabio.

<Si, insomma, per non saper né leggere né scrivere - contento? - io vorrei daver, davvero farmi una bella doccia bollente.>

Gazzi non era d'accordo. <Lav, Deni, vi prego>, mugolò a bassa voce, mascherando il su oevidente terrore con un'estrema cautela. <Dobbiamo andare alla polizia. Il Fontanelli è appena ricomparso dopo mesi, e mi ha quasi ammazzato! E' armato, è pazzo! E' pericoloso, non ve ne potete andare come se niente fosse!>

Fabio gli si avvicinò lentamente, il volto imperscrutabile, la pistola sempre al sicuro nei pantaloni. 

<Ragazze> pigolò il Gazzi <ragazze, fermmmmmph

<Vecchio mio> cinguettò Fabio, tappandogli la bocca con la mano e parlando direttamente al suo orecchio-ì. <Ci andremo dalla polizia, non preoccuparti. Gli diremo tutto e il contrario di tutto, ci arresteranno e poi, se siamo fortunati ci farrano un bl processo che durerà anni>

Vittoria fece per dire qualcosa, ma Denise la interruppe mettendosi a ridere. <Scusate> fece dopo un po', <è che non ce lo vedo prorip Fabio a diventare il nuovo Pacciani>

<No?> fece Fabio, deluso. <Potrei fare il Vanni allora, ma un  Pacciani della situazione ci vuole. Gazzi, mi sa che ti tocca a farllo a te>

Giacomo Gazzi si liberò alla presa di Fabio, ma non disse niente, lo sguardo fisso su un pinto lontano. 

<Sù>, lo incalzò Lavinia. <Tu se' perfetto, merda tu se' merda uguale. 'gnamo, prova: Se n'ì mondo esistesse un po' di bene...>

Denise progesuì <...e ognun si considerasse su' fratello...>

Nonostante le imbeccate, Gazzi non proseguì.

Vittoria si limitò ad squadrarli stranita per un po', poi si guardò intorno, squittì e si voltò di scatto, come se si fosse tutt'a un tratto imbronciata.

<Ooh, non ci mettere il muso!> le fece Lavinia. <Si scherzava. Te non sei di qui, vero? Sai una sega te del mostro!>

La ragazza parlò in fretta. <No, ma non è per - cioè, sì, è anche per quello che stavate dicendo, ma è che sta arrivando qualcuno e non credo che noi dovremmo farci trovare qui.> 

In effetti, Due persone si stavano avvicinando alla combriccola. Uno di loro era magro e molto alto, l'altro un po' più basso ma comunque molto imponente, con il volto nascosto da un ampio cappuccio.

<Che mattinata...> sbuffò pesantemente Fabio.

Un tetro silenzio scese sul quintetto. L'aria si fece molto strana strana; come se fosse troppo stanca per caricarsi di tensione, si caricò di un totale, opprimente sensod i rassegnazione.

Gazzi finalmente parlò. <Non ci posso credere>, bisbigliò a sé stesso.

<Ah vecchio mio, io ormai non mi stupisco più di niente> disse Fabio.

<Io lui lo conosco!> esclamò Vittoria a voce alta, indicando la figura più alta. <Govi, mi pare? Govodi?>

<Ma chi, l'azzeccagarbugli di Bruno?> chiese sospettosa Denise. <Che ci fa qui a quest'ora?>

<L'ho chiamato io, però un secolo fa> le rispose Vittoria. <Mi ha dato una specie di cellulare enorme con un bottone solo.>

Fabio emise un imprecisato suono monotono, a metà fra un ringhio e un lamento.

<Vittoria> gorgogliò non appena ne fu in grado. <Vittoria, Vittoria santoddio *Vittoria* quandopensavididirmelocristoddio>

<Oh stai calmo eh, eri tutto preso da altro, e anche io ero tutta presa... dai ma cazzotene ho fatto quello che mi hai detto, eh!>

<Che mattinata, che mattinata!> ripeté lui, alzando le braccia al cielo. <Ragazzacci, date retta a un bischero. Ora voi state calmi, soprattutto voi ragazze, e ci penso io. Ormai ho una certa esperienza, fidatevi di me, queste cose vanno così: arriva qualcuno in questo cimitero del cazzo, io do di matto, litighiamo, ci sconvolgiamo, sbalordiamo e tutto quanto, e poi finisce tutto a tarallucci e vino. Solo che stavolta è un casino, io ve lo dico, se deve succedere qualcosa succede ora, mi spiego?>

<Fabietto ma che dici?> sbottò Lavinia. <Chi sta arrivando, perché devi dare di matto?>

Fabio, il volto più serio che mai, la guardò per dei lunghi istanti, e poi si soffermò con lo sguardo su tutti gli altri i presenti, uno dopo l'altro. <Darò di matto> dichiarò infine <perché sono quello più qualificato per farlo. In quanto a chi stia per arrivare, beh, ragazzacci, chi altro potrebbe mai essere? Nel giro di - quanto, due ore? - insomma, mi sono confrontato con Lavinia e con il caro Gazzi, e questa storia ormai va avanti da un bel po', ma manca ancora qualche tassello, non vi pare? Forza, non ci arrivate?>

Nessuno gli rispose.

<Non importa> mormorò deluso. <L'importante è che restiate calmi. Lavinia, Denise, soprattutto voi, mi raccomando.>

La coppia di figuri ormai era vicina.

<Gazzi, tu hai capito chi è, vero?> chiese fabio al suo acerrimo nemico. 

Egli annuì, il volto pietrificato da un'orrore senza precedenti.

<Lo sapevi?> insisté Fabio.

Lui scosse timidamente la testa.

Denise e Lavinia si limitarono a guardarsi confuse, ma Vittoria intervenne vivacemente. <Cosa?> domandò impaziente. <Non ci ho capito niente, chi è di tanto importante quel tizio insieme al Govi?>

<E' Govidi, signorina Meis> le fece la figura più alta, non appena fu sufficientemente vicino da poter comunicare senza urlare. <E riguardo al mio compagno io non rivelerò né accennerò in alcun modo alla sua identità.>

La figura incappucciata rise brevemente. <E' sempre sul pezzo lei, ingegnere>.

Anche Fabio rise; una risata lunga e sguaiata, senza alcun freno, degna del migliore dei pazzi. <Che vi dicevo? Ci ho dato! Chi altri poteva essere?> strillò, tentando vistosamente di controllarsi.

<Hai poco da ridere te, caro il mio Fontanelli.> lo ammonì la misteriosa figura incappucciata. <Dovevi fare una cosa sola, ed era tutto predisposto affinché tu la facessi; eppure *costui*> gesticolò verso il Gazzi <è ancora vivo, pronto a mettere le mani su tutto ciò che è mio>

<No, sei te che hai poco da ridere> ringhiò Fabio, tornato improvvisamente in sé. Si frugò nei pantaloni e ne estrasse teatralmente la pistola. <Togliti quel cappuccio, se hai il coraggio. Fatti vedere dai tuoi vecchi amici, lasciati riabbracciare.>

Repentinamente, anche Govidi estrasse una pistola.

Denise e Lavinia sospirarono all'unisono, evidentemente rassegnate ad dover sventare l'ennesima minaccia di carneficina nel corso della stessa mattinata. 

Vittoria mugolò <cominciano ad esserci un po' troppe pistole qui. il gioco è bello quando dura poco>

Fabio proseguì, per niente turbato dal fatto che non era più l'unica persona armata di quel bizzarro gruppo. <Bene, bene, bene> annunciò con l'aria di chi vuole tirare un'importante somma. <Provo a indovinare: tu, persona imprecisata che non identificherò per non rovinare la sorpresa alle nostre care ragazze, incoraggi la mia follia spedendomi gentilmente a fare in culo in un posto dove *sai* che c'è anche il Brogelli, magari sperando che mi faccia coinvolgere nell'abuso di qualche sostanza che mi faccia perdere definitivamente il poco senno che mi era rimasto. Fin qui ci sono?>

Govidi intervenne: <Questa parte non è andata proprio a buon fine. Nonostante tutto, lei mi sembra ancora piuttosto perspicace>

Fabio produsse un rabbioso grigno. <Se fossi stato perspicace>, si disse a denti stretti, <avrei mangiato la foglia quella domenica in quel cazzo di pub, invece che solo adesso.> Un puro disprezzo filtrò nella sua voce. <Com'è che dicevi? Eravamo rimasti soltanto io e te del vecchio gruppo, no?> Accennò con la testa verso Denise. <Non dovevo cascare nelle braccia di nessuno, vero?>

All'improvviso, la ragazza chiamata in causa si aggrappò forte a Lavinia, come se si sentisse travolta da una terribile rivelazione.

<Ma lasciamo perdere> concluse Fabio. <Questo è solo l'inizio, giusto? Mi hai fatto promettere di tornare per farmi scoprire che eri morto e darmi il colpo di grazia. Volevi proprio che fossi fuori di me, che non avessi più niente da perdere e che cercassi a tutti i costi un capro espiatorio per sfigare tutto l'odio, la rabbia e il dolore ->

<Lui voleva che tu mi uccidessi!> intervenne il Gazzi, esclamando incredulo. <E' stato lui ad aizzarti contro di m->

<Lo so bene, *sta' zitto!*> lo interruppe Fabio, puntandogli l'arma in faccia senza tante cerimonie. <Non ci è voluto poi cos' tanto, io già ti odiavo per conto mio, perché mi stai sul cazzo e pèerché avevo una fottuta paura che l'unica persona che mi amava arrivasse a preferire *te* a me!>

Fabio prese fiato, e Lavinia ne approfittò per rivolgersi a lui. <Noi si farà i conti più tardi> sbottò fin troppo dolcemente per risultare minacciosa. <Ora però credo che sverrò, perché le cose sono due: o non ci ho capito una bella sega, e spero proprio sia così, oppure mi sa che sarò io ad ammazzare qualcuno prima della fine.>

La figura incappucciata ridacchiò. <Mi dispiace, Lavinia, ma era proprio necessario che nessuno sapesse niente.>

<Era proprio necessario> gli fece eco la ragazza, sussurrando. <Hai una vaga idea - sì, certo che ce l'hai che grulla che sono> continuò, tremante di rabbia. <Ce l'hai perfettamente un'idea di quello che mi hai fatto passare - che CI hai fatto passare, a tutti quanti noi, E PER QUALE CAZZO DI MOTIVO!> 

La ragazza cadde sui ginocchi e cominciò a piangere copiose lacrime di rabbia. <Te sei morto, MORTO!> urlò. <Non puoi essere te, IO TI HO PIANTO! ORA TE TI SCAPPUCCI, TI FAI VEDERE, NOI TI SI FA DUE CAREZZINE E DUE MORMORII E POI TI SI METTE DOVE TU DEVI ->

<Ferma!> le ordinò Fabio, stringendo ossessivamente la sua pistola e mantenendola puntata fermamente contro l'ingegner Govidi. 

<Fermati Lavi, non ammattire anche te, ci penso io, non ti preoccupare.> Sospirò vistosamente, cercando di raccogliere la sua totale rassegnazione e di proiettarla sul resto del gruppo sotto forma di confidenza. < Ricapitoliamo, che la situazione è delicata. Te, mia cara Persona-Ormai-Non-Più-Tanto-Misteriosa, hai fatto tutto questo per farmi uccidere il Gazzi. Neghi di non aver tramato per evitare di non farmelo fare?>

<L'ho fatto> rispose tranquillo lui, evitando di confondersi con la quadrupla negazione.

Fabio abbassò l'arma, incredulo.

<Hai davvero messo in moto un piano così complicato, che non aveva quasi nessuna probabilità di funzionare, che avrebbe ferito così tante persone, solo per... solo per uccidere *lui*?>

<Proprio>

Fabio si grattò un improvviso prurito dietro la nuca. <Va bene, ho capito> bofonchiò. <C'è solo un'ultima cosa che non mi torna.>

<Forza, spara>

Tutto il dolore dell'ultimo anno penetrò in quell'unica parola: <Perché?>

Una risata proruppe da dietro quel cappuccio.

Fabio lo ignorò e si rivolse al Gazzi. <Giacomino> gli fece, sconsolato. <Ma che gli hai fatto?>

Altre risate emersero dal cappuccio, ma nessuna risposta arrivò dal Gazzi.

<Ragazzaccio> lo incalzò Fabio, gesticolando con la sua pistola <Ricordati che non te lo sto chiedendo per favore.>

Anche la pistola del Govidi finì puntata contro di lui. <Sottoscrivo la richiesta del buon Fontanelli> flautò, <ovviamente noialtri abbiamo già predetto e neutralizzato alcune delle sue mosse, ma sarebbe assai comodo se Giacomo Gazzi confessasse tutto e ci togliesse qualche dubbio su alcune delle sue trame.>

Il Gazzi ci mise un sacco di tempo a rispondere, e quando finalmente lo fece, usò un tono di voce monotono e straziato, come se gli stessero estorcendo la verità col fuoco. <Niente, in realtà.>

Fabio lo fissò per un po'. <Come *niente*?>, fece infine.

<Non ho *fatto* niente, se non dire le cose giuste alle persone giuste. Non avete niente - *niente* vi dico - contro di me, perché io non ho fatto assolutamente niente se non esprimere certe opinioni a certe persone.>

Fabio scoppiò a ridere, completamente spiazzato.

<col cazzo> eruppe Bruno Bagonghi, togliendosi finalmente il cappuccio. Sul suo volto come sempre ben curato, si dipingeva una selvaggia rabbia.

Si avvicinò minacciosamente al Gazzi, che si ritirò su sé stesso quasi a voler concentrare tutta la sua massa in un unico, minuscolo punto.

<Il tuo vero cognome non è Gazzi, dico bene?> sputò velenosamente. <Gazzi è il cognome di tua madre, il tuo vero cognome, quello che porti all'anagrafe, lo hai preso dal secondo marito di tua madre, è Gori. Ma non lo hai mai usato, forse per non associarti ad altri Gori, ad esempio lei.> Il suo sguardo guizzò verso Lavinia per un solo istante. <Ma non importa. Gazzi non è il tuo vero cognome, e allo stesso modo Gaetano Gori non è il tuo vero padre.>

Fabio intervenne, soffocato da un nuovo attacco di ilarità. <Ora gli dirai - ah! - sì, gli dirai che sei suo padre?>

Bruno lo degnò di un solo sguardo, l'espressione indecifrabile. <Quasi>, spifferò a bocca stretta.

<Vedi, caro il mio Giacomo>, proseguì, <quando è morto mio padre, io non ho potuto ereditare tutte le sue quote del lanificio ->

Il Gazzi, intuito improvvisamente dove il Bagonghi stava andando a parare, vinse la paura e intervenne. <Non sono tuo fratello, coglione!> gli sputò contro. <E anche se lo fossi, non fregherebbe niente a nessuno. Non ho io le quote che ti mancano.>

Govidi parlò, la pistola sempre puntata contro Fabio. <Al giorno d'oggi, lei, Giacomo Bagonghi, è socio del Lanificio Bagonghi; questo è un fatto. Tuttavia, data la recente trasformazione della forma societaria e dati gli ultimi... sviluppi, diciamo così, immaginavamo che lei potesse beneficiare dal cedere le sue quote di partecipazione a qualcuno di meno coinvolto. Ci dica a chi progetta di venderle, donarle o cederle in qualunque altro modo, e noi non avremo più da importunarla.>

<Io non ho ceduto proprio niente, e vi ripeto che non sono suo fratello!> ribadì il Gazzi, animandosi sempre di più. <Sapete solo quello che avete sentito dire, non avete niente, niente su di me! E poi la vostra società di merda fra qualche giorno porterà i libri in tribunale. Non saranno certo quelle quote a salvarvi.>

<Lo so bene> ringhiò il Bagonghi. <La società non può essere salvata, con buona pace dell'anima di nostro padre. Ma vedi, la mia morte ha sbloccato la situazione in un modo piuttosto interessante, e anche se non potrò salvare il lanificio dalle tue grinfie o da quelle del curatore che lo svenderà, almeno ho potuto salvare gran parte del suo patrimonio.>

Fabio rise un'altra volta.

<Che c'è?> lo sgridò il Bagonghi. <La nostra faida familiare tutt'a un tratto ti appassiona?>

<No, è che - cioè, sì, certo> sghignazzò Fabio. <Una faida familiare in cui te ti fingi morto per mettere alle strette uno che credi essere tuo fratellastro, beh, è sicuramente appassionante - a dire poco. Certo, non raggiunge le vette di follia che tocca il tuo piano per farmi uccidere il Gazzi - che poi, scusa, perché lo volevi morto se sei morto tu stesso per salvare i tuoi soldi? Lasciamo stare, non è che ho capito proprio tutto>

<E meno male che quello pazzo è il Fontanelli>, intervenne Denise, che aveva stampato in faccia lo sconforto più totale.

<Sì, infatti> concordò Fabio. <Ma vedete, la cosa divertente è un'altra. Mi sono improvvisamente accorto che io non ho più il mio cellulare, e che Vittoria è sparita.>

Lavinia e Denise si guardarono intorno; l'altra ragazza, in effetti, non c'era più.

Sul volto di Fabio si fece largo un'espressione che non vi compariva da tanto tempo: un sorriso. <Che spettacolo!>, esclamò con calore. <E così tutto si chiude. Era ovvio che il colpo di scena finale sarebbe dovuto essere il suo. E con il mio telefono, che ho comprato vendendo il suo! Che meraviglia.>

Tutti i presenti lo guardarono confusi.

<Eh sì, ragazzacci, torna tutto. Tutto! Persino le mie visioni avevano quasi senso, pensate un po'!>

Denise esternò la sua scetticità con un pretestuoso colpo di tosse.

Fabio la ignorò. Si avvicinò Bruno Bagonghi, la pistola sempre salda in mano, e lo guardò dritto negli occhi.

<Sei sempre stato tu il mio nemico> gli disse con estrema tranquillità. <Ma sei stati uno sciocco a venire qui stanotte, Bruno. La polizia sta per arrivare.>
