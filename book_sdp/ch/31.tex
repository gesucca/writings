\chapter{Occhio Non Vede}

// un po' troppo repentini questi passaggi, legare un po' meglio

Il sole fece capolino da dietro la coltre di nubi, inondando tutta Prato di un piacevole sole invernale.

<E ora che si fa?> chiese finalmente Vittoria.

<Che si fa...> ripeté Denise, meditabonda e imbarazzata. <Se Fabio... no, senza se: Fabio ha ragione. Non ci si può mettere a giustiziare un figlio di troia in un cimitero.>

<Anche se sono pazzo, comincio ad avere ragione con un'allarmante frequenza.> intervenne Fabio allegro. <Siete sicuri di stare bene, voialtri?>

Nessuno gli rispose.

Ad un certo punto, Vittoria si animò. <Io - com'è che dici te? Senza saper leggere o scrivere - >

<Per non saper né leggere né scrivere> la corresse al volo Fabio.

<Si, insomma, per non saper né leggere né scrivere - contento? - io vorrei daver, davvero farmi una bella doccia bollente.>

Gazzi non era d'accordo. <Lav, Deni, vi prego>, mugolò a bassa voce, mascherando il su oevidente terrore con un'estrema cautela. <Dobbiamo andare alla polizia. Il Fontanelli è appena ricomparso dopo mesi, e mi ha quasi ammazzato! E' armato, è pazzo! E' pericoloso, non ve ne potete andare come se niente fosse!>

Fabio gli si avvicinò lentamente, il volto imperscrutabile, la pistola sempre al sicuro nei pantaloni. 

<Ragazze> pigolò il Gazzi <ragazze, fermmmmmph

<Vecchio mio> cinguettò Fabio, tappandogli la bocca con la mano e parlando direttamente al suo orecchio-ì. <Ci andremo dalla polizia, non preoccuparti. Gli diremo tutto e il contrario di tutto, ci arresteranno e poi, se siamo fortunati ci farrano un bl processo che durerà anni>

Vittoria fece per dire qualcosa, ma Denise la interruppe mettendosi a ridere. <Scusate> fece dopo un po', <è che non ce lo vedo prorip Fabio a diventare il nuovo Pacciani>

<No?> fece Fabio, deluso. <Potrei fare il Vanni allora, ma un  Pacciani della situazione ci vuole. Gazzi, mi sa che ti tocca a farllo a te>

Giacomo Gazzi si liberò alla presa di Fabio, ma non disse niente, lo sguardo fisso su un pinto lontano. 

<Sù>, lo incalzò LAvinia. <Tu se' perfetto, merda tu se' merda uguale. 'gnamo, prova: Se n'ì mondo esistesse un po' di bene...>

Denise progesuì <...e ognun si considerasse su' fratello...>

Nonostante le imbeccate, Gazzi non proseguì.

Vittoria si limitò a guardarli stranita. <Sei troppo giovane> le disse lavinia. <E poi non sei toscana, vero? Sai una sega te del mostro>

<No, ma non è per - cioè, sì, è anche per quello che state dicendo, ma è che sta arrivando qualcuno e non credo che noi dovremmo farci trovare qui.> 

Due persone si stavano avvicinando alla combriccola. Uno di loro era magro e molto alto, l'altro un po' più basso ma comunque molto imponente, con il volto nascosto da un ampio cappuccio.

Un tetro silenzio scese sul quintetto. L'aria si fece molto strana strana; era troppo stanca per caricarsi di tensione, cercò di ...

Gazzi finalmente parlò. <Non ci posso credere>, bisbigliò a sé stesso.

<Ah vecchio mio, io ormai non mi stupisco più di niente> disse fabio

<Io lui lo conosco!> esclamò Vittoria a voce alta, indicando la figura più alta. <Govi, mi pare? Govodi?>

<Ma chi, l'azzeccagarbugli di Bruno?> chiese sospettosa Denise. <Che ci fa qui a quest'ora?>

<L'ho chiamato io, però un secolo fa> le rispose Vittoria. <Mi ha dato una specie di cellulare enorme con un bottone solo.>

Fabio emise un imprecisato suono monotono, a metà fra un ringhio e un lamento.<Vittoria> gorgogliò. <Vittoria, Vittoria santoddio Vittoria quandopensavididirmelo ->

<Oh stai calmo eh, eri tutto preso da altro, e anche io ero tutta presa... dai ma cazzotene ho fatto quello che mi hai detto, eh!>

Fabio guardò tutti i presenti, uno dopo l'altro. <Me ne frega> dichiarò infine <perché io con quello incappucciato stanotte ci ho parlato.>

<Stanotte?> sbuffò incredula Denise. <Ma se s'era tutti fatti come fegatelli!>

<Sì, d'accordo, forse l'ho sognato> ammise Fabio. <Di sicuro ero ancora drogato, non lo so, lì per lì  mi è sembrato familiare ma niente di strano, ma ora che ci penso *cazzo*. Sarò pazzo, ma non sono scemo. Lo so fare due più due>

// cosa come ma che dici eccetera
// sentite, lo so, io sono grullo e cmq vi pare strano che per la terxza volta da quando siamo qui arrivi qualcuno e succeda il casino, cmq questo vi dovrebbe tranquillizzare no? se ci ho dato la rivelazione sarà sconvolgente, ma non preoccupatevi, ci penso io a dare di matto

La coppia di figuri ormai era vicina.

<Gazzi, tu hai capito chi è, vero?> chiese fabio al suo acerrimo nemico. 

Egli annuì, il volto pietrificato da un'orrore senza precedenti.

<Lo sapevi?> insisté Fabio.

Lui scosse la testa. 

Denise e Lavinia si limitarono a guardarsi confuse, ma Vittoria intervenne vivacemente. <Cosa?> domandò Vittoria, impaziente. <Non ci ho capito niente, chi è di tanto importante quel tizio insieme al Govi?>

<E' Govidi, signorina Meis> le fece la figura più alta, non appena fu sufficientemente vicino da poter comunicare senza urlare. <E riguardo al mio compagno io non rivelerò né accennerò in alcun modo alla sua identità.>

La figura incappucciata rise brevemente. <E' sempre sul pezzo lei, ingegnere>.

Anche Fabio rise, ma la sua ilarità non fu affatto breve.

<Hai poco da ridere te, caro il mio Fontanelli.> lo ammonì la misteriosa figura incappucciata. <Dovevi fare una cosa sola, ed era tutto predisposto affinché tu la facessi; eppure *costui*> gesticolò verso il Gazzi <è ancora vivo, pronto a mettere le mani su tutto ciò che è mio>

<No, sei te che hai poco da ridere> ringhiò Fabio, frugandosi nei pantaloni ed estraendo teatralmente la pistola. <Togliti quel cappuccio, se hai il coraggio. Fatti vedere dai tuoi vecchi amici, lasciati riabbracciare.>

Repentinamente, anche Govidi estrasse una pistola.

Denise e Lavinia sospirarono all'unisono, evidentemente rassegnate ad dover sventare l'ennesima minaccia di carneficina nel corso della stessa mattinata.

Fabio proseguì, per niente turbato dal fatto che non era più l'unica persona armata di quel bizzarro gruppo. <Bene, bene, bene> annunciò con l'aria di chi vuole tirare un'importante somma. <Provo a indovinare: tu, persona imprecisata che non identificherò per non rovinare la sorpresa alle nostre care ragazze, incoraggi la mia follia spedendomi gentilmente a fare in culo in un posto dove *sai* che c'è anche il Brogelli, magari sperando che mi faccia coinvolgere nell'abuso di qualche sostanza che mi faccia perdere definitivamente il poco senno che mi era rimasto. Fin qui ci sono?>

Govidi intervenne: <Questa parte non è andata prprio a buon fine. Nonostante tutto, lei mi sembra ancora piuttosto perspicace>

Fabio produsse un rabbioso grigno. <Se fossi stato perspicace>, si disse a denti stretti, <avrei mangiato la foglia quella domenica in quel cazzo di pub, invece che solo adesso.> Un puro disprezzo filtrò nella sua voce. <Com'è che dicevi? Eravamo rimasti soltanto io e te del vecchio gruppo, no?> Accennò con la testa verso Denise. <Non dovevo cascare nelle braccia di nessuno, vero?>

All'improvviso, la ragazza chiamata in causa si aggrappò forte a Lavinia, come se si sentisse travolta da una terribile rivelazione.

<Ma lasciamo perdere> concluse Fabio. <Questo è solo l'inizio, giusto? Mi hai fatto promettere di tornare per farmi scoprire che eri morto e darmi il colpo di grazia. Volevi proprio che fossi fuori di me, che non avessi più niente da perdere e che cercassi a tutti i costi un capro espiatorio per sfigare tutto l'odio, la rabbia e il dolore ->

<Lui voleva che tu mi uccidessi!> intervenne il Gazzi, esclamando incredulo. <E' stato lui ad aizzarti contro di m->

<Lo so bene, *sta' zitto!*> lo interruppe Fabio, puntandogli l'arma in faccia senza tante cerimonie. <Non ci è voluto poi cos' tanto, io già ti odiavo per conto mio, perché mi stai sul cazzo e pèerché avevo una fottuta paura che l'unica persona che mi amava arrivasse a preferire *te* a me!>

Fabio prese fiato, e Lavinia ne approfittò per rivolgersi a lui. <Noi si farà i conti più tardi> sbottò fin troppo dolcemente per risultare minacciosa. <Ora però credo che sverrò, perché le cose sono due: o non ci ho capito una bella sega, e spero proprio sia così, oppure mi sa che sarò io ad ammazzare qualcuno prima della fine.>

La figura incappucciata ridacchiò. <Mi dispiace, Lavinia, ma era proprio necessario che nessuno sapesse niente.>

<Fermi, fermi!> ordinò Fabio, stringendo ossessivamente la sua pistola. <Ricapitoliamo, che la situazione è delicata. Te, persona ormai non più tanto misteriosa, hai fatto tutto questo per farmi uccidere il Gazzi. Neghi di non aver tramato per evitare di non farmelo fare?>

<L'ho fatto> rispose tranquillo lui, evitando di confondersi con la quadrupla negazione.

Fabio abbassò l'arma, incredulo.

<Hai davvero messo in moto un piano così complicato, che non aveva quasi nessuna probabilità di funzionare, che avrebbe ferito così tante persone, solo per... solo per uccidere *lui*?>

<Proprio>

ok ho capito, va bene quindi ho solo un ultima domanda...
perchééééééééééééé

***

Fabio cominciò a sbracciarsi
arriva bagonghi un casino guarda
