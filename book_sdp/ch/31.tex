\chapter{Occhio Non Vede}

// un po' troppo repentini questi passaggi, legare un po' meglio

Il sole fece capolino da dietro la coltre di nubi, inondando tutta Prato di un piacevole sole invernale.

<E ora che si fa?> chiese finalmente Vittoria.

<Che si fa...> ripeté Denise, meditabonda e imbarazzata. <Se Fabio... no, senza se: Fabio ha ragione.>

<Anche se sono pazzo, comincio ad avere ragione con un'allarmante frequenza.> intervenne Fabio allegro. <Siete sicuri di stare bene, voialtri?>

Nessuno gli rispose.

Ad un certo punto, Vittoria si animò. <Io - com'è che dici te? Senza saper leggere o scrivere - >

<Per non saper né leggere né scrivere> la corresse al volo Fabio.

<Si, insomma, per non saper né leggere né scrivere - contento? - io vorrei daver, davvero farmi una bella doccia bollente.>

Gazzi non era d'accordo. <Lav, Deni, vi prego>, mugolò a bassa voce, ostentando un'estrema cautela. <Dobbiamo andare alla polizia! Il Fontanelli è appena ricomparso dopo mesi, e mi ha quasi ammazzato! E' armato, è pazzo! E' pericoloso, non ve ne potete andare come se niente fosse!>

Fabio gli si avvicinò lentamente, il volto imperscrutabile, la pistola sempre al sicuro nei pantaloni. 

<Ragazze> pigolò il Gazzi <ragazze, fermmmmmph

<Vecchio mio> gli disse FAbio, tappandogli la bocca con la mano e parlando direttamente al suo orecchio-ì. <Ci andremo dalla polizia, non preoccuparti. Gli diremo tutto e il contrario di tutto, ci arresteranno e poi, se siamo fortunati ci farrano un bl processo che durerà anni>

Vittoria fece per dire qualcosa, ma Denise la interruppe mettendosi a ridere. <Scusate> fece dopo un po', <è che non ce lo vedo prorip Fabio a diventare il nuovo Pacciani>

<No?> fece Fabio, deluso. <Potrei fare il Vanni allora, ma un  Pacciani della situazione ci vuole. Gazzi, mi sa che ti tocca a farllo a te>

Giacomo Gazzi si liberò alla presa di Fabio, ma non disse niente, lo sguardo fisso su un pinto lontano. 

<Sù>, lo incalzò LAvinia. <Tu se' perfetto, merda tu se' merda uguale. 'gnamo, prova: Se n'ì mondo esistesse un po' di bene...>

Denise progesuì <...e ognun si considerasse su' fratello...>

Nonostante le imbeccate, Gazzi non proseguì.

Vittoria si limitò a guardarli stranita. <Sei troppo giovane> le disse lavinia. <E poi non sei toscana, vero? Sai una sega te del mostro>

<No, ma non è per - cioè, sì, è anche per quello che state dicendo, ma è che sta arrivando qualcuno e noi non credo che dovremmo essere qui.> 

Gazzi finalmente parlò. <Non ci posso credere>, bisbigliò a sé stesso.

Due persone si stavano avvicinando alla combriccola. Uno di loro era magro e molto alto, l'altro un po' più basso ma comunque molto imponente, con il volto nascosto da un ampio cappuccio.

<Io lui lo conosco!> esclamò Vittoria a voce alta, indicando la figura più alta. <Govi, mi pare? Govodi?>

<Ma chi, l'azzeccagarbugli di Bruno?> chiese sospettosa Denise. <Che ci fa qui a quest'ora?>

<L'ho chiamato io, però un secolo fa> le rispose Vittoria. <Mi ha dato una specie di cellulare enorme con un bottone solo.>

Fabio emise un imprecisato suono monotono, a metà fra un ringhio e un lamento.<Vittoria> gorgogliò. <Vittoria, Vittoria santoddio Vittoria quandopensavididirmelo ->

<Oh stai calmo eh, eri tutto preso da altro, e anche io ero tutta presa... dai ma cazzotene ho fatto quello che mi hai detto, eh!>

Fabio guardò tutti i presenti, uno dopo l'altro. <Me ne frega> dichiarò infine <perché io con quello incappucciato stanotte ci ho parlato. Forse l'ho sognato, forse ero ancora drogato, non lo so, lì per lì  mi è sembrato familiare ma niente di strano, ma cazzo. Sarò pazzo, ma non sono scemo. Lo so fare due più due>

La coppia di figuri ormai era vicina.

<Gazzi, tu lo sapevi?> chiese fabio al suo acerrimo nemico. Lui scosse la testa. <Nemmeno lo sospettavi?> <Ovviamente lo sospettavo> rispose lui.

Denise e Lavinia si guardarono, confuse.

<Cosa?> domandò Vittoria, impaziente. <Non ci ho cpaito niente, chi è di tanto importante quel tizio insieme al Govi?>

<E' Govidi, signorina Meis> le fece la figura più alta, non appena fu sufficientemente vicino da poter comunicare senza urlare. <E riguardo al mio compagno io non rivelerò né accennerò in alcun modo alla sua identità.>



Fabio cominciò a sbracciarsi
arriva bagonghi un casino guarda
