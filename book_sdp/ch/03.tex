\chapter{Flashback: il Fuoco}

\begin{chapquote}{Simone Cipriani, \textit{Venga il Buio}}
	Quelle luci nei locali\newline
	Morte liquida da bere\newline
	Nubi tossiche mortali\newline
	E la vita può aspettare
\end{chapquote}

% pulled
% tex syntax

Turbinii colorati, lampeggii arzigogolati. Sferzavano lo spazio, rimanendo impressi nella realtà per diversi istanti, prima di venire sovrascritti da altri guizzi colorati. Il danzare frenetico degli artefatti psichedelici era così regolare da essere ipnotizzante: tutto accadeva e smetteva di esistere armonicamente, rispettando proporzioni temporali sconosciute ma perfette. Sembrava una danza destinata ad essere eterna: tutta quella perfezione non poteva che durare per sempre. Nonostante l'eternità intrinseca, l'insieme aveva una sua progressività: ogni ciclo cromatico faceva acquisire solidità al tutto. Piano piano, senza perdere il moto perpetuo che il sistema aveva acquisito, comparve qualcosa di molto simile alla realtà concreta: c'erano tavolini bassi, da fumo, e divanetti di finta pelle scadente.

Fabio era seduto su uno di essi. Osservava i resti del bicchiere di plastica che aveva appena rotto, maledicendo la sua stretta compulsiva. Sembrava visibilmente stanco, ma più probabilmente era solo un po' brillo. Voleva prendere altro da bere, ma gli faceva tanta fatica: il divanetto era così avvolgente, la palpebra così pesante\ldots

La sua sete lo costrinse a vincere la stanchezza. Si alzò, attraversò la stanza gremita di gente e si accinse a preparare un cocktail al tavolo del buffet. ``Che gin di merda'', pensò mischiando l'alcolico scadente con rum chiaro e vodka. Mentre esternava il suo disappunto con un suono a metà fra un grugnito ed un ringhio, si accorse che mancava il triple sec.

«\textit{`io serpente maiale!}» imprecò a denti stretti, mangiandosi una consonante.

Buttò via il suo abbozzo di Long Island e, esasperato, scrutò il tavolo, maledicendo chi avesse scelto le bottiglie da portare. Concluse che l'unica cosa che avrebbe potuto preparare senza scendere a compromessi disumani fosse un Vodka Tonic  ``Almeno la vodka è decente'', si convinse, mentre cercava con crescente disperazione un bicchiere più o meno pulito.

Preparato il suo misero drink, ne bevve un sorso e si guardò intorno. Il suo sguardo scorse su varie persone, senza esserne rapito; dribblò accuratamente una ragazza bionda e si soffermò infine su un tizio che somigliava ad un becchino, appoggiato al muro dalla parte opposta della stanza.

``Gazzi, quanto sei patetico'' pensò Fabio. ``Sarò disperato, cornuto, magari anch'io patetico, ma almeno non sono te''.

Il Gazzi gli restituì lo sguardo ammiccando, poi riprese ad osservare gli altri con distacco. Era un ragazzo di media statura, aveva la pelle olivastra e i capelli neri; il suo volto presentava dei tratti estremamente comuni, quasi addirittura banali. Fabio continuò a guardarlo, incapace di fermare il flusso suo di pensieri, che ormai scaturiva libero dalla sua mente non proprio lucida.

Le scarpe del Gazzi lo infastidivano parecchio. A dire il vero, odiava ogni cosa che quell'idiota aveva addosso, dalla camicia di sartoria con la cifratura sul colletto, all'orologio Panerai da diverse migliaia di euro. Però le sue scarpe\ldots le sue maledette scarpe lo mandavano proprio in bestia. Cosa diavolo aveva nella testa? Quale assurdo processo logico gli aveva permesso di discernere che indossare delle scarpe del genere in quel contesto fosse una buona idea? In fondo, c'era un motivo per cui nove persone su dieci portavano gli anfibi, e non era solo il fatto di essere ad una festa di simil-rockettari in un locale underground. La stanza era in condizioni pietose, così come metà dei suoi occupanti. Per terra c'era di tutto, dalla cenere al vomito, perfino qualche persona troppo provata per mantenere la propria dignità a livelli accettabili. Ogni passo rischiava di far perdere valore per diverse centinaia di euro a quelle scarpe. ``Gazzi io ti ammazzerò'' pensò Fabio, mentre con un furioso sorso seccava la sua bevuta, ``sei troppo stupido per vivere''.

«Fontanelli! Tutto a posto? Vuoi uccidere il buon Gazzi?»

Fabio trasalì e si voltò: un ragazzo biondo dal volto pallido e scavato apparve davanti a lui.

«Dani» prese a rispondere, un po' turbato dal fatto che l'inquietante figuro avesse indovinato i suoi pensieri, «certo che lo voglio uccidere, chi non vorrebbe?»

«Beh, zio, chi non vorrebbe non lo so, ma l'aria assassina che avevi mentre lo scrutavi non lasciava dubbi. Potevo vedere il fuoco del nazionalsocialismo avvamparti nello sguardo mentre progettavi la spedizione in Siberia di Giacomo Gazzi. Se ti avessi lasciato elucubrare un altro po' si sarebbe sentita la colonna sonora dei tuoi pensieri».

«Cioè?» chiese Fabio, divertito. Adorava gli sproloqui totalitaristi di Daniele Brogelli, discorsi privi di qualsiasi significato ma esilaranti e terrificanti allo stesso tempo.

Lui gonfiò il petto e proclamò in tono solenne, gesticolando pomposamente:

«Un rumore\ldots anzi no, un coro, perché è musica\ldots Un coro di anfibi che marciano, calpestando facce bolsceviche; urla di prigionieri politici che, martoriati dalle fruste, poggiano le pietre per la costruzione della ferrovia transiberiana, sulla quale i leader della rivoluzione sfileranno trionfanti godendo della sofferenza che ne permise la costruzione! Allora Egli si desterà dal suo sonno, la divisa marmorea sulla fronte lambita dal vento, e marcerà trionfante sulla Piazza Rossa, acclamato da pugni alzati e saluti romani!»

La risata di Fabio eruppe.

«Dani, sei troppo sobrio», gli disse non appena riuscì a riprendere fiato, «fai discorsi fin troppo sensati. Fatti fare un beverino prima che il Leka ti senta e ti prenda sul serio.»

Fabio preparò altri due Vodka Tonic, pregando che Daniele non facesse caso al fatto che per lui aveva preso un bicchiere vuoto abbandonato sul tavolo da chissacchì.

«Insomma zio», parlò Daniele, «questo Gazzi non la vuole smettere di fare il dandy nemmeno qui al Cipher, eh?»

Fabio osservò per un attimo il volto sciupato dalla droga prima di rispondere. Daniele, con tutti i difetti che poteva avere, era affidabile. Il suo passato da tossicodipendente e le cicatrici che aveva lasciato nel suo presente lo rendevano un sicuro confessore. Se mai avesse deciso di divulgare segreti in giro, non sarebbe stato preso molto sul serio. A Fabio bastava: decise che poteva intraprendere con lui questa conversazione.

«Era proprio quello a cui stavo pensando quando lo guardavo con aria omicida», gli rispose.

«Zio, io non lo biasimo troppo. E' vuoto dentro, mi capisci vero? Ha bisogno di fare così.»

«Ma dove prende i soldi per vestirsi a quel modo? Voglio dire, io non potrei mai permettermi tutta quella roba, se mai avessi voglia di comprarmela.»

«Fa sacrifici, sicuramente. Vedi che va sempre a scrocco: non si compra mai le bombe, i beverini li gattona agli altri, cose così. Sinceramente dopo un po' scoccia anche, ma è una persona così piccola che non me la sento di dirgli mai niente. Tu lo odi, è normale che lo odi\ldots»

«Quindi, praticamente quasi non mangia per andare in giro vestito da milionario?»

«Sì, secondo me sì. Lavora e vive per i pochi momenti in cui può sentirsi davvero vivo, mi capisci? L'ho fatto anche io, so cosa significa\ldots»

Fabio pensò. Praticamente Gazzi si drogava con la sua apparenza: gli piaceva così tanto credersi migliore dei quattro stronzi che frequentava da sacrificare ogni altra cosa nella sua vita in favore di quella sensazione. Non riusciva proprio a provare pietà per lui, solo puro disgusto.

«Dai zio, basta parlare di cose brutte, fanne tre.»

Si misero a sedere e cominciarono a lavorare.

«Con la Denise come va?» chiese Fabio, mentre trafficava con il suo borsello in cerca dell'occorrente.

«Come sempre, zio. Lo sai come siamo fatti, non credo cambieremo mai. Godiamo dei nostri pregi e sopportiamo i nostri difetti. Certo, a volte sopportiamo più di quanto godiamo, ma viviamo per quando succede il contrario. Abbiamo imparato ad andarci bene così, non ci chiediamo niente di ciò che non possiamo darci. Siamo liberi.»

«Beati voi» sospirò Fabio, soppesando la mista e decidendo che la materia prima era sufficiente.

«Te e la tua bella ormai siete al capolinea?»

«Magari la situazione fosse così definita. O forse lo è, ma io non riesco a trovarne un senso. Lei ormai\ldots È palese, ma fa finta di niente. Fa sempre finta di niente\ldots Ho cominciato a farlo anch'io. E anche lui fa finta di niente, il nostro caro Gazzi! Accidenti a lui e alla sua stirpe! Si comportano come se non stesse succedendo niente, tutti quanti, ma nel frattempo tutto il mondo sa che ho le corna!»

«Zio, io non ti ho mai detto niente, ma le voci come arrivano a me arrivano anche a te, è chiaro.»

«Non è solo questo, comunque. Sta andando tutto a puttane Dani, veramente tutto: ormai non ho più un lavoro, o una qualsiasi fonte di reddito, né una famiglia che mi possa aiutare\ldots Non ho più voglia di trovare giustificazioni assurde alla merda che mi sta arrivando addosso, voglio\ldots Voglio smettere.»

«Stai pensando al suicidio?»

Il volto di Daniele appariva ancora più sciupato del solito, ma nei suoi occhi c'era un risoluto fervore. Fabio sostenne il suo sguardo.

«No» rispose amaramente. «Piuttosto mi armo fino ai denti e comincio ad ammazzare gente a caso, finché non mandano l'esercito a fermarmi. Poi erre uno, erre uno, tondo, erre due, su giù tre volte e ricomincio. Cazzate a parte, perché dovrei? Ce l'ho con gli altri, non con me. Chi si suicida\ldots È un debole.»

Ci fu un lungo istante in cui l'atmosfera si addensò, facendo precipitare sulla scena un assordante silenzio.

Infine, Daniele parlò. «Non diresti così, amico, se avessi passato tutto quello che ho passato io.»

«Scusa» disse Fabio, vergognandosi un po' per la sua mancanza di tatto.

Il tema era molto delicato per Daniele. A causa della sua tossicodipendenza --- e dei motivi che lo avevano condotto ad essa --- aveva attraversato momenti di pura disperazione.

«Il suicidio può essere generosità, non resa», spiegò. «Può essere il sacrificio della tua esistenza in favore di quella dei tuoi cari. Se tu fossi un peso economico per la tua famiglia, un motivo di vergogna ed un pericolo per la loro incolumità, toglierti la vita sarebbe la cosa più altruista che tu potresti fare. Ma sarebbe anche la cosa più facile e più povera di amore per te stesso.»

Gli occhi di Daniele si inumidirono un po'. Fabio abbassò lo sguardo in forma di rispetto: Daniele era più grande di lui ed infinitamente più saggio. L'oscurità dalla quale era riemerso sminuiva qualunque altro tormento. Non avrebbe dovuto permettersi di apostrofarlo in quel modo.

«Dani, scusa, davvero. Sono stato indelicato.»

«No, Fontanelli, sei stato onesto. Sei fatto così, non ti interessa se quello che dici ferisce il prossimo. Ma almeno sei vero, perciò ti perdono; di finta commiserazione ne ho avuta abbastanza. \emph{Appizza `sta fiamma, zio.}»

Il fumo vorticava turbolento verso il soffitto, prima di ricadere dolcemente avvolgendo i due ceffi in una mistica nebbia. Lo stordimento penetrava deliziosamente la corazza di ostentata sobrietà di Fabio, facendo capitolare la sua patetica maschera e costringendolo ad apparire com'era veramente: ubriaco, stupefatto e stravolto dalla stanchezza. Neanche il tempo di godersi quello stato di nuda trascendenza, che un evento catturò l'attenzione di tutti i presenti. Le risse erano all'ordine del giorno in locali come quello, eppure il gruppo della festa era sempre stato piuttosto tranquillo, relativamente al tipo di gente da cui era composto.

«Ebreo di merda, t'ammazzo!» sbraitò un ragazzo enorme dai capelli rossi, gettandosi contro un tizio alto e un po' torto.

I due si aggrovigliarono in una pietosa lotta a terra. Parecchie persone accorsero per rimetterli in piedi, anche se nessuno aveva la reale intenzione di interrompere le ostilità. Fabio non si mosse nemmeno dal suo divanetto, del quale ormai faceva parte, ma osservò con vivo interesse la scena: i due combattenti erano suoi amici.

«Abbozzala» ansimò il ragazzo torto, piantando un sonoro montante nello stomaco al suo avversario, stendendolo, ma scivolando di nuovo a terra per lo sforzo.

Neanche il tempo di riprendere fiato che il rosso si rialzò inviperito, assestandogli due pedate sulle gengive, probabilmente rompendogli qualche dente.

«Che la fa'e finita!», urlò furiosa la ragazza bionda di cui Fabio tentava disperatamente di ignorare l'esistenza.

Udita la sua voce, la coppia di combattenti si congelò all'istante.

«Che c'ave'e?» continuò lei. «O andae a' ammazzavvi fòri, o l'abbozzate! Che ave'e `nteso?»

I due si lasciarono andare, non osando disubbidire alla ragazza, anche se dai loro sguardi si intuiva che avrebbero voluto combattere fino all'ultimo respiro.

«Sei un cazzo di ebreo, infame di merda!» cominciò a sbraitare il ragazzo dai capelli rossi, schifosamente ubriaco. «Che cazzo ti ho fatto?! Eh?! Sei un infame, un infame\ldots Fai schifo, speriamo tu muoia, sempre a pensare ai soldi, ebreo di merda\ldots»

«O' Anton, la fa' finita? Bada `ome tu l'ha' conciato! Oh'icché t'arà ma' fatto i' Bagonghi?» sbottò ancora la ragazza, esasperata.

Anton non sembrò minimamente intenzionato a rispondere: prese ed andò via, senza smettere di borbottare confuse maledizioni.

Il Bagonghi, senza osare rialzarsi da terra, si mise a sedere a fatica, sputò una boccata di sangue misto a denti e mugolò:

«Carabinieri\ldots La bamba\ldots Non potevo proteggerlo, mi sono tirato fuori\ldots Avrei chiuso\ldots Forse chiuderò comunque, ma ci mancava soltanto lui\ldots»

«I carabinieri hanno trovato la droga al Leka? E te? L'hai licenziato? `o cane, e ci credo l'è `ncazza'o!» lo incalzò la ragazza.

Bagonghi annuì, sputò, sillabò una bestemmia e poi rispose:

«Lavinia\ldots Che potevo fare\ldots Non li ho mai corrotti i carabinieri, non ho avuto scelta, non potevo impedirgli di entrare e di frugare dove volevano, c'era il mandato\ldots»

Fabio, vinta la repulsione che aveva ad avvicinarsi a Lavinia, si scollò dal divanetto e, insieme al Brogelli, alzò il suo amico da terra.

Dopo essersi rimesso in piedi, egli continuò:

«Quell'idiota del Leka crede che li abbia chiamati io i caramba, l'altra settimana abbiamo litigato\ldots Si fa troppo i cazzi suoi, mi ha fatto degli errori che mi hanno fatto perdere diversi soldi\ldots Crede che volessi una scusa per licenziarlo e che non mi importasse di rovinargli la vita\ldots»

«Non posso credere che tu abbia fatto una cosa del genere!» esclamò Daniele.

«No che non l'ho fatta!» replicò brusco il Bagonghi, sputacchiando sangue e saliva ovunque. «L'ho solo rimproverato perché mi aveva fatto perdere dei soldi, non lo voglio far schedare! Non so come cazzo hanno fatto i caramba ad avere un mandato\ldots»

«Certo è proprio stupido» intervenne Fabio, incapace di trattenersi. «Cosa ti porti a fare la coca al lavoro? Una bottarella in pausa caffè, così la giornata prende tutta un'altra piega?»

«Credo la spacciasse a qualche mio dipendente», disse il Bagonghi, ammiccando in una certa direzione. «Ma non me ne frega niente, se non l'avessero beccato non lo avrei licenziato, che cazzo me ne frega della droga!»

«Già, che cazzo te ne frega della droga\ldots» ripeté meccanicamente Fabio, assorto nell'osservare qualcuno a cui, nella più completa ipocrisia, della droga fregava eccome, soprattutto quando poteva ostentare il fatto di potersela permettere. Perché non riusciva a smettere di tirarlo in ballo in qualsiasi suo pensiero?

Le cose stavano cominciando ad assumere nuovamente la loro forma eterea ed i suoni si stavano ovattando sempre di più. Ma la scena non svanì abbastanza in fretta: Lavinia si accorse che Fabio stava guardando il Gazzi, probabilmente sfoggiando il suo sguardo trucidatore. Fece per avvicinarsi, forse stava per parlare, ma Fabio la bloccò.

«Lascia stare, lascia stare\ldots», le sussurrò.

Il suo sguardo incrociò quello della ragazza: amaramente, ci sprofondò dentro. Come una dolce morfina che neutralizza il dolore al malato, l'ambiente psichedelico riapparve prepotente, cancellando ogni barlume di realtà.
